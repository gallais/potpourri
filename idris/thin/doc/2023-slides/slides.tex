\PassOptionsToPackage{x11names}{xcolor}
\documentclass{beamer}

\bibliographystyle{alpha}
\beamertemplatenavigationsymbolsempty

\usepackage{tikz}
\usetikzlibrary{shapes.geometric}

\usepackage{listings}

\lstset{ %
  language=C,
  numbers=left,
  numberstyle=\tiny,
  stepnumber=1,
  numbersep=5pt,
  breaklines=true,
}

\usepackage{idris2}
\usepackage{catchfilebetweentags}
\makeatletter

\newrobustcmd*\OrigExecuteMetaData[2][\jobname]{%
\CatchFileBetweenTags\CatchFBT@tok{#1}{#2}%
\global\expandafter\CatchFBT@tok\expandafter{%
\expandafter}\the\CatchFBT@tok
}%\OrigExecuteMetaData

\newrobustcmd*\ChkExecuteMetaData[2][\jobname]{%
\CatchFileBetweenTags\CatchFBT@tok{#1}{#2}%
\edef\mytokens{\detokenize\expandafter{\the\CatchFBT@tok}}
\ifx\mytokens\empty\PackageError{catchfilebetweentags}{the tag #2 is not found\MessageBreak in file #1 \MessageBreak called from \jobname.tex}{use a different tag}\fi%
}%\ChkExecuteMetaData

\renewrobustcmd*\ExecuteMetaData[2][\jobname]{%
\ChkExecuteMetaData[#1]{#2}%
\OrigExecuteMetaData[#1]{#2}%
}

\makeatother


\title{Builtin Types viewed as Inductive Families}
\author{Guillaume Allais}
\institute{University of Strathclyde \\ Glasgow, UK}
\date{ESOP'23 \\ $24^{th}$ April 2023}

\begin{document}

\begin{frame}
  \maketitle
\end{frame}

\begin{frame}
  \frametitle{Table of Contents}
  \tableofcontents
\end{frame}

\section{Compiling Inductive Families: State of the Art}

\begin{frame}{Safe Lookup}
\ExecuteMetaData[Lookup.idr.tex]{vect}

\ExecuteMetaData[Lookup.idr.tex]{fin}

\ExecuteMetaData[Lookup.idr.tex]{vectlookup}
\end{frame}

\begin{frame}{Compiling Safe Lookup}

Brady, McBride, and McKinna~\cite{DBLP:conf/types/BradyMM03}

\ExecuteMetaData[Lookup.idr.tex]{erasedvectlookup}

\vfill

Tejiščák~\cite{DBLP:journals/pacmpl/Tejiscak20}

\ExecuteMetaData[Lookup.idr.tex]{finallookup}
\end{frame}

\begin{frame}{No Magic Solution}

Quoting from the Coq reference manual:\bigskip

\begin{quote}
Translating an inductive type to an arbitrary ML type
does \textbf{not} magically improve the asymptotic complexity of
functions, even if the ML type is an efficient representation.
For instance, when extracting \texttt{nat} to OCaml \texttt{int}, the
function \texttt{Nat.mul} stays quadratic.
\end{quote}
\end{frame}

\AtBeginSection[]
{
    \begin{frame}
        \frametitle{Table of Contents}
        \tableofcontents[currentsection]
    \end{frame}
}

\section{Motivation: Co-De Bruijn syntax}

\begin{frame}{Named: $\lambda{}g.\lambda{}f.\lambda{}x.g\,x\,(f\,x)$}
\vfill
\ExecuteMetaData[ast.tex]{named}

\vfill
Hard: $\alpha$-equivalence
\end{frame}

\begin{frame}{De Bruijn: $\lambda{}g.\lambda{}f.\lambda{}x.g\,x\,(f\,x)$}

\ExecuteMetaData[ast.tex]{debruijn}

\vfill
Hard: thickening
\end{frame}

\begin{frame}{Co-De Bruijn~\cite{DBLP:journals/corr/abs-1807-04085}: $\lambda{}g.\lambda{}f.\lambda{}x.g\,x\,(f\,x)$}

\ExecuteMetaData[ast.tex]{codebruijn}
\end{frame}

\section{Motivation: Thinning Representations}

\begin{frame}{Safe: Inductive Family}
\ExecuteMetaData[Thinnings.idr.tex]{compactfamily}

\vfill

Compiled to:

\ExecuteMetaData[Thinnings.idr.tex]{bare}

i.e. essentially \IdrisType{List} \IdrisType{Bool}
\end{frame}

\begin{frame}{Unsafe: Tuple}
\ExecuteMetaData[Thinnings.idr.tex]{unsafe}

\vfill

$$
\text{\IdrisFunction{bitPattern}} = \cdots0000\underbrace{1001\cdots11}_{\text{\IdrisFunction{bigEnd} bits}}
$$

\end{frame}

\section{Solution: Quantitative Type Theory}

\begin{frame}{Thinnings}

\ExecuteMetaData[Thin.idr.tex]{thin}
\vfill
\uncover<2->{\ExecuteMetaData[Thin/Internal.idr.tex]{thinning}}

\end{frame}

\begin{frame}{Smart constructors}

\ExecuteMetaData[Thin.idr.tex]{done}

\end{frame}

\begin{frame}{Smart constructors}
\ExecuteMetaData[Thin.idr.tex]{compactkeep}

\end{frame}

\begin{frame}{Smart constructors}
\ExecuteMetaData[Thin.idr.tex]{compactdrop}

\end{frame}

\begin{frame}{View~\cite{DBLP:conf/popl/Wadler87}: ``un-applying'' the smart constructors}
\ExecuteMetaData[Thin.idr.tex]{view}

\ExecuteMetaData[Thin.idr.tex]{viewtotal}
\end{frame}

\begin{frame}{Using the view}
\ExecuteMetaData[Thin.idr.tex]{kept}
\end{frame}


\section{Conclusion}

\begin{frame}{Results \& Limitations}

Results:

\begin{itemize}
  \item Functional, proven correct, library
  \item Generates good code
  \item Choose your own abstraction level
\end{itemize}

\vfill

Limitations:

\begin{itemize}
  \item Smart constructors \& view are only \emph{provably} inverses
  \item \IdrisType{Invariant} is proof irrelevant but it may not always be possible
  \item Inverting proofs with a \IdrisKeyword{0} quantity is currently painful
\end{itemize}
\end{frame}

\begin{frame}{Future Work}
\begin{itemize}
  \item A generic bitvector library
  \item Building such representations in a compositional manner
  \item Imperative structures with destructive updates
\end{itemize}
\end{frame}



\begin{frame}{Bibliography}
\bibliography{thin}
\end{frame}
\end{document}
