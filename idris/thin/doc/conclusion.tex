
%%%%%%%%%%%%%%%%%%%%%%%%%%%%%%%%%%%%%%%%%%%%%%%%%%%%%%%%%%%%%%%%%%%%%%%%%%%%%%
%% conclusion
%%%%%%%%%%%%%%%%%%%%%%%%%%%%%%%%%%%%%%%%%%%%%%%%%%%%%%%%%%%%%%%%%%%%%%%%%%%%%%

\section{Conclusion}\label{sec:conclusion}

We have seen that inductive families provide programmers with ways to root out bugs
by enforcing strong invariants. Unfortunately these families can get in the way of
producing performant code despite existing optimisation passes erasing redundant
or runtime irrelevant data.
%
This tension has led us to take advantage of Quantitative Type Theory
in order to design a library
combining the best of both worlds: the strong invariants and ease of use of inductive
families together with the runtime performance of explicit bit manipulations.

\subsection{Related Work}

For historical and ergonomic reasons, idiomatic code in \coq{} tends to center programs
written in a subset of the language quite close to OCaml and then prove properties
about these programs in the runtime irrelevant \texttt{Prop} fragment.
%
This can lead to awkward encodings when the unrefined inputs force the user to consider
cases which ought to be impossible. Common coping strategies involve relaxing the types
to insert a modicum of partiality e.g. returning an option type or taking an additional
input to be used as the default return value.
%
This approach completely misses the point of type-driven development.
%
We benefit a lot from having as much information as possible available during
interactive editing.
%
This information not only helps tremendously getting the definitions right by
ensuring we always maintain vital invariants thus making invalid states
unrepresentable, it also gives programmers access to type-driven tools and automation.
%
Thankfully libraries such as Equations~\cite{DBLP:conf/itp/Sozeau10,DBLP:journals/pacmpl/SozeauM19}
can help users write more dependently typed programs, by taking care of the complex
encoding required in \coq{}. A view-based approach similar to ours but using \texttt{Prop}
instead of the zero quantity ought to be possible.

Prior work on erasure~\cite{DBLP:journals/pacmpl/Tejiscak20} has the advantage of
offering a fully automated analysis of the code. The main inconvenience is that users
cannot state explicitly that a piece of data ought to be runtime irrelevant and so
they may end up inadvertently using it which would prevent its erasure.
%
Quantitative Type Theory allows us users to explicitly choose what is and is not
runtime relevant, with the quantity checker keeping us true to our word.
%
This should ensure that the resulting program has a much more predictable complexity.

A somewhat related idea was explored by Brady, McKinna, and Hammond in the context of
circuit design~\cite{DBLP:conf/sfp/BradyMH07}. In their verification work they index
an efficient representation (natural numbers as a list of bits) by its meaning as a
unary natural number. All the operations are correct by construction as witnessed by
the use of their unary counterparts acting as type-level specifications.
%
In the end their algorithms still process the inductive family instead of working
directly with binary numbers. This makes sense in their setting where they construct
circuits and so are explicitly manipulating wires carrying bits.
%
By contrast, in our motivating example we really want to get down to actual (unbounded)
integers rather than linked lists of bits.

%\todo{IIRC Iris does pointer manipulations. What about bit masks?
%  High level invariants linking memory-mapped data to high level concepts?}

\subsection{Limitations and Future Work}

Overall we found this case study using \idris{}, a state of the art language
based on Quantitative Type Theory, very encouraging.
%
The language implementation is still experimental (see for instance
\cref{appendix:limitations} for some of the bugs we found) but none of
the issues are intrinsic limitations.
%
We hope to be able to push this line of work further, tackling the following
limitations and exploring more advanced use cases.

Unfortunately it is only \emph{propositionally} true that
(\IdrisFunction{view} (\IdrisFunction{keep} \IdrisBound{th} \IdrisBound{x}))
computes to (\IdrisData{Keep} \IdrisBound{th} \IdrisBound{x}) (and similarly for
\IdrisFunction{done}/\IdrisData{Done} and \IdrisFunction{drop}/\IdrisData{Drop}).
%
This means that users may need to manually deploy these lemmas when proving the
properties of functions defined by pattern matching on the result of calling the
\IdrisFunction{view} function.
%
This annoyance would disappear if we had the ability to extend \idris{}'s reduction rules
with user-proven equations as implemented in Agda and formally studied
by Cockx, Tabareau, and Winterhalter~\cite{DBLP:journals/pacmpl/CockxTW21}.

In this paper's case study, we were able to design the core \IdrisType{Invariant}
relation making the invariants explicit in such a way that it would be provably
proof irrelevant.
%
This may not always be possible given the type theory currently implemented by
\idris{}. Adding support for a proof-irrelevant sort of propositions (see e.g.
Altenkirch, McBride, and Swierstra's work~\cite{DBLP:conf/plpv/AltenkirchMS07})
could solve this issue once and for all.

Our objectives with this work and that to come are twofold: we would like to
explore more memory-mapped representations equipped with a high level interface
and extend language support for this style of programming.

The \idris{} standard library thankfully gave us access to a polished pure interface
to explicitly manipulate an integer's bits.
%
However these built-in operations came with no built-in properties whatsoever.
%
And so we had to postulate a (minimal) set of axioms (see \cref{appendix:postulated})
and prove a lot of useful corollaries ourselves.
%
There is even less support for other low-level operations such as reading from
a read-only array, or manipulating pointers.

We also found the use of runtime irrelevance (the \IdrisKeyword{0} quantity)
sometimes somewhat frustrating.
%
Pattern-matching on a runtime irrelevant value in a runtime relevant context
is currently only possible if it is manifest for the compiler that the value
could only arise using one of the family's constructors.
%
In non-trivial cases this is unfortunately only merely provable rather than
self-evident.
%
Consequently we are forced to jump through hoops to appease the quantity
checker, and end up defining complex inversion lemmas to bypass these
limitations.
%
This could be solved by a mix of improvements to the typechecker and
meta-programming using prior ideas on automating
inversion~\cite{DBLP:conf/types/CornesT95,DBLP:conf/types/McBride96,monin:inria-00489412}.
