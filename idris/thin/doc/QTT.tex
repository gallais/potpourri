
\section{Some Key Features of \idris}\label{sec:quantitativeTT}

\idris{} implements Quantitative Type Theory,
a Martin-Löf type theory enriched with a semiring of quantities
classifying the ways in which values may be used.
%
In a type, each binder is annotated with the quantity by which its
argument must abide.

\subsection{Quantities}

A value may be \emph{runtime irrelevant}, \emph{linear}, or \emph{unrestricted}.

\emph{Runtime irrelevant} values (\IdrisKeyword{0} quantity) cannot possibly influence
control flow as they will be erased entirely during compilation.
%
This forces the language to impose strong restrictions on pattern-matching over these
values.
%
Typical examples are types like the \IdrisBound{a} parameter in (\IdrisType{List} \IdrisBound{a}),
or indices like the natural number \IdrisBound{n} in
(\IdrisType{Vect} \IdrisBound{n} \IdrisBound{a}).
%
These are guaranteed to be erased at compile time. The advantage over a Tejiščák-style
analysis is that users can state their intent that an argument ought to be runtime
irrelevant and the language will insist that it needs to be convinced it indeed is.

\emph{Linear} values (\IdrisKeyword{1} quantity) have to be used exactly once.
%
Typical examples include the \IdrisData{\%World} token used by \idris{} to implement the
\IdrisType{IO} monad à la Haskell, or file handles that cannot be discarded without first explicitly
closing the file.
%
At runtime these values can be updated destructively. We will not use linearity in this paper.

Last, \emph{unrestricted} values (denoted by no quantity annotation) can flow into any
position, be duplicated or thrown away.
%
They are the usual immutable values of functional programming.

The most basic of examples mobilising both the runtime irrelevance and unrestricted
quantities is the identity function.

\ExecuteMetaData[QuantitativeTT.idr.tex]{identity}

Its type starts with a binder using curly braces.
%
This means it introduces an implicit variable that does not need to be filled in by
the user at call sites and will be reconstructed by unification.
%
The variable it introduces is named \IdrisBound{a} and
has type \IdrisType{Type}. It has the \IdrisKeyword{0} quantity annotation which means
that this argument is runtime irrelevant and so will be erased during compilation.

The second binder uses parentheses. It introduces an explicit variable whose name
is \IdrisBound{x} and whose type is the type \IdrisBound{a} that was just bound. It has
no quantity annotation which means it will be an unrestricted variable.

Finally the return type is the type \IdrisBound{a} bound earlier. This is, as expected,
a polymorphic function from \IdrisBound{a} to \IdrisBound{a}. It is implemented using
a single clause that binds \IdrisBound{x} on the left-hand side and immediately returns
it on the right-hand side.


If we were to try to annotate the binder for \IdrisBound{x} with a \IdrisKeyword{0}
quantity to make it runtime irrelevant then \idris{} would
rightfully reject the definition.
%
The following \IdrisKeyword{failing} block shows part of the error message complaining
that \IdrisBound{x} cannot be used at an unrestricted quantity on the right-hand side.

\ExecuteMetaData[QuantitativeTT.idr.tex]{invalididentity}

%\ExecuteMetaData[QuantitativeTT.idr.tex]{vect}

% \todo{more on quantitative TT}

\subsection{Proof Search}\label{sec:proofsearch}

In \idris{}, Haskell-style ad-hoc polymorphism~\cite{DBLP:conf/popl/WadlerB89}
is superseded by a more general proof search mechanism.
%
Instead of having blessed notions of type classes, instances and constraints,
the domain of any dependent function type can be marked as \IdrisKeyword{auto}.
%
This signals to the compiler that the corresponding argument will be an implicit
argument and that it should not be reconstructed by unification alone but rather by
proof search.
%
The search algorithm will use the appropriate user-declared hints as well as
the local variables in scope.

By default, a datatype's constructors are always added to the database of hints.
And so the following declaration brings into scope both an indexed family
\IdrisType{So} of proofs that a given boolean is \IdrisData{True}, and a unique
constructor \IdrisData{Oh} that is automatically added as a hint.

\ExecuteMetaData[QuantitativeTT.idr.tex]{so}

As a consequence, we can for instance define a record type specifying what it
means for \IdrisBound{n} to be an even number by storing its \IdrisFunction{half}
together with a proof that is both runtime irrelevant and filled in by proof search.
%
Because (\IdrisData{2} \IdrisFunction{*} \IdrisData{3} \IdrisFunction{==} \IdrisData{6})
computes to \IdrisData{True}, \idris{} is able to fill-in the missing proof in the
definition of \IdrisFunction{even6} using the \IdrisData{Oh} hint.

\noindent
\begin{minipage}[t]{0.55\textwidth}
\ExecuteMetaData[QuantitativeTT.idr.tex]{even}
\end{minipage}\hfill
\begin{minipage}[t]{0.4\textwidth}
\ExecuteMetaData[QuantitativeTT.idr.tex]{four}
\end{minipage}

We will use both \IdrisType{So} and the \IdrisKeyword{auto} mechanism in
\cref{sec:thininginternal}.

\subsection{Application: \IdrisType{Vect}, as \IdrisType{List}}\label{sec:vectaslist}

We can use the features of Quantitative Type Theory to give an implementation
of \IdrisType{Vect} that is guaranteed to erase to a \IdrisType{List} at runtime
independently of the optimisation passes implemented by the compiler.
%
The advantage over the optimisation passes described in \cref{sec:optimisation-example}
is that the user has control over the runtime representation and does not need to
rely on these optimisations being deployed by the compiler.

The core idea is to make the slogan `a vector is a length-indexed list' a reality
by defining a record packing together the \IdrisFunction{encoding} as a list and
a proof its length is equal to the expected \IdrisType{Nat} index.
%
This proof is marked as runtime irrelevant to ensure that the list is the only
thing remaining after compilation.

\ExecuteMetaData[VectAsList.idr.tex]{vect}

\paragraph{Smart constructors}
Now that we have defined vectors,
we can recover the usual building blocks for vectors by defining smart
constructors, that is to say functions \IdrisFunction{Nil} and
\IdrisFunction{(::)} that act as replacements for the inductive
family's data constructors.

\ExecuteMetaData[VectAsList.idr.tex]{nil}

The smart constructor \IdrisFunction{Nil} returns an empty vector.
It is, unsurprisingly, encoded as the empty list (\IdrisData{[]}).
%
Because (\IdrisFunction{length} \IdrisData{[]}) statically computes to
\IdrisData{Z}, the proof that the encoding is valid can be discharged by
reflexivity.

\ExecuteMetaData[VectAsList.idr.tex]{cons}

Using \IdrisFunction{(::)} we can combine a head and a tail of size \IdrisBound{n}
to obtain a vector of size (\IdrisData{S} \IdrisBound{n}).
%
The encoding is obtained by consing the head in front of the tail's encoding
and the proof this is valid
{(\IdrisFunction{cong} \IdrisData{S} \IdrisBound{eq})}
uses the fact that propositional
equality is a congruence and that
(\IdrisFunction{length} (\IdrisBound{x} \IdrisData{::} \IdrisBound{xs}))
computes to (\IdrisData{S} (\IdrisFunction{length} \IdrisBound{xs})).

\paragraph{View}
Now that we know how to build vectors, we demonstrate that we can also take
them apart using a view.

A view for a type $T$, in the sense of Wadler~\cite{DBLP:conf/popl/Wadler87},
and as refined by McBride and McKinna~\cite{DBLP:journals/jfp/McBrideM04},
is an inductive family $V$ indexed by $T$ together with a total function
mapping every element $t$ of $T$ to a value of type ($V t$).
%
This simple gadget provides a powerful, user-extensible, generalisation of
pattern-matching.
%
Patterns are defined inductively as either a pattern variable, a forced term
(i.e. an arbitrary expression that is determined by a constraint arising from
another pattern), or a data constructor fully applied to subpatterns.
%
In contrast, the return indices of an inductive family's constructors can be
arbitrary expressions.

In the case that interests us, the view allows us to emulate `matching'
on which of the two smart constructors \IdrisFunction{Nil} or \IdrisFunction{(::)}
was used to build the vector being taken apart.

\ExecuteMetaData[VectAsList.idr.tex]{dataview}

The inductive family \IdrisType{View} is indexed by a vector and has two
constructors corresponding to the two smart constructors.
%
We use \idris{}'s overloading capabilities to give each of the
\IdrisType{View}'s constructors the name of the smart constructor
it corresponds to.
%
By pattern-matching on a value of type (\IdrisType{View} \IdrisBound{xs}),
we will be able to break \IdrisBound{xs} into its constitutive parts and
either observe it is equal to \IdrisFunction{Nil} or recover its head
and its tail.

\ExecuteMetaData[VectAsList.idr.tex]{view}

The function \IdrisFunction{view} demonstrates that we can always tell which
constructor was used by inspecting the \IdrisFunction{encoding} list. If it
is empty, the vector was built using the \IdrisFunction{Nil} smart constructor.
If it is not then we got our hands on the head and the tail of the encoding
and (modulo some re-wrapping of the tail) they are effectively the head and the
tail that were combined using the smart constructor.

\subsubsection{Application: \IdrisFunction{map}}

We can then use these constructs to implement the function \IdrisFunction{map}
on vectors without ever having to explicitly manipulate the encoding.
%
The maximally sugared version of \IdrisFunction{map} is as follows:

\ExecuteMetaData[VectAsList.idr.tex]{map}

On the left-hand side the view lets us seamlessly pattern-match on the input
vector.
%
Using the \IdrisKeyword{with} keyword we have locally modified the function
definition so that it takes an extra argument, here the result of the intermediate
computation (\IdrisFunction{view} \IdrisBound{xs}).
%
Correspondingly, we have two clauses matching on this extra argument;
the symbol \IdrisKeyword{|} separates the original left-hand side
(here elided using \IdrisKeyword{\KatlaUnderscore{}} because it is exactly the
same as in the parent clause) from the additional pattern.
%
This pattern can
either have the shape \IdrisData{[]} or (\IdrisBound{hd} \IdrisData{::} \IdrisBound{tl})
and, correspondingly, we learn that \IdrisBound{xs} is either \IdrisFunction{[]} or
(\IdrisBound{hd} \IdrisFunction{::} \IdrisBound{tl}).

On the right-hand side the smart constructors let us build the output vector.
Mapping a function over the empty vector yields the empty vector while mapping
over a cons node yields a cons node whose head and tail have been appropriately
modified.


This sugared version of \IdrisFunction{map} is equivalent to the following more
explicit one:

\ExecuteMetaData[WithExpanded.idr.tex]{map}

In the parent clause we have explicitly bound \IdrisBound{xs}
instead of merely introducing an alias for it by writing
{(\IdrisBound{xs}\IdrisKeyword{@}\IdrisKeyword{\_})}
and so we will need to be explicit about the ways in which this
pattern is refined in the two with-clauses.

In the with-clauses, we have explicitly repeated the refined version
of the parent clause's left-hand side. In particular we have used dotted
patterns to insist that \IdrisBound{xs} is now entirely \emph{forced}
by the match on the result of (\IdrisFunction{view} \IdrisBound{xs}).

We have seen that by matching on the result of the
(\IdrisFunction{view} \IdrisBound{xs}) call,
we get to `match' on \IdrisBound{xs} as if \IdrisType{Vect} were an
inductive type.
%
This is precisely the power of views.

\subsubsection{Application: \IdrisFunction{lookup}}

The type (\IdrisType{Fin} \IdrisBound{n}) can similarly be represented by a
single natural number and a runtime irrelevant proof that it is bound by
\IdrisBound{n}.
%
We leave these definitions out, and invite the curious reader
to either attempt to implement them for themselves or look at the accompanying code.

Bringing these definitions together, we can define a \IdrisFunction{lookup}
function which is  similar to the one defined in \cref{sec:optimisation-example}.

\ExecuteMetaData[LookupRefactor.idr.tex]{lookup}

We are seemingly using \IdrisFunction{view} at two different types (\IdrisType{Vect}
and \IdrisType{Fin} respectively) but both occurrences actually refer to separate
functions: \idris{} lets us overload functions and performs type-directed disambiguation.

For pedagogical purposes, this sugared version of \IdrisFunction{lookup} can
also be expanded to a more explicit one that demonstrates the views' power.

\ExecuteMetaData[WithExpanded.idr.tex]{lookup}


The main advantage of this definition is that, based on its type alone, we know
that this function is guaranteed to be processing a list and a single natural
number at runtime.
%
This efficient runtime representation does not rely on the assumption that state
of the art optimisation passes will be deployed.


We have seen some of \idris{}'s powerful features and how they can be leveraged
to empower users to control the runtime representation of the inductive families
they manipulate.
%
This simple example only allowed us to reproduce the performance that could already
be achieved by compilers deploying state of the art optimisation passes.
%
In the following sections, we are going to see how we can use the same core ideas
to compile an inductive family to a drastically different runtime representation
while keeping good high-level ergonomics.
