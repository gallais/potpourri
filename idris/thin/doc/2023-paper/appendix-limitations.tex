\section{Current Limitations of \idris{}}\label{appendix:limitations}

This challenge, suggested by Jacques Carette, highlights some of the current
limitations of \idris{}.

\subsection{Problem statement}

The goal is to use the \IdrisType{Vect} type defined in~\cref{sec:vectaslist}
and define a view that un-does vector-append.
%
This is a classic exercise in dependently-typed programming, the interesting
question being whether we can implement the function just as seamlessly with
our encoding.

Vector append can easily be defined by induction over the first vector.

\ExecuteMetaData[VectAsList.idr.tex]{append}

If the first vector is empty we can readily return the second vector.
%
If it is cons-headed, we can return the head and compute the tail by performing
a recursive call.

Equipped with this definition, we can declare the view type which we call
\IdrisType{SplitAt} by analogy with its weakly typed equivalent processing
lists.
%
It states that a vector \IdrisBound{xs} of length \IdrisBound{p} can be split
at \IdrisBound{m} if
\IdrisBound{p} happens to be
(\IdrisBound{m} \IdrisFunction{+} \IdrisBound{n})
and \IdrisBound{xs} happens to be
(\IdrisBound{pref} \IdrisFunction{++} \IdrisBound{suff})
where \IdrisBound{pref} and \IdrisBound{suff}'s
respective lengths are \IdrisBound{m} and \IdrisBound{n}.

\ExecuteMetaData[VectAsList.idr.tex]{splitAtrel}

The challenge is to define the function proving that a vector of size
(\IdrisBound{m} \IdrisFunction{+} \IdrisBound{n}) can be split at
\IdrisBound{m}.

\subsection{Failing attempts}

The proof will necessarily go by induction on \IdrisBound{m}, followed by
a case analysis on the input vector and a recursive call in the non-zero
case.

Our first failing attempt successfully splits the natural number, calls the
view on the vector \IdrisBound{xs} to take it apart but then fails when
performing the recursive call to \IdrisFunction{splitAt}.

\ExecuteMetaData[VectAsList.idr.tex]{splitAtFail1}

This reveals an issue in \idris{}'s handling of the interplay between
\IdrisKeyword{@}-patterns and quantities: the compiler arbitrarily decided
to make the alias \IdrisBound{tl} runtime irrelevant only to then complain
that \IdrisBound{tl} is not accessible when we want to perform the recursive
call (\IdrisFunction{splitAt} \IdrisBound{m} \IdrisBound{tl})!


In order to work around this limitation, we decided to let go of
\IdrisKeyword{@}-patterns and write the fully explicit clause ourselves,
using dotted patterns to mark the forced expressions.

\ExecuteMetaData[VectAsList.idr.tex]{splitAtFail2}

The left-hand side now typechecks but the case tree builder fails with a
perplexing error.
%
This reveals a bug in \idris{}'s implementation of elaboration of
pattern-matching functions to case trees.
%
Instead of ignoring dotted expressions when building the case tree (these
expressions are forced and so the variables they mention will have necessarily
been bound in another pattern), it attempts to use them to drive the case-splitting
strategy.
%
This is a well-studied problem and should be fixable by referring to
Cockx and Abel's work~\cite{DBLP:journals/jfp/CockxA20}.

\subsection{Working Around \idris{}'s Limitations}

This leads us to our working solution.
%
Somewhat paradoxically, working around these \idris{} bugs led us to a more
principled solution whereby the pattern-matching step needed to adjust the
result returned by the recursive call is abstracted away in an auxiliary
function whose type clarifies what is happening.

From an \IdrisBound{m} split on \IdrisBound{xs}, we can easily compute an
(\IdrisData{S} \IdrisBound{m}) split on (\IdrisBound{x} \IdrisFunction{::} \IdrisBound{xs})
by cons-ing \IdrisBound{x} on the prefix.

\ExecuteMetaData[VectAsList.idr.tex]{splitAtcons}

In this auxiliary function, \IdrisBound{xs} is clearly runtime irrelevant and
so the case-splitter will not attempt to inspect it, thus generating the correct
case tree.
%
We are forced to match further on \IdrisBound{pref} (in particular by making
the equality proof \IdrisData{Refl}) so that just enough computation happens
at the type level for the typechecker to see that things do line up.
%
A proof irrelevant type of propositional equality would have helped us here.

We can put all of these pieces together and finally get our
\IdrisFunction{splitAt} view.

\ExecuteMetaData[VectAsList.idr.tex]{splitAt}

We do want to reiterate that these limitations are not intrinsic limitations of
the approach, there are just flaws in the current experimental implementation of
the \idris{} language and can and should be remedied.
