
\subsubsection{A simple example of a view}

For instance, it is trivial to check whether a (\IdrisType{List} \IdrisBound{a})
is empty or constructed from a head and a tail.
%
In the first case, we would use the constructor \IdrisData{Nil} which is a valid
pattern according to our above definition.
%
In the second case, we would use the constructor \IdrisData{(::)} applied to the
variables \IdrisBound{head} and \IdrisBound{tail} acting as binding sites, thus
forming another valid pattern.
%
But it is impossible to look at the list from its end: the expression
(\IdrisBound{xs} \IdrisFunction{++} \IdrisData{[}\IdrisBound{x}\IdrisData{]})
does not fall in the pattern fragment because it mentions the function (\IdrisFunction{++}).
But we can define a view showing what it means to have this shape.

\ExecuteMetaData[Snoc.idr.tex]{AsSnoc}

We call this inductive family \IdrisType{AsSnoc} because it allows us to get
a handle on the input list as if it were a \IdrisType{Snoclist}. If the list is
empty then it corresponds to an empty snoclist (constructor \IdrisData{Lin})
but if it is non-empty, we get our hands on the initial segment \IdrisBound{xs},
the last element \IdrisBound{x} and a proof that the list is indeed
(\IdrisBound{xs} \IdrisFunction{++} \IdrisData{[}\IdrisBound{x}\IdrisData{]}).
%
We can finish the view definition by implementing the function \IdrisFunction{asSnoc}
that shows that every list \IdrisBound{xs} can be taken apart in an \IdrisType{AsSnoc}
manner.

\ExecuteMetaData[Snoc.idr.tex]{asSnoc}

The interesting case is the one where the input list is non-empty.
%
Using the \IdrisKeyword{with} keyword we can locally modify the function
definition to have it take an extra argument, here the recursive call to
(\IdrisFunction{asSnoc} \IdrisBound{xs})).
%
Correspondingly, we locally extend the pattern-matching definition with
an additional pattern on the left-hand side of the indented with-clauses.
%
By matching on the result of the recursive call, we get to learn about the
structure of the tail. This is precisely the power of views.

In the clause with the pattern \IdrisData{[<]}, we learn that \IdrisBound{xs} is empty
(hence the \emph{forced} pattern denoted by a full stop
\IdrisKeyword{.(}\IdrisData{[]}\IdrisKeyword{)} that replaces \IdrisBound{xs})
and therefore \IdrisBound{x} is the last element.
%
We can return (\IdrisData{[]} \IdrisData{:<} \IdrisBound{x}) whose index is
(\IdrisData{[]} \IdrisFunction{++} \IdrisData{[}\IdrisBound{x}\IdrisData{]})
which computes to \IdrisData{[}\IdrisBound{x}\IdrisData{]}.

%
In the clause with the pattern (\IdrisBound{ys} \IdrisData{:<} \IdrisBound{y}), we learn
that \IdrisBound{xs} has the shape (\IdrisBound{ys} \IdrisFunction{++} \IdrisBound{y})
which would not be a valid pattern where it not forced.
%
We can conclude by pushing \IdrisBound{x} back onto \IdrisBound{ys}.

Note that compared to a function returning a value of type
{\IdrisType{Maybe} (\IdrisType{List} \IdrisBound{a}\IdrisType{,} \IdrisBound{a})},
we have here the guarantee that the result actually reflects the structure of the
input list.
%
For instance, if we had tried to return \IdrisData{Lin} instead of
(\IdrisData{[]} \IdrisData{:<} \IdrisBound{x}) in the singleton list case,
the program would not have typechecked.

We can now use the view to pattern-match on the list from its end almost as
seamlessly as we can match on its head as is demonstrated by the following
comparison between \IdrisFunction{first} and \IdrisFunction{last}.
%
We have left an empty line in the definition of the former to line up each
equation with the matching one in the latter.

\noindent
\begin{minipage}[t]{0.45\textwidth}
\ExecuteMetaData[Snoc.idr.tex]{first}
\end{minipage}\hfill
\begin{minipage}[t]{0.45\textwidth}
\ExecuteMetaData[Snoc.idr.tex]{last}
\end{minipage}

Our goal is to define a view for \IdrisType{Th} that demonstrates
that any thinning has been built using one of the three smart
constructors introduced in the previous section.
