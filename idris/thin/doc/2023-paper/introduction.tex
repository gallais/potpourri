
\section{Introduction}

Dependently typed languages have empowered users to precisely describe their domain
of discourse by using inductive families~\cite{DBLP:journals/fac/Dybjer94}.
%
Programmers can bake crucial invariants directly into their definitions thus refining
both their functions' inputs and outputs.
%
The constrained inputs allow them to only consider the relevant cases during pattern
matching, while the refined outputs guarantee that client code can safely rely on the
invariants being maintained.
%
This programming style is dubbed `correct by construction'.

However, relying on inductive families can have a non-negligible runtime cost if
the host language is compiling them naïvely. And even state of the art optimisation
passes for dependently typed languages cannot make miracles: if the source code is
not efficient, the executable will not be either.

A state of the art compiler will for instance successfully compile length-indexed
lists to mere lists thus reducing the space complexity from quadratic to linear
in the size of the list.
%
But, confronted with a list of booleans whose length is statically known to be
less than 64, it will fail to pack it into a single machine word thus spending
linear space when constant would have sufficed.

In \cref{sec:optimisation-example}, we will look at an optimisation example
that highlights both the strengths and the limitations of the current state
of the art when it comes to removing the runtime overheads potentially
incurred by using inductive families.

In \cref{sec:quantitativeTT} we will give a quick introduction to Quantitative
Type Theory, the expressive language that grants programmers the ability
to have both strong invariants and, reliably, a very efficient runtime
representation.

In \cref{sec:codebruijn} we will look at an inductive family that
\iftoggle{BLIND}
  {the \typos{} project~\cite{MANUAL:talk/types/Allais22}
    uses in a performance-critical way}
  {we use in a performance-critical way in the
    \typos{} project~\cite{MANUAL:talk/types/Allais22}}
and whose compilation suffers from the limitations highlighted in
\cref{sec:optimisation-example}.
%
\iftoggle{BLIND}{The project's authors}{Our}
current and unsatisfactory approach is to rely on the safe and convenient
inductive family when experimenting in Agda and then replace it with an unsafe
but vastly more efficient representation in
\iftoggle{BLIND}{their}{our} actual Haskell implementation.

Finally in \cref{sec:efficient}, we will study the actual implementation of
our efficient and invariant-rich solution implemented in \idris{}. We will
also demonstrate that we can recover almost all the conveniences of programming
with inductive families thanks to smart constructors and views.
