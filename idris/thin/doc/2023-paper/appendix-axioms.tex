\section{Postulated Lemmas for the \IdrisType{Bits} Interface}\label{appendix:postulated}

It is often more convenient to reason about integers in terms of their bits.
We define the notion of bitwise equality as the pointwise equality according
to the \IdrisFunction{testBit}.

\ExecuteMetaData[Data/Bits/Integer/Postulated.idr.tex]{extensionalEq}

Our first postulate is a sort of extensionality principle stating that bitwise
equality implies propositional equality.

\ExecuteMetaData[Data/Bits/Integer/Postulated.idr.tex]{extensionally}

This gives us a powerful reasoning principle once combined with axioms
explaining the behaviour of various primitives at the bit level.
%
This is why almost all of the remaining axioms are expressed in terms
of \IdrisFunction{testBit} calls.

\subsection{Logical Operations}

Our first batch of axioms relates logical operations on integers to their
boolean counterparts. This is essentially stating that these operations are
bitwise.

\ExecuteMetaData[Data/Bits/Integer/Postulated.idr.tex]{testBitAnd}
\ExecuteMetaData[Data/Bits/Integer/Postulated.idr.tex]{testBitOr}
\ExecuteMetaData[Data/Bits/Integer/Postulated.idr.tex]{testBitComplement}

Together with the extensionality principle mentioned above this already
allows us to prove for instance that the binary operators are commutative
and associative, that the de Morgan laws hold, or that conjunction distributes
over disjunction.

\subsection{Bit Shifting}

The second set of axiom describes the action of left and right shifting on
bit patterns.

A right shift of size \IdrisBound{k} will drop the \IdrisBound{k} least
significant bits; consequently testing the bit \IdrisBound{i} on the
right-shifted integer amounts to testing the bit
(\IdrisBound{k} \IdrisFunction{+} \IdrisBound{i}) on the original integer.

\ExecuteMetaData[Data/Bits/Integer/Postulated.idr.tex]{testBitShiftR}

A left shift will add \IdrisBound{k} new least significant bits initialised
at \IdrisBound{0}; consequently testing a bit \IdrisBound{i} on the left-shifted
integer will either return \IdrisData{False} if \IdrisBound{i} is strictly less than
\IdrisBound{k}, or the bit at position (\IdrisBound{i} \IdrisFunction{-} \IdrisBound{k})
in the original integer.

For simplicity we state these results without mentioning the `strictly less than'
relation, by considering on the one hand the effect of a non-zero left shift,
and on the other the fact that a left-shift by 0 bits is the identity function.

\ExecuteMetaData[Data/Bits/Integer/Postulated.idr.tex]{testBit0ShiftL}
\ExecuteMetaData[Data/Bits/Integer/Postulated.idr.tex]{testBitSShiftL}
\ExecuteMetaData[Data/Bits/Integer/Postulated.idr.tex]{shiftL0}

\subsection{Bit Testing}

The last set of axioms specifies what happens when a bit is set.

Testing a bit other than the one that was set amounts to testing it on the
original integer.

\ExecuteMetaData[Data/Bits/Integer/Postulated.idr.tex]{testSetBitOther}

Finally, we have an axiom stating that the integer
(\IdrisFunction{bit} \IdrisBound{i}) (i.e. $2^i$) is non-zero.

\ExecuteMetaData[Data/Bits/Integer/Postulated.idr.tex]{bitNonZero}
