
\section{An Optimisation Example}\label{sec:optimisation-example}

The prototypical examples of the naïve compilation of inductive families being
inefficient are probably the types of vectors (\IdrisType{Vect})
and finite numbers (\IdrisType{Fin}).
%
Their interplay is demonstrated by the \IdrisFunction{lookup} function.
Let us study this example and how successive optimisation passes can, in this
instance, get rid of the overhead introduced by using indexed families over
plain data.

A vector is a length-indexed list.
%
The type \IdrisType{Vect} is parameterised by the type of values it stores
and indexed over a natural number corresponding to its length.
%
More concretely, its \IdrisData{Nil} constructor builds an empty vector of size
\IdrisData{Z} (i.e. zero),
%
and its \IdrisData{(::)} (pronounced `cons') constructor combines a
value of type \IdrisBound{a} (the head) and a subvector of size \IdrisBound{n}
(the tail) to build a vector of size (\IdrisData{S} \IdrisBound{n})
(i.e. successor of n).

\ExecuteMetaData[Lookup.idr.tex]{vect}

The size \IdrisBound{n} is not explicitly bound in the type of \IdrisData{(::)}.
In \idris{}, this means that it is automatically generalised over in a prenex
manner reminiscent of the handling of free type variables in languages in the
ML family.
%
This makes it an implicit argument of the constructor. Consequently, given that
\IdrisType{Nat} is a type of \emph{unary} natural numbers, a naïve runtime
representation of a {(\IdrisType{Vect} \IdrisBound{n} \IdrisBound{a})} would
have a size quadratic in \IdrisBound{n}.

A finite number is a number known to be strictly smaller than a given natural
number. The type \IdrisType{Fin} is indexed by said bound.
%
Its \IdrisData{Z} constructor models \IdrisData{0} and is bound by any
non-zero bound, and its \IdrisData{S} constructor takes a number bound by
\IdrisBound{n} and returns its successor, bound by
(\IdrisData{1} \IdrisFunction{+} \IdrisBound{n}).
%
A naïve compilation would here also lead to a runtime representation suffering
from a quadratic blowup.

\ExecuteMetaData[Lookup.idr.tex]{fin}

This leads us to the definition of the \IdrisFunction{lookup} function.
%
Provided a vector of size \IdrisBound{n} and a finite number \IdrisBound{k} bound
by this same \IdrisBound{n}, we can define a \emph{total} function looking up the
value stored at position \IdrisBound{k} in the vector.
%
It is guaranteed to return a value.
%
Note that we do not need to consider the case of the empty vector in the pattern
matching clauses as all of the return types of the \IdrisType{Fin} constructors force
the index to be non-zero and, because the vector and the finite number talk about the
same \IdrisBound{n}, having an empty vector would automatically imply having a value
of type (\IdrisType{Fin} \IdrisData{0}) which is self-evidently impossible.

\ExecuteMetaData[Lookup.idr.tex]{vectlookup}

Thanks to our indexed family, we have gained the ability to define a function that cannot
possibly fail, as well as the ability to only talk about the pattern matching clauses
that make sense.
This seemed to be at the cost of efficiency but luckily for us there has already been
extensive work on erasure to automatically detect redundant
data~\cite{DBLP:conf/types/BradyMM03} or data that will not be used at
runtime~\cite{DBLP:journals/pacmpl/Tejiscak20}.

\subsection{Optimising \IdrisType{Vect}, \IdrisType{Fin}, and \IdrisFunction{lookup}}

An analysis in the style of Brady, McBride, and McKinna's~\cite{DBLP:conf/types/BradyMM03}
can solve the quadratic blowup highlighted above by observing
that the natural number a vector is indexed by is entirely determined by the spine of
the vector. In particular, the length of the tail does not need to be stored
as part of the constructor: it can be reconstructed as the predecessor of the length
of the overall vector. As a consequence, a vector can be adequately represented at
runtime by a pair of a natural number and a list. Similarly a bounded number can be
adequately represented by a pair of natural numbers. Putting all of this together and
remembering that the vector and the finite number share the same \IdrisBound{n},
\IdrisFunction{lookup} can be compiled to a function taking two natural numbers and a list.
In \idris{} we would write the optimised \IdrisFunction{lookup} as follows (we use the
\IdrisKeyword{partial} keyword because this transformed version is not total at that type).

\ExecuteMetaData[Lookup.idr.tex]{erasedvectlookup}

We can see in the second clause that the recursive call is performed on the tail of
the list (formerly vector) and so the first argument to \IdrisFunction{lookup}
corresponding to the vector's size is decreased by one. The invariant, despite not
being explicit anymore, is maintained.

A Tejiščák-style analysis~\cite{DBLP:journals/pacmpl/Tejiscak20} can additionally
notice that the lookup function never makes
use of the bound's value and drop it entirely. This leads to the lookup function on
vectors being compiled to its partial-looking counterpart acting on lists.

\ExecuteMetaData[Lookup.idr.tex]{finallookup}

Even though this is in our opinion a pretty compelling example of erasing away the
apparent complexity introduced by inductive families, this approach has two drawbacks.

Firstly, it relies on the fact that the compiler can and will automatically perform
these optimisations.
%
But nothing in the type system prevents users from inadvertently using a value they
thought would get erased, thus preventing the Tejiščák-style optimisation from firing.
%
In performance-critical settings, users may rather want to state their intent
explicitly and be kept to their word by the compiler in exchange for predictable
and guaranteed optimisations.

Secondly, this approach is intrinsically limited to transformations that preserve the
type's overall structure: the runtime data structures are simpler but very similar still.
%
We cannot expect much better than that e.g. a change of representation to use a
balanced binary tree instead of a list in order to get logarithmic lookups rather
than linear ones.

\subsection{No Magic Solution}

Even if we are able to obtain a more compact representation of the inductive
family at runtime through enough erasure, this does not guarantee runtime efficiency.
As the \coq{} manual~\cite{Coq:manual} reminds its users, extraction does not magically
optimise away a user-defined quadratic multiplication algorithm when extracting unary
natural numbers to an efficient machine representation.
%
In a pragmatic move, \coq{}, \agda{}, and \idris{} all have ad-hoc rules to replace
convenient but inefficiently implemented numeric functions with asymptotically faster
counterparts in the target language.

However this approach is not scalable: if we may be willing to extend our trusted core to a
high quality library for unbounded integers, we do not want to replace
our code only proven correct thanks to complex invariants with a wildly different
untrusted counterpart purely for efficiency reasons.

In this paper we use Quantitative Type
Theory~\cite{DBLP:conf/birthday/McBride16,DBLP:conf/lics/Atkey18}
as implemented in \idris{}~\cite{DBLP:conf/ecoop/Brady21} to bridge the gap between
an invariant-rich but inefficient representation based on an inductive family and
an unsafe but efficient implementation using low-level primitives.
%
Inductive families allow us to
\emph{view}~\cite{DBLP:conf/popl/Wadler87,DBLP:journals/jfp/McBrideM04}
the runtime relevant information encoded in the low-level and efficient representation
as an information-rich compile time data structure. Moreover the quantity annotations
guarantee that this additional information will be erased away during compilation.
