\documentclass{article}

\usepackage{catchfilebetweentags}
\makeatletter

\newrobustcmd*\OrigExecuteMetaData[2][\jobname]{%
\CatchFileBetweenTags\CatchFBT@tok{#1}{#2}%
\global\expandafter\CatchFBT@tok\expandafter{%
\expandafter}\the\CatchFBT@tok
}%\OrigExecuteMetaData

\newrobustcmd*\ChkExecuteMetaData[2][\jobname]{%
\CatchFileBetweenTags\CatchFBT@tok{#1}{#2}%
\edef\mytokens{\detokenize\expandafter{\the\CatchFBT@tok}}
\ifx\mytokens\empty\PackageError{catchfilebetweentags}{the tag #2 is not found\MessageBreak in file #1 \MessageBreak called from \jobname.tex}{use a different tag}\fi%
}%\ChkExecuteMetaData

\renewrobustcmd*\ExecuteMetaData[2][\jobname]{%
\ChkExecuteMetaData[#1]{#2}%
\OrigExecuteMetaData[#1]{#2}%
}

\makeatother


\usepackage{idris2}

\usepackage{minted}
\usepackage{cleveref}

\usepackage{todonotes}
\setuptodonotes{inline}

\usepackage{tikz}

\newcommand{\typos}{TypOS}
\newcommand{\idris}{Idris 2}
\newcommand{\coq}{Coq}
\newcommand{\agda}{Agda}

\bibliographystyle{alpha}

\begin{document}

\title{Builtin Types viewed as Inductive Families \\
{\large Efficiently Representing Thinnings in \idris, a case study}}
\author{Guillaume Allais}

\maketitle

\begin{abstract}
  Remembering that types give us a way to make sense of unstructured data stored
  in memory, we demonstrate how to use Quantitative Type Theory to define an
  invariant-rich typechecking time data structure that is guaranteed to compile
  to an efficient runtime one.

  Unlike other approaches, the resulting complexity is entirely predictable, we do
  not require both representations to have the same structure, and yet we are able
  to seamlessly program as if we were using the high-level structure.
\end{abstract}

%%%%%%%%%%%%%%%%%%%%%%%%%%%%%%%%%%%%%%%%%%%%%%%%%%%%%%%%%%%%%%%%%%%%%%%%%%%%%%
%% introduction
%%%%%%%%%%%%%%%%%%%%%%%%%%%%%%%%%%%%%%%%%%%%%%%%%%%%%%%%%%%%%%%%%%%%%%%%%%%%%%

\section{Introduction}

Dependently typed languages have empowered users to precisely describe their domain
of discourse by using inductive families~\cite{DBLP:journals/fac/Dybjer94}.
%
Programmers can bake crucial invariants directly into their definitions thus refining
both their functions' inputs and outputs.
%
The constrained inputs allow them to only consider the relevant cases during pattern
matching, while the refined outputs guarantee that client code can safely rely on the
invariants being maintained.
%
This programming style is dubbed `correct by construction'.

However relying on inductive families can have a non-negligible runtime cost if
the host language is compiling them naïvely. And even state of the art optimisation
passes for dependently typed languages cannot make miracles: if the source code is
not efficient, the executable will not be either.

In \cref{sec:optimisation-example}, we will look at an optimisation example
that highlights both the strengths and the limitations of the current state
of the art when it comes to removing the runtime overheads potentially
incurred by using inductive families.

In \cref{sec:codebruijn} we will then look at an inductive family that we
use in a performance-critical way in the \typos{}
project and whose compilation suffers from said limitations.
%
Our current and unsatisfactory approach is to rely on the safe and convenient
inductive family when experimenting in Agda and then replace it with an unsafe
but vastly more efficient representation in our actual Haskell implementetation.

In \cref{sec:quantitativeTT} we will give a quick introduction to Quantitative
Type Theory, the expressive language that grants programmers the ability
to have both strong invariants and, reliably, a very efficient runtime
representation.

Finally in \cref{sec:efficient}, we will study the actual implementation of
our efficient and invariant-rich solution implemented in \idris{}. We will
also demonstrate that we can recover almost all the conveniences of programming
with inductive families thanks to smart constructors and views.

%%%%%%%%%%%%%%%%%%%%%%%%%%%%%%%%%%%%%%%%%%%%%%%%%%%%%%%%%%%%%%%%%%%%%%%%%%%%%%
%% imports
%%%%%%%%%%%%%%%%%%%%%%%%%%%%%%%%%%%%%%%%%%%%%%%%%%%%%%%%%%%%%%%%%%%%%%%%%%%%%%


\section{An Optimisation Example}\label{sec:optimisation-example}

The prototypical examples of the naïve compilation of inductive families being
inefficient are probably the types of vectors (\IdrisType{Vect})
and finite numbers (\IdrisType{Fin}).
%
Their interplay is demonstrated by the \IdrisFunction{lookup} function.
Let us study this example and how successive optimisation passes can, in this
instance, get rid of the overhead introduced by using indexed families over
plain data.

A vector is a length-indexed list. The type \IdrisType{Vect} is parameterised
by the type of values it stores and indexed over a natural number corresponding
to its length.
%
More concretely, its \IdrisData{Nil} constructor builds an empty vector of size
\IdrisData{0}, and its \IdrisData{(::)} (pronounced `cons') constructor combines a
value of type \IdrisBound{a} (the head) and a subvector of size \IdrisBound{n}
(the tail) to build a vector of size (\IdrisData{1} \IdrisFunction{+} \IdrisBound{n}).

\ExecuteMetaData[Lookup.idr.tex]{vect}

The size \IdrisBound{n} is not explicitly bound in the type of \IdrisData{(::)}.
In \idris{}, this means that it is automatically generalised over in a prenex
manner reminiscent of the handling of free type variables in languages in the
ML family.
%
This makes it an implicit argument of the constructor. Consequently, given that
\IdrisType{Nat} is a type of \emph{unary} natural numbers, a naïve runtime
representation of a {(\IdrisType{Vect} \IdrisBound{n} \IdrisBound{a})} would
have a size quadratic in \IdrisBound{n}.

A finite number is a number known to be strictly smaller than a given natural
number. The type \IdrisType{Fin} is indexed by said bound.
%
Its \IdrisData{Z} constructor models \IdrisData{0} and is bound by any
non-zero bound, and its \IdrisData{S} constructor takes a number bound by
\IdrisBound{n} and returns its successor, bound by
(\IdrisData{1} \IdrisFunction{+} \IdrisBound{n}).
%
A naïve compilation would here also lead to a runtime representation suffering
from a quadratic blowup.

\ExecuteMetaData[Lookup.idr.tex]{fin}

This leads us to the definition of the \IdrisFunction{lookup} function.
%
Provided a vector of size \IdrisBound{n} and a finite number \IdrisBound{k} bound
by this same \IdrisBound{n}, we can define a \emph{total} function looking up the
value stored at position \IdrisBound{k} in the vector.
%
It is guaranteed to return a value.
%
Note that we do not need to consider the case of the empty vector in the pattern
matching clauses as all of the return types of the \IdrisType{Fin} constructors force
the index to be non-zero and, because the vector and the finite number talk about the
same \IdrisBound{n}, having an empty vector would automatically imply having a value
of type (\IdrisType{Fin} \IdrisData{0}) which is self-evidently impossible.

\ExecuteMetaData[Lookup.idr.tex]{vectlookup}

Thanks to our indexed family, we have gained the ability to define a function that cannot
possibly fail, as well as the ability to only talk about the pattern matching clauses
that make sense.
This seemed to be at the cost of efficiency but luckily for us there has already been
extensive work on erasure to automatically detect redundant
data~\cite{DBLP:conf/types/BradyMM03} or data that will not be used at
runtime~\cite{DBLP:journals/pacmpl/Tejiscak20}.

\subsection{Optimising \IdrisType{Vect}, \IdrisType{Fin}, and \IdrisFunction{lookup}}

A Brady-style analysis~\cite{DBLP:conf/types/BradyMM03}
can solve the quadratic blowup highlighted above by observing
that the natural number a vector is indexed by is entirely determined by the spine of
the vector. In particular, the length of the predecessor does not need to be stored
as part of the constructor: it can be reconstructed as the predecessor of the length
of the overall vector. As a consequence, a vector can be adequately represented at
runtime by a pair of a natural number and a list. Similarly the bounded number can be
adequately represented by a pair of natural numbers. Putting all of this together and
remembering that the vector and the finite number share the same \IdrisBound{n},
\IdrisFunction{lookup} can be compiled to a function taking two natural numbers and a list.
In \idris{} we would write the optimised \IdrisFunction{lookup} as follows (we use the
\IdrisKeyword{partial} keyword because this transformed version is not total at that type).

\ExecuteMetaData[Lookup.idr.tex]{erasedvectlookup}

We can see in the second clause that the recursive call is performed on the tail of
the list (formerly vector) and so the first argument to \IdrisFunction{lookup}
corresponding to the vector's size is decreased by one. The invariant, despite not
being explicit anymore, is maintained.

A Tejiščák-style analysis~\cite{DBLP:journals/pacmpl/Tejiscak20} can additionally
notice that the lookup function never makes
use of the bound's value and drop it entirely. This leads to the lookup function on
vectors being compiled to its partial-looking counterpart acting on lists.

\ExecuteMetaData[Lookup.idr.tex]{finallookup}

Even though this is in our opinion a pretty compelling example of erasing away the
apparent complexity introduced by inductive families, we need to stay
realistic.
%
For this approach to work, the compiler needs to be able to automatically map the
indexed family to a simpler datatype that can then be extracted to the target language.
%
It is particularly important here to observe that if the types have been simplified,
they still have very much the same structure. We cannot expect much better than that.

\subsection{No Magic Solution}

Even if we are able to obtain a more compact representation of the inductive
family at runtime through enough erasure, this does not guarantee runtime efficiency.
As the \coq{} manual~\cite{Coq:manual} reminds its users, extraction does not magically
optimises away a user-defined quadratic multiplication algorithm when extracting unary
natural numbers to an efficient machine representation.
%
In a pragmatic move, \coq{}, \agda{}, and \idris{} all have ad-hoc rules to replace
convenient but inefficiently implemented numeric functions with asymptotically faster
counterparts in the target language.

However this is not scalable: if we may be willing to extend our trusted core to a
high quality library for unbounded integers, we cannot possibly contemplate replacing
our code only proven correct thanks to complex invariants with a wildly different
untrusted counterpart purely for efficiency reasons.

In this paper we use Quantitative Type
Theory~\cite{DBLP:conf/birthday/McBride16,DBLP:conf/lics/Atkey18}
as implemented in \idris{}~\cite{DBLP:conf/ecoop/Brady21} to bridge the gap between
an invariant-rich but inefficient representation based on an inductive family and
an unsafe but efficient implementation using low-level primitives.
%
Inductive families allow us to \emph{view}~\cite{DBLP:journals/jfp/McBrideM04} the
runtime relevant information encoded in the low-level and efficient representation
as an information-rich compile time data structure. Moreover the quantity annotations
guarantee that this additional information will be erased away during compilation.

\section{Co-de Bruijn representation, cooked two ways}\label{sec:codebruijn}

We experienced one such limitation during the development of
\typos~\cite{MANUAL:talk/types/Allais22}, a domain specific language
to define typecheckers and elaborators.
%
Core to this project is the definition of actors manipulating terms

uses a co-de Bruijn
representation internally. \todo{explain why}


\subsection{Named, de Bruijn, and co-de Bruijn syntaxes}

In this section we will use the $S$ combinator as a running example and represent
terms using a syntax tree whose constructor nodes are circles and variable nodes
are squares.
%
To depict the $S$ combinator we will only need $\lambda{}$-abstraction and
application (rendered \$) nodes. A constructor's arguments become its children
in the tree and they are layed out top-to-bottom.

The first representation is using explicit names. Each binder has an associated
name and each variable node carries a name. It refers to the closest englobing
binder which happens to be using the same name.

\ExecuteMetaData[ast.tex]{named}

To check whether two terms are structurally equivalent (\emph{$\alpha$-equivalence})
potentially requires renaming bound names.
%
In order to have a simple and cheap $\alpha$-equivalence check we can instead opt
for a nameless representation.

An abstract syntax tree based on de Bruijn indices~\cite{MANUAL:journals/math/debruijn72}
replaces names with natural numbers counting the number of binders separating a variable
from its binding site.
%
You can see in the following graphical depiction of the $S$ combinator that
$\lambda$-abstractions do not carry a name and that variables are simply pointing
to the binder that introduced them. We have left the squares empty but in practice
the various coloured arrows would be represented by a natural number.
%
For instance the {\color{magenta}magenta} one corresponds to $1$ because you need
to ignore one $\lambda{}$-abstraction on your way towards the root of the tree
before your reach the corresponding magenta binder.

\ExecuteMetaData[ast.tex]{debruijn}

To check whether a subterm does not mention a given variable (\emph{thickening}
test, the opposite of \emph{thinning} which extends the current scope with unused
variables), you need to traverse it in its entirety.
%
In order to have a simple and cheap thickening test we can ensure that each subterms
knows precisely what its precise scope (i.e. its \emph{support}) is and how it embeds
in its parent's.

In a co-de Bruijn
representation~\cite{DBLP:journals/corr/abs-1807-04085} each subterm
selects exactly the variables that stay in scope for that term,
and so a variable constructor ultimately refers to the only variable still
in scope by the time it is reached.
%
This representation ensures that we know precisely what the scope of a given term
currently is.

In the following graphical rendering, we represent thinnings as lists of full
($\bullet$) or empty ($\circ$) discs depending on whether the corresponding
variable is either kept or discarded.
For instance the thinning represented by
$\color{blue}{\circ}\color{magenta}{\bullet}\color{orange}{\bullet}$
throws the {\color{blue}blue} variable away, and keeps both the
{\color{magenta}magenta} and {\color{orange}orange} ones.

\ExecuteMetaData[ast.tex]{codebruijn}

We can see that in such a representation, each node in the tree stores one
thinning per subterm. This will not be tractable unless we have an efficient
representation of thinnings.

We can recover terms living in arbitrary scopes by pairing a co-de Bruijn term
with a thinning~\cite{MANUAL:phd/nott/Chapman09}
embedding its tight support into the given englobing scope.
\todo{defined CdB}

\subsection{The Performance Challenges of co-de Bruijn}

\todo{application e.g. shrinking}

In order to efficiently represent and traverse terms in co-de Bruijn representation,
we need a compact encoding of thinnings and a cheap composition operator.
\todo{example: opening an application node}

The implementation of \typos~\cite{MANUAL:talk/types/Allais22} uses a co-de Bruijn
representation internally. \todo{explain why}
%
The developpement of the \typos{} language highlights a glaring gap between on the
one hand the experiments done in Agda and on the other the actual implementation
in Haskell.
%
The Agda-based experiments use inductive families that make the key invariants explicit
which helps tracking complex constraints and catches design flaws. The indices guarantee
that we always transform the thinnings appropriately when we add or remove bound variables.
%
The Haskell implementation represents a thinning as a pair of integers and resorts to
explicitly manipulating individual bits. It is not indexed and thus all the invariant
tracking has to be done by hand. This has led to numerous and hard to diagnose bugs.

\idris{} is a bootstrapped language. If we were to use such a co-deBruijn representation
of terms as a replacement for \idris{}'s current core language we would want, and should
be able, to have the best of both worlds: a safe and efficient representation!


\section{Some Key Features of \idris}\label{sec:quantitativeTT}

\idris{} implements Quantitative Type Theory,
a Martin-Löf type theory enriched with a semiring of quantities
classifying the ways in which values may be used.
%
In a type, each binder is annotated with the quantity by which its
argument must abide.

\subsection{Quantities}

A value may be \emph{runtime irrelevant}, \emph{linear}, or \emph{unrestricted}.

\emph{Runtime irrelevant} values (\IdrisKeyword{0} quantity) cannot possibly influence
control flow as they will be erased entirely during compilation.
%
This forces the language to impose strong restrictions on pattern-matching over these
values.
%
Typical examples are types like the \IdrisBound{a} parameter in (\IdrisType{List} \IdrisBound{a}),
or indices like the natural number \IdrisBound{n} in
(\IdrisType{Vect} \IdrisBound{n} \IdrisBound{a}).
%
These are guaranteed to be erased at compile time. The advantage over a Tejiščák-style
analysis is that users can state their intent that an argument ought to be runtime
irrelevant and the language will insist that it needs to be convinced it indeed is.

\emph{Linear} values (\IdrisKeyword{1} quantity) have to be used exactly once.
%
Typical examples include the \IdrisData{\%World} token used by \idris{} to implement the
\IdrisType{IO} monad à la Haskell, or file handles that cannot be discarded without first explicitly
closing the file.
%
At runtime these values can be updated destructively. We will not use linearity in this paper.

Last, \emph{unrestricted} values (denoted by no quantity annotation) can flow into any
position, be duplicated or thrown away.
%
They are the usual immutable values of functional programming.

The most basic of examples mobilising both the runtime irrelevance and unrestricted
quantities is the identity function.

\ExecuteMetaData[QuantitativeTT.idr.tex]{identity}

Its type starts with a binder using curly braces.
%
This means it introduces an implicit variable that does not need to be filled in by
the user at call sites and will be reconstructed by unification.
%
The variable it introduces is named \IdrisBound{a} and
has type \IdrisType{Type}. It has the \IdrisKeyword{0} quantity annotation which means
that this argument is runtime irrelevant and so will be erased during compilation.

The second binder uses parentheses. It introduces an explicit variable whose name
is \IdrisBound{x} and whose type is the type \IdrisBound{a} that was just bound. It has
no quantity annotation which means it will be an unrestricted variable.

Finally the return type is the type \IdrisBound{a} bound earlier. This is, as expected,
a polymorphic function from \IdrisBound{a} to \IdrisBound{a}. It is implemented using
a single clause that binds \IdrisBound{x} on the left-hand side and immediately returns
it on the right-hand side.


If we were to try to annotate the binder for \IdrisBound{x} with a \IdrisKeyword{0}
quantity to make it runtime irrelevant then \idris{} would
rightfully reject the definition.
%
The following \IdrisKeyword{failing} block shows part of the error message complaining
that \IdrisBound{x} cannot be used at an unrestricted quantity on the right-hand side.

\ExecuteMetaData[QuantitativeTT.idr.tex]{invalididentity}

%\ExecuteMetaData[QuantitativeTT.idr.tex]{vect}

% \todo{more on quantitative TT}

\subsection{Proof Search}\label{sec:proofsearch}

In \idris{}, Haskell-style ad-hoc polymorphism~\cite{DBLP:conf/popl/WadlerB89}
is superseded by a more general proof search mechanism.
%
Instead of having blessed notions of type classes, instances and constraints,
the domain of any dependent function type can be marked as \IdrisKeyword{auto}.
%
This signals to the compiler that the corresponding argument will be an implicit
argument and that it should not be reconstructed by unification alone but rather by
proof search.
%
The search algorithm will use the appropriate user-declared hints as well as
the local variables in scope.

By default, a datatype's constructors are always added to the database of hints.
And so the following declaration brings into scope both an indexed family
\IdrisType{So} of proofs that a given boolean is \IdrisData{True}, and a unique
constructor \IdrisData{Oh} that is automatically added as a hint.

\ExecuteMetaData[QuantitativeTT.idr.tex]{so}

As a consequence, we can for instance define a record type specifying what it
means for \IdrisBound{n} to be an even number by storing its \IdrisFunction{half}
together with a proof that is both runtime irrelevant and filled in by proof search.
%
Because (\IdrisData{2} \IdrisFunction{*} \IdrisData{3} \IdrisFunction{==} \IdrisData{6})
computes to \IdrisData{True}, \idris{} is able to fill-in the missing proof in the
definition of \IdrisFunction{even6} using the \IdrisData{Oh} hint.

\noindent
\begin{minipage}[t]{0.55\textwidth}
\ExecuteMetaData[QuantitativeTT.idr.tex]{even}
\end{minipage}\hfill
\begin{minipage}[t]{0.4\textwidth}
\ExecuteMetaData[QuantitativeTT.idr.tex]{four}
\end{minipage}

We will use both \IdrisType{So} and the \IdrisKeyword{auto} mechanism in
\cref{sec:thininginternal}.

\subsection{Application: \IdrisType{Vect}, as \IdrisType{List}}\label{sec:vectaslist}

We can use the features of Quantitative Type Theory to give an implementation
of \IdrisType{Vect} that is guaranteed to erase to a \IdrisType{List} at runtime
independently of the optimisation passes implemented by the compiler.
%
The advantage over the optimisation passes described in \cref{sec:optimisation-example}
is that the user has control over the runtime representation and does not need to
rely on these optimisations being deployed by the compiler.

The core idea is to make the slogan `a vector is a length-indexed list' a reality
by defining a record packing together the \IdrisFunction{encoding} as a list and
a proof its length is equal to the expected \IdrisType{Nat} index.
%
This proof is marked as runtime irrelevant to ensure that the list is the only
thing remaining after compilation.

\ExecuteMetaData[VectAsList.idr.tex]{vect}

\paragraph{Smart constructors}
Now that we have defined vectors,
we can recover the usual building blocks for vectors by defining smart
constructors, that is to say functions \IdrisFunction{Nil} and
\IdrisFunction{(::)} that act as replacements for the inductive
family's data constructors.

\ExecuteMetaData[VectAsList.idr.tex]{nil}

The smart constructor \IdrisFunction{Nil} returns an empty vector.
It is, unsurprisingly, encoded as the empty list (\IdrisData{[]}).
%
Because (\IdrisFunction{length} \IdrisData{[]}) statically computes to
\IdrisData{Z}, the proof that the encoding is valid can be discharged by
reflexivity.

\ExecuteMetaData[VectAsList.idr.tex]{cons}

Using \IdrisFunction{(::)} we can combine a head and a tail of size \IdrisBound{n}
to obtain a vector of size (\IdrisData{S} \IdrisBound{n}).
%
The encoding is obtained by consing the head in front of the tail's encoding
and the proof this is valid
{(\IdrisFunction{cong} \IdrisData{S} \IdrisBound{eq})}
uses the fact that propositional
equality is a congruence and that
(\IdrisFunction{length} (\IdrisBound{x} \IdrisData{::} \IdrisBound{xs}))
computes to (\IdrisData{S} (\IdrisFunction{length} \IdrisBound{xs})).

\paragraph{View}
Now that we know how to build vectors, we demonstrate that we can also take
them apart using a view.

A view for a type $T$, in the sense of Wadler~\cite{DBLP:conf/popl/Wadler87},
and as refined by McBride and McKinna~\cite{DBLP:journals/jfp/McBrideM04},
is an inductive family $V$ indexed by $T$ together with a total function
mapping every element $t$ of $T$ to a value of type ($V t$).
%
This simple gadget provides a powerful, user-extensible, generalisation of
pattern-matching.
%
Patterns are defined inductively as either a pattern variable, a forced term
(i.e. an arbitrary expression that is determined by a constraint arising from
another pattern), or a data constructor fully applied to subpatterns.
%
In contrast, the return indices of an inductive family's constructors can be
arbitrary expressions.

In the case that interests us, the view allows us to emulate `matching'
on which of the two smart constructors \IdrisFunction{Nil} or \IdrisFunction{(::)}
was used to build the vector being taken apart.

\ExecuteMetaData[VectAsList.idr.tex]{dataview}

The inductive family \IdrisType{View} is indexed by a vector and has two
constructors corresponding to the two smart constructors.
%
We use \idris{}'s overloading capabilities to give each of the
\IdrisType{View}'s constructors the name of the smart constructor
it corresponds to.
%
By pattern-matching on a value of type (\IdrisType{View} \IdrisBound{xs}),
we will be able to break \IdrisBound{xs} into its constitutive parts and
either observe it is equal to \IdrisFunction{Nil} or recover its head
and its tail.

\ExecuteMetaData[VectAsList.idr.tex]{view}

The function \IdrisFunction{view} demonstrates that we can always tell which
constructor was used by inspecting the \IdrisFunction{encoding} list. If it
is empty, the vector was built using the \IdrisFunction{Nil} smart constructor.
If it is not then we got our hands on the head and the tail of the encoding
and (modulo some re-wrapping of the tail) they are effectively the head and the
tail that were combined using the smart constructor.

\subsubsection{Application: \IdrisFunction{map}}

We can then use these constructs to implement the function \IdrisFunction{map}
on vectors without ever having to explicitly manipulate the encoding.
%
The maximally sugared version of \IdrisFunction{map} is as follows:

\ExecuteMetaData[VectAsList.idr.tex]{map}

On the left-hand side the view lets us seamlessly pattern-match on the input
vector.
%
Using the \IdrisKeyword{with} keyword we have locally modified the function
definition so that it takes an extra argument, here the result of the intermediate
computation (\IdrisFunction{view} \IdrisBound{xs}).
%
Correspondingly, we have two clauses matching on this extra argument;
the symbol \IdrisKeyword{|} separates the original left-hand side
(here elided using \IdrisKeyword{\KatlaUnderscore{}} because it is exactly the
same as in the parent clause) from the additional pattern.
%
This pattern can
either have the shape \IdrisData{[]} or (\IdrisBound{hd} \IdrisData{::} \IdrisBound{tl})
and, correspondingly, we learn that \IdrisBound{xs} is either \IdrisFunction{[]} or
(\IdrisBound{hd} \IdrisFunction{::} \IdrisBound{tl}).

On the right-hand side the smart constructors let us build the output vector.
Mapping a function over the empty vector yields the empty vector while mapping
over a cons node yields a cons node whose head and tail have been appropriately
modified.


This sugared version of \IdrisFunction{map} is equivalent to the following more
explicit one:

\ExecuteMetaData[WithExpanded.idr.tex]{map}

In the parent clause we have explicitly bound \IdrisBound{xs}
instead of merely introducing an alias for it by writing
{(\IdrisBound{xs}\IdrisKeyword{@}\IdrisKeyword{\_})}
and so we will need to be explicit about the ways in which this
pattern is refined in the two with-clauses.

In the with-clauses, we have explicitly repeated the refined version
of the parent clause's left-hand side. In particular we have used dotted
patterns to insist that \IdrisBound{xs} is now entirely \emph{forced}
by the match on the result of (\IdrisFunction{view} \IdrisBound{xs}).

We have seen that by matching on the result of the
(\IdrisFunction{view} \IdrisBound{xs}) call,
we get to `match' on \IdrisBound{xs} as if \IdrisType{Vect} were an
inductive type.
%
This is the power of views.

\subsubsection{Application: \IdrisFunction{lookup}}

The type (\IdrisType{Fin} \IdrisBound{n}) can similarly be represented by a
single natural number and a runtime irrelevant proof that it is bound by
\IdrisBound{n}.
%
We leave these definitions out, and invite the curious reader
to either attempt to implement them for themselves or look at the accompanying code.

Bringing these definitions together, we can define a \IdrisFunction{lookup}
function which is  similar to the one defined in \cref{sec:optimisation-example}.

\ExecuteMetaData[LookupRefactor.idr.tex]{lookup}

We are seemingly using \IdrisFunction{view} at two different types (\IdrisType{Vect}
and \IdrisType{Fin} respectively) but both occurrences actually refer to separate
functions: \idris{} lets us overload functions and performs type-directed disambiguation.

For pedagogical purposes, this sugared version of \IdrisFunction{lookup} can
also be expanded to a more explicit one that demonstrates the views' power.

\ExecuteMetaData[WithExpanded.idr.tex]{lookup}


The main advantage of this definition is that, based on its type alone, we know
that this function is guaranteed to be processing a list and a single natural
number at runtime.
%
This efficient runtime representation does not rely on the assumption that state
of the art optimisation passes will be deployed.


We have seen some of \idris{}'s powerful features and how they can be leveraged
to empower users to control the runtime representation of the inductive families
they manipulate.
%
This simple example only allowed us to reproduce the performance that could already
be achieved by compilers deploying state of the art optimisation passes.
%
In the following sections, we are going to see how we can use the same core ideas
to compile an inductive family to a drastically different runtime representation
while keeping good high-level ergonomics.


\section{An Efficient Invariant-Rich Representation}\label{sec:efficient}

We can combine both approaches highlighted in \cref{sec:thinningsintypos}
by defining a record parameterised by a source
(\IdrisBound{sx}) and target (\IdrisBound{sy}) scopes corresponding to the two
ends of the thinnings, just like we would for the inductive family. This record
packs two numbers and a runtime irrelevant proof.

Firstly, we have a natural number called \IdrisFunction{bigEnd} corresponding
to the size of the big end of the thinning (\IdrisBound{sy}).
%
We are happy to use a (unary) natural number here because we know that \idris{}
will compile to an unbounded integer.

Secondly, we have an integer called \IdrisFunction{encoding} corresponding to
the thinning represented as a bit vector stating, for each variable, whether
it is kept or dropped. We only care about the integer's \IdrisFunction{bigEnd}
first bits and assume the rest is set to 0.

Thirdly, we have a runtime irrelevant proof \IdrisFunction{thinning} that
\IdrisFunction{encoding} is indeed a valid encoding of size \IdrisFunction{bigEnd}
of a thinning from \IdrisBound{sx} to \IdrisBound{sy}. We will explore the
definition of the relation \IdrisType{Thinning} later on
in \cref{sec:thininginternal}.

\ExecuteMetaData[Thin.idr.tex]{thin}

The first sign that this definition is adequate is our ability to construct
any valid thinning. We demonstrate it is the case by introducing functions
that act as smart constructor analogues for the inductive family's data
constructors.

\subsection{Smart Constructors for \IdrisType{Th}}

The first and simplest one is \IdrisFunction{done}, a function that packs a pair of
\IdrisData{0} (the size of the big end, and the empty encoding) together with a proof
that it is an adequate encoding of the thinning from the empty scope to itself.

\ExecuteMetaData[Thin.idr.tex]{done}

The \IdrisFunction{keep} smart constructor demonstrates that from a thinning from
\IdrisBound{sx} to \IdrisBound{sy} and a runtime irrelevant variable \IdrisBound{x}
we can compute a thinning from the extended source scope
(\IdrisBound{sx} \IdrisData{:<} \IdrisBound{x}) to the target scope
(\IdrisBound{sy} \IdrisData{:<} \IdrisBound{x}) where \IdrisBound{x} was kept.

\ExecuteMetaData[Thin.idr.tex]{keep}

Similarly the \IdrisFunction{drop} function demonstrates that we can compute a
thinning getting rid of the variable \IdrisBound{x} freshly added to the target
scope.

\ExecuteMetaData[Thin.idr.tex]{drop}

We can already deploy these smart constructors to implement functions producing
thinnings. We use \IdrisFunction{which} as our example. It is a filter-like
function that returns the elements that satisfy a boolean predicate together with
a proof that there is a thinning embedding them back into the input snoclist.
%
If the input snoclist is empty then the output shall also be, and
\IdrisFunction{done} builds a thinning from \IdrisData{[<]} to itself.
%
If it is not empty we can perform a recursive call on the tail of the snoclist
and then depending on whether the predicates holds true of the head we can either
\IdrisFunction{keep} or \IdrisFunction{drop} it.

\ExecuteMetaData[Thin.idr.tex]{which}

We are now equipped with these smart constructors that allow us to seamlessly
build thinnings.
%
To recover the full expressive power of the inductive family, we also need to
be able to take these thinnings apart. We are now going to tackle this issue.

\subsection{Pattern Matching on \IdrisType{Th}}

A view for a type $T$ is, in the sense of McBride and
McKinna~\cite{DBLP:journals/jfp/McBrideM04}, an inductive family
$V$ indexed by $T$ together with a total function which maps every element $t$ of $T$
to a value of type ($V t$).
%
This simple gadget provides a powerful, user-extensible, generalisation of
pattern-matching.
%
Patterns are defined inductively as either a binding position, a forced term
(i.e. an arbitrary expression that is determined by a constraint arising from
another pattern), or a data constructor fully applied to subpatterns.
%
In contrast, the return indices of an inductive family's constructors can be
arbitrary expressions.

\subsubsection{A simple example of a view}

For instance, it is trivial to check whether a (\IdrisType{List} \IdrisBound{a})
is empty or constructed from a head and a tail.
%
In the first case, we would use the constructor \IdrisData{Nil} which is a valid
pattern according to our above definition.
%
In the seconc case, we would use the constructor \IdrisData{(::)} applied to the
variables \IdrisBound{head} and \IdrisBound{tail} acting as binding sites, thus
forming another valid pattern.
%
But it is impossible to look at the list from its end: the expression
(\IdrisBound{xs} \IdrisFunction{++} \IdrisData{[}\IdrisBound{x}\IdrisData{]})
does not fall in the pattern fragment because it mentions the function (\IdrisFunction{++}).
But we can define a view showing what it means to have this shape.

\ExecuteMetaData[Snoc.idr.tex]{AsSnoc}

We call this inductive family \IdrisType{AsSnoc} because it allows us to get
a handle on the input list as if it were a \IdrisType{Snoclist}. If the list is
empty then it corresponds to an empty snoclist (constructor \IdrisData{Lin})
but if it is non-empty, we get our hands on the initial segment \IdrisBound{xs},
the last element \IdrisBound{x} and a proof that the list is indeed
(\IdrisBound{xs} \IdrisFunction{++} \IdrisData{[}\IdrisBound{x}\IdrisData{]}).
%
We can finish the view definition by implementing the function \IdrisFunction{asSnoc}
that shows that every list \IdrisBound{xs} can be taken apart in an \IdrisType{AsSnoc}
manner.

\ExecuteMetaData[Snoc.idr.tex]{asSnoc}

The interesting case is the one where the input list is non-empty.
%
Using the \IdrisKeyword{with} keyword we can locally modify the function
definition to have it take an extra argument, here the recursive call to
(\IdrisFunction{asSnoc} \IdrisBound{xs})).
%
Correspondingly, we locally extend the pattern-matching definition with
an additional pattern on the left-hand side of the indented with-clauses.
%
By matching on the result of the recursive call, we get to learn about the
structure of the tail. This is precisely the power of views.

In the clause with the pattern \IdrisData{[<]}, we learn that \IdrisBound{xs} is empty
(hence the \emph{forced} pattern denoted by a full stop
\IdrisKeyword{.(}\IdrisData{[]}\IdrisKeyword{)} that replaces \IdrisBound{xs})
and therefore \IdrisBound{x} is the last element.
%
We can return (\IdrisData{[]} \IdrisData{:<} \IdrisBound{x}) whose index is
(\IdrisData{[]} \IdrisFunction{++} \IdrisData{[}\IdrisBound{x}\IdrisData{]})
which computes to \IdrisData{[}\IdrisBound{x}\IdrisData{]}.

%
In the clause with the pattern (\IdrisBound{ys} \IdrisData{:<} \IdrisBound{y}), we learn
that \IdrisBound{xs} has the shape (\IdrisBound{ys} \IdrisFunction{++} \IdrisBound{y})
which would not be a valid pattern where it not forced.
%
We can conclude by pushing \IdrisBound{x} back onto \IdrisBound{ys}.

Note that compared to a function returning a value of type
{\IdrisType{Maybe} (\IdrisType{List} \IdrisBound{a}\IdrisType{,} \IdrisBound{a})},
we have here the guarantee that the result actually reflects the structure of the
input list.
%
For instance, if we had tried to return \IdrisData{Lin} instead of
(\IdrisData{[]} \IdrisData{:<} \IdrisBound{x}) in the singleton list case,
the program would not have typechecked.

We can now use the view to pattern-match on the list from its end almost as
seamlessly as we can match on its head as is demonstrated by the following
comparison between \IdrisFunction{first} and \IdrisFunction{last}.
%
We have left an empty line in the definition of the former to line up each
equation with the matching one in the latter.

\noindent
\begin{minipage}[t]{0.45\textwidth}
\ExecuteMetaData[Snoc.idr.tex]{first}
\end{minipage}\hfill
\begin{minipage}[t]{0.45\textwidth}
\ExecuteMetaData[Snoc.idr.tex]{last}
\end{minipage}

Our goal is to define a view for \IdrisType{Th} that demonstrates
that any thinning has been built using one of the three smart
constructors introduced in the previous section.

\subsubsection{A View for \IdrisType{Th}}

The \IdrisType{View} family is a sum type indexed by a thinning. It has one
data constructor associated to each smart constructor and storing its arguments.

\ExecuteMetaData[Thin.idr.tex]{view}

The accompanying \IdrisFunction{view} function witnesses the fact that any
thinning arises as one of these three cases.
%
We leave out the slightly technical definition of \IdrisFunction{view}, the
interested reader can find it in the accompanying material.
%
We will however discuss the code it compiles to after erasure in
\cref{sec:compiledview}.

\ExecuteMetaData[Thin.idr.tex]{viewtotal}

We can readily use this function to implement pattern matching functions taking
a thinning apart. We can for instance define \IdrisFunction{kept}, the function
that counts the number of \IdrisFunction{keep} smart constructors used when
manufacturing the input thinning and returns a proof that this is exactly the
length of the source scope \IdrisBound{sx}.

\ExecuteMetaData[Thin.idr.tex]{kept}

We proceed by calling the \IdrisFunction{view} function on the input thinning
which immediately tells us that we only have three cases to consider.
%
The \IdrisData{Done} case is easily handled because the branche's refined
types inform us that both \IdrisBound{sx} and \IdrisBound{sy} are the
empty snoclist \IdrisData{[<]} whose length is evidently \IdrisData{0}.
%
In the \IdrisData{Keep} branch we learn that \IdrisBound{sx} has the shape
(\IdrisBound{\KatlaUnderscore} \IdrisData{:<} \IdrisBound{x}) and so we must return the
successor of whatever the result of the recursive call gives us.
%
Finally in the \IdrisData{Drop} case, \IdrisBound{sx} is untouched and so a
simple recursive call suffices.
%
Note that the function is correctly detected as total because the target scope
\IdrisBound{sy} is indeed getting structurally smaller at every single recursive
call.
%
It is runtime irrelevant but it can still be successfully used as a termination
measure by the compiler.

\subsection{The \IdrisType{Thinning} Relation}\label{sec:thininginternal}

We have shown the user-facing \IdrisType{Th} and have claimed that it is possible
to define smart constructors \IdrisFunction{done}, \IdrisFunction{keep},
and \IdrisFunction{drop}, as well as a \IdrisFunction{view} function.
%
This should become apparent once we show the actual definition of \IdrisType{Thinning}.

\subsubsection{Definition of \IdrisType{Thinning}}

The relation maintains the invariant between the record's
fields \IdrisFunction{bigEnd} (a \IdrisType{Nat})
and \IdrisFunction{encoding} (an \IdrisType{Integer})
and the index scopes \IdrisBound{sx} and \IdrisBound{sy}.

\ExecuteMetaData[Thin/Internal.idr.tex]{thinning}

As always, the \IdrisData{Done} constructor is the simplest.
%
It states that the thinning of size \IdrisData{Z} and encoded as the bit
pattern \IdrisData{0} is the empty thinning.

The \IdrisData{Keep} constructor guarantees that the thinning of
size (\IdrisData{S} \IdrisBound{i}) and encoding \IdrisBound{bs}
represents an injection
from (\IdrisBound{sx} \IdrisData{:<} \IdrisBound{x})
to (\IdrisBound{sy} \IdrisData{:<} \IdrisBound{x})
provided that the bit at position \IdrisData{Z} of \IdrisBound{bs}
is set, and that the rest of the bit pattern (obtained by a right shift
on \IdrisBound{bs}) is a valid thinning of size \IdrisBound{i} from
\IdrisBound{sx} to \IdrisBound{sy}.

The \IdrisData{Drop} constructor is structured the same way, except that
it insists the bit at position \IdrisData{Z} should \emph{not} be set.

We can readily use this relation to prove that some basic encoding are
valid representations of useful thinnings.

\subsubsection{Examples of \IdrisType{Thinning} proofs}

For instance, we can always define a thinning from the empty scope to
an arbitrary scope \IdrisBound{sy}.
%
The \IdrisFunction{encoding} of this thinning is \IdrisData{0} because
every variable is being discarded and its \IdrisFunction{bigEnd} is
the length of the outer scope \IdrisBound{sy}.

\ExecuteMetaData[Thin/Internal.idr.tex]{none}

The proof proceeds by induction over the outer scope \IdrisBound{sy}. If it
is empty, we can simply use the constructor for the empty thinning.
%
Otherwise we can invoke \IdrisData{Drop} on the induction hypothesis.
%
This all typechecks because (\IdrisFunction{testBit} \IdrisData{0} \IdrisData{Z})
computes to \IdrisData{False} and so the \IdrisBound{nb} proof can be constructed
automatically by \idris{}'s proof search (cf. \cref{sec:proofsearch}),
%
and (\IdrisData{0} \IdrisFunction{`shiftR`} \IdrisData{1}) evaluates to \IdrisData{0}
which means the induction hypothesis has exactly the right type.


The definition of the identity thinning is a bit more involved.
%
For a scope of size $k$, we are going to need to generate a bit pattern with
$k$ ones followed by zeros.
%
We define it in two steps.
%
First, \IdrisFunction{cofull} defines a bit pattern of $k$ zeros followed by ones
by shifting $k$ places to the left a bit pattern of ones only.
%
Then, we obtain \IdrisFunction{full} by taking the complement of \IdrisFunction{cofull}.

\noindent
\begin{minipage}{.45\textwidth}
\ExecuteMetaData[Data/Bits/Integer.idr.tex]{cofull}
\end{minipage}\hfill
\begin{minipage}{.45\textwidth}
\ExecuteMetaData[Data/Bits/Integer.idr.tex]{full}
\end{minipage}

We can then prove that the identity thinning for a scope of size \IdrisBound{n} is
represented by the pairing of (\IdrisFunction{full} \IdrisBound{n}) as the
\IdrisFunction{encoding} and \IdrisBound{n} as the \IdrisFunction{bigEnd}.

\ExecuteMetaData[Thin/Internal.idr.tex]{ones}


This proof proceeds once more by induction on the scope.
%
If the scope is empty then once again the constructor for the empty thinning will do.
%
In the non-empty case, we first appeal to an auxiliary lemma (not shown here) to
construct \IdrisBound{nb} a proof that the bit at position \IdrisData{Z} for a
non-zero \IdrisFunction{full} integer is known to be \IdrisData{True}.
%
We then need to use another lemma to cast the induction hypothesis which mentions
(\IdrisFunction{full} (\IdrisFunction{length} \IdrisBound{sx})) so that it may be
used in a position where we expect a proof talking about
(\IdrisFunction{full} (\IdrisFunction{length} (\IdrisBound{sx} \IdrisData{:<} \IdrisBound{x}))
\IdrisFunction{`shiftR`} \IdrisData{1}).

\subsubsection{Properties of the \IdrisType{Thinning} relation}

This relation has a lot of convenient properties.

First, it is proof irrelevant: any two proofs that the same
\IdrisBound{i}, \IdrisBound{bs}, \IdrisBound{sx}, and \IdrisBound{sy} are
related are provably equal.
%
Consequently, equality on \IdrisType{Th} values amounts to equality of
the \IdrisFunction{bigEnd} and \IdrisFunction{encoding} values. In particular
it is cheap to test whether a given thinning is the empty or the
identity thinning.

Second, it can be inverted~\cite{DBLP:conf/types/CornesT95} knowing only two bits:
whether the natural number is empty and what the value of the bit at position
\IdrisData{Z} of the encoding is.
%
This means that \IdrisFunction{view} can be efficiently implemented by using
these two checks and then inverting the \IdrisType{Thinning} proof to gain access
to the proof that the remainder of the thinning's encoding is valid.
%
We will see in \cref{sec:compiledview} that this leads to efficient runtime code for the view.

\subsection{Choose Your Own Abstraction Level}

Access to both the high-level \IdrisType{View} and the internal \IdrisType{Thinning}
relation means that programmers can pick the level of abstraction at which they
want to work.
%
They may need to explicitly manipulate bits to implement key operators that are
used in performance-critical paths but can also stay at the highest level for
more negligible operations, or when proving runtime irrelevant properties.

In the previous section we saw simple examples of these bit manipulations when
defining \IdrisFunction{none} (using the constant 0 bit pattern) and
\IdrisFunction{ones} using bit shifting and complement to form an initial segment
of 1s followed by 0s.

Other natural examples include the \emph{meet} and \emph{join} of two thinnings
sharing the same wider scope.
%
The join can for instance be thought of either as a function defined by induction
on the first thinning and case analysis on the second, emitting a \IdrisData{Keep}
constructor whenever either of the inputs does.
%
Or we can observe that the bit pattern in the join is exactly the disjunction of
the inputs' respective bit patterns and prove a lemma about the \IdrisType{Thinning}
relation instead.
%
This can be visualised as follows. In each column, the meet is a
$\bullet$ whenever either of the inputs is.

\[
\begin{array}{c@{~}l}
& \color{blue}{\circ}\color{orange}{\circ}\color{magenta}{\bullet}\color{teal}{\bullet}\color{lightgray}{\circ} \\
  \vee & \color{blue}{\bullet}\color{orange}{\circ}\color{magenta}{\circ}\color{teal}{\bullet}\color{lightgray}{\bullet} \\
  \hline
  & \color{blue}{\bullet}\color{orange}{\circ}\color{magenta}{\bullet}\color{teal}{\bullet}\color{lightgray}{\bullet} \\
\end{array}
\]

The join is of particular importance because it appears when we convert and `opened'
view of a term into its co-de Bruijn counterpart.
%
As we mentioned earlier, co-de Bruijn terms in an arbitrary scope are represented by
the pairing of a term indexed by its precise support with a thinning embedding this
support back into the wider scope.
%
When working with such a representation, it is convenient to have access to an
`opened' view where the outer thinning has been pushed inside therefore exposing
the term's top-level constructor, ready for case-analysis.


The following diagram shows the correspondence between an `opened' application node
using the view (the diamond `\$' node) with two subterms both living in the outer scope
and its co-de Bruijn form (the circular `\$' node) with an outer thinning selecting the
term support.

\noindent
\begin{minipage}{.45\textwidth}\center
  \ExecuteMetaData[ast.tex]{opened}
\end{minipage}\hfill
\begin{minipage}{.45\textwidth}\center
  \ExecuteMetaData[ast.tex]{opening}
\end{minipage}

The outer thinning of the co-de Bruijn term is obtained precisely by
computing the join of the respective outer thinnings of the `opened'
application's function and argument.

These explicit bit manipulations will be preserved during compilation and
thus deliver more efficient code.


\subsection{Compiled Code}\label{sec:compiledview}

The following code block shows the javascript code that is produced when compiling the
\IdrisFunction{view} function. We chose to use the javascript backend rather than e.g.
the chezscheme one because it produces fairly readable code.
%
We have modified the backend to also write comments reminding the reader of the type
of the function being defined and the data constructors the natural number tags
correspond to.

The only manual modifications we have performed are the inlining of a function
corresponding to a \IdrisKeyword{case} block, renaming variables and property names
to make them human-readable, introducing the \texttt{\$tail} definitions to make
lines shorter, and slightly changing the layout.


%% \begin{minted}{javascript}
%% function view($th) {
%%   switch($th.bigEnd) {
%%   // empty thinning, use Done
%%   case 0n: return {tag: 0};
%%   default: {
%%     // bigEnd is (S predBE)
%%     const $predBE = ($th.bigEnd-1n);
%%     // Test whether the bit at index 0 of encoding is non-zero by
%%     // using 1n as a bit mask
%%     // and checking whether the result is not equal to 0
%%     const $bitTest = choose(notEq($th.encoding&1n, 0n));
%%     switch($bitTest.tag) {
%%       case 0: { // The test was true, use Keep
%%         const $tail = $th.encoding>>1n;
%%         return {tag: 1, val: {bigEnd: $predBE, encoding: $tail}}; }
%%       case 1: { // The test was false, use Drop
%%         const $tail = $th.encoding>>1n;
%%         return {tag: 2, val: {bigEnd: $predBE, encoding: $tail}}; }
%% }}}}
%% \end{minted}


\begin{minted}{javascript}
/* Thin.Smart.view : (th : Th sx sy) -> View th */
function Thin_Smart_view($th) {
  switch($th.bigEnd) {
    case 0n: return {h: 0 /* Done */};
    default: {
      const $predBE = ($th.bigEnd-1n);
      const $test = choose(notEq(($th.encoding&1n), 0n)));
      switch($test.tag) {
        case 0: /* Left  */ {
          const $tail = $th.encoding>>1n;
          return { tag: 1 /* Keep */
                 , val: {bigEnd: $predBE, encoding: $tail}}; }
        case 1: /* Right */ {
          const $tail = $th.encoding>>1n;
          return { tag: 2 /* Drop */
                 , val: {bigEnd: $predBE, encoding: $tail}}; }
}}}}
\end{minted}


Readers can see that the compilation process has erased all of the indices
and the proofs
showing that the invariant tying the efficient runtime representation to the
high-level specification is maintained.
%
A thinning is represented at runtime by a javascript object with two properties
corresponding to \IdrisType{Th}'s runtime relevant fields: \IdrisFunction{bigEnd}
and \IdrisFunction{encoding}.
%
Both are storing a javascript \texttt{bigInt} (one corresponding to the
\IdrisType{Nat}, the other to the \IdrisType{Integer}).
%
For instance the thinning [01101] would be at runtime
\mintinline{javascript}{{ bigEnd: 5n, encoding: 13n }}.
%

The view proceeds in two steps. First if the \texttt{bigEnd} is \texttt{0n}
then we know the thinning is empty and can immediately return the \IdrisData{Done}
constructor.
%
Otherwise we know the thinning to be non-empty and so we can compute the big end
of its tail (\texttt{\$predBE}) by subtracting one to the non-zero \texttt{bigEnd}.
We can then inspect the bit at position \texttt{0} to decide whether to return a
\IdrisData{Keep} or a \IdrisData{Drop} constructor. This is performed by using a
bit mask to 0-out all the other bits (\texttt{\$th.bigEnd\&1n}) and checking whether
the result is zero.
%
If it is not equal to 0 then we emit \IdrisData{Keep} and compute the \texttt{\$tail}
of the thinning by shifting the original encoding to drop the 0th bit. Otherwise we
emit \IdrisData{Drop} and compute the same tail.

By running \IdrisFunction{view} on this [01101] thinning, we would get
back (\IdrisData{Keep} [0110]), that is to say
\mintinline{javascript}{{ tag: 1, val: { bigEnd: 4n, encoding: 6n } }}.

Thanks to \idris{}'s implementation of Quantitative Type Theory we have managed
to manufacture a high level representation that can be manipulated like a classic
inductive family using smart constructors and views without giving up an inch of
control on its runtime representation.


%%%%%%%%%%%%%%%%%%%%%%%%%%%%%%%%%%%%%%%%%%%%%%%%%%%%%%%%%%%%%%%%%%%%%%%%%%%%%%
%% conclusion
%%%%%%%%%%%%%%%%%%%%%%%%%%%%%%%%%%%%%%%%%%%%%%%%%%%%%%%%%%%%%%%%%%%%%%%%%%%%%%

\section{Conclusion}\label{sec:conclusion}

We have seen that inductive families provide programmers with ways to root out bugs
by enforcing strong invariants. Unfortunately these families can get in the way of
producing performant code despite existing optimisation passes erasing redundant
or runtime irrelevant data.
%
This has led us to take advantage of Quantitative Type Theory in order to design a library
combining the best of both worlds: the strong invariants and ease of use of inductive
families together with the runtime performance of explicit bit manipulations.

\subsection{Related Work}

For historical and ergonomic reasons, idiomatic code in \coq{} tends to center programs
written in a subset of the language quite close to OCaml and then prove properties
about these programs in the runtime irrelevant \texttt{Prop} fragment.
%
This can lead to awkward encodings when the unrefined inputs force the user to consider
cases which ought to be impossible. Common coping strategies involve relaxing the types
to insert a modicum of partiality e.g. returning an option type or taking an additional
input to be used as the default return value.
%
This approach completely misses the point of type-driven development. We benefit a lot
from having as much information as possible available during interactive editing. This
helps tremendously getting the definitions right by ensuring we always maintain vital
invariants thus making invalid states unrepresentable.
%
Thankfully libraries such as Equations~\cite{DBLP:conf/itp/Sozeau10,DBLP:journals/pacmpl/SozeauM19}
can help users write more dependently typed programs, by taking care of the complex
encoding required in \coq{}. A view-based approach similar to ours but using \texttt{Prop}
instead of the zero quantity ought to be possible.

Prior work on erasure~\cite{DBLP:journals/pacmpl/Tejiscak20} has the advantage of
offering a fully automated analysis of the code. The main inconvenient is that users
cannot state explicitly that a piece of data ought to be runtime irrelevant and so
they may end up inadvertently using it which would prevent its erasure.
%
Quantitative Type Theory allows us users to explicitly choose what is and is not
runtime relevant, with the quantity checker keeping us true to our word.
%
This should ensure that the resulting program has a much more predictable complexity.

A somewhat related idea was explored by Brady, McKinna, and Hammond in the context of
circuit design~\cite{DBLP:conf/sfp/BradyMH07}. In their verification work they index
an efficient representation (natural numbers as a list of bits) by its meaning as a
unary natural number. All the operations are correct by construction as witnessed by
the use of their unary counterparts acting as type-level specifications.
%
In the end their algorithms still process the inductive family instead of working
directly with binary numbers. This makes sense in their setting where they construct
circuits and so are explicitly manipulating wires carrying bits.
%
By contrast, in our motivating example we really want to get down to actual (unbounded)
integers rather than linked lists of bits.

\todo{Iris}

\subsection{Limitations and Future Work}

Unfortunately it is only \emph{propositionally} true that
(\IdrisFunction{view} (\IdrisFunction{keep} \IdrisBound{th} \IdrisBound{x}))
computes to (\IdrisData{Keep} \IdrisBound{th} \IdrisBound{x}) (and similarly for
\IdrisFunction{done}/\IdrisData{Done} and \IdrisFunction{drop}/\IdrisData{Drop}).
%
This means that users may need to manually deploy these lemmas when proving the
properties of functions defined by pattern matching on the result of calling the
\IdrisFunction{view} function.
%
This annoyance would disappear if we had the ability to extend \idris{}'s reduction rules
with user-proven equations as suggested by Cockx, Tabareau, and
Winterhalter~\cite{DBLP:journals/pacmpl/CockxTW21}.
\todo{re-read Taming}

Our objectives are twofold: we would like to explore more low-level operations and
extend language support for this style of programming.

The \idris{} standard library gave us access to a pure interface to explicitly
manipulate an integer's bits. There is no such thing yet for interacting with a
read-only array for instance.


\todo{Explain annoying inversion lemmas}

Our library of efficient operations on thinnings currently suffers from having to
implement annoyingly verbose inversion lemmas. This is caused by the fact that
\idris{} will not let you pattern-match on a runtime irrelevant value in a runtime
relevant context. It should however be safe to do so if all but one branches can
be proven to be impossible, or even more generally, if the information thus obtained
only flows into runtime irrelevant arguments.


\bibliography{thin}

\end{document}
