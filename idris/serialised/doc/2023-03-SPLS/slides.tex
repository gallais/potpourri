\PassOptionsToPackage{x11names}{xcolor}
\documentclass{beamer}

\beamertemplatenavigationsymbolsempty

\usepackage{tikz}
\usetikzlibrary{shapes.geometric}

\usepackage{listings}
\usepackage{lstautogobble} % Provides autogobble, which is useful to remove indentation based on first line

\lstset{ %
  language=C,
  numbers=left,
  numberstyle=\tiny,
  stepnumber=1,
  numbersep=5pt,
  breaklines=true,
  autogobble=true, % Removes indentation based on first line
}


\usepackage{idris2}
\usepackage{catchfilebetweentags}
\makeatletter

\newrobustcmd*\OrigExecuteMetaData[2][\jobname]{%
\CatchFileBetweenTags\CatchFBT@tok{#1}{#2}%
\global\expandafter\CatchFBT@tok\expandafter{%
\expandafter}\the\CatchFBT@tok
}%\OrigExecuteMetaData

\newrobustcmd*\ChkExecuteMetaData[2][\jobname]{%
\CatchFileBetweenTags\CatchFBT@tok{#1}{#2}%
\edef\mytokens{\detokenize\expandafter{\the\CatchFBT@tok}}
\ifx\mytokens\empty\PackageError{catchfilebetweentags}{the tag #2 is not found\MessageBreak in file #1 \MessageBreak called from \jobname.tex}{use a different tag}\fi%
}%\ChkExecuteMetaData

\renewrobustcmd*\ExecuteMetaData[2][\jobname]{%
\ChkExecuteMetaData[#1]{#2}%
\OrigExecuteMetaData[#1]{#2}%
}

\makeatother


\newenvironment{hexdump}{\medskip\ttfamily\obeyspaces\obeylines\noindent}{\medskip}
\newcommand{\hexadata}[1]{\texttt{\IdrisData{#1}}}

\title{
  \[
  \left.
    \begin{tabular}{l}
      Seamless \\
      Correct \\
      Generic
    \end{tabular}
  \right\}
  \text{Programming over Serialised Data}
  \]}
\author{Guillaume Allais}
\institute{University of St Andrews}
\date{SPLS, March 8th 2023}

\begin{document}

\begin{frame}
  \maketitle
\end{frame}

\newcommand{\mknode}[3]{\draw (#1,#2)  circle (.27cm) node[align=center] {\IdrisData{#3}};}
\newcommand{\mkleaf}[2]{\draw[fill=black] (#1,#2) node[align=center] {} +(-.1cm,-.1cm) rectangle +(.1cm,.1cm);}

\begin{frame}{Trees and Pattern Matching}
\begin{minipage}{.5\textwidth}
\begin{tikzpicture}
\mknode{0}{0}{10}
  \mknode{-1}{-1}{5};
    \mknode{-2}{-2}{1};
      \mkleaf{-2.7}{-3};
      \mkleaf{-1.3}{-3};
  \mkleaf{-.2}{-2}
  \mknode{1}{-1}{20}
    \mkleaf{.2}{-2}
    \mkleaf{1.8}{-2}

\draw [->] (-0.27,0) to [out=180,in=90] (-1,-.73);
  \draw [->] (-1.27,-1) to [out=180,in=90] (-2,-1.73);
    \draw [->] (-2.27,-2) to [out=180,in=90] (-2.7,-2.9);
    \draw [->] (-1.73,-2) to [out=0,in=90] (-1.3,-2.9);
  \draw [->] (-.73,-1) to [out=0,in=90] (-.2,-1.9);
\draw [->] (0.27,0) to [out=0,in=90] (1,-.73);
  \draw [->] (.73,-1) to [out=180,in=90] (.2,-1.9);
  \draw [->] (1.27,-1) to [out=0,in=90] (1.8,-1.9);
\end{tikzpicture}
\end{minipage}\hfill
\begin{minipage}{.45\textwidth}
  \ExecuteMetaData[Motivating.idr.tex]{motivation}
\end{minipage}
\end{frame}

\begin{frame}[fragile]{Serialised Data and Pointer Manipulations}
\begin{hexdump}
01 01 01 00 \hexadata{01} 00 \hexadata{05} 00 \hexadata{0a} 01 00 \hexadata{14} 00
\end{hexdump}

\begin{lstlisting}
int sumAt (int buf[], int *ptr, int *res) {
  int tag = buf[*ptr]; (*ptr)++;
  switch (tag) {
    case 0:
      return 0;
    case 1:
      int lsum = sumAt(buf, ptr, res);
      int val  = buf[*ptr]; (*ptr)++;
      *res += val;
      int rsum = sumAt(buf, ptr, res);
      return 0;
    default: return -1;}}
\end{lstlisting}
\end{frame}

\begin{frame}{Seamless}
  \noindent
  \begin{minipage}{.7\textwidth}
    \ExecuteMetaData[SaferIndexed.idr.tex]{tsum}
  \end{minipage}\hfill

  \vfill

  \hfill\begin{minipage}{.7\textwidth}
    \ExecuteMetaData[SaferIndexed.idr.tex]{rsum}
  \end{minipage}
\end{frame}


\begin{frame}{Correct}
  \ExecuteMetaData[Data/Singleton.idr.tex]{singleton}
  \vfill
  \ExecuteMetaData[SaferIndexed.idr.tex]{csum}
\end{frame}


\begin{frame}{Generic}
  \ExecuteMetaData[SaferIndexed.idr.tex]{viewfun}
\end{frame}


\end{document}
