\section{Serialised Representation}\label{sec:hexdump}

Before we can give a meaning to descriptions as pointers into a buffer we
need to decide on a serialisation format.
%
The format we have opted for is split in two parts: a header containing
data that can be used to check that a user's claim that a given file
contains a serialised tree of a given type is correct, followed by the
actual representation of the tree.


For instance, the following binary snippet is a hex dump of a file
containing the serialised representation of a binary tree belonging to
the type we have been using as our running example.
%
The raw data is semantically highlighted:
8-bytes-long \hexaoffset{offsets},
a \hexadesc{type} description of the stored data,
some \hexacons{nodes} of the tree
and the \hexadata{data} stored in the nodes.

\begin{hexdump}
87654321\hphantom{:} 00 11 22 33 44 55 66 77 88 99 AA BB CC DD EE FF
00000000: \hexaoffset{07 00 00 00 00 00 00 00} \hexadesc{02 00 02 03 02 01 03} \hexacons{01}
00000010: \hexaoffset{17 00 00 00 00 00 00 00} \hexacons{01} \hexaoffset{0c 00 00 00 00 00 00}
00000020: \hexaoffset{00} \hexacons{01} \hexaoffset{01 00 00 00 00 00 00 00} \hexacons{00} \hexadata{01} \hexacons{00} \hexadata{05} \hexacons{00} \hexadata{0a}
00000030: \hexacons{01} \hexaoffset{01 00 00 00 00 00 00 00} \hexacons{00} \hexadata{14} \hexacons{00}
\end{hexdump}

More spefically, this block is the encoding of the \IdrisFunction{example}
given in the previous section and,
%
knowing that a \IdrisFunction{leaf} is represented here by \hexacons{00}
and a \IdrisFunction{node} is represented by \hexacons{01}
%
the careful reader can check
(modulo ignoring the type description and offsets for now)
that the data is stored in a depth-first, left-to-right traversal of the tree.


\subsection{Header}

The header consists of an offset allowing us to jump past it in case we do
not care to inspect it, followed by a representation of the \IdrisType{Data}.
%
It is encoded by a byte giving us the number of constructors, followed by
these constructors serialised one after the other.

\IdrisData{None} is represented by \hexadesc{00},
\IdrisData{Byte} is represented by \hexadesc{01},
(\IdrisData{Prod} \IdrisBound{d} \IdrisBound{e}) is represented by
\hexadesc{02} followed by the representation of \IdrisBound{d} and then that of \IdrisBound{e},
and \IdrisData{Rec} is represented by \hexadesc{03}.


In our example, the header was as follows:
\begin{hexdump}
\hexaoffset{07 00 00 00 00 00 00 00} \hexadesc{02 00 02 03 02 01 03}
\end{hexdump}
\noindent and the data description is indeed 7 bytes long. It starts with \hexadesc{02}
meaning that the type has two constructors.
The first one is \hexadesc{00} i.e. \IdrisData{None} (that is to say \IdrisFunction{Leaf}),
and the second one is \hexadesc{02 03 02 01 03} i.e.
\IdrisKeyword{(}\IdrisData{Prod} \IdrisData{Rec}
\IdrisKeyword{(}\IdrisData{Prod} \IdrisData{Byte} \IdrisData{Rec}\IdrisKeyword{))}
(that is to say \IdrisFunction{Node}).
%
According to the header, we do have a \IdrisFunction{Tree} value stored in this file.

\subsection{Tree Serialisation}

Our main focus in the definition of this format is that we should be able
to process the right subtree of a node without having to look at the left
subtree first.
%
This will allow us to, for instance, implement a function looking up the
value stored in the rightmost node in the tree (if it exists) in time linear
in the depth of the tree rather than exponential.
%
To this end each node needs to store an offset measuring the size of the left subtree.
