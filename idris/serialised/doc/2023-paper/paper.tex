\documentclass{article}

\usepackage{idris2}

\title{A Universe for Serialised Data}
\author{Guillaume Allais}

\begin{document}

\maketitle

\begin{abstract}
In typed functional languages, one can typically only manipulate data
in a type-safe manner if it first has been deserialised into an in-memory
tree represented as a graph of nodes-as-structs and subterms-as-pointers.

We demonstrate how we can use QTT as implemented in Idris 2 to define
a small universe of serialised datatypes, and provide generic programs
allowing users to process values stored contiguously in buffers.

Our approach allows implementors to prove the full functional correctness,
in a correct by construction manner, of the IO functions processing the
data stored in the buffer.
\end{abstract}

\section{Introduction}


\end{document}
