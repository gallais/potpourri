\documentclass[10pt]{article}

\bibliographystyle{alpha}

\usepackage{etoolbox}
\usepackage{idris2}
\usepackage{catchfilebetweentags}
\makeatletter

\newrobustcmd*\OrigExecuteMetaData[2][\jobname]{%
\CatchFileBetweenTags\CatchFBT@tok{#1}{#2}%
\global\expandafter\CatchFBT@tok\expandafter{%
\expandafter}\the\CatchFBT@tok
}%\OrigExecuteMetaData

\newrobustcmd*\ChkExecuteMetaData[2][\jobname]{%
\CatchFileBetweenTags\CatchFBT@tok{#1}{#2}%
\edef\mytokens{\detokenize\expandafter{\the\CatchFBT@tok}}
\ifx\mytokens\empty\PackageError{catchfilebetweentags}{the tag #2 is not found\MessageBreak in file #1 \MessageBreak called from \jobname.tex}{use a different tag}\fi%
}%\ChkExecuteMetaData

\renewrobustcmd*\ExecuteMetaData[2][\jobname]{%
\ChkExecuteMetaData[#1]{#2}%
\OrigExecuteMetaData[#1]{#2}%
}

\makeatother

\usepackage{cleveref}

%\usepackage{fullpage}

\usepackage{todonotes}
\setuptodonotes{inline}

\usepackage{listings}

\lstset{ %
  language=C,
  numbers=left,
  numberstyle=\tiny,
  stepnumber=1,
  numbersep=5pt,
  breaklines=true,
}

%%%%%%%%%%%%% BLINDING

\newtoggle{BLIND}
\togglefalse{BLIND}

%%%%%%%%%%%%%%%%%%%%%%%%%%%

\title{Seamless, Correct, Generic Programming over Serialised Data}
\author{\iftoggle{BLIND}{ANONYMOUS}{Guillaume Allais}}

\newcommand{\idris}{Idris 2}
\newcommand{\assertTotal}{\IdrisPostulate{assert\KatlaUnderscore{}total}}
\newcommand{\hexadesc}[1]{\texttt{\IdrisType{#1}}}
\newcommand{\hexadata}[1]{\texttt{\IdrisData{#1}}}
\newcommand{\hexacons}[1]{\texttt{\IdrisFunction{#1}}}
\newcommand{\hexaoffset}[1]{\texttt{{\color{lightgray}#1}}}

\newenvironment{hexdump}{\medskip\ttfamily\obeyspaces\obeylines\noindent}{\medskip}

\begin{document}

\maketitle

\begin{abstract}
In typed functional languages, one can typically only manipulate data
in a type-safe manner if it first has been deserialised into an in-memory
tree represented as a graph of nodes-as-structs and subterms-as-pointers.

We demonstrate how we can use QTT as implemented in \idris{} to define
a small universe of serialised datatypes, and provide generic programs
allowing users to process values stored contiguously in buffers.

Our approach allows implementors to prove the full functional correctness,
in a correct-by-construction manner, of the IO functions processing the
data stored in the buffer.
\end{abstract}

\section{Introduction}

In (typed) functional language we are used to manipulating
structured data by pattern-matching on it.
We include an illustrative example below.

\begin{minipage}{.5\textwidth}

\newcommand{\mknode}[3]{\draw (#1,#2)  circle (.27cm) node[align=center] {\IdrisData{#3}};}
\newcommand{\mkleaf}[2]{\draw[fill=black] (#1,#2) node[align=center] {} +(-.1cm,-.1cm) rectangle +(.1cm,.1cm);}

\begin{tikzpicture}
\mknode{0}{0}{10}
  \mknode{-1}{-1}{5};
    \mknode{-2}{-2}{1};
      \mkleaf{-2.7}{-3};
      \mkleaf{-1.3}{-3};
  \mkleaf{-.2}{-2}
  \mknode{1}{-1}{20}
    \mkleaf{.2}{-2}
    \mkleaf{1.8}{-2}

\draw [->] (-0.27,0) to [out=180,in=90] (-1,-.73);
  \draw [->] (-1.27,-1) to [out=180,in=90] (-2,-1.73);
    \draw [->] (-2.27,-2) to [out=180,in=90] (-2.7,-2.9);
    \draw [->] (-1.73,-2) to [out=0,in=90] (-1.3,-2.9);
  \draw [->] (-.73,-1) to [out=0,in=90] (-.2,-1.9);
\draw [->] (0.27,0) to [out=0,in=90] (1,-.73);
  \draw [->] (.73,-1) to [out=180,in=90] (.2,-1.9);
  \draw [->] (1.27,-1) to [out=0,in=90] (1.8,-1.9);
\end{tikzpicture}

\end{minipage}\hfill
\begin{minipage}{.45\textwidth}
  \ExecuteMetaData[Motivating.idr.tex]{motivation}
\end{minipage}

On the left, an example of a binary tree storing bytes in its nodes and
nothing at its leaves.
%
On the right, a small \idris{} snippet defining the corresponding
inductive type and declaring a function summing up all of the
nodes' content.
%
It proceeds by pattern-matching:
%
if the tree is a leaf then we immediately return 0,
otherwise we start by summing up the left and right subtrees, cast the
byte to a natural number and add everything up.
%
Simply by virtue of being accepted by the typechecker, we know that
this function is covering (all the possible patterns have been handled)
and total (all the recursive calls are performed on smaller trees).

At runtime, the tree will quite probably be represented by
constructors-as-structs and substructures-as-pointers:
%
each constructor is a struct with a tag indicating which
constructor is represented and subsequent fields storing
the constructors' arguments.
%
Each argument will be either a value (e.g. a byte) or a pointer
to either a boxed value or a substructure.
%
If we were to directly write a function processing a value in this
encoding, proving that a dispatch over a tag is covering, and that
the pointer-chasing is terminating relies on global invariants
tying the encoding to the inductive type.
%
Crucially, the functional language allows us to ignore all of these
details and program at a higher level of abstraction where we can
benefit from strong guarantees.

Unfortunately not all data comes structured as inductive values
abstracting over a constructors-as-structs and substructures-as-pointers
runtime representation.
%
Data that is stored in a file or received over the network is typically
represented in a contiguous format.
%
Using a naïve serialisation format (a left-to-right in-order traversal
of the tree where leaves are represented by the byte 0, and nodes by
the byte 1), the above example tree could be represented by the
following list of bytes
(each byte is represented by two hexadecimal character,
we have additionally \IdrisData{highlighted} the bytes stored in the nodes):

\begin{hexdump}
01 01 01 00 \hexadata{01} 00 \hexadata{05} 00 \hexadata{0a} 01 00 \hexadata{14} 00
\end{hexdump}

The idiomatic way to process such data in a functional language
is to first deserialise it as an inductive type and then call
the \IdrisFunction{sum} function we defined above.
%
If we were using a lower-level language however, we could directly
process the serialised data without the need to fully deserialise it.
%
A basic C implementation of such a sum function could look something
like the following:

\begin{lstlisting}
int sumAt (int buf[], int *ptr, int *acc) {
  int tag = buf[*ptr]; (*ptr)++;
  switch (tag) {
    case 0: return 0;
    case 1:
      sumAt(buf, ptr, acc);
      int val = buf[*ptr]; (*ptr)++;
      *acc += val;
      sumAt(buf, ptr, acc);
      return 0;
    default: exit(-1); }}
\end{lstlisting}


This function takes a buffer, a pointer currently indicating the start of
a tree, and an accumulator in which we add up all the values.
%
We start (line 2) by reading the byte the pointer is referencing and
immediately move the pointer past it.
%
This is the tag indicating which constructor is at the root of the tree
and so we inspect it (line 3).
%
If the tag is 0 (line 4), the tree is a leaf and so we return.
%
If the tag is 1 (line 5), then the tree starts with a node and the rest
of the buffer contains the left subtree, the byte stored in the node,
and then the right subtree.
%
We start by summing the left subtree (line 6),
after which the pointer has been moved past its end and is now pointing
at the byte stored in the node.
We can therefore dereference the byte and move the pointer past it (line 7),
increment the accumulator by the value thus obtained (line 8)
and finally sum the right subtree (line 9).
%
At the end the accumulator contains the sum of the nodes' content.

As we can readily see, this program
directly performs pointer arithmetic,
explicitly mentions buffer reads,
and relies on undocumented global invariants
such as the structure of the data stored in the buffer,
or the fact the pointer is being moved along and points directly past
the end of a subtree once \texttt{sumAt} has finished computing
its sum.

Our goal with this work is to completely hide away all of these
dangerous aspects
and offer the user the ability to program over serialised data
just as seamlessly and correctly as
if they were processing inductive values.
%
We will see that
Quantitative Type Theory (QTT)~\cite{DBLP:conf/birthday/McBride16, DBLP:conf/lics/Atkey18}
as implemented in \idris{}~\cite{DBLP:conf/ecoop/Brady21}
empowers us to do just that purely in library code.

\subsection{Seamless Programming over Serialised Data}\label{sec:seamless}

Forgetting about correctness for now, this can be summed up by the
two following code snippets describing how we can compute the sum
of the values stored in the type of tree we are using as a running
example.

\noindent
\begin{minipage}{.4\textwidth}
  \ExecuteMetaData[SaferIndexed.idr.tex]{tsum}
\end{minipage}
\hfill\begin{minipage}{.55\textwidth}
  \ExecuteMetaData[SaferIndexed.idr.tex]{rsum}
\end{minipage}

We reserve for later our detailed explanations of the representations
used in these snippets
(\IdrisType{Data.Mu} in \Cref{sec:trees} and
\IdrisType{Pointer.Mu} in \Cref{sec:pointers}).
%
For now, it is enough to understand that the function on the left hand
side is analogous to the one presented at the start of this article: it
is a pure function taking a tree apart by pattern-matching and returning
a natural number corresponding to the sum of the values stored in its nodes;
%
and that the one on the right hand side is an \IdrisType{IO} process
inspecting a buffer that contains such a tree stored in serialised format
and computing the same sum.

In both cases, if we have uncovered a leaf
({\IdrisData{"Leaf"} \IdrisData{\#}} \IdrisKeyword{\KatlaUnderscore{}})
then we return zero,
and if we have uncovered a node
({\IdrisData{"Node"} \IdrisData{\#}} \IdrisBound{l} \IdrisData{\#} \IdrisBound{b} \IdrisData{\#} \IdrisBound{r})
with
a left branch \IdrisBound{l},
a stored byte \IdrisBound{b},
and a right branch \IdrisBound{r},
then we recursively compute the sums for the left and right subtrees,
cast the byte to a natural number and add everything up.
%
Crucially, the two functions look eerily similar, and the one operating on
serialised data does not explicitly perform error-prone pointer arithmetic,
or low-level buffer reads.
%
This is the first way in which our approach shines.

One major difference between the two functions comes from the fact that
we can easily prove some of the pure function's properties by a structural
induction on the input tree whereas we
cannot prove anything about the \IdrisType{IO} process without first
explicitly postulating the \IdrisType{IO} monad's properties.
%
Our second contribution tackles this issue.

\subsection{Correct Programming over Serialised Data}

We will see that we can refine that second definition to obtain
a correct-by-construction version of
\IdrisFunction{sum}, with almost exactly the same code.

\begin{center}
  \begin{minipage}{.7\textwidth}
    \ExecuteMetaData[SaferIndexed.idr.tex]{csum}
  \end{minipage}
\end{center}

In the above snippet, we can see that the \IdrisType{Pointer.Mu} is indexed
by a phantom parameter: a runtime irrelevant \IdrisBound{t} which has type
(\IdrisType{Data.Mu} \IdrisFunction{Tree}).
%
And so the return type is able to mention the pure \IdrisFunction{Data.sum}
function that consumes inductively defined trees.
%
\IdrisType{Singleton} is, as its name suggests, a singleton type. That is
to say that the natural number we compute is now proven to be equal to the
one computed by the pure \IdrisFunction{sum} function.
%
The implementation itself only differs in that we had to use idiom
brackets~\cite{DBLP:journals/jfp/McbrideP08}, something we will explain
in \todo{ref to singleton section}

In other words, our approach also allows us to prove the functional
correctness of the \IdrisType{IO} procedures processing trees stored
in buffer in serialised format. This is our second main contribution.

\subsection{Generic Programming over Serialised Data}

Last but not least, as Altenkirch and McBride
demonstrated~\cite{DBLP:conf/ifip2-1/AltenkirchM02}:
``With dependently (sic) types, generic programming is just programming:
it is not necessary to write a new compiler each time a useful
universe presents itself.''

All of our results are obtained by generic programming, meaning that
we are not limited to the type of binary trees with bytes stored
in the nodes we used in the examples above: we capture an entire
universe of inductive types.

This is our third contribution.

\subsection{Plan}

In summary, we are going to define a library for the
generic,
seamless,
and correct-by-construction
manipulation of algebraic types in serialised format.


\Cref{sec:desc} introduces the language of descriptions capturing the
subset of inductively defined types that our work can handle.
It differs slightly from usual presentations in that it ensures the
types can be serialised and tracks crucial invariants towards that goal.

\Cref{sec:trees} gives a standard meaning to these data descriptions
as strictly positive endofunctors whose fixpoints give us the expected
inductive types.
%
We will use this standard meaning in the specification layer of our work.

\Cref{sec:hexdump} explores the serialisation format we have picked
for these trees and the definition of a generic serialisation function.

\Cref{sec:pointers} defines IO primitives that operate on serialised
trees stored in an underlying buffer.
%
They encapsulate all the unsafe low-level operations and offer a
high-level interface that allows users to implement correct-by-construction
procedures.


\todo{Add benchmarks?}


\section{Our Universe of Descriptions}\label{sec:desc}

We first need to pin down the domain of our discourse.
%
To talk generically about an entire class of datatypes
without needing to modify the host language
we have decided to perform a universe
construction~\cite{benke-ugpp, DBLP:phd/ethos/Morris07, DBLP:conf/icfp/LohM11}.
%
That is to say that we are going to introduce an inductive type
defining a set of codes together
with an interpretation of these codes as bona fide
host-language types.
%
We will then be able to program generically over the universe of
datatypes by performing induction on the type of
codes~\cite{DBLP:conf/tphol/PfeiferR99}.

The universe we define is in the tradition of
a sums-of-products vision of inductive types~\cite{DBLP:conf/popl/JanssonJ97}
where the data description records additional information about
the static and dynamic size of the data being stored.
%
In our setting, constructors are essentially arbitrarily nested tuples of
values of type unit,
bytes,
and recursive substructures.
%
A datatype is given by listing a choice of constructors.

\subsection{Descriptions}

We start with these constructor descriptions;
they are represented internally by an inductive family \IdrisType{Desc}
declared below.

\ExecuteMetaData[Serialised/Desc.idr.tex]{desctype}

This family has three indices corresponding to three crucial
invariants being tracked.
%
First, an index telling us whether the current description
is being used in the \IdrisBound{rightmost} branch of the overall
constructor description.
%
Second, the \IdrisBound{static}ally known size of the described data
in the number of bytes it occupies.
%
Third, the number of \IdrisBound{offsets} that need to be stored to
compensate for subterms not having a statically known size.
%
The reader should think of \IdrisBound{rightmost} as an `input' index
whereas \IdrisBound{static} and \IdrisBound{offsets} are `output' indices.

Next we define the family proper by giving its four constructors.

\ExecuteMetaData[Serialised/Desc.idr.tex]{desc}

Each constructor can be used anywhere in a description so their return
\IdrisBound{rightmost} index can be an arbitrary boolean.

\IdrisData{None} is the description of values of type unit. The static
size of these values is zero as no data is stored in a value of type unit.
Similarly, they do not require an offset to be stored as we statically
know their size.

\IdrisData{Byte} is the description of bytes.
%
Their static size is precisely one byte, and they do not require an
offset to be stored either.

\IdrisData{Prod} gives us the ability to pair two descriptions together.
Its static size and the number of offsets are the respective sums of the
static sizes and numbers of offsets of each subdescription.
%
The description of the left element of the pair will never be in the
rightmost branch of the overall constructors description and so its
index is \IdrisData{False} while the description of the right element
of the pair is in the rightmost branch precisely whenever the whole pair
is; hence the propagation of the \IdrisBound{r} arbitrary value from the
return index into the description of the right component.

Last but not least, \IdrisData{Rec} is a position for a subtree.
We cannot know its size in bytes statically and so we decide to store
an offset unless we are in the rightmost branch of the overall description.
%
Indeed, there are no additional constructor arguments behind the rightmost
one and so we have no reason to skip past the subterm. Consequently we
do not bother recording an offset for it.


\subsection{Constructors}

We represent a constructor as a record packing together
a name for the constructor,
the description of its arguments (which is, by virtue of
being used at the toplevel, in rightmost position),
and the values of the \IdrisFunction{static} and
\IdrisFunction{offsets} invariants.
%
The two invariants are stored as implicit fields
because their value is easily reconstructed by \idris{}
using unification and so users do not need
to spell them out explicitly.

\ExecuteMetaData[Serialised/Desc.idr.tex]{constructor}

Note that we used \IdrisData{(::)} as the name of the
constructor for records of type \IdrisType{Constructor}.
This allows us to define constructors by forming an
expression reminiscent of Haskell's type declarations:
\IdrisBound{name} \IdrisData{::} \IdrisBound{type}.
%
Returning to our running example, this gives us the following encodings for
leaves that do not store anything
and nodes that contain a left branch, a byte, and a right branch.

\noindent
\begin{minipage}[t]{.38\textwidth}
  \ExecuteMetaData[Serialised/Desc.idr.tex]{treeleaf}
\end{minipage}\hfill
\begin{minipage}[t]{.58\textwidth}
  \ExecuteMetaData[Serialised/Desc.idr.tex]{treenode}
\end{minipage}

\subsection{Datatypes}

A datatype description is given by a number of constructors together with
a vector (also known as a length-indexed list) associating a description
to each of these constructors.

\ExecuteMetaData[Serialised/Desc.idr.tex]{data}

We can then encode our running example as a simple \IdrisType{Data}
declaration: a binary tree whose node stores bytes is described by the choice
of either a \IdrisFunction{Leaf} or \IdrisFunction{Node}, as defined above.

\ExecuteMetaData[Serialised/Desc.idr.tex]{treedesc}

Now that we have a language that allows us to give a description of our
inductive types, we are going to give these descriptions a meaning as trees.

\section{Meaning as Trees}\label{sec:trees}

This is a standard semantics of description akin to
\todo{cite}
and does not make any use of the invariants that \IdrisType{Desc}
tracks.
%
In our work it will be used primarily to allow users to give
a precise specification of the functions they actually want to
write on values stored in buffers.

\subsection{\IdrisType{Desc}s as Functors}

We define the meaning of descriptions as strictly positive
endofunctors on \IdrisType{Type} by induction on said descriptions.
%
\IdrisFunction{Meaning} gives us the action of the functors on objects.

\ExecuteMetaData[Serialised/Desc.idr.tex]{meaning}

Both \IdrisData{None} and \IdrisData{Byte} are interpreted by constant
functors (respectively the one returning the unit type, and the one returning
the type of bytes).

\IdrisData{Rec} is the identity functor.


Finally (\IdrisData{Prod} \IdrisBound{d} \IdrisBound{e})
is interpreted as the pairing of the interpretation of
\IdrisBound{d} and \IdrisBound{e} respectively.
We use our own definition of pairing rather than the
standard library's because it gives us better syntactic sugar:

\ExecuteMetaData[Lib.idr.tex]{pair}

This gives us the action of descriptions on types, let us now
see their action on morphisms.
%
We once again proceed by induction on the description.

\ExecuteMetaData[Serialised/Desc.idr.tex]{fmap}

All cases but the one for \IdrisData{Rec} are structural.
%
Verifying that these definitions respect the functor laws is left as
an exercise for the reader.

\subsection{\IdrisType{Data} as Trees}

Given a datatype description \IdrisBound{cs}, our first goal is
to define what it means to pick a constructor.
%
The \IdrisType{Index} record is a thin layer around a finite
natural number known to be smaller than the number of constructors
this type provides.

\ExecuteMetaData[Serialised/Desc.idr.tex]{index}

We use this type rather than \IdrisType{Fin} directly because it
plays well with inference and allows us to provide users with
syntactic sugar enabling them to use the constructors' names
directly rather than confusing numeric indices.
%
The following function runs a decision procedure
\IdrisFunction{isConstructor} at the type level
in order to turn any raw string \IdrisBound{str}
into the corresponding \IdrisType{Index}.

\ExecuteMetaData[Serialised/Desc.idr.tex]{fromString}

If the name is valid then the \IdrisFunction{isConstructor} will
return a valid \IdrisType{Index} and \idris{} will be able to
\IdrisKeyword{auto}matically filling-in the implicit proof.
%
If the name is not valid then idris will not be able to
find the index and will raise a compile time error.
%
We include below a successful example on the left and a failing test
on the right hand side (\IdrisKeyword{failing} blocks are only
accepted in \idris{} if their body leads to an error).

\begin{minipage}[t]{0.3\textwidth}
  \ExecuteMetaData[Serialised/Desc.idr.tex]{indexleaf}
\end{minipage}\hfill
\begin{minipage}[t]{0.5\textwidth}
\ExecuteMetaData[Serialised/Desc.idr.tex]{notindexcons}
\end{minipage}

Once equipped with the ability to pick constructors, we can define
the type of algebras for the functor described by a \IdrisType{Data}
description. For each possible construtor, we demand an algebra for
the functor corresponding to the meaning of the  constructor's description.

\ExecuteMetaData[Serialised/Desc.idr.tex]{alg}

We can then define the fixpoint of data descriptions as the following
inductive type.

\ExecuteMetaData[Serialised/Desc.idr.tex]{mu}


Note that here we are forced to use \assertTotal{} to convince \idris{}
to accept the definition.
%
Indeed, unlike Agda, \idris{} does not (yet!) track whether a function's
arguments are used in a strictly positive manner.
%
Consequently the positivity checker
is unfortunately unable to see that \IdrisFunction{Meaning} uses its second
argument in a strictly positive manner
and that this is therefore a legal definition.

Now that we can build whole trees as fixpoints of the
meaning of descriptions, we can define convenient aliases for
the \IdrisFunction{Tree} constructors.
%
Note that the leftmost \IdrisData{(\#)} use in each definition corresponds
to the \IdrisType{Mu} constructror while later ones are \IdrisType{Tuple}
constructors.
%
\idris{}'s type-directed disambiguation of constructors allows us to use
this uniform notation for all of these pairing notions.

\ExecuteMetaData[Serialised/Desc.idr.tex]{leaf}
\ExecuteMetaData[Serialised/Desc.idr.tex]{node}

\subsection{Generic Fold}

\IdrisType{Mu} give us the initial fixpoint for these algebras i.e.
we have a \IdrisFunction{fold} function. Here we only use \assertTotal{}
for convenience but this could easily be bypassed by mutually defining
an inlined and specialised version of
(\IdrisFunction{fmap} (\IdrisFunction{fold} \IdrisBound{alg})).

\ExecuteMetaData[Serialised/Desc.idr.tex]{fold}

This enables us to define our running example.

\ExecuteMetaData[Serialised/Desc.idr.tex]{example}

\section{Serialised Representation}\label{sec:hexdump}

Before we can give a meaning to descriptions as pointers into a buffer we
need to decide on a serialisation format.
%
The format we have opted for is split in two parts: a header containing
data that can be used to check that a user's claim that a given file
contains a serialised tree of a given type is correct, followed by the
actual representation of the tree.


For instance,the following binary snippet is a hex dump of a file
containing the serialised representation of a binary tree belonging to
the type we have been using as our running example.
%
The raw data is semantically highlighted:
8-bytes-long \hexaoffset{offsets},
a \hexadesc{type} description of the stored data,
some \hexacons{nodes} of the tree
and the \hexadata{data} stored in the nodes.

\begin{hexdump}
87654321\hphantom{:} 00 11 22 33 44 55 66 77 88 99 AA BB CC DD EE FF
00000000: \hexaoffset{07 00 00 00 00 00 00 00} \hexadesc{02 00 02 03 02 01 03} \hexacons{01}
00000010: \hexaoffset{17 00 00 00 00 00 00 00} \hexacons{01} \hexaoffset{0c 00 00 00 00 00 00}
00000020: \hexaoffset{00} \hexacons{01} \hexaoffset{01 00 00 00 00 00 00 00} \hexacons{00} \hexadata{01} \hexacons{00} \hexadata{05} \hexacons{00} \hexadata{0a}
00000030: \hexacons{01} \hexaoffset{01 00 00 00 00 00 00 00} \hexacons{00} \hexadata{14} \hexacons{00}
\end{hexdump}

More spefically, this block is the encoding of the \IdrisFunction{example}
given in the previous section and,
%
knowing that a \IdrisFunction{leaf} is represented here by \hexacons{00}
and a \IdrisFunction{node} is represented by \hexacons{01}
%
the careful reader can check
(modulo ignoring the type description and offsets for now)
that the data is stored in a depth-first, left-to-right traversal of the tree.


\subsection{Header}

The header consists of an offset allowing us to jump past it in case we do
not care to inspect it, followed by a representation of the \IdrisType{Data}.
%
It is encoded by a byte giving us the number of constructors, followed by
these constructors serialised one after the other.

\IdrisData{None} is represented by \hexadesc{00},
\IdrisData{Byte} is represented by \hexadesc{01},
(\IdrisData{Prod} \IdrisBound{d} \IdrisBound{e}) is represented by
\hexadesc{02} followed by the representation of \IdrisBound{d} and then that of \IdrisBound{e},
and \IdrisData{Rec} is represented by \hexadesc{03}.


In our example, the header was as follows:
\begin{hexdump}
\hexaoffset{07 00 00 00 00 00 00 00} \hexadesc{02 00 02 03 02 01 03}
\end{hexdump}
\noindent and the data description is indeed 7 bytes long. It starts with \hexadesc{02}
meaning that the type has two constructors.
The first one is \hexadesc{00} i.e. \IdrisData{None} (that is to say \IdrisFunction{Leaf}),
and the second one is \hexadesc{02 03 02 01 03} i.e.
\IdrisKeyword{(}\IdrisData{Prod} \IdrisData{Rec}
\IdrisKeyword{(}\IdrisData{Prod} \IdrisData{Byte} \IdrisData{Rec}\IdrisKeyword{))}
(that is to say \IdrisFunction{Node}).
%
According to the header, we do have a \IdrisFunction{Tree} value stored here.

\subsection{Tree Serialisation}

Our main focus in the definition of this format is that we should be able
to process the right subtree of a node without having to look at the left
subtree first.
%
This will allow us to, for instance, implement a function looking up the
value stored in the rightmost node in the tree (if it exists) in time linear
in the depth of the tree rather than exponential.
%
To this end each node needs to store an offset measuring the size of the left subtree.

Before we can define the meaning of our descriptions as pointers
into buffers, we need to decide on a serialised representation of
our trees.



offsets are 8 bytes each and record just enough information to
allow us to jump past parts of the data that we do not require (e.g. finding
the rightmost node can be done without deserialising any left branch),

the type of the data being stored is encoded so that loading
can be done in safe mode

\section{Meaning as Pointers into a Buffer}\label{sec:pointers}

Now that we know the serialisation format, we can give a meaning
to constructor and data descriptions as pointers into a buffer.
%
For reasons that will become apparent in \Cref{sec:bufferfold}
when we start programming over serialised data in a correct-by-construction
manner, our types of `pointers' will be parameterised not only
by the description of the type of the data stored but also by a
runtime-irrelevant inductive value of that type.
%
For now, it is enough to think of these indices as a lightweight
version of the `points to' assertions used in separation
logic~\citep{DBLP:conf/lics/Reynolds02}
when reasoning about imperative programs.
%
We expand on this analogy in \Cref{appendix:hoare} where we
also discuss the connection with the combinators defined
in \Cref{sec:poking}.

\subsection{Tracking Buffer Positions}

We start with the definition of \IdrisType{Pointer.Mu},
the counterpart to \IdrisType{Data.Mu} for serialised values.

\ExecuteMetaData[SaferIndexed.idr.tex]{pointermu}

A tree sitting in a buffer is represented
by a record packing the buffer, the position at which the tree's
root node is stored, and the size of the tree.
%
The record is indexed by \IdrisBound{cs} a \IdrisType{Data} description
and \IdrisBound{t} the tree of type (\IdrisType{Data.Mu} \IdrisBound{cs})
which is represented by the buffer's content.
Neither are mentioned in the types of the record's fields, making them
\emph{phantom types}~\citep{DBLP:conf/dsl/LeijenM99}.
%
Note that according to our serialisation format the size is not stored
in the file but using the size of the buffer, the stored offsets,
and the size of the static data we will always
be able to compute a value corresponding to it.

\ExecuteMetaData[SaferIndexed.idr.tex]{elem}

The counterpart to a \IdrisFunction{Meaning} stores additional information.
%
For a description of type (\IdrisType{Desc} \IdrisBound{r} \IdrisBound{s} \IdrisBound{o})
on top of the buffer, the position at which the root of the meaning resides,
and the size of the layer we additionally have a vector of \IdrisBound{o} offsets
that allow us to efficiently access any value we want.

\subsection{Writing a Tree to a File}\label{sec:writetofile}

Once we have a pointer to a tree \IdrisBound{t} of type \IdrisBound{cs}
(\IdrisType{Pointer.Mu} \IdrisBound{cs} \IdrisBound{t} in the type below)
in a buffer, we can easily write it to a file be it for safekeeping
or sending over the network.


\ExecuteMetaData[SaferIndexed.idr.tex]{writeToFile}

\begin{remark}[Forall Quantifier]
  The \IdrisKeyword{forall} quantifier is sugar for an implicit
  binder at quantity \IdrisKeyword{0}.
  %
  It can be useful to introduce variables that cannot be automatically
  bound in a prenex manner because they have a type dependency over an
  explicitly bound argument.
\end{remark}


We first start by reading the size of the header stored in the buffer.
%
This allows us to compute both the \IdrisBound{start} of the data block
as well as the size of the buffer (\IdrisBound{bufSize}) that will
contain the header followed by the tree we want to write to a file.
%
We then check whether the position of the pointer is exactly the beginning
of the data block.
%
If it is then we are pointing to the whole tree and the current buffer can
be written to a file as is.
%
Otherwise we are pointing to a subtree and need to separate it from its
surrounding context first.
%
To do so we allocate a new buffer of the right size and use the
standard library's \IdrisFunction{copyData} primitive to copy the raw bytes
corresponding to the header first, and the tree of interest second.
%
We can then write the buffer we have picked to a file and happily succeed.



Now that we have pointers and can save the tree they are standing for,
we are only missing the ability to look at the content they are pointing to.
%
But first we need to introduce some basic tools
to be able to talk precisely about this stored content.



\subsection{Related Work}

\subsubsection{Data-Generic Programming}

There is a long tradition of data-generic
programming~\cite{DBLP:conf/ssdgp/Gibbons06} and we will mostly focus here
on the approach based on the careful reification of a precise universe of
discourse as an inductive family in a host type theory,
and the definition of generic programs by induction over this family.

One early such instance is Pfeifer and Rue{\ss}'
`polytypic proof construction'~\cite{DBLP:conf/tphol/PfeiferR99}
meant to replace unsafe meta-programs deriving recursors
(be they built-in support, or user-written tactics).

\todo{See refs. in n-ary polymorphic functions}

In his PhD thesis, Morris~\cite{DBLP:phd/ethos/Morris07} declares various
universes for strictly positive types and families and defines by generic
programming
further types (the type of one-hole contexts),
modalities (the universal and existential predicate lifting over the functors he considers),
and functions (map, boolean equality).


\todo{\cite{DBLP:conf/icfp/LohM11}}

\subsubsection{Efficient Runtime Representation of Inductive Values}

Although not dealing explicitly with programming over serialised data,
Monnier's work~\cite{DBLP:conf/icfp/Monnier19} with its focus on performance and
in particular on the layout of inductive values at runtime,
partially motivated our endeavour.
%
Provided that we find a way to get the specialisation and partial evaluation
of the generically defined views, we ought to be able to achieve --purely in
user code-- Monnier's vision of a representation where n-ary tuples have
constant-time access to each of their component.

\subsubsection{Working on Serialised Data}

LoCal~\cite{DBLP:conf/pldi/VollmerKRS0N19} is the work that originally
motivated the design of this library.
%
We have demonstrated that generic programming within a dependently typed
language can yield the sort of benefits other language can only achieve
by inventing entirely new intermediate languages and compilation schemes.

LoCal was improved upon with a re-thinking of the serialisation scheme
making the approach compatible with parallel
programming~\cite{DBLP:journals/pacmpl/KoparkarRVKN21}.
This impressive improvement is a natural candidate for future work on our
part: the authors demonstrate it is possible to reap the benefits of
both programming over serialised data
and dividing up the work over multiple processors
with almost no additional cost in a purely sequential scenario \todo{double check}.

\subsubsection{Serialisation Formats}

The PADS project~\cite{DBLP:conf/popl/MandelbaumFWFG07} aims to let users
quickly, correctly, and compositionally describe existing formats they
have no control over.
%
As they reminds us, ad-hoc serialisation formats abound be it in
networking, logging, billing, or as output of measurement equipments
in e.g. gene sequencing or molecular biology.
%
Our current project is not offering this kind of versatility as we have
decided to focus on a specific serialisation format with strong
guarantees about the efficient access to subtrees.
%
But our approach to defining correct-by-construction components could
be leveraged in that setting too and bring users strong guarantees about
the traversals they write.

ASN.1\cite{MANUAL:book/larmouth1999} provides users the ability to define a high-level
specification of the exchange format to be used in communications without
the need to concern themselves with the actual encoding as bit patterns.
%
The parsing and encoding implementations are then derived entirely automatically
from the specification just like in our system.
%
The main difference is once again our ability to program in a
correct-by-construction over the values thus represented.



\todo{protobuf}


%%%%%%%%%%%%%%%%%%%%%%%%%%%%%%%%%%%%%%%%%%%%%%%%%%%%%%%%%%%%%%%%%%%%%%%%%%%%%%
%% conclusion
%%%%%%%%%%%%%%%%%%%%%%%%%%%%%%%%%%%%%%%%%%%%%%%%%%%%%%%%%%%%%%%%%%%%%%%%%%%%%%

\section{Conclusion}\label{sec:conclusion}

We have seen that inductive families provide programmers with ways to root out bugs
by enforcing strong invariants. Unfortunately these families can get in the way of
producing performant code despite existing optimisation passes erasing redundant
or runtime irrelevant data.
%
This tension has led us to take advantage of Quantitative Type Theory
in order to design a library
combining the best of both worlds: the strong invariants and ease of use of inductive
families together with the runtime performance of explicit bit manipulations.

\subsection{Related Work}

For historical and ergonomic reasons, idiomatic code in \coq{} tends to center programs
written in a subset of the language quite close to OCaml and then prove properties
about these programs in the runtime irrelevant \texttt{Prop} fragment.
%
This can lead to awkward encodings when the unrefined inputs force the user to consider
cases which ought to be impossible. Common coping strategies involve relaxing the types
to insert a modicum of partiality e.g. returning an option type or taking an additional
input to be used as the default return value.
%
This approach completely misses the point of type-driven development.
%
We benefit a lot from having as much information as possible available during
interactive editing.
%
This information not only helps tremendously getting the definitions right by
ensuring we always maintain vital invariants thus making invalid states
unrepresentable, it also gives programmers access to type-driven automation.
%
Thankfully libraries such as Equations~\cite{DBLP:conf/itp/Sozeau10,DBLP:journals/pacmpl/SozeauM19}
can help users write more dependently typed programs, by taking care of the complex
encoding required in \coq{}. A view-based approach similar to ours but using \texttt{Prop}
instead of the zero quantity ought to be possible.

Prior work on erasure~\cite{DBLP:journals/pacmpl/Tejiscak20} has the advantage of
offering a fully automated analysis of the code. The main inconvenient is that users
cannot state explicitly that a piece of data ought to be runtime irrelevant and so
they may end up inadvertently using it which would prevent its erasure.
%
Quantitative Type Theory allows us users to explicitly choose what is and is not
runtime relevant, with the quantity checker keeping us true to our word.
%
This should ensure that the resulting program has a much more predictable complexity.

A somewhat related idea was explored by Brady, McKinna, and Hammond in the context of
circuit design~\cite{DBLP:conf/sfp/BradyMH07}. In their verification work they index
an efficient representation (natural numbers as a list of bits) by its meaning as a
unary natural number. All the operations are correct by construction as witnessed by
the use of their unary counterparts acting as type-level specifications.
%
In the end their algorithms still process the inductive family instead of working
directly with binary numbers. This makes sense in their setting where they construct
circuits and so are explicitly manipulating wires carrying bits.
%
By contrast, in our motivating example we really want to get down to actual (unbounded)
integers rather than linked lists of bits.

%\todo{IIRC Iris does pointer manipulations. What about bit masks?
%  High level invariants linking memory-mapped data to high level concepts?}

\subsection{Limitations and Future Work}

Overall we found this case study using \idris{}, a state of the art language
based on Quantitative Type Theory, very encouraging.
%
The language implementation is still experimental (see for instance
\cref{appendix:limitations} for some of the bugs we found) but none of
the issues are intrinsic limitations.
%
We hope to be able to push this line of work further, tackling the following
limitations and exploring more advanced use cases.

Unfortunately it is only \emph{propositionally} true that
(\IdrisFunction{view} (\IdrisFunction{keep} \IdrisBound{th} \IdrisBound{x}))
computes to (\IdrisData{Keep} \IdrisBound{th} \IdrisBound{x}) (and similarly for
\IdrisFunction{done}/\IdrisData{Done} and \IdrisFunction{drop}/\IdrisData{Drop}).
%
This means that users may need to manually deploy these lemmas when proving the
properties of functions defined by pattern matching on the result of calling the
\IdrisFunction{view} function.
%
This annoyance would disappear if we had the ability to extend \idris{}'s reduction rules
with user-proven equations as implemented in Agda and formally studied
by Cockx, Tabareau, and Winterhalter~\cite{DBLP:journals/pacmpl/CockxTW21}.

In this paper's case study, we were able to design the core \IdrisType{Invariant}
relation making the invariants explicit in such a way that it would be provably
proof irrelevant.
%
This may not always be possible given the type theory currently implemented by
\idris{}. Adding support for a proof-irrelevant sort of propositions (see e.g.
Altenkirch, McBride, and Swierstra's work~\cite{DBLP:conf/plpv/AltenkirchMS07})
could solve this issue once and for all.

Our objectives with this work and that to come are twofold: we would like to
explore more memory-mapped representations equipped with a high level interface
and extend language support for this style of programming.

The \idris{} standard library thankfully gave us access to a polished pure interface
to explicitly manipulate an integer's bits.
%
However these built-in operations came with no built-in properties whatsoever.
%
And so we had to postulate a (minimal) set of axioms (see \cref{appendix:postulated})
and prove a lot of useful corollaries ourselves.
%
There is even less support for other low-level operations such as reading from
a read-only array, or manipulating pointers.

We also found the use of runtime irrelevance (the \IdrisKeyword{0} quantity)
sometimes somewhat frustrating.
%
Pattern-matching on a runtime irrelevant value in a runtime relevant context
is currently only possible if it is manifest for the compiler that the value
could only arise using one of the family's constructors.
%
In non-trivial cases this is unfortunately only merely provable rather than
self-evident.
%
Consequently we are forced to jump through hoops to appease the quantity
checker, and end up defining complex inversion lemmas to bypass these
limitations.
%
This could be solved by a mix of improvements to the typechecker and
meta-programming using prior ideas on automating
inversion~\cite{DBLP:conf/types/CornesT95,DBLP:conf/types/McBride96,monin:inria-00489412}.


\section*{Acknowledgements}

\iftoggle{BLIND}{remarks from friends}{
  We would like to thank Wouter Swierstra for his helpful comments on a
  draft of this paper.
}

\iftoggle{BLIND}{funding and data}{
This research was partially funded by the Engineering and Physical Sciences
Research Council (grant number EP/T007265/1).
}


\newpage
\bibliography{paper}

\end{document}
