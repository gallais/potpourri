\documentclass{article}

\usepackage{idris2}
\usepackage{catchfilebetweentags}
\makeatletter

\newrobustcmd*\OrigExecuteMetaData[2][\jobname]{%
\CatchFileBetweenTags\CatchFBT@tok{#1}{#2}%
\global\expandafter\CatchFBT@tok\expandafter{%
\expandafter}\the\CatchFBT@tok
}%\OrigExecuteMetaData

\newrobustcmd*\ChkExecuteMetaData[2][\jobname]{%
\CatchFileBetweenTags\CatchFBT@tok{#1}{#2}%
\edef\mytokens{\detokenize\expandafter{\the\CatchFBT@tok}}
\ifx\mytokens\empty\PackageError{catchfilebetweentags}{the tag #2 is not found\MessageBreak in file #1 \MessageBreak called from \jobname.tex}{use a different tag}\fi%
}%\ChkExecuteMetaData

\renewrobustcmd*\ExecuteMetaData[2][\jobname]{%
\ChkExecuteMetaData[#1]{#2}%
\OrigExecuteMetaData[#1]{#2}%
}

\makeatother


\usepackage{fullpage}

\title{A Universe for Serialised Data}
\author{Guillaume Allais}

\newcommand{\idris}{Idris 2}
\newcommand{\assertTotal}{\IdrisPostulate{assert\KatlaUnderscore{}total}}

\begin{document}

\maketitle

\begin{abstract}
In typed functional languages, one can typically only manipulate data
in a type-safe manner if it first has been deserialised into an in-memory
tree represented as a graph of nodes-as-structs and subterms-as-pointers.

We demonstrate how we can use QTT as implemented in \idris{} to define
a small universe of serialised datatypes, and provide generic programs
allowing users to process values stored contiguously in buffers.

Our approach allows implementors to prove the full functional correctness,
in a correct by construction manner, of the IO functions processing the
data stored in the buffer.
\end{abstract}

\section{Introduction}

\section{Our Universe of Descriptions}\label{sec:desc}

We first need to pin down the domain of our discourse.
%
To talk generically about an entire class of datatypes
without needing to modify the host language
we have decided to perform a universe
construction~\cite{benke-ugpp, DBLP:phd/ethos/Morris07, DBLP:conf/icfp/LohM11}.
%
That is to say that we are going to introduce an inductive type
defining a set of codes together
with an interpretation of these codes as bona fide
host-language types.
%
We will then be able to program generically over the universe of
datatypes by performing induction on the type of
codes~\cite{DBLP:conf/tphol/PfeiferR99}.

The universe we define is in the tradition of
a sums-of-products vision of inductive types~\cite{DBLP:conf/popl/JanssonJ97}
where the data description records additional information about
the static and dynamic size of the data being stored.
%
In our setting, constructors are essentially arbitrarily nested tuples of
values of type unit,
bytes,
and recursive substructures.
%
A datatype is given by listing a choice of constructors.

\subsection{Descriptions}

We start with these constructor descriptions;
they are represented internally by an inductive family \IdrisType{Desc}
declared below.

\ExecuteMetaData[Serialised/Desc.idr.tex]{desctype}

This family has three indices corresponding to three crucial
invariants being tracked.
%
First, an index telling us whether the current description
is being used in the \IdrisBound{rightmost} branch of the overall
constructor description.
%
Second, the \IdrisBound{static}ally known size of the described data
in the number of bytes it occupies.
%
Third, the number of \IdrisBound{offsets} that need to be stored to
compensate for subterms not having a statically known size.
%
The reader should think of \IdrisBound{rightmost} as an `input' index
whereas \IdrisBound{static} and \IdrisBound{offsets} are `output' indices.

Next we define the family proper by giving its four constructors.

\ExecuteMetaData[Serialised/Desc.idr.tex]{desc}

Each constructor can be used anywhere in a description so their return
\IdrisBound{rightmost} index can be an arbitrary boolean.

\IdrisData{None} is the description of values of type unit. The static
size of these values is zero as no data is stored in a value of type unit.
Similarly, they do not require an offset to be stored as we statically
know their size.

\IdrisData{Byte} is the description of bytes.
%
Their static size is precisely one byte, and they do not require an
offset to be stored either.

\IdrisData{Prod} gives us the ability to pair two descriptions together.
Its static size and the number of offsets are the respective sums of the
static sizes and numbers of offsets of each subdescription.
%
The description of the left element of the pair will never be in the
rightmost branch of the overall constructors description and so its
index is \IdrisData{False} while the description of the right element
of the pair is in the rightmost branch precisely whenever the whole pair
is; hence the propagation of the \IdrisBound{r} arbitrary value from the
return index into the description of the right component.

Last but not least, \IdrisData{Rec} is a position for a subtree.
We cannot know its size in bytes statically and so we decide to store
an offset unless we are in the rightmost branch of the overall description.
%
Indeed, there are no additional constructor arguments behind the rightmost
one and so we have no reason to skip past the subterm. Consequently we
do not bother recording an offset for it.


\subsection{Constructors}

We represent a constructor as a record packing together
a name for the constructor,
the description of its arguments (which is, by virtue of
being used at the toplevel, in rightmost position),
and the values of the \IdrisFunction{static} and
\IdrisFunction{offsets} invariants.
%
The two invariants are stored as implicit fields
because their value is easily reconstructed by \idris{}
using unification and so users do not need
to spell them out explicitly.

\ExecuteMetaData[Serialised/Desc.idr.tex]{constructor}

Note that we used \IdrisData{(::)} as the name of the
constructor for records of type \IdrisType{Constructor}.
This allows us to define constructors by forming an
expression reminiscent of Haskell's type declarations:
\IdrisBound{name} \IdrisData{::} \IdrisBound{type}.
%
Returning to our running example, this gives us the following encodings for
leaves that do not store anything
and nodes that contain a left branch, a byte, and a right branch.

\noindent
\begin{minipage}[t]{.38\textwidth}
  \ExecuteMetaData[Serialised/Desc.idr.tex]{treeleaf}
\end{minipage}\hfill
\begin{minipage}[t]{.58\textwidth}
  \ExecuteMetaData[Serialised/Desc.idr.tex]{treenode}
\end{minipage}

\subsection{Datatypes}

A datatype description is given by a number of constructors together with
a vector (also known as a length-indexed list) associating a description
to each of these constructors.

\ExecuteMetaData[Serialised/Desc.idr.tex]{data}

We can then encode our running example as a simple \IdrisType{Data}
declaration: a binary tree whose node stores bytes is described by the choice
of either a \IdrisFunction{Leaf} or \IdrisFunction{Node}, as defined above.

\ExecuteMetaData[Serialised/Desc.idr.tex]{treedesc}

Now that we have a language that allows us to give a description of our
inductive types, we are going to give these descriptions a meaning as trees.

\section{Meaning as Trees}\label{sec:trees}

This is a standard semantics of description akin to
\todo{cite}
and does not make any use of the invariants that \IdrisType{Desc}
tracks.
%
In our work it will be used primarily to allow users to give
a precise specification of the functions they actually want to
write on values stored in buffers.

\subsection{\IdrisType{Desc}s as Functors}

We define the meaning of descriptions as strictly positive
endofunctors on \IdrisType{Type} by induction on said descriptions.
%
\IdrisFunction{Meaning} gives us the action of the functors on objects.

\ExecuteMetaData[Serialised/Desc.idr.tex]{meaning}

Both \IdrisData{None} and \IdrisData{Byte} are interpreted by constant
functors (respectively the one returning the unit type, and the one returning
the type of bytes).

\IdrisData{Rec} is the identity functor.


Finally (\IdrisData{Prod} \IdrisBound{d} \IdrisBound{e})
is interpreted as the pairing of the interpretation of
\IdrisBound{d} and \IdrisBound{e} respectively.
We use our own definition of pairing rather than the
standard library's because it gives us better syntactic sugar:

\ExecuteMetaData[Lib.idr.tex]{pair}

This gives us the action of descriptions on types, let us now
see their action on morphisms.
%
We once again proceed by induction on the description.

\ExecuteMetaData[Serialised/Desc.idr.tex]{fmap}

All cases but the one for \IdrisData{Rec} are structural.
%
Verifying that these definitions respect the functor laws is left as
an exercise for the reader.

\subsection{\IdrisType{Data} as Trees}

Given a datatype description \IdrisBound{cs}, our first goal is
to define what it means to pick a constructor.
%
The \IdrisType{Index} record is a thin layer around a finite
natural number known to be smaller than the number of constructors
this type provides.

\ExecuteMetaData[Serialised/Desc.idr.tex]{index}

We use this type rather than \IdrisType{Fin} directly because it
plays well with inference and allows us to provide users with
syntactic sugar enabling them to use the constructors' names
directly rather than confusing numeric indices.
%
The following function runs a decision procedure
\IdrisFunction{isConstructor} at the type level
in order to turn any raw string \IdrisBound{str}
into the corresponding \IdrisType{Index}.

\ExecuteMetaData[Serialised/Desc.idr.tex]{fromString}

If the name is valid then the \IdrisFunction{isConstructor} will
return a valid \IdrisType{Index} and \idris{} will be able to
\IdrisKeyword{auto}matically filling-in the implicit proof.
%
If the name is not valid then idris will not be able to
find the index and will raise a compile time error.
%
We include below a successful example on the left and a failing test
on the right hand side (\IdrisKeyword{failing} blocks are only
accepted in \idris{} if their body leads to an error).

\begin{minipage}[t]{0.3\textwidth}
  \ExecuteMetaData[Serialised/Desc.idr.tex]{indexleaf}
\end{minipage}\hfill
\begin{minipage}[t]{0.5\textwidth}
\ExecuteMetaData[Serialised/Desc.idr.tex]{notindexcons}
\end{minipage}

Once equipped with the ability to pick constructors, we can define
the type of algebras for the functor described by a \IdrisType{Data}
description. For each possible construtor, we demand an algebra for
the functor corresponding to the meaning of the  constructor's description.

\ExecuteMetaData[Serialised/Desc.idr.tex]{alg}

We can then define the fixpoint of data descriptions as the following
inductive type.

\ExecuteMetaData[Serialised/Desc.idr.tex]{mu}


Note that here we are forced to use \assertTotal{} to convince \idris{}
to accept the definition.
%
Indeed, unlike Agda, \idris{} does not (yet!) track whether a function's
arguments are used in a strictly positive manner.
%
Consequently the positivity checker
is unfortunately unable to see that \IdrisFunction{Meaning} uses its second
argument in a strictly positive manner
and that this is therefore a legal definition.

Now that we can build whole trees as fixpoints of the
meaning of descriptions, we can define convenient aliases for
the \IdrisFunction{Tree} constructors.
%
Note that the leftmost \IdrisData{(\#)} use in each definition corresponds
to the \IdrisType{Mu} constructror while later ones are \IdrisType{Tuple}
constructors.
%
\idris{}'s type-directed disambiguation of constructors allows us to use
this uniform notation for all of these pairing notions.

\ExecuteMetaData[Serialised/Desc.idr.tex]{leaf}
\ExecuteMetaData[Serialised/Desc.idr.tex]{node}

\subsection{Generic Fold}

\IdrisType{Mu} give us the initial fixpoint for these algebras i.e.
we have a \IdrisFunction{fold} function. Here we only use \assertTotal{}
for convenience but this could easily be bypassed by mutually defining
an inlined and specialised version of
(\IdrisFunction{fmap} (\IdrisFunction{fold} \IdrisBound{alg})).

\ExecuteMetaData[Serialised/Desc.idr.tex]{fold}

This enables us to define our running example.

\ExecuteMetaData[Serialised/Desc.idr.tex]{example}

\section{Meaning as Pointers into a Buffer}\label{sec:pointers}

Now that we know the serialisation format, we can give a meaning
to constructor and data descriptions as pointers into a buffer.
%
For reasons that will become apparent in \Cref{sec:bufferfold}
when we start programming over serialised data in a correct-by-construction
manner, our types of `pointers' will be parameterised not only
by the description of the type of the data stored but also by a
runtime-irrelevant inductive value of that type.
%
For now, it is enough to think of these indices as a lightweight
version of the `points to' assertions used in separation
logic~\citep{DBLP:conf/lics/Reynolds02}
when reasoning about imperative programs.
%
We expand on this analogy in \Cref{appendix:hoare} where we
also discuss the connection with the combinators defined
in \Cref{sec:poking}.

\subsection{Tracking Buffer Positions}

We start with the definition of \IdrisType{Pointer.Mu},
the counterpart to \IdrisType{Data.Mu} for serialised values.

\ExecuteMetaData[SaferIndexed.idr.tex]{pointermu}

A tree sitting in a buffer is represented
by a record packing the buffer, the position at which the tree's
root node is stored, and the size of the tree.
%
The record is indexed by \IdrisBound{cs} a \IdrisType{Data} description
and \IdrisBound{t} the tree of type (\IdrisType{Data.Mu} \IdrisBound{cs})
which is represented by the buffer's content.
Neither are mentioned in the types of the record's fields, making them
\emph{phantom types}~\citep{DBLP:conf/dsl/LeijenM99}.
%
Note that according to our serialisation format the size is not stored
in the file but using the size of the buffer, the stored offsets,
and the size of the static data we will always
be able to compute a value corresponding to it.

\ExecuteMetaData[SaferIndexed.idr.tex]{elem}

The counterpart to a \IdrisFunction{Meaning} stores additional information.
%
For a description of type (\IdrisType{Desc} \IdrisBound{r} \IdrisBound{s} \IdrisBound{o})
on top of the buffer, the position at which the root of the meaning resides,
and the size of the layer we additionally have a vector of \IdrisBound{o} offsets
that allow us to efficiently access any value we want.

\subsection{Writing a Tree to a File}\label{sec:writetofile}

Once we have a pointer to a tree \IdrisBound{t} of type \IdrisBound{cs}
(\IdrisType{Pointer.Mu} \IdrisBound{cs} \IdrisBound{t} in the type below)
in a buffer, we can easily write it to a file be it for safekeeping
or sending over the network.


\ExecuteMetaData[SaferIndexed.idr.tex]{writeToFile}

\begin{remark}[Forall Quantifier]
  The \IdrisKeyword{forall} quantifier is sugar for an implicit
  binder at quantity \IdrisKeyword{0}.
  %
  It can be useful to introduce variables that cannot be automatically
  bound in a prenex manner because they have a type dependency over an
  explicitly bound argument.
\end{remark}


We first start by reading the size of the header stored in the buffer.
%
This allows us to compute both the \IdrisBound{start} of the data block
as well as the size of the buffer (\IdrisBound{bufSize}) that will
contain the header followed by the tree we want to write to a file.
%
We then check whether the position of the pointer is exactly the beginning
of the data block.
%
If it is then we are pointing to the whole tree and the current buffer can
be written to a file as is.
%
Otherwise we are pointing to a subtree and need to separate it from its
surrounding context first.
%
To do so we allocate a new buffer of the right size and use the
standard library's \IdrisFunction{copyData} primitive to copy the raw bytes
corresponding to the header first, and the tree of interest second.
%
We can then write the buffer we have picked to a file and happily succeed.



Now that we have pointers and can save the tree they are standing for,
we are only missing the ability to look at the content they are pointing to.
%
But first we need to introduce some basic tools
to be able to talk precisely about this stored content.




\end{document}
