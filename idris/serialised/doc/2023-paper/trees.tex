\section{Meaning as Trees}\label{sec:trees}

\subsection{\IdrisType{Desc}s as Functors}

We can define the meaning of descriptions as strictly positive
endofunctors on \IdrisType{Type} by induction on said descriptions.
%
\IdrisFunction{Meaning} gives us the action of the functors on objects
while \IdrisFunction{fmap} gives us their action on morphisms.
%
Verifying that these definitions respect the functor laws is left as
an exercise for the reader.

Both \IdrisData{None} and \IdrisData{Byte} are interpreted by constant
functors (respectively the one returning the unit type, and the one returning
the type of bytes),
%
(\IdrisData{Prod} \IdrisBound{d} \IdrisBound{e}) is interpreted as the
pairing of the interpretation of \IdrisBound{d} and \IdrisBound{e} respectively,
%
and \IdrisData{Rec} is the identity functor.

\ExecuteMetaData[Serialised/Desc.idr.tex]{meaning}

This gives us the action of descriptions on types, let us now
see their action on morphisms.
%
We once again proceed by induction on the description.
%
All cases but the one for \IdrisData{Rec} are structural.

\ExecuteMetaData[Serialised/Desc.idr.tex]{fmap}

\subsection{\IdrisType{Data} as Trees}

We can define the type of algebras for the functor described by a
\IdrisType{Data} description as follows
%: for each choice of constructor
%(\IdrisType{Fin} (\IdrisFunction{length} \IdrisBound{cs})),
%we can collapse a layer

\ExecuteMetaData[Serialised/Desc.idr.tex]{alg}

We can then define the fixpoints of functors as


\ExecuteMetaData[Serialised/Desc.idr.tex]{mu}

%
Note that here we are forced to use \assertTotal{} to force \idris{}
to accept the definition.
%
Indeed, unlike Agda, \idris{} does not (yet!) track whether a function's
arguments are used in a strictly positive manner.
%
Consequently the positivity checker
is unfortunately unable to see that \IdrisFunction{Meaning} uses its second
argument in a strictly positive manner
and that this is therefore a legal definition.

Now that we are able to build whole trees as fixpoints of the
meaning of descriptions, we can define convenient aliases for
the \IdrisFunction{Tree} constructors.

\ExecuteMetaData[Serialised/Desc.idr.tex]{leaf}
\ExecuteMetaData[Serialised/Desc.idr.tex]{node}


\IdrisType{Mu} give us the initial fixpoint for these algebras i.e.
we have a \IdrisFunction{fold} function. Here we only use \assertTotal
for convenience but this could easily be bypassed by mutually defining
an inlined and specialised version of
(\IdrisFunction{fmap} (\IdrisFunction{fold} \IdrisBound{alg})).

\ExecuteMetaData[Serialised/Desc.idr.tex]{fold}

This enables us to define our running example.

\ExecuteMetaData[Serialised/Desc.idr.tex]{example}
