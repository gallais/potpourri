\section{Meaning as Pointers into a Buffer}\label{sec:pointers}

Now that we know the serialisation format, we can give a meaning
to constructor and data descriptions as pointers into a buffer.

\subsection{Tracking Buffer Positions}

We start with the definition of the counterpart to \IdrisType{Mu}
for serialised values. A tree sitting in a buffer is represented
by the pairing of the buffer and the offset at which the tree's
root node is stored.

\ExecuteMetaData[SaferIndexed.idr.tex]{pointermu}

For reasons that will become apparent in \Cref{sec:bufferfold}
when we start programming over serialised data in a correct-by-construction
manner the record \IdrisType{Mu} is parameterised not only by the description
of the data type stored but also by a runtime-irrelevant inductive value of
that type.

\ExecuteMetaData[SaferIndexed.idr.tex]{elem}

Now that we have pointers, we can use them to look at the content
they are referring to.

\subsection{Poking the Buffer}

The precise indexing of pointers by a runtime-irrelevant copy of the value
they are pointing to means that inspecting the buffer's content should
not only return runtime information but also refine the index to reflect
that information at the type-level.
%
As a consequence, the functions we are going to define in the following
subsections are views.


\section{Interlude: Views and Singletons}\label{sec:view}

The precise indexing of pointers by a runtime-irrelevant copy of the value
they are pointing to means that inspecting the buffer's content should
not only return runtime information but also refine the index to reflect
that information at the type-level.
%
As a consequence, the functions we are going to define in the following
subsections are views.

\subsection{Views}

A view
in the sense of Wadler~\citep{DBLP:conf/popl/Wadler87},
and subsequently refined by McBride and McKinna~\citep{DBLP:journals/jfp/McBrideM04}
for a type $T$ is a type family $V$ indexed by $T$ together
with a function which maps values $t$ of type $T$ to values of type
$V\,t$.
%
By inspecting the $V\,t$ values we can learn something about the
$t$ input.
%
The prototypical example is perhaps the `snoc` (`cons' backwards) view
of right-nested lists as if they were left-nested.
We present the \IdrisType{Snoc} family below.

\ExecuteMetaData[Snoc.idr.tex]{Snoc}

By matching on a value of type
(\IdrisType{Snoc} \IdrisBound{xs}) we get to learn
either that \IdrisBound{xs} is empty (\IdrisData{Lin}, nil backwards)
or that it has an initial segment \IdrisBound{init} and a last element
\IdrisBound{last} (\IdrisBound{init} \IdrisData{:<} \IdrisBound{last}).
%
The function \IdrisFunction{unsnoc} demonstrates that we can always
\emph{view} a \IdrisType{List} in a \IdrisType{Snoc}-manner.

\ExecuteMetaData[Snoc.idr.tex]{unsnoc}

\begin{remark}[With-Abstraction]
  The \IdrisKeyword{with} construct allows programmers to locally
  define an anonymous auxiliary function taking an extra argument
  compared to its parent.
  %
  By writing (\IdrisKeyword{with} \IdrisKeyword{(}$e$\IdrisKeyword{)})
  we introduce such an auxiliary function an immediately apply it to $e$.
  %
  The nested clauses that immediately follow each take an extra pattern
  which matches over the possible values of $e$.
  %
  If the left-hand side of the auxiliary function is the same as that of
  its parents bar the pattern for the newly added argument, we can use the
  elision notation (\IdrisKeyword{\_} \IdrisKeyword{|}) to avoid having to
  repeat ourselves.

  In other words the following definition of \IdrisFunction{f} using a
  \IdrisKeyword{with} construct with the elision notation

  \ExecuteMetaData[With.idr.tex]{fsugar}

  \noindent is equivalent to the following desugared version where
  the auxiliary function taking an extra argument has been lifted
  to the toplevel.

  \ExecuteMetaData[With.idr.tex]{fdesugar}

\end{remark}

In the above code we performed a recursive call on
(\IdrisFunction{unsnoc} \IdrisBound{xs}) and distinguished
two cases: when the view returns the empty snoclist \IdrisData{[<]}
and when it returns an \IdrisBound{init}ial segment together with the
\IdrisBound{last} element.
%
Because we are using a view, matching on these constructors actually
refined the shape of the parent clause's argument \IdrisBound{xs}.
We do not need to spell out its exact shape in each branch because
we were careful to only introduce \IdrisBound{xs} as a name alias
using an as-pattern while letting the actual pattern be a catch-all
pattern (\IdrisBound{xs}\IdrisKeyword{@}\IdrisKeyword{\_}).
%
This is a common trick to make working with views as lightweight as
possible.

Here we defined \IdrisType{Snoc} as an inductive family but it can
sometimes be convenient to define the family recursively instead,
in which case the \IdrisType{Singleton} inductive family can
help us connect runtime values to their
runtime-irrelevant type-level counterparts.


\subsection{The Singleton type}\label{sec:datasingleton}

The \IdrisType{Singleton} family has a single constructor
which takes an argument \IdrisBound{x} of type \IdrisBound{a},
its return type is indexed precisely by this \IdrisBound{x}.

\ExecuteMetaData[Data/Singleton.idr.tex]{singleton}

More concretely this means that a value of type
(\IdrisType{Singleton} $t$) has to be a runtime relevant
copy of the term $t$.
%
Note that \idris{} performs an optimisation similar to Haskell's
\texttt{newtype} unwrapping: every data type that has a single
non-recursive constructor with only one non-erased argument
is unwrapped during compilation.
%
This means that at runtime the
\IdrisType{Singleton} / \IdrisData{MkSingleton} indirections
will have disappeared.

We can define some convenient combinators to manipulate
singletons.
%
We reuse the naming conventions typical of applicative
functors which will allow us to rely on \idris{}'s automatic
desugaring of \emph{idiom brackets}~\cite{DBLP:journals/jfp/McbrideP08}
into expressions using these combinators.

\ExecuteMetaData[Data/Singleton.idr.tex]{pure}

First \IdrisFunction{pure} is a simple alias for \IdrisData{MkSingleton},
it turns a runtime-relevant value \IdrisBound{x} into a singleton for
this value.

\ExecuteMetaData[Data/Singleton.idr.tex]{fmap}

Next, we can `map' a function under a \IdrisType{Singleton} layer: given
a pure function \IdrisBound{f} and a runtime copy of \IdrisBound{t} we
can get a runtime copy of (\IdrisBound{f} \IdrisBound{t}).

\ExecuteMetaData[Data/Singleton.idr.tex]{ap}

Finally, we can apply a runtime copy of a function \IdrisBound{f}
to a runtime copy of an argument \IdrisBound{t}
to get a runtime copy of the result (\IdrisBound{f} \IdrisBound{t}).

As we mentioned earlier, \idris{} automatically desugars idiom brackets
using these combinators. That is to say that
\IdrisKeyword{[|} \IdrisBound{x} \IdrisKeyword{|]} will be elaborated to
(\IdrisFunction{pure} \IdrisBound{x}) while
\IdrisKeyword{[|} \IdrisBound{f} \IdrisBound{t1} $\cdots$ \IdrisBound{tn} \IdrisKeyword{|]}
will become
(\IdrisBound{f} \IdrisFunction{<\$>} \IdrisBound{t1} \IdrisFunction{<*>} $\cdots$ \IdrisFunction{<*>} \IdrisBound{tn}).
%
This lets us apply \IdrisType{Singleton}-wrapped values almost as seamlessly as pure values.


We are now equipped with the appropriate notions and definitions to
look at a buffer's content.




But first let us have a look at the \IdrisType{Singleton} family
which will help us connect the values we read or compute to their
runtime-irrelevant type-level counterparts.


\subsection{The Singleton type}\label{sec:datasingleton}

The \IdrisType{Singleton} family has a single constructor
which takes an argument \IdrisBound{x} of type \IdrisBound{a},
its return type is indexed precisely by this \IdrisBound{x}.

\ExecuteMetaData[Data/Singleton.idr.tex]{singleton}

More concretely this means that a value of type
(\IdrisType{Singleton} $t$) has to be a runtime relevant
copy of the term $t$.
%
Note that \idris{} performs an optimisation similar to Haskell's
\texttt{newtype} unwrapping: every data type that has a single
non-recursive constructor with only one non-erased argument
is unwrapped during compilation.
%
This means that at runtime the
\IdrisType{Singleton} / \IdrisData{MkSingleton} indirections
will have disappeared.

We can define some convenient combinators to manipulate
singletons.
%
We reuse the naming conventions typical of applicative
functors which will allow us to rely on \idris{}'s automatic
desugaring of \emph{idiom brackets}~\cite{DBLP:journals/jfp/McbrideP08}
into expressions using these combinators.

\ExecuteMetaData[Data/Singleton.idr.tex]{pure}

First \IdrisFunction{pure} is a simple alias for \IdrisData{MkSingleton},
it turns a runtime-relevant value \IdrisBound{x} into a singleton for
this value.

\ExecuteMetaData[Data/Singleton.idr.tex]{fmap}

Next, we can `map' a function under a \IdrisType{Singleton} layer: given
a pure function \IdrisBound{f} and a runtime copy of \IdrisBound{t} we
can get a runtime copy of (\IdrisBound{f} \IdrisBound{t}).

\ExecuteMetaData[Data/Singleton.idr.tex]{ap}

Finally, we can apply a runtime copy of a function \IdrisBound{f}
to a runtime copy of an argument \IdrisBound{t}
to get a runtime copy of the result (\IdrisBound{f} \IdrisBound{t}).

As we mentioned earlier, \idris{} automatically desugars idiom brackets
using these combinators. That is to say that
\IdrisKeyword{[|} \IdrisBound{x} \IdrisKeyword{|]} will be elaborated to
(\IdrisFunction{pure} \IdrisBound{x}) while
\IdrisKeyword{[|} \IdrisBound{f} \IdrisBound{t1} $\cdots$ \IdrisBound{tn} \IdrisKeyword{|]}
will become
(\IdrisBound{f} \IdrisFunction{<\$>} \IdrisBound{t1} \IdrisFunction{<*>} $\cdots$ \IdrisFunction{<*>} \IdrisBound{tn}).
%
This lets us apply \IdrisType{Singleton}-wrapped values almost as seamlessly as pure values.


Our most basic operation consists in poking the buffer to unfold
the description by exactly one step.

\ExecuteMetaData[SaferIndexed.idr.tex]{pokefun}

The result type of this operation is defined by case-analysis on the
description. In order to keep the notations user-friendly, we
mutually define
a recursive function \IdrisFunction{Poke} interpreting the straightforward type constructors
and an inductive family \IdrisType{Poke'} with interesting return indices.

\ExecuteMetaData[SaferIndexed.idr.tex]{pokedatafun}

Poking a buffer containing \IdrisData{None} will return a value of
the unit type as no information whatsoever is stored there.

If we access a \IdrisData{Byte} then we expect that inspecting the
buffer will yield a runtime-relevant copy of the type-level byte we
have for reference. Hence the use of \IdrisType{Singleton}.

If we access a \IdrisData{Prod} of two descriptions then the type-level term
better be a pair and we better be able to obtain a \IdrisType{Pointer.Meaning}
to each of the sub-meanings.
%
Because \idris{} does not currently support definitional eta equality
for records, it will me more ergonomic for users if we introduce
\IdrisType{Poke'} rather than yielding a \IdrisType{Tuple} of values.

Last but not least if the description is \IdrisData{Rec} this means
we have a substructure. In this case we simply demand a pointer to it.

\ExecuteMetaData[SaferIndexed.idr.tex]{pokedatadata}

The implementation of this operation proceeds by case analysis
on the description.
%
As we are going to see shortly, it is necessarily somewhat unsafe
as we claim to be able to connect a type-level value to whatever
it is that we read from the buffer.

\ExecuteMetaData[SaferIndexed.idr.tex]{pokefunNone}

If the description is \IdrisData{None} we do not need to fetch any
information from the buffer.

\ExecuteMetaData[SaferIndexed.idr.tex]{pokefunByte}

If the description is \IdrisData{Byte} then we read a byte at the
determined position. The only way we can connect this value we just
read to the type index is to use the unsafe combinator
\IdrisPostulate{unsafeMkSingleton} to manufacture a value of type
(\IdrisType{Singleton} \IdrisBound{t}) instead of the value of type
(\IdrisType{Singleton} \IdrisBound{bs})
we would expect from wrapping \IdrisBound{bs} in the \IdrisData{MkSingleton} constructor.


\ExecuteMetaData[SaferIndexed.idr.tex]{pokefunProd}

If the description is the product of two sub-descriptions then we
want to compute the \IdrisType{Pointer.Meaning} corresponding to
each of them.
%
We start by splitting the vector of offsets to distribute them between
the left and right subtrees.
%
We can readily build the pointer for the \IdrisBound{left} subdescription:
it takes the left offsets, the buffer, and has the same starting position
as the whole description of the product as they are stored one after the other.
%
We then compute the starting position of the right subtree: we need to
move past the whole of the left subtree, that is to say past the space
reserved by all of the left offsets (\IdrisFunction{sum} \IdrisBound{subl})
but also past the content whose size is known statically (\IdrisBound{sl}).
%
We can finally use an eta-equality lemma to turn \IdrisBound{t} into
(\IdrisFunction{fst} \IdrisBound{t} \IdrisData{\#} \IdrisFunction{snd} \IdrisBound{t})
which lets us use the \IdrisType{Poke'} constructor \IdrisData{(\#)} to return our
pair of pointers.

\ExecuteMetaData[SaferIndexed.idr.tex]{pokefunRec}

Lastly, when we reach a \IdrisData{Rec} description, we can discard the
vector of offsets and return a \IdrisType{Pointer.Mu} with the same buffer
and starting position as our input pointer.

\subsection{Extracting one layer}

By repeatedly poking the buffer, we can unfold a full layer.
The result of this operation is defined by induction
on the description. It is identical to the definition of
\IdrisFunction{Poke} except for the \IdrisData{Prod} case:
here, instead of being content with a pointer for each of the
subdescriptions, we demand a full layer for them too.

\ExecuteMetaData[SaferIndexed.idr.tex]{layerdata}

This function can easily be implemented by induction on the description
and repeatedly calling \IdrisFunction{poke} to expose the values one by
one.

\ExecuteMetaData[SaferIndexed.idr.tex]{layerfun}

\subsection{Exposing the top constructor}

\todo{remind people of the serialisation format here}

Now that we can deserialise an entire layer of \IdrisFunction{Meaning},
the only thing we are missing to be able to generically manipulate trees
is the ability to expose the top constructor of a tree stored at a
\IdrisType{Pointer.Mu} position.
%
The \IdrisType{Out} family describes what it means to get your hands on
the index of a tree's constructor: you obtain an \IdrisType{Index},
and a \IdrisType{Pointer.Meaning} to the constructor's payload.
%
The return type ensures that the structure of the tree is adequately
described by combining both components.

\ExecuteMetaData[SaferIndexed.idr.tex]{outdata}

The type of the \IdrisFunction{out} is as expected: given a pointer
to a tree \IdrisBound{t} of type \IdrisBound{cs} we can get a value
of type (\IdrisType{Out} \IdrisBound{cs} \IdrisBound{t}).
%
That is to say, we can get a view allowing us to reveal what the
index of the tree's head constructor is.

\ExecuteMetaData[SaferIndexed.idr.tex]{outfun}

The implementation is fairly straightforward except for another
unsafe step meant to reconcile the information we read in the buffer
with the runtime-irrelevant tree index.

\ExecuteMetaData[SaferIndexed.idr.tex]{outfunbody}

We start by reading the tag \IdrisBound{k}
corresponding to the constructor choice:
we obtain a byte by calling \IdrisFunction{getBits8}, cast it to a
natural number and then make sure that it is in the range
$[0 \cdots \text{\IdrisFunction{consNumber} \IdrisBound{cs}}[$ using
\IdrisFunction{natToFin}.
%
We then use the unsafe \IdrisPostulate{unfoldAs} primitive to step the
type-level \IdrisBound{t} to something of the form
(\IdrisBound{k} \IdrisData{\#} \IdrisBound{val}).
The declaration of \IdrisPostulate{unfoldAs} is marked as runtime
irrelevant because it cannot possibly be implemented
(\IdrisBound{t} is runtime irrelevant and so cannot be inspected)
and so should its output should not be relied upon in runtime-relevant
computations.

\ExecuteMetaData[Serialised/Desc.idr.tex]{unfoldAs}

Now that we know the head constructor we want to deserialise and that
we have the ability to step the runtime irrelevant tree to match the
actual content of the buffer, we can use \IdrisFunction{getConstructor}
to build such a value.

\ExecuteMetaData[SaferIndexed.idr.tex]{getConstructor}

\todo{explain}

\ExecuteMetaData[SaferIndexed.idr.tex]{getOffsets}

The implementation of \IdrisType{getOffsets} is straightforward: given
a continuation that expect \IdrisBound{n} offsets as well as the
position past the last of these offsets, we read the 8-bytes-long
offsets one by one and pass them to the continuation, making sure
that we move the current position accordingly before every recursive call.

\subsection{Offering a convenient \IdrisType{View}}

We can combine \IdrisFunction{out} and \IdrisFunction{layer} to obtain
the \IdrisFunction{view} function we used in our introductory examples
in~\Cref{sec:seamless}.
%
A (\IdrisType{View} \IdrisBound{cs} \IdrisBound{t}) value gives us
access to the (\IdrisType{Index} \IdrisBound{cs}) of
\IdrisBound{t}'s top constructor together with the corresponding
\IdrisFunction{Layer} of deserialised values and pointers to subtrees.

\ExecuteMetaData[SaferIndexed.idr.tex]{viewdata}

The implementation of \IdrisFunction{view} is unsurprising: we use
\IdrisFunction{out} to expose the top constructor index and a
\IdrisType{Pointer.Meaning} to the constructor's payload.
%
We then user \IdrisFunction{layer} to extract the full
\IdrisFunction{Layer} of deserialised values that the
pointer references.

\ExecuteMetaData[SaferIndexed.idr.tex]{viewfun}
\ExecuteMetaData[SaferIndexed.idr.tex]{viewfunbody}

It is worth noting that although a \IdrisFunction{view} may be
convenient to consume, a performance-minded user may decide to
directly use the \IdrisFunction{out} and \IdrisFunction{poke}
combinators to avoid deserialising values that they do not need.
%
For instance, given a binary tree storing strings in its nodes and
natural numbers at its leaves, if the user only wants to extract the
rightmost natural number in the tree there is no need to deserialise
all of the strings encountered on the way down.

\todo{Add rightmost and rightmost' as case studies in an appendix?}
