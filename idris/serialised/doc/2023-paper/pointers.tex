\section{Meaning as Pointers into a Buffer}\label{sec:pointers}

Now that we know the serialisation format, we can give a meaning
to constructor and data descriptions as pointers into a buffer.

\subsection{Tracking Buffer Positions}

We start with the definition of the counterpart to \IdrisType{Mu}
for serialised values. A tree sitting in a buffer is represented
by the pairing of the buffer and the offset at which the tree's
root node is stored.

\ExecuteMetaData[SaferIndexed.idr.tex]{pointermu}

For reasons that will become apparent in \todo{ref}
when we start programming over serialised data in a correct-by-construction
manner the record \IdrisType{Mu} is parameterised not only by the description
of the data type stored but also by a runtime-irrelevant inductive value of
that type.

\ExecuteMetaData[SaferIndexed.idr.tex]{elem}

Once we have a type of pointers into values stored in the buffer
we can start looking at what is actually stored.
%
But first let us have a look at the \IdrisType{Singleton} family
which will help us connect the values we read or compute to their
runtime-irrelevant type-level counterparts.


\subsection{The Singleton type}\label{sec:datasingleton}

The \IdrisType{Singleton} family has a single constructor
which takes an argument \IdrisBound{x} of type \IdrisBound{a},
its return type is indexed precisely by this \IdrisBound{x}.

\ExecuteMetaData[Data/Singleton.idr.tex]{singleton}

More concretely this means that a value of type
(\IdrisType{Singleton} $t$) has to be a runtime relevant
copy of the term $t$.
%
Note that \idris{} performs an optimisation similar to Haskell's
\texttt{newtype} unwrapping: every data type that has a single
non-recursive constructor with only one non-erased argument
is unwrapped during compilation.
%
This means that at runtime the
\IdrisType{Singleton} / \IdrisData{MkSingleton} indirections
will have disappeared.

We can define some convenient combinators to manipulate
singletons.
%
We reuse the naming conventions typical of applicative
functors which will allow us to rely on \idris{}'s automatic
desugaring of \emph{idiom brackets}~\cite{DBLP:journals/jfp/McbrideP08}
into expressions using these combinators.

\ExecuteMetaData[Data/Singleton.idr.tex]{pure}

First \IdrisFunction{pure} is a simple alias for \IdrisData{MkSingleton},
it turns a runtime-relevant value \IdrisBound{x} into a singleton for
this value.

\ExecuteMetaData[Data/Singleton.idr.tex]{fmap}

Next, we can `map' a function under a \IdrisType{Singleton} layer: given
a pure function \IdrisBound{f} and a runtime copy of \IdrisBound{t} we
can get a runtime copy of (\IdrisBound{f} \IdrisBound{t}).

\ExecuteMetaData[Data/Singleton.idr.tex]{ap}

Finally, we can apply a runtime copy of a function \IdrisBound{f}
to a runtime copy of an argument \IdrisBound{t}
to get a runtime copy of the result (\IdrisBound{f} \IdrisBound{t}).

As we mentioned earlier, \idris{} automatically desugars idiom brackets
using these combinators. That is to say that
\IdrisKeyword{[|} \IdrisBound{x} \IdrisKeyword{|]} will be elaborated to
(\IdrisFunction{pure} \IdrisBound{x}) while
\IdrisKeyword{[|} \IdrisBound{f} \IdrisBound{t1} $\cdots$ \IdrisBound{tn} \IdrisKeyword{|]}
will become
(\IdrisBound{f} \IdrisFunction{<\$>} \IdrisBound{t1} \IdrisFunction{<*>} $\cdots$ \IdrisFunction{<*>} \IdrisBound{tn}).
%
This lets us apply \IdrisType{Singleton}-wrapped values almost as seamlessly as pure values.


\subsection{Poking the Buffer}

Our most basic operation consists in poking the buffer to unfold
the description by exactly one step.

\ExecuteMetaData[SaferIndexed.idr.tex]{pokefun}

The result type of this operation is defined by induction on the
description. In order to keep the notations user-friendly, we
mutually define
a recursive function \IdrisFunction{Poke} interpreting the straightforward type constructors
and an inductive family \IdrisType{Poke'} with interesting return indices.

\ExecuteMetaData[SaferIndexed.idr.tex]{pokedatafun}

Poking a buffer containing \IdrisData{None} will return a value of
the unit type as no information whatsoever is stored there.

If we access a \IdrisData{Byte} then we expect that inspecting the
buffer will yield a runtime-relevant copy of the type-level byte we
have for reference. Hence the use of \IdrisType{Singleton}.

If we access a \IdrisData{Prod} of two descriptions then the type-level term
better be a pair and we better be able to obtain a \IdrisType{Pointer.Meaning}
to each of the sub-meanings.
%
Because \idris{} does not currently support definitional eta equality
for records, it will me more ergonomic for users if we introduce
\IdrisType{Poke'} rather than yielding a \IdrisType{Tuple} of values.

Last but not least if the description is \IdrisData{Rec} this means
we have a substructure. In this case we simply demand a pointer to it.



\ExecuteMetaData[SaferIndexed.idr.tex]{pokedatadata}


The implementation of this operation is necessarily unsafe as we claim
to be able to connect a type-level value to whatever it is that we read
from the buffer. Hence the use of \believeMe{} in the \IdrisData{Byte}
case.

\ExecuteMetaData[SaferIndexed.idr.tex]{pokefunbody}


\subsection{Extracting one layer}

By repeatedly poking the buffer, we can unfold the full layer.
The result of this operation is once again definded by induction
on the description. It is identical to the definition of
\IdrisFunction{Poke} except for the \IdrisData{Prod} case:
here, instead of being content with a pointer for each of the
subdescriptions, we demand a full layer for them too.

\ExecuteMetaData[SaferIndexed.idr.tex]{layerdata}

This function can easily be implemented by repeatedly calling
\IdrisFunction{poke}.

\ExecuteMetaData[SaferIndexed.idr.tex]{layerfun}

\subsection{Exposing the top constructor}



\ExecuteMetaData[SaferIndexed.idr.tex]{outdata}
\ExecuteMetaData[SaferIndexed.idr.tex]{outfun}


\subsection{Offering a convenient \IdrisType{View}}

We can combine \IdrisFunction{out} and \IdrisFunction{layer} to obtain
the \IdrisFunction{view} function we used in our introductory examples
in~\Cref{sec:seamless}.


\ExecuteMetaData[SaferIndexed.idr.tex]{viewdata}
\ExecuteMetaData[SaferIndexed.idr.tex]{viewfun}

These are views
(in the sense of Wadler~\cite{DBLP:conf/popl/Wadler87},
and McBride and McKinna~\cite{DBLP:journals/jfp/McBrideM04})


It is worth noting that although a \IdrisFunction{view} may be
convenient to consume, a performance-minded user may decide to
directly use the \IdrisFunction{out} and \IdrisFunction{poke}
combinators to avoid deserialising values that they do not need.
%
For instance, given a binary tree storing strings in its nodes and
natural numbers at its leaves, if the user only wants to extract the
rightmost natural number in the tree there is no need to deserialise
all of the strings encountered on the way down.

\todo{Add rightmost and rightmost' as case studies in an appendix?}

\subsection{Generic Fold}

The implementation of the generic \IdrisFunction{fold} over a tree stored
in a buffer is going to have the same structure as the generic fold over
inductive values: first match on the top constructor, then use \IdrisFunction{fmap}
to apply the fold to all the substructures and, finally, apply the algebra to
the result.
%
We start by implementing the buffer-based counterpart to \IdrisFunction{fmap}.
Let us go through the details of its type first.

\ExecuteMetaData[SaferIndexed.idr.tex]{fmaptype}

The first two arguments to \IdrisFunction{fmap} are similar to its pure
counterpart: a description \IdrisBound{d} and a runtime-irrelevant function
\IdrisBound{f} to map over a \IdrisFunction{Meaning}.
%
Next we take a function which is the buffer-aware counterpart to \IdrisBound{f}:
given any runtime-irrelevant term \IdrisBound{t} and a pointer to it in a buffer,
it returns an \IdrisType{IO} process computing the value (\IdrisBound{f} \IdrisBound{t}).
%
Finally, we take a runtime-irrelevant meaning \IdrisBound{t}
as well as a pointer to its representation in a buffer and compute
an \IdrisType{IO} process which will return a value equal to
(\IdrisFunction{Data.fmap} \IdrisBound{d} \IdrisBound{f} \IdrisBound{t}).

We can now look at the definition of \IdrisFunction{fmap}.

\ExecuteMetaData[SaferIndexed.idr.tex]{fmapfun}

We poke the buffer to reveal the value the \IdrisType{Pointer.Meaning}
named \IdrisBound{ptr} is pointing at and then dispatch over the description
\IdrisBound{d} using the \IdrisFunction{go} auxiliary function.

If the description is \IdrisData{None} we rewrite by the lemma
\IdrisFunction{etaUnit} which proves that all the values of the unit type
are equal to \IdrisData{()}. This then allows us to return \IdrisData{()}.

If the description is \IdrisData{Byte}, the value is left untouched and so
we can simply return it immediately.

If we have a \IdrisData{Prod} of two descriptions, we recursively apply
\IdrisFunction{fmap} to each of them and pair the results back.

Finally, if we have a \IdrisData{Rec} we apply the function operating
on buffers that we know performs the same computation as \IdrisBound{f}.


We can now combine \IdrisFunction{out} and \IdrisFunction{fmap} to compute
the correct-by-construction \IdrisFunction{fold}: provided an algebra for
a datatype \IdrisBound{cs} and a pointer to a tree of type \IdrisBound{cs}
stored in a buffer, we return an \IdrisType{IO} process computing the fold.

\ExecuteMetaData[SaferIndexed.idr.tex]{fold}

We first use \IdrisFunction{out} to reveal the constructor choice in the
tree's top node, we then recursively apply (\IdrisFunction{fold} \IdrisBound{alg})
to all the substructures by calling \IdrisFunction{fmap}, and we conclude by
applying the algebra to this result.

We once again (cf. \Cref{sec:genericfoldinductive}) had to
use \assertTotal{} because it is not obvious to
\idris{} that \IdrisFunction{fmap} only uses its argument on subterms.
%
This could have also been avoided by mutually defining \IdrisFunction{fold}
and a specialised version of
(\IdrisFunction{fmap} \IdrisKeyword{(}\IdrisFunction{fold} \IdrisBound{alg}\IdrisKeyword{)})
at the cost of code duplication and obfuscation.
%
We once again include such a definition in \Cref{sec:safefold}.
