\section{Meaning as Pointers into a Buffer}\label{sec:pointers}

Now that we know the serialisation format, we can give a meaning
to constructor and data descriptions as pointers into a buffer.

\subsection{Tracking Buffer Positions}

We start with the definition of the counterpart to \IdrisType{Mu}
for serialised values. A tree sitting in a buffer is represented
by the pairing of the buffer and the offset at which the tree's
root node is stored.

\ExecuteMetaData[SaferIndexed.idr.tex]{pointermu}

For reasons that will become apparent in \todo{ref}
when we start programming over serialised data in a correct-by-construction
manner the record \IdrisType{Mu} is parameterised not only by the description
of the data type stored but also by a runtime-irrelevant inductive value of
that type.



\ExecuteMetaData[SaferIndexed.idr.tex]{elem}

Once we have a type of pointers into values stored in the buffer
we can start looking at what is actually stored.

\subsection{Poking the Buffer}

Our most basic operation consists in poking the buffer to unfold
the description by exactly one step.

\ExecuteMetaData[SaferIndexed.idr.tex]{pokefun}

The result type of this operation is defined by induction on the
description.

\ExecuteMetaData[SaferIndexed.idr.tex]{pokedata}


The implementation of this operation is necessarily unsafe.

\ExecuteMetaData[SaferIndexed.idr.tex]{pokefunbody}

By repeatedly poking the buffer, we can unfold the full layer.
The result of this operation is once again definded by induction
on the description. It is identical to the definition of
\IdrisFunction{Poke} except for the \IdrisData{Prod} case:
here, instead of being content with a pointer for each of the
subdescriptions, we demand a full layer for them too.

\ExecuteMetaData[SaferIndexed.idr.tex]{layerdata}

This function can easily be implemented by repeatedly calling
\IdrisFunction{poke}.

\ExecuteMetaData[SaferIndexed.idr.tex]{layerfun}


\ExecuteMetaData[SaferIndexed.idr.tex]{outdata}
\ExecuteMetaData[SaferIndexed.idr.tex]{outfun}




We can combine \IdrisFunction{out} and \IdrisFunction{layer} to obtain
the \IdrisFunction{view} function we used in our introductory examples
in~\Cref{sec:seamless}.


\ExecuteMetaData[SaferIndexed.idr.tex]{viewdata}
\ExecuteMetaData[SaferIndexed.idr.tex]{viewfun}






These are views
(in the sense of Wadler~\cite{DBLP:conf/popl/Wadler87},
and McBride and McKinna~\cite{DBLP:journals/jfp/McBrideM04})
