\section{Limitations and Future Work}

\subsection{A More Robust Library}

For sake of ease of presentation we have not dealt with issues necessiting
buffer resizing: in \Cref{sec:serialising},
we defined \IdrisFunction{execSerialising}
by allocating a fixed size buffer and not worrying whether the whole content
would fit.
%
A real library would need to adopt a more robust approach
akin to the one used in the implementation of \idris{}'s own serialisation
code: whenever we are about to write a byte to the buffer, we make sure there
is either enough space left or we grow it.

\subsection{A More Efficient Library}

Partial evaluation and program specialisation to remove the interpretation
overhead incurred by working generically.
\todo{Look into 2LTT}

We currently do not support in-place updates to the data stored in a buffer.
This could however be beneficial for functions like \IdrisFunction{map}.
%
We see two possible approaches here. The first one leverages \idris{}'s
linear quantity annotation to provide users with serialised value that
can be safely updated in place: as they are used to take values apart,
pointers become unavailable and so the underlying data in the buffer can
be modified.
%
Another approach is to build separation logic into our correct-by-construction
combinators and thus ensure that the system keeps track of the modifications
being made.
\todo{Iris}



Our current approach allows us to define a correct-by-construction
\IdrisFunction{sum} operating directly on serialised data but it
does not eliminate the call stack used in the naïve functional
implementation.
%
Converting a fold to a tail recursive function in a generic manner
is a well studied problem and the existing
solutions~\cite{DBLP:conf/popl/McBride08,DBLP:conf/icfp/CortinasS18}
should be fairly straightforward to port to our setting.


\subsection{A More Expressive Library}

We have used a minimal universe to demonstrate our approach but a practical
application would require the ability to store more than just raw bytes.
An easy extension is to add support for all of the numeric types of
known size that \idris{} offers
(\IdrisType{Bits\{8,16,32,64\}}, \IdrisType{Int\{8,16,32,64\}})
as well as a unbounded data such as \IdrisType{Nat}, or \IdrisType{String}
as long as an extra offset is provided for each value.

A natural next candidate is a universe allowing the definition of parametrised
types: we should be able to implement functions over arbitrary
(\IdrisType{List} \IdrisBound{a}) values stored in a buffer,
provided that we know that \IdrisBound{a} is serialisable.

Next, we will want to consider a universe of indexed data: we can currently
model lists or trees but we cannot model vectors, or correct-by-construction
red-black trees.\todo{correct RB trees: cite (Weirich?)}
