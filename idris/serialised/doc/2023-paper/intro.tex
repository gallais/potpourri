\section{Introduction}

Our goal with this work is to be able to seamlessly and correctly
operate on algebraic data stored in a buffer in serialised format
(i.e. a linear representation laid out in contiguous memory blocks),
instead of trees
(i.e. data represented by nodes-as-structs and subtrees-as-pointers).
%
Forgetting about correctness for now, this can be summed up by the
two following code snippets describing how we can compute the sum
of the value stored in a tree of numbers.

\noindent
\begin{minipage}{.4\textwidth}
  \ExecuteMetaData[SaferIndexed.idr.tex]{tsum}
\end{minipage}
\hfill\begin{minipage}{.55\textwidth}
  \ExecuteMetaData[SaferIndexed.idr.tex]{rsum}
\end{minipage}

On the left, we have a function taking apart a tree and either returning
\IdrisData{0} if it is a leaf or, when faced with a node, recursively
summing the left and right subtrees before adding these results to value
stored in the node.
%
On the right, we have a procedure using \IdrisType{IO} actions to inspect
positions in a buffer and compute the same result.
%
Crucially, both functions look eerily similar, and the one operating on
serialised data does not explicitly perform error-prone pointer arithmetic.
