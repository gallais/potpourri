\section{Introduction}

\todo{include sum.c}


QTT~\cite{DBLP:conf/birthday/McBride16, DBLP:conf/lics/Atkey18}
as implemented in \idris{}~\cite{DBLP:conf/ecoop/Brady21}



\subsection{Seamless Programming over Serialised Data}

Our goal with this work is to be able to seamlessly and correctly
operate on algebraic data stored in a buffer in serialised format
(i.e. a linear representation laid out in contiguous memory blocks),
instead of trees
(i.e. data represented by nodes-as-structs and subtrees-as-pointers).
%
Forgetting about correctness for now, this can be summed up by the
two following code snippets describing how we can compute the sum
of the values stored in a tree of non-negative numbers.

\noindent
\begin{minipage}{.4\textwidth}
  \ExecuteMetaData[SaferIndexed.idr.tex]{tsum}
\end{minipage}
\hfill\begin{minipage}{.55\textwidth}
  \ExecuteMetaData[SaferIndexed.idr.tex]{rsum}
\end{minipage}

We reserve for later our detailed explanations of the representations
used in these snippets
(\IdrisType{Data.Mu} in \cref{sec:trees} and
\IdrisType{Pointer.Mu} in \cref{sec:pointers}).
%
For now, it is enough to understand that the function on the left hand
side is taking a tree apart by pattern-matching and returning a natural
number corresponding to the sum of the values stored in its nodes;
%
and that the one on the right hand side is an \IdrisType{IO} process
inspecting a buffer that contains such a tree stored in serialised format
and computing the same sum.

In both cases, if we have uncovered a leaf
({\IdrisData{MkMu} \IdrisData{0}} \IdrisKeyword{\KatlaUnderscore{}})
then we return zero,
and if we have uncovered a node
({\IdrisData{MkMu} \IdrisData{1}} \IdrisKeyword{(}\IdrisBound{l} \IdrisData{\#} \IdrisBound{b}  \IdrisData{\#} \IdrisBound{r}\IdrisKeyword{)})
with
a left branch \IdrisBound{l},
a stored byte \IdrisBound{b},
and a right branch \IdrisBound{r},
then we recursively compute the sums for the left and right subtrees,
cast the byte to a natural number and add everything up.

Crucially, the two functions look eerily similar, and the one operating on
serialised data does not explicitly perform error-prone pointer arithmetic,
or low-level buffer reads. This is the first way in which our approach shines.

Coming back to correctness, we will see that we can actually refine that
second definition and obtain a correct-by-construction version of
\IdrisFunction{sum}, with almost exactly the same code.

\begin{center}
  \begin{minipage}{.7\textwidth}
    \ExecuteMetaData[SaferIndexed.idr.tex]{csum}
  \end{minipage}
\end{center}

In the above snippet, we can see that the \IdrisType{Pointer.Mu} is indexed
by a phantom parameter: a runtime irrelevant \IdrisBound{t} which has type
(\IdrisType{Data.Mu} \IdrisFunction{Tree}).
%
And so the return type is able to mention the pure \IdrisFunction{sum}
function that consumes inductively defined trees.
%
\IdrisType{Singleton} is, as its name suggests, a singleton type. That is
to say that the natural number we compute is now proven to be equal to the
one computed by the pure \IdrisFunction{sum} function.
%
The implementation itself only differs in that we had to use idiom
brackets~\cite{DBLP:journals/jfp/McbrideP08}, something we will explain
in \todo{ref to singleton section}

In other words, our approach also allows us to prove the functional
correctness of the \IdrisType{IO} procedures processing trees stored
in buffer in serialised format. This is our second main contribution.

Last but not least, all of our results are obtained by generic programming,
meaning that we are not limited to the type of binary trees with bytes stored
in the nodes we used in the examples above: we capture an entire universe of
inductive types.


In summary, we are going to define a library for the
generic,
seamless,
and correct-by-construction
manipulation of algebraic types in serialised format.

\subsection{Index}

\todo{Add index here}
