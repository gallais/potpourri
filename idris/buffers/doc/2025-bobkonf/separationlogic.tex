\newcommand{\sepconj}{\ensuremath{\mathrel{*}}}

\section{Separation logic}


\newcommand{\centered}[1]{\begin{tabular}{l} #1 \end{tabular}}

\begin{frame}{What we \textit{build}: a {\bf separating} conjunction}

  \begin{itemize}
    \item Make predicates support-aware
    \item Disallow claims over overlapping memory regions
  \end{itemize}

  \vfill


\newcommand{\Pointssupport}{{
  \begin{bytefield}[bitwidth=3mm, bitheight=3mm]{4}
    \\
    \bitbox[lt]{1}[bgcolor=white]{ }
    \bitbox[t]{1}[bgcolor=white]{ }
    \bitbox[t]{1}[bgcolor=white]{ }
    \bitbox[t]{1}[bgcolor=white]{ }
    \bitbox[rt]{1}[bgcolor=white]{ }
    \vspace{-1.5mm}\\
    \bitbox[l]{1}[bgcolor=white]{ }
    \bitbox[]{1}[bgcolor=white]{ }
    \bitbox[]{1}[bgcolor=white]{ }
    \bitbox[]{1}[bgcolor=blue]{ }
    \bitbox[r]{1}[bgcolor=white]{ }
    \vspace{-1.5mm}\\
    \bitbox[lb]{1}[bgcolor=white]{ }
    \bitbox[b]{1}[bgcolor=white]{ }
    \bitbox[b]{1}[bgcolor=white]{ }
    \bitbox[b]{1}[bgcolor=white]{ }
    \bitbox[rb]{1}[bgcolor=white]{ }
  \end{bytefield}}}

\newcommand{\Propsupport}{{
  \begin{bytefield}[bitwidth=3mm, bitheight=3mm]{4}
    \\
    \bitbox[lt]{1}[bgcolor=white]{ }
    \bitbox[t]{1}[bgcolor=white]{ }
    \bitbox[t]{1}[bgcolor=white]{ }
    \bitbox[t]{1}[bgcolor=white]{ }
    \bitbox[rt]{1}[bgcolor=white]{ }
    \vspace{-1.5mm}\\
    \bitbox[l]{1}[bgcolor=white]{ }
    \bitbox[]{1}[bgcolor=white]{ }
    \bitbox[]{1}[bgcolor=white]{ }
    \bitbox[]{1}[bgcolor=white]{ }
    \bitbox[r]{1}[bgcolor=white]{ }
    \vspace{-1.5mm}\\
    \bitbox[lb]{1}[bgcolor=white]{ }
    \bitbox[b]{1}[bgcolor=white]{ }
    \bitbox[b]{1}[bgcolor=white]{ }
    \bitbox[b]{1}[bgcolor=white]{ }
    \bitbox[rb]{1}[bgcolor=white]{ }
  \end{bytefield}}
}

\begin{center}
\begin{tabular}{l|ll}
    & \centered{Predicate}
    & \centered{Support} \\\hline
  \uncover<2->{\centered{Purely logical}
    & \centered{$m + n = 3$}
    & \centered{\Propsupport{}}} \\
  \uncover<3->{\centered{Points to}
    & \centered{$\ell \mapsto v$}
    & \centered{\Pointssupport{}}} \\
  \uncover<4->{\centered{Conjunction}
    & \centered{$P \sepconj Q$}
    & \centered{?}}
\end{tabular}
\end{center}
\end{frame}

\begin{frame}{What we \textit{build}: a {\bf separating} conjunction}

\newcommand{\Psupport}{
\begin{bytefield}[bitwidth=3mm, bitheight=3mm]{4}
  \bitbox[lt]{1}[bgcolor=blue]{ }
  \bitbox[t]{1}[bgcolor=blue]{ }
  \bitbox[t]{1}[bgcolor=white]{ }
  \bitbox[t]{1}[bgcolor=white]{ }
  \bitbox[rt]{1}[bgcolor=blue]{ }
  \vspace{-1.5mm}\\
  \bitbox[l]{1}[bgcolor=white]{ }
  \bitbox[]{1}[bgcolor=white]{ }
  \bitbox[]{1}[bgcolor=white]{ }
  \bitbox[]{1}[bgcolor=blue]{ }
  \bitbox[r]{1}[bgcolor=white]{ }
  \vspace{-1.5mm}\\
  \bitbox[lb]{1}[bgcolor=white]{ }
  \bitbox[b]{1}[bgcolor=blue]{ }
  \bitbox[b]{1}[bgcolor=blue]{ }
  \bitbox[b]{1}[bgcolor=white]{ }
  \bitbox[rb]{1}[bgcolor=white]{ }
\end{bytefield}}

\newcommand{\Qsupport}{\begin{bytefield}[bitwidth=3mm, bitheight=3mm]{4}
  \bitbox[lt]{1}[bgcolor=white]{ }
  \bitbox[t]{1}[bgcolor=white]{ }
  \bitbox[t]{1}[bgcolor=white]{ }
  \bitbox[t]{1}[bgcolor=white]{ }
  \bitbox[rt]{1}[bgcolor=white]{ }
  \vspace{-1.5mm}\\
  \bitbox[l]{1}[bgcolor=carandache]{ }
  \bitbox[]{1}[bgcolor=carandache]{ }
  \bitbox[]{1}[bgcolor=white]{ }
  \bitbox[]{1}[bgcolor=white]{ }
  \bitbox[r]{1}[bgcolor=carandache]{ }
  \vspace{-1.5mm}\\
  \bitbox[lb]{1}[bgcolor=white]{ }
  \bitbox[b]{1}[bgcolor=white]{ }
  \bitbox[b]{1}[bgcolor=white]{ }
  \bitbox[b]{1}[bgcolor=carandache]{ }
  \bitbox[rb]{1}[bgcolor=carandache]{ }
\end{bytefield}}

\newcommand{\BadQsupport}{\begin{bytefield}[bitwidth=3mm, bitheight=3mm]{4}
  \bitbox[lt]{1}[bgcolor=carandache]{ }
  \bitbox[t]{1}[bgcolor=carandache]{ }
  \bitbox[t]{1}[bgcolor=white]{ }
  \bitbox[t]{1}[bgcolor=white]{ }
  \bitbox[rt]{1}[bgcolor=white]{ }
  \vspace{-1.5mm}\\
  \bitbox[l]{1}[bgcolor=carandache]{ }
  \bitbox[]{1}[bgcolor=carandache]{ }
  \bitbox[]{1}[bgcolor=white]{ }
  \bitbox[]{1}[bgcolor=white]{ }
  \bitbox[r]{1}[bgcolor=carandache]{ }
  \vspace{-1.5mm}\\
  \bitbox[lb]{1}[bgcolor=white]{ }
  \bitbox[b]{1}[bgcolor=carandache]{ }
  \bitbox[b]{1}[bgcolor=white]{ }
  \bitbox[b]{1}[bgcolor=carandache]{ }
  \bitbox[rb]{1}[bgcolor=carandache]{ }
\end{bytefield}}

\newcommand{\PQsupport}{\begin{bytefield}[bitwidth=3mm, bitheight=3mm]{4}
  \bitbox[lt]{1}[bgcolor=blue]{ }
  \bitbox[t]{1}[bgcolor=blue]{ }
  \bitbox[t]{1}[bgcolor=white]{ }
  \bitbox[t]{1}[bgcolor=white]{ }
  \bitbox[rt]{1}[bgcolor=blue]{ }
  \vspace{-1.5mm}\\
  \bitbox[l]{1}[bgcolor=carandache]{ }
  \bitbox[]{1}[bgcolor=carandache]{ }
  \bitbox[]{1}[bgcolor=white]{ }
  \bitbox[]{1}[bgcolor=blue]{ }
  \bitbox[r]{1}[bgcolor=carandache]{ }
  \vspace{-1.5mm}\\
  \bitbox[lb]{1}[bgcolor=white]{ }
  \bitbox[b]{1}[bgcolor=blue]{ }
  \bitbox[b]{1}[bgcolor=blue]{ }
  \bitbox[b]{1}[bgcolor=carandache]{ }
  \bitbox[rb]{1}[bgcolor=carandache]{ }
\end{bytefield}}

\newcommand{\halfandhalf}{\tikz{
    \fill[fill=blue](0,0)--(3.1mm,0)--(0,3mm)--cycle;
    \fill[fill=carandache](3.1mm,0)--(3mm,3mm)--(0,3mm)--cycle;
}}

\newcommand{\BadPQsupport}{\begin{bytefield}[bitwidth=3mm, bitheight=3mm]{4}
  \bitbox[lt]{1}{\halfandhalf}
  \bitbox[t]{1}{\halfandhalf}
  \bitbox[t]{1}[bgcolor=white]{ }
  \bitbox[t]{1}[bgcolor=white]{ }
  \bitbox[rt]{1}[bgcolor=blue]{ }
  \vspace{-1.5mm}\\
  \bitbox[l]{1}[bgcolor=carandache]{ }
  \bitbox[]{1}[bgcolor=carandache]{ }
  \bitbox[]{1}[bgcolor=white]{ }
  \bitbox[]{1}[bgcolor=blue]{ }
  \bitbox[r]{1}[bgcolor=carandache]{ }
  \vspace{-1.5mm}\\
  \bitbox[lb]{1}[bgcolor=white]{ }
  \bitbox[b]{1}{\halfandhalf}
  \bitbox[b]{1}[bgcolor=blue]{ }
  \bitbox[b]{1}[bgcolor=carandache]{ }
  \bitbox[rb]{1}[bgcolor=carandache]{ }
\end{bytefield}}



\begin{itemize}
  \item Non-overlapping:

\begin{center}
\begin{minipage}{.5\textwidth}
\tikz[tstyle]{\node[nstyle](psupport){\Psupport{}}}
\raisebox{2mm}{\scalebox{2}{$\sepconj$}}
\tikz[tstyle]{\node[nstyle](qsupport){\Qsupport{}}}
\raisebox{1mm}{\scalebox{2.5}{\,=\,}}
\uncover<4->{\tikz[tstyle]{\node[nstyle](pqsupport){\PQsupport{}}}}
\end{minipage}
\end{center}

  \vfill

  \item<5-> Overlapping

\begin{center}
\begin{minipage}{.5\textwidth}
\tikz[tstyle]{\node[nstyle](badpsupport){\Psupport{}}}
\raisebox{2mm}{\scalebox{2}{$\sepconj$}}
\tikz[tstyle]{\node[nstyle](badqsupport){\BadQsupport{}}}
\raisebox{1mm}{\scalebox{2.5}{\,=\,}}
\uncover<8->{\tikz[tstyle]{\node[nstyle](empty){\BadPQsupport{}}}} %%\scalebox{2.5}{$\bot$}}}}
\end{minipage}
\end{center}
\end{itemize}

\begin{tikzpicture}[tpstyle]
  \draw<2-4>[arrow,->] ([yshift=2pt]psupport.north) to[bend left] +(+.5,+1) node[anchor=south]
       {\hand If this is $P$'s support};
  \draw<3-4>[arrow,->] ([yshift=2pt]qsupport.north) to[bend left] +(1.8,+1) node[anchor=west]
       {\hand and this $Q$'s support};
  \draw<4>[arrow,->] ([yshift=-2pt]pqsupport.south) to[bend left] +(-1.6,-1)
       node[anchor=north, align=center]
           {\hand Then $P \sepconj Q$ is defined as $P \wedge Q$ and has this support};
  \draw<6->[arrow,->] ([yshift=2pt]badpsupport.north) to[bend left] +(+.5,+1) node[anchor=south]
       {\hand If this is $P$'s support};
  \draw<7->[arrow,->] ([yshift=2pt]badqsupport.north) to[bend left] +(1.8,+1) node[anchor=west]
       {\hand and this $Q$'s support};
  \draw<8>[arrow,->] ([yshift=-2pt]empty.south) to[bend left] +(-1.6,-1)
       node[anchor=north, align=center]
           {\hand Then $P \sepconj Q$ collapses to the absurd predicate $\bot$};
\end{tikzpicture}
\end{frame}


\begin{frame}{What we \textit{obtain}: the frame rule}
{\huge
\begin{mathpar}
 \inferrule{
   \lbrace P \rbrace c \lbrace Q \rbrace
 }{\lbrace \tikz[tstyle]{\node[nstyle](premise){$P \sepconj R$}} \rbrace c
   \lbrace \tikz[tstyle]{\node[nstyle](conclusion){$Q \sepconj R$}} \rbrace}
\end{mathpar}
}
\begin{tikzpicture}[tpstyle]
    \draw<2->[arrow,->] ([xshift=-4pt,yshift=-2pt]premise.south east) to[bend left] +(-1.5,-1) node[anchor=north]
       {\hand If $R$ is true and non-overlapping};
  \draw<3->[arrow,->] ([xshift=-4pt,yshift=-2pt]conclusion.south east) to[bend left] +(-1,-2.5) node[anchor=north]
       {\hand Then $R$ remains true and non-overlapping};
\end{tikzpicture}

\end{frame}

\begin{frame}{Revisiting our footgun}
  \begin{minipage}{.7\textwidth}
    \uncover<2->{We have: $P = \ell \mapsto 1$ and $Q = \ell \mapsto 0$}

    \uncover<3->{We pick: $R = \ell \mapsto 1$}

    \bigskip

    {\huge
    $$
    \left\lbrace \alt<-4>{\begin{array}{c@{\,}l}
      & \ell \mapsto 1 \only<2->{\\
          \sepconj & \only<3->{\tikz[tstyle]{\node[nstyle](lnode){$\ell$}} \mapsto 1}}

    \end{array}}{\bot \\ \\} \right\rbrace \,
    \begin{array}{c}
      \ell \mathbin{:=} 0 \, \only<2-4>{\\
      \\}
    \end{array}
    \left\lbrace \alt<-4>{\begin{array}{c@{\,}l}
      & \ell \mapsto 0 \only<2->{\\
      \sepconj & \only<3->{\tikz[tstyle]{\node[nstyle](rnode){$\ell$}} \mapsto 1}}
    \end{array}}{\bot \\ \\} \right\rbrace
    $$}
  \end{minipage}

\begin{tikzpicture}[remember picture, overlay]
  \node[anchor=north east] at ($(current page.north east)-(.15,.15)$){
    \begin{minipage}{.3\textwidth}\NB{
      \large%
              \begin{mathpar}
                \inferrule{
                  \lbrace P \rbrace c \lbrace Q \rbrace
                }{\lbrace P \sepconj R \rbrace c \lbrace Q \sepconj R \rbrace}
            \end{mathpar}}
  \end{minipage}};
\end{tikzpicture}

\begin{tikzpicture}[tpstyle]
  \node<4>[pencil,ultra thick,draw, minimum width=.8cm, minimum height=1.9cm, xshift=-.1cm, yshift=.5cm, ellipse] (box1) at (lnode) {};
  \node<4>[pencil,ultra thick,draw, minimum width=.8cm, minimum height=1.9cm, xshift=-.1cm, yshift=.5cm, ellipse] (box2) at (rnode) {};
  \node<4>[pencil] (speech) at ($(lnode) +(3,-2)$)
          {\hand Overlapping! $P \sepconj{} R$ and $Q \sepconj{} R$ both equal $\bot$};
  \draw<4>[arrow,->] (box1.south) to[bend right] ([xshift=-1.5cm]speech.north);
  \draw<4>[arrow,->] (box2.south) to[bend left] ([xshift=1.5cm]speech.north);

\end{tikzpicture}

\uncover<6>{\raisebox{-2pt}{\smiley{}} Garbage in; garbage out}
\end{frame}

\begin{frame}{From Points-to to Ownership}
  {\huge
    $$\ell \mapsto v$$
  }

  Meaning:
  \begin{itemize}
    \item used to be ``$\ell$ points to $v$''
    \item now is ``I {\bf own} $\ell$ and it points to $v$''
  \end{itemize}

  \vfill

  \uncover<2->{
  Ownership:
  \begin{itemize}
    \item is \tikz[tstyle]{\node[nstyle](paradox){globally}} unique
    \item is transferrable
    \item allows destructive updates
  \end{itemize}}

  \begin{tikzpicture}[tpstyle]
    \draw<3>[arrow,->] ([xshift=5pt, yshift=2pt]paradox.north) to[bend left] +(1.5,+.5) node[anchor=west]
       {\hand Somewhat paradoxically, this allows {\bf local} reasoning};
  \end{tikzpicture}

  \uncover<4>{All of this is implicitly enforced by the rules of the logic}

\end{frame}
