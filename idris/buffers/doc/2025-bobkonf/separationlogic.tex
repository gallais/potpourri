\newcommand{\sepconj}{\ensuremath{\mathrel{*}}}

\section{Separation logic}


\newcommand{\centered}[1]{\begin{tabular}{l} #1 \end{tabular}}

\begin{frame}{What we need: a {\bf separating} conjunction}

\newcommand{\Pointssupport}{{
  \begin{bytefield}[bitwidth=3mm, bitheight=3mm]{4}
    \\
    \bitbox[lt]{1}[bgcolor=white]{ }
    \bitbox[t]{1}[bgcolor=white]{ }
    \bitbox[t]{1}[bgcolor=white]{ }
    \bitbox[t]{1}[bgcolor=white]{ }
    \bitbox[rt]{1}[bgcolor=white]{ }
    \vspace{-1.5mm}\\
    \bitbox[l]{1}[bgcolor=white]{ }
    \bitbox[]{1}[bgcolor=white]{ }
    \bitbox[]{1}[bgcolor=white]{ }
    \bitbox[]{1}[bgcolor=blue]{ }
    \bitbox[r]{1}[bgcolor=white]{ }
    \vspace{-1.5mm}\\
    \bitbox[lb]{1}[bgcolor=white]{ }
    \bitbox[b]{1}[bgcolor=white]{ }
    \bitbox[b]{1}[bgcolor=white]{ }
    \bitbox[b]{1}[bgcolor=white]{ }
    \bitbox[rb]{1}[bgcolor=white]{ }
  \end{bytefield}}}

\newcommand{\Propsupport}{{
  \begin{bytefield}[bitwidth=3mm, bitheight=3mm]{4}
    \\
    \bitbox[lt]{1}[bgcolor=white]{ }
    \bitbox[t]{1}[bgcolor=white]{ }
    \bitbox[t]{1}[bgcolor=white]{ }
    \bitbox[t]{1}[bgcolor=white]{ }
    \bitbox[rt]{1}[bgcolor=white]{ }
    \vspace{-1.5mm}\\
    \bitbox[l]{1}[bgcolor=white]{ }
    \bitbox[]{1}[bgcolor=white]{ }
    \bitbox[]{1}[bgcolor=white]{ }
    \bitbox[]{1}[bgcolor=white]{ }
    \bitbox[r]{1}[bgcolor=white]{ }
    \vspace{-1.5mm}\\
    \bitbox[lb]{1}[bgcolor=white]{ }
    \bitbox[b]{1}[bgcolor=white]{ }
    \bitbox[b]{1}[bgcolor=white]{ }
    \bitbox[b]{1}[bgcolor=white]{ }
    \bitbox[rb]{1}[bgcolor=white]{ }
  \end{bytefield}}
}

\begin{center}
\begin{tabular}{l|ll}
    & \centered{Predicate}
    & \centered{Support} \\\hline
  \uncover<2->{\centered{Purely logical}
    & \centered{$m + n = 3$}
    & \centered{\Propsupport{}}} \\
  \uncover<3->{\centered{Points to}
    & \centered{$\ell \mapsto v$}
    & \centered{\Pointssupport{}}} \\
  \uncover<4->{\centered{Conjunction}
    & \centered{$P \sepconj Q$}
    & \centered{?}}
\end{tabular}
\end{center}
\end{frame}

\begin{frame}{What we need: a {\bf separating} conjunction}

\newcommand{\Psupport}{
\begin{bytefield}[bitwidth=3mm, bitheight=3mm]{4}
  \bitbox[lt]{1}[bgcolor=blue]{ }
  \bitbox[t]{1}[bgcolor=blue]{ }
  \bitbox[t]{1}[bgcolor=white]{ }
  \bitbox[t]{1}[bgcolor=white]{ }
  \bitbox[rt]{1}[bgcolor=blue]{ }
  \vspace{-1.5mm}\\
  \bitbox[l]{1}[bgcolor=white]{ }
  \bitbox[]{1}[bgcolor=white]{ }
  \bitbox[]{1}[bgcolor=white]{ }
  \bitbox[]{1}[bgcolor=blue]{ }
  \bitbox[r]{1}[bgcolor=white]{ }
  \vspace{-1.5mm}\\
  \bitbox[lb]{1}[bgcolor=white]{ }
  \bitbox[b]{1}[bgcolor=blue]{ }
  \bitbox[b]{1}[bgcolor=blue]{ }
  \bitbox[b]{1}[bgcolor=white]{ }
  \bitbox[rb]{1}[bgcolor=white]{ }
\end{bytefield}}

\newcommand{\Qsupport}{\begin{bytefield}[bitwidth=3mm, bitheight=3mm]{4}
  \bitbox[lt]{1}[bgcolor=white]{ }
  \bitbox[t]{1}[bgcolor=white]{ }
  \bitbox[t]{1}[bgcolor=white]{ }
  \bitbox[t]{1}[bgcolor=white]{ }
  \bitbox[rt]{1}[bgcolor=white]{ }
  \vspace{-1.5mm}\\
  \bitbox[l]{1}[bgcolor=carandache]{ }
  \bitbox[]{1}[bgcolor=carandache]{ }
  \bitbox[]{1}[bgcolor=white]{ }
  \bitbox[]{1}[bgcolor=white]{ }
  \bitbox[r]{1}[bgcolor=carandache]{ }
  \vspace{-1.5mm}\\
  \bitbox[lb]{1}[bgcolor=white]{ }
  \bitbox[b]{1}[bgcolor=white]{ }
  \bitbox[b]{1}[bgcolor=white]{ }
  \bitbox[b]{1}[bgcolor=carandache]{ }
  \bitbox[rb]{1}[bgcolor=carandache]{ }
\end{bytefield}}

\newcommand{\BadQsupport}{\begin{bytefield}[bitwidth=3mm, bitheight=3mm]{4}
  \bitbox[lt]{1}[bgcolor=carandache]{ }
  \bitbox[t]{1}[bgcolor=carandache]{ }
  \bitbox[t]{1}[bgcolor=white]{ }
  \bitbox[t]{1}[bgcolor=white]{ }
  \bitbox[rt]{1}[bgcolor=white]{ }
  \vspace{-1.5mm}\\
  \bitbox[l]{1}[bgcolor=carandache]{ }
  \bitbox[]{1}[bgcolor=carandache]{ }
  \bitbox[]{1}[bgcolor=white]{ }
  \bitbox[]{1}[bgcolor=white]{ }
  \bitbox[r]{1}[bgcolor=carandache]{ }
  \vspace{-1.5mm}\\
  \bitbox[lb]{1}[bgcolor=white]{ }
  \bitbox[b]{1}[bgcolor=white]{ }
  \bitbox[b]{1}[bgcolor=white]{ }
  \bitbox[b]{1}[bgcolor=carandache]{ }
  \bitbox[rb]{1}[bgcolor=carandache]{ }
\end{bytefield}}

\newcommand{\PQsupport}{\begin{bytefield}[bitwidth=3mm, bitheight=3mm]{4}
  \bitbox[lt]{1}[bgcolor=blue]{ }
  \bitbox[t]{1}[bgcolor=blue]{ }
  \bitbox[t]{1}[bgcolor=white]{ }
  \bitbox[t]{1}[bgcolor=white]{ }
  \bitbox[rt]{1}[bgcolor=blue]{ }
  \vspace{-1.5mm}\\
  \bitbox[l]{1}[bgcolor=carandache]{ }
  \bitbox[]{1}[bgcolor=carandache]{ }
  \bitbox[]{1}[bgcolor=white]{ }
  \bitbox[]{1}[bgcolor=blue]{ }
  \bitbox[r]{1}[bgcolor=carandache]{ }
  \vspace{-1.5mm}\\
  \bitbox[lb]{1}[bgcolor=white]{ }
  \bitbox[b]{1}[bgcolor=blue]{ }
  \bitbox[b]{1}[bgcolor=blue]{ }
  \bitbox[b]{1}[bgcolor=carandache]{ }
  \bitbox[rb]{1}[bgcolor=carandache]{ }
\end{bytefield}}

\begin{itemize}
  \item Non-overlapping:

\begin{center}
\begin{minipage}{.5\textwidth}
\tikz[tstyle]{\node[nstyle](psupport){\Psupport{}}}
\raisebox{2mm}{\scalebox{2}{$\sepconj$}}
\tikz[tstyle]{\node[nstyle](qsupport){\Qsupport{}}}
\raisebox{1mm}{\scalebox{2.5}{\,=\,}}
\uncover<4->{\tikz[tstyle]{\node[nstyle](pqsupport){\PQsupport{}}}}
\end{minipage}
\end{center}

  \vfill

  \item<5-> Overlapping

\begin{center}
\begin{minipage}{.5\textwidth}
\tikz[tstyle]{\node[nstyle](badpsupport){\Psupport{}}}
\raisebox{2mm}{\scalebox{2}{$\sepconj$}}
\tikz[tstyle]{\node[nstyle](badqsupport){\BadQsupport{}}}
\raisebox{1mm}{\scalebox{2.5}{\,=\,}}
\uncover<7>{\tikz[tstyle]{\node[nstyle](empty){\scalebox{2.5}{$\bot$}}}}
\end{minipage}
\end{center}
\end{itemize}

\begin{tikzpicture}[tpstyle]
  \draw<2-4>[arrow,->] ([yshift=2pt]psupport.north) to[bend left] +(+.5,+1) node[anchor=south]
       {\hand If this is $P$'s support};
  \draw<3-4>[arrow,->] ([yshift=2pt]qsupport.north) to[bend left] +(1.8,+1) node[anchor=west]
       {\hand and this $Q$'s support};
  \draw<4>[arrow,->] ([yshift=-2pt]pqsupport.south) to[bend left] +(-1.6,-1)
       node[anchor=north, align=center]
           {\hand Then $P \sepconj Q$ is defined and has this support};
  \draw<5->[arrow,->] ([yshift=2pt]badpsupport.north) to[bend left] +(+.5,+1) node[anchor=south]
       {\hand If this is $P$'s support};
  \draw<6->[arrow,->] ([yshift=2pt]badqsupport.north) to[bend left] +(1.8,+1) node[anchor=west]
       {\hand and this $Q$'s support};
  \draw<7>[arrow,->] ([yshift=-2pt]empty.south) to[bend left] +(-1.6,-1)
       node[anchor=north, align=center]
           {\hand Then $P \sepconj Q$ is the absurd predicate};
\end{tikzpicture}
\end{frame}
