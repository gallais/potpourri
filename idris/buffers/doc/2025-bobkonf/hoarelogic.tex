\section{Hoare logic}

\newcommand{\addra}{\ensuremath{\ell_1}}
\newcommand{\addrb}{{\color{green} \ensuremath{\ell_2}}}

\begin{frame}{Hoare logic}

A logic for imperative programs.

A memory model.

\end{frame}

\begin{frame}{Assignment Axiom}
\begin{center}
    \huge
    \tikz[tstyle]{\node[nstyle](precondition){$\lbrace \ell \mapsto \text{\_} \rbrace$}}
    \quad
    \tikz[tstyle]{\node[nstyle](command){$\ell \mathbin{:=} 0$}}
    \quad
    \tikz[tstyle]{\node[nstyle](postcondition){$\lbrace \ell \mapsto 0 \rbrace$}}
\end{center}

\begin{tikzpicture}[tpstyle]
  \draw<2->[arrow,->] ([yshift=2pt]precondition.north) to[bend left] +(+.5,+2) node[anchor=south]
       {\hand Assuming that $\ell$ is a valid location};
  \draw<3->[arrow,->] ([yshift=2pt]command.north) to[bend left] +(1.6,+1) node[anchor=west]
       {\hand Assigning $0$ to $\ell$};
  \draw<4->[arrow,->] ([yshift=-2pt]postcondition.south) to[bend left] +(-1.6,-2) node[anchor=east]
       {\hand Ensures that $\ell$ points to $0$};
\end{tikzpicture}
\end{frame}


\begin{frame}{Sequential Execution Axiom}

{
  \huge
\begin{mathpar}
 \inferrule{
      \tikz[tstyle]{\node[nstyle](premise1){$\lbrace P \rbrace c_1 \lbrace Q \rbrace$}}
      \and \tikz[tstyle]{\node[nstyle](premise2){$\lbrace Q \rbrace c_2 \lbrace R \rbrace$}}
    }{\tikz[tstyle]{\node[nstyle](conclusion){$\lbrace P \rbrace c_1; c_2 \lbrace R \rbrace$}}}
\end{mathpar}}

\begin{tikzpicture}[tpstyle]
  \draw<2->[arrow,->] ([yshift=2pt]premise1.north) to[bend left] +(+0,+2) node[anchor=south]
       {\hand If $c_1$ takes us from $P$ to $Q$};
  \draw<3->[arrow,->] ([yshift=2pt]premise2.north) to[bend right] +(0,+1) node[anchor=south]
       {\hand And $c_2$ takes us from $Q$ to $R$};
  \draw<4->[arrow,->] ([yshift=-2pt]conclusion.south) to[bend left] +(+0,-1) node[anchor=north]
       {\hand Then the composition $c_1; c_2$ takes us from $P$ to $R$};
\end{tikzpicture}
\end{frame}

\begin{frame}{A proof: swap with no allocation}
  \Large
\[
\begin{array}{lr}
  & \uncover<2-9>{\lbrace \addra \mapsto a \wedge \addrb \mapsto b \rbrace}\\
  \addra \mathbin{:=} xor(\addra, \addrb);\\
  & \uncover<3->{\lbrace \addra \mapsto xor(a, b) \wedge \addrb \mapsto b \rbrace}\\
  \addrb \mathbin{:=} xor(\addra, \addrb);\\
  & \uncover<4->{
   \alt<4-5>
   {\lbrace \addra \mapsto xor(a, b)
        \wedge \addrb \mapsto \tikz[tstyle]{\node[nstyle](xorabb){$xor(xor(a,b),b)$}} \rbrace}
   {\lbrace \addra \mapsto xor(a, b) \wedge \addrb \mapsto a \rbrace}}\\
  \addra \mathbin{:=} xor(\addra, \addrb);\\
  & \uncover<7->{
   \alt<7-8>
   {\lbrace \addra \mapsto \tikz[tstyle]{\node[nstyle](xoraba){$xor(xor(a, b), a)$}}
        \wedge \addrb \mapsto a \rbrace}
   {\lbrace \addra \mapsto b \wedge \addrb \mapsto a \rbrace}}\\
 & \hphantom{\hphantom{\lbrace \addra \mapsto xor(a, b) \wedge \addrb \mapsto xor(xor(a,b),b) \rbrace}}
\end{array}
\]

\begin{tikzpicture}[tpstyle]
  \draw<5>[arrow,->] ([yshift=2pt]xorabb.south) to[bend left=10] +(-2.5,-1) node[anchor=north]
       {\hand $xor(xor(a,b),b)$ equals $a$};
  \draw<8>[arrow,->] ([yshift=-2pt]xoraba.south) to[bend left=10] +(-2.5,-1) node[anchor=north]
       {\hand $xor(xor(a,b),a)$ equals $b$};
\end{tikzpicture}
\end{frame}

\begin{frame}{Combining proofs}
{
  \huge
\begin{mathpar}
 \inferrule{
   \lbrace \tikz[tstyle]{\node[nstyle](nodeP0){$P$}} \rbrace c_1
   \lbrace \tikz[tstyle]{\node[nstyle](nodeQ0){$Q$}} \rbrace
 \and
   \lbrace \tikz[tstyle]{\node[nstyle](nodeQ1){$Q$}} \rbrace c_2
   \lbrace \tikz[tstyle]{\node[nstyle](nodeR0){$R$}} \rbrace
 }{\lbrace \tikz[tstyle]{\node[nstyle](nodeP1){$P$}} \rbrace c_1; c_2
   \lbrace \tikz[tstyle]{\node[nstyle](nodeR1){$R$}} \rbrace}
\end{mathpar}
\begin{tikzpicture}[tpstyle]
  \node<2->[pencil,ultra thick,draw, minimum height=1cm, minimum width=1.2cm, ellipse,xscale=1.1] (box0) at (nodeP0) {};
  \node<2->[pencil,ultra thick,draw, minimum height=1cm, minimum width=1.2cm, ellipse,xscale=1.1] (box1) at (nodeP1) {};
  \node<3->[pencil,ultra thick,draw, minimum height=1cm, minimum width=1.2cm, ellipse,xscale=1.1] (box2) at (nodeQ0) {};
  \node<3->[pencil,ultra thick,draw, minimum height=1cm, minimum width=1.2cm, ellipse,xscale=1.1] (box3) at (nodeQ1) {};
  \node<4->[pencil,ultra thick,draw, minimum height=1cm, minimum width=1.2cm, ellipse,xscale=1.1] (box4) at (nodeR0) {};
  \node<4->[pencil,ultra thick,draw, minimum height=1cm, minimum width=1.2cm, ellipse,xscale=1.1] (box5) at (nodeR1) {};
\end{tikzpicture}
}

  The sequential composition rule is, ironically, anti-compositional:
  each subprogram needs to talk about
  the {\bf entire} world no matter what they {\bf actually} use!
\end{frame}

\begin{frame}{What we \textit{want}: lifting results}
{\huge
\begin{mathpar}
 \inferrule{
   \lbrace P \rbrace c \lbrace Q \rbrace
 }{\lbrace \tikz[tstyle]{\node[nstyle](premise){$P \wedge R$}} \rbrace c
   \lbrace \tikz[tstyle]{\node[nstyle](conclusion){$Q \wedge R$}} \rbrace}
\end{mathpar}
}
\begin{tikzpicture}[tpstyle]
    \draw<2->[arrow,->] ([xshift=-4pt,yshift=-2pt]premise.south east) to[bend left] +(-2.5,-1) node[anchor=east]
       {\hand If $R$ is also true};
  \draw<3->[arrow,->] ([xshift=-4pt,yshift=-2pt]conclusion.south east) to[bend left] +(-1,-2) node[anchor=north]
       {\hand Then $R$ remains true};
\end{tikzpicture}

\vfill
\uncover<4->{In a sense $R$ is {\bf independent} from $P$ \& $Q$}
\end{frame}

\begin{frame}{What we \textit{get}: nonsense!}
  \begin{minipage}{.7\textwidth}
    \uncover<2->{We have: $P = \ell \mapsto 1$ and $Q = \ell \mapsto 0$}

    \uncover<3->{We pick: $R = \ell \mapsto 1$}

    \bigskip

    {\huge
    $$
    \left\lbrace \begin{array}{c@{\,}l}
      & \ell \mapsto 1 \only<2->{\\
      \wedge & \only<3->{\ell \mapsto 1}}
    \end{array} \right\rbrace \,
    \begin{array}{c}
      \ell \mathbin{:=} 0 \, \only<2->{\\
      \\}
    \end{array}
    \left\lbrace \begin{array}{c@{\,}l}
      & \ell \mapsto 0 \only<2->{\\
      \wedge & \only<3->{\ell \mapsto \tikz[tstyle]{\node[nstyle](node){$1$}}}}
    \end{array} \right\rbrace
    $$}
  \end{minipage}

\begin{tikzpicture}[remember picture, overlay]
  \node[anchor=north east] at ($(current page.north east)-(.15,.15)$){
    \begin{minipage}{.3\textwidth}\NB{
      \large%
              \begin{mathpar}
                \inferrule{
                  \lbrace P \rbrace c \lbrace Q \rbrace
                }{\lbrace P \wedge R \rbrace c \lbrace Q \wedge R \rbrace}
            \end{mathpar}}
  \end{minipage}};
\end{tikzpicture}

\begin{tikzpicture}[tpstyle]
  \node<4->[pencil,ultra thick,draw, minimum width=.8cm, minimum height=1.9cm, xshift=.1cm, yshift=.5cm, ellipse] (box1) at (node) {};
  \draw<4->[arrow,->] (box1.south) to[bend left] +(-2,-1) node[anchor=east]
       {\hand \raisebox{-2pt}{\scream{}} $0$ is equal to $1$?!};

\end{tikzpicture}

\vfill

\uncover<5>{Nothing actually enforces that $R$ is {\bf independent} from $P$ \& $Q$!}

\end{frame}
