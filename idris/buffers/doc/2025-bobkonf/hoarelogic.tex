\section{Hoare logic}

\newcommand{\addra}{\ensuremath{\ell_1}}
\newcommand{\addrb}{{\color{green} \ensuremath{\ell_2}}}

\begin{frame}{Hoare logic}

A logic for imperative programs.

A memory model.

\end{frame}

\begin{frame}{Assignment Axiom}
\begin{center}
    \huge
    \tikz[tstyle]{\node[nstyle](precondition){$\lbrace \ell \mapsto \text{\_} \rbrace$}}
    \quad
    \tikz[tstyle]{\node[nstyle](command){$\ell \mathbin{:=} 0$}}
    \quad
    \tikz[tstyle]{\node[nstyle](postcondition){$\lbrace \ell \mapsto 0 \rbrace$}}
\end{center}

\begin{tikzpicture}[tpstyle]
  \draw<2->[arrow,->] ([yshift=2pt]precondition.north) to[bend left] +(+.5,+2) node[anchor=south]
       {\hand Assuming that $\ell$ is a valid location};
  \draw<3->[arrow,->] ([yshift=2pt]command.north) to[bend left] +(1.6,+1) node[anchor=west]
       {\hand Assigning $0$ to $\ell$};
  \draw<4->[arrow,->] ([yshift=-2pt]postcondition.south) to[bend left] +(-1.6,-2) node[anchor=east]
       {\hand Ensures that $\ell$ points to $0$};
\end{tikzpicture}
\end{frame}


\begin{frame}{Sequential Execution Axiom}

{
  \huge
\begin{mathpar}
 \inferrule{
      \tikz[tstyle]{\node[nstyle](premise1){$\lbrace P \rbrace c_1 \lbrace Q \rbrace$}}
      \and \tikz[tstyle]{\node[nstyle](premise2){$\lbrace Q \rbrace c_2 \lbrace R \rbrace$}}
    }{\tikz[tstyle]{\node[nstyle](conclusion){$\lbrace P \rbrace c_1; c_2 \lbrace R \rbrace$}}}
\end{mathpar}}

\begin{tikzpicture}[tpstyle]
  \draw<2->[arrow,->] ([yshift=2pt]premise1.north) to[bend left] +(+0,+2) node[anchor=south]
       {\hand If $c_1$ takes us from $P$ to $Q$};
  \draw<3->[arrow,->] ([yshift=2pt]premise2.north) to[bend right] +(0,+1) node[anchor=south]
       {\hand And $c_2$ takes us from $Q$ to $R$};
  \draw<4->[arrow,->] ([yshift=-2pt]conclusion.south) to[bend left] +(+0,-1) node[anchor=north]
       {\hand Then the composition $c_1; c_2$ takes us from $P$ to $R$};
\end{tikzpicture}
\end{frame}

\begin{frame}{A proof: swap with no allocation}
  \Large
\[
\begin{array}{lr}
  & \uncover<2-9>{\lbrace \addra \mapsto a \wedge \addrb \mapsto b \rbrace}\\
  \addra \mathbin{:=} xor(\addra, \addrb);\\
  & \uncover<3->{\lbrace \addra \mapsto xor(a, b) \wedge \addrb \mapsto b \rbrace}\\
  \addrb \mathbin{:=} xor(\addra, \addrb);\\
  & \uncover<4->{
   \alt<4-5>
   {\lbrace \addra \mapsto xor(a, b)
        \wedge \addrb \mapsto \tikz[tstyle]{\node[nstyle](xorabb){$xor(xor(a,b),b)$}} \rbrace}
   {\lbrace \addra \mapsto xor(a, b) \wedge \addrb \mapsto a \rbrace}}\\
  \addra \mathbin{:=} xor(\addra, \addrb);\\
  & \uncover<7->{
   \alt<7-8>
   {\lbrace \addra \mapsto \tikz[tstyle]{\node[nstyle](xoraba){$xor(xor(a, b), a)$}}
        \wedge \addrb \mapsto a \rbrace}
   {\lbrace \addra \mapsto b \wedge \addrb \mapsto a \rbrace}}\\
 & \hphantom{\hphantom{\lbrace \addra \mapsto xor(a, b) \wedge \addrb \mapsto xor(xor(a,b),b) \rbrace}}
\end{array}
\]

\begin{tikzpicture}[tpstyle]
  \draw<5>[arrow,->] ([yshift=2pt]xorabb.south) to[bend left=10] +(-2.5,-1) node[anchor=north]
       {\hand $xor(xor(a,b),b)$ equals $a$};
  \draw<8>[arrow,->] ([yshift=-2pt]xoraba.south) to[bend left=10] +(-2.5,-1) node[anchor=north]
       {\hand $xor(xor(a,b),a)$ equals $b$};
\end{tikzpicture}
\end{frame}
