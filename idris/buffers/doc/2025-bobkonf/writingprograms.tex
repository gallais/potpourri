\section{Correct by Construction}

\newcommand{\mathidris}[1]{{\bf{\mathtt{#1}}}}

\begin{frame}[fragile]{Old School Verification: Write, Test, Fix loop}
\begin{lstlisting}{Fortran}
  10 WRITE CODE
  20 DO FORMALISATION
  30 IF (CONTAINS BUG) THEN
  40   GOTO 10
  50 END IF
\end{lstlisting}
\end{frame}

\begin{frame}{Correct by Construction: Specify, Implement Correctly, Keep}
  Sometimes known as goal-driven development
  \vfill

  \begin{enumerate}
    \item Write a specification
    \item In a dialogue with the compiler interactively refine it
      \begin{itemize}
        \item Each step produces part of the program
        \item Some step introduce some further goals too
       \end{itemize}
    \item Keep refining until all goals are trivials
  \end{enumerate}
\end{frame}

\begin{frame}{In This Talk: Idris 2}
  \begin{itemize}
    \item<1-> Functional (lambdas, pure functions, inductive types)
      \only<1>{\ExecuteMetaData[Intro.idr.tex]{swap}}
    \item<2-> First class types (i.e. types are standard values)
      \only<2>{\ExecuteMetaData[Intro.idr.tex]{fileloc}}
    \item<3-> Resource-aware (separation of specification vs. runtime)
      \only<3-4>{\tikz[tstyle]{\node[nstyle](target){}}\ExecuteMetaData[Intro.idr.tex]{resource}}
    \item<5-> Strict (with explicit Laziness annotations)
    \item<6-> Compiled to ChezScheme (great target for a functional language)
    \item<7-> Self-hosted (reasonably fast!)
  \end{itemize}


\begin{tikzpicture}[tpstyle]
  \draw<4>[arrow,->] ([xshift=-8.2cm,yshift=-.9cm]target.south east) to[bend right] +(1.5,-1.2)
    node[anchor=west] {\hand Quantity $0$: erased during compilation};
\end{tikzpicture}
\end{frame}


\begin{frame}{In This Talk: Core Idea}
  Define a Domain Specific Language internalising
  Separation logic ideas

  \vfill
  \begin{itemize}
    \item<2-> Linearity (ab)used to ensure global uniqueness
    \item<3-> Ownership proofs instead of raw pointers
    \item<4-> Erasure to get rid of specification data (values showing up in $P$s, $Q$s, $R$s)
  \end{itemize}

\end{frame}

\begin{frame}{Ownership Type}


  $$\mathit{region}[\mathit{start}, \mathit{end}] \mapsto \mathit{vs}$$

  \ExecuteMetaData[Data/Buffer/Indexed.idr.tex]{owned}

\end{frame}

\newcommand{\listappend}{\mathop{+\!\!\!+}}

\begin{frame}{Read}
  $$\left\lbrace
    \uncover<2->{\begin{array}{cl}
      & \mathit{region}[\mathit{start}, \mathit{end}] \mapsto \mathit{vs}\\
      \uncover<3->{\sepconj & 0 \le \mathit{idx} < | \mathit{vs} |}
    \end{array}}
    \right\rbrace$$
  $$v = \mathidris{getBits8}(\mathit{idx});$$
  $$\left\lbrace
    \uncover<4->{\begin{array}{cl}
      & \mathit{region}[\mathit{start}, \mathit{end}] \mapsto \mathit{vs}\\
      \sepconj & v = \mathit{vs}[\mathit{idx}]
    \end{array}}
    \right\rbrace$$

  \uncover<5->{\ExecuteMetaData[Data/Buffer/Indexed.idr.tex]{read}}
\end{frame}

\begin{frame}{Write}
  $$\left\lbrace
    \uncover<2->{\begin{array}{cl}
      & \mathit{region}[\mathit{start}, \mathit{end}] \mapsto \mathit{vs}\\
      \uncover<3->{\sepconj & 0 \le \mathit{idx} < | \mathit{vs} |}
    \end{array}}
    \right\rbrace$$
  $$\mathidris{setBits8}(\mathit{idx}, \mathit{val});$$
  $$\left\lbrace
    \uncover<4->{\begin{array}{cl}
      & \mathit{region}[\mathit{start}, \mathit{end}] \mapsto \mathit{vs}[\mathit{idx} := \mathit{val}]\\
    \end{array}}
    \right\rbrace$$

  \uncover<5->{\ExecuteMetaData[Data/Buffer/Indexed.idr.tex]{write}}
\end{frame}

\begin{frame}{Split}
  $$\left\lbrace
    \uncover<2->{\begin{array}{cl}
      & \mathit{region}[\mathit{start}, \mathit{end}] \mapsto \mathit{vs} \listappend{} \mathit{ws}\\
      \uncover<3->{\sepconj & | \mathit{vs} | = \mathit{m}}
    \end{array}}
    \right\rbrace$$
  $$\mathidris{splitAt}(\mathit{m});$$
  $$\left\lbrace
    \uncover<4->{\begin{array}{cl}
        & \mathit{region}[\mathit{start}, \mathit{start} + \mathit{m}] \mapsto \mathit{vs}\\
        \sepconj & \mathit{region}[\mathit{start} + \mathit{m}, \mathit{end}] \mapsto \mathit{ws}
    \end{array}}
    \right\rbrace$$

    \uncover<5->{\ExecuteMetaData[Data/Buffer/Indexed.idr.tex]{split}}
\end{frame}

\begin{frame}{Combine}
  $$\left\lbrace
    \uncover<2->{\begin{array}{cl}
      & \mathit{region}[\mathit{start}, \mathit{middle}] \mapsto \mathit{vs}\\
      \sepconj & \mathit{region}[\mathit{middle}, \mathit{end}] \mapsto \mathit{ws}
    \end{array}}
    \right\rbrace$$
  $$\mathidris{combine}();$$
  $$\left\lbrace
    \uncover<3->{\begin{array}{cl}
      & \mathit{region}[\mathit{start}, \mathit{end}] \mapsto \mathit{vs} \mathop{+\!\!\!+} \mathit{ws}
    \end{array}}
    \right\rbrace$$

    \uncover<4->{\ExecuteMetaData[Data/Buffer/Indexed.idr.tex]{combine}}
\end{frame}


\begin{frame}{Map Type}
  \ExecuteMetaData[Data/Buffer/Indexed.idr.tex]{mapType}
\end{frame}

\begin{frame}{Sequential Map - Loop Type}
\hspace*{.1\textwidth}
\begin{bytefield}[bitwidth=6mm, bitheight=8mm]{4}
    \bitbox[]{1}{}
    \bitbox[]{1}{}
    \bitbox[]{1}{}
    \bitbox[]{1}{}
    \bitbox[]{1}{}
    \bitbox[]{1}{\smiley}
    \bitbox[]{1}{}
    \bitbox[]{1}{}
    \bitbox[]{1}{}
    \bitbox[]{1}{}
    \bitbox[]{1}{}
    \bitbox[]{1}{}
    \bitbox[]{1}{}
    \bitbox[]{1}{}
    \bitbox[]{1}{}
    \bitbox[]{1}{}
    \bitbox[]{1}{}
    \bitbox[]{1}{}
    \bitbox[]{1}{}
    \bitbox[]{1}{}
    \bitbox[]{1}{}

 \\ \bitbox[tbl]{1}{\tikz[tstyle]{\node[nstyle](target){\target}}}
    \bitbox[tb]{1}{\target}
    \bitbox[tb]{1}{\target}
    \bitbox[tb]{1}{\target}
    \bitbox[tb]{1}{\target}
    \bitbox[tb]{1}{\target}
    \bitbox[tb]{1}{\tikz[tstyle]{\node[nstyle](source){\source}}}
    \bitbox[tb]{1}{\source}
    \bitbox[tb]{1}{\source}
    \bitbox[tb]{1}{\source}
    \bitbox[tb]{1}{\source}
    \bitbox[tb]{1}{\source}
    \bitbox[tb]{1}{\source}
    \bitbox[tb]{1}{\source}
    \bitbox[tb]{1}{\source}
    \bitbox[tb]{1}{\source}
    \bitbox[tb]{1}{\source}
    \bitbox[tbr]{1}{\source}
\end{bytefield}

\vfill

\uncover<2->{
  \tikz[tstyle]{\node[nstyle](codepos){}}
  \ExecuteMetaData[Data/Buffer/Indexed.idr.tex]{loopty}
}

\begin{tikzpicture}[tpstyle]
  \node<3>[pencil,ultra thick,draw, minimum width=3.75cm, xshift=1.5cm, minimum height=.8cm, ellipse] (boxtarget) at (target) {};
  \node<4>[pencil,ultra thick,draw, minimum width=7.25cm, xshift=3.4cm, minimum height=.8cm, ellipse] (boxsource) at (source) {};
  \node<3>[pencil,ultra thick,draw, minimum width=3.5cm, xshift=8.5cm, yshift=-.65cm, minimum height=.8cm, ellipse] (codeboxtarget) at (codepos) {};
  \node<4>[pencil,ultra thick,draw, minimum width=1.55cm, xshift=11.9cm, yshift=-.65cm, minimum height=.8cm, ellipse] (codeboxsource) at (codepos) {};
\end{tikzpicture}

\end{frame}

\begin{frame}{Parallel Map}

{
    \uncover<2->{\ExecuteMetaData[Data/Buffer/Indexed.idr.tex]{halve}}
    \uncover<3->{\ExecuteMetaData[Data/Buffer/Indexed.idr.tex]{par1}}
    \uncover<4->{\ExecuteMetaData[Data/Buffer/Indexed.idr.tex]{parmaprec}}
}

\end{frame}

\begin{frame}{Parallel Reduce}
  Apply the same principles to get a parallel reduce

  Relying on monoid laws to prove correctness
\end{frame}
