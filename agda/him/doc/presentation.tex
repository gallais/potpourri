\PassOptionsToPackage{x11names}{xcolor}
\documentclass{beamer}

\usepackage{agda}
\usepackage{graphicx}
\usepackage{catchfilebetweentags}
\makeatletter

\newrobustcmd*\OrigExecuteMetaData[2][\jobname]{%
\CatchFileBetweenTags\CatchFBT@tok{#1}{#2}%
\global\expandafter\CatchFBT@tok\expandafter{%
\expandafter}\the\CatchFBT@tok
}%\OrigExecuteMetaData

\newrobustcmd*\ChkExecuteMetaData[2][\jobname]{%
\CatchFileBetweenTags\CatchFBT@tok{#1}{#2}%
\edef\mytokens{\detokenize\expandafter{\the\CatchFBT@tok}}
\ifx\mytokens\empty\PackageError{catchfilebetweentags}{the tag #2 is not found\MessageBreak in file #1 \MessageBreak called from \jobname.tex}{use a different tag}\fi%
}%\ChkExecuteMetaData

\renewrobustcmd*\ExecuteMetaData[2][\jobname]{%
\ChkExecuteMetaData[#1]{#2}%
\OrigExecuteMetaData[#1]{#2}%
}

\makeatother

\input{unicode}

\beamertemplatenavigationsymbolsempty

\title{A Quick Tour of Agda}
\author{Guillaume Allais}
\institute{Hausdorff Research Institute for Mathematics}
\date{May 13$^{th}$ 2024}

\begin{document}

\begin{frame}
    \begin{center}
      \includegraphics[width=.5\textwidth]{agda-logo.png}
    \end{center}
    \maketitle
\end{frame}

\begin{frame}{First, an ad break}

  {\Large Scottish PL and Verification Summer School}

  {\large July 29th to August 2nd}

  \bigskip

  2 core courses, 1 invited lecturer, 2*3 advanced courses
  \begin{itemize}
    \item Category Theory (core, applied, semantics of PL)
    \item Type Theory (core, for implicit complexity)
    \item PL (distributed systems, effects and handlers)
    \item Verification (protocols, syntaxes with binding)
  \end{itemize}

  \bigskip

  \url{https://scottish-pl-institute.github.io/splv/2024-strathclyde/}

\end{frame}

\begin{frame}
    \begin{center}
      \includegraphics[width=.5\textwidth]{agda-logo.png}
    \end{center}
    \maketitle
\end{frame}

\begin{frame}{Agda -- A community}
    \begin{itemize}
      \item<1-> Originated in 1999, designed by Catarina Coquand
      \item<2-> Complete rewrite in 2007 by Ulf Norell
      \item<3-> Developped openly on Github
      \item<4-> Core developpers in Sweden and Belgium
      \item<5-> Two week-long hacking meetings per year
    \end{itemize}
\end{frame}

\begin{frame}{Agda -- A programming language}
    \begin{itemize}
      \item<1-> Pluggable backends (the main ones are all Haskell-based)
      \item<2-> Powerful FFI (the whole Haskell ecosystem)
      \item<3-> Escape hatches
    \end{itemize}
\end{frame}

\begin{frame}{Agda -- A theorem prover}
  \begin{itemize}
    \item<1-> Based on type \textbf{theories} (MLTT, 2 level TT, with \& without K, cubical, strict prop, sized types, TT in colour, ...)
    \item<2-> Foundations à la carte through \texttt{OPTIONS}
    \item<3-> A more conservative \texttt{--safe} mode
  \end{itemize}


  \vfill
  \uncover<4->{
  Some soundness bugs, almost all outside of \texttt{--safe}:
  \url{https://github.com/agda/agda/issues?q=is\%3Aissue+is\%3Aopen+sort\%3Aupdated-desc+label\%3Afalse}}
\end{frame}

\begin{frame}{Small kernel?}
  \begin{center}
    \Huge No!
  \end{center}

\vfill
Well, that's not completely true but it's a hard problem:\\
\url{https://jesper.cx/posts/agda-core.html}\\
\url{https://github.com/jespercockx/agda-core}
\end{frame}

\begin{frame}{Automation?}
  \begin{center}
    \Huge No!
  \end{center}

\vfill
Well, that's not true:
\begin{itemize}
  \item \textbf{Extremely} powerful elaboration machinery
    \begin{itemize}
      \item Dependent pattern matching
      \item Unification to reconstruct missing information
      \item Parametrised modules
    \end{itemize}
  \item A meta-programming mechanism (``typed'' tactics)
  \item A form of dependent type providers
\end{itemize}
\end{frame}

\begin{frame}{Flavours}
  \begin{itemize}
    \item Standard Library (\url{https://github.com/agda/agda-stdlib})
    \item Category Theory (\url{https://github.com/agda/agda-categories})
    \item Cubical Agda (\url{https://github.com/agda/cubical})
    \item A Logical Relation for MLTT (\url{https://github.com/mr-ohman/logrel-mltt})
    \item Type Topology (\url{https://github.com/martinescardo/TypeTopology})
    \item Invoking SMT solvers (\url{https://github.com/wenkokke/schmitty})
  \end{itemize}
\end{frame}

\begin{frame}{First, an anecdote}
  \ExecuteMetaData[stringlytyped]{type}
  \ExecuteMetaData[stringlytyped]{hastype}
  \ExecuteMetaData[stringlytyped]{constructors}
\end{frame}

\begin{frame}{Why?!}
  Agda takes seriously the idea that we should use the
  right data structure to represent the objects of study.

  \begin{itemize}
    \item<1-> Inductive Types
    \item<2-> Inductive Families
    \item<3-> Induction Induction
    \item<4-> Induction Recursion
    \item<5-> Higher Inductive Types
    \item<6-> Higher Inductive Inductive Types
  \end{itemize}
\end{frame}

\end{document}
