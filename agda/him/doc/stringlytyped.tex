\documentclass[twocolumn]{article}

\usepackage{fullpage}
\usepackage{nopageno}
\usepackage{hyperref}
\usepackage{agda}
\setlength{\mathindent}{5pt}

\usepackage{todonotes}

\usepackage[T1]{fontenc}
\usepackage[utf8]{inputenc}
\usepackage{newunicodechar}
\usepackage{newtxmath}
%\usepackage{newtxtext}

% Misc symbols
\newunicodechar{⌞}{\ensuremath{\llcorner}}
\newunicodechar{⌟}{\ensuremath{\lrcorner}}
\newunicodechar{′}{\ensuremath{\prime}}
\newunicodechar{−}{\ensuremath{-}}
\newunicodechar{─}{\ensuremath{-}}
\newunicodechar{◆}{\ensuremath{\Diamondblack}}
\newunicodechar{⧫}{\ensuremath{\blacklozenge}}
\newunicodechar{∷}{\ensuremath{::}}
\newunicodechar{∙}{\ensuremath{\bullet}}
\newunicodechar{□}{\ensuremath{\Box}}
\newunicodechar{∎}{\ensuremath{\blacksquare}}
\newunicodechar{⋆}{\ensuremath{\star}}
\newunicodechar{∣}{\ensuremath{|}}

% indices
\newunicodechar{₀}{\ensuremath{_0}}
\newunicodechar{₁}{\ensuremath{_1}}
\newunicodechar{₂}{\ensuremath{_2}}
\newunicodechar{₃}{\ensuremath{_3}}

\newunicodechar{ₑ}{\ensuremath{_e}}
\newunicodechar{ᵢ}{\ensuremath{_i}}
\newunicodechar{ₖ}{\ensuremath{_k}}
\newunicodechar{ₘ}{\ensuremath{_m}}
\newunicodechar{ₙ}{\ensuremath{_n}}
\newunicodechar{ᵣ}{\ensuremath{_r}}
\newunicodechar{ₛ}{\ensuremath{_s}}

\newunicodechar{₊}{\ensuremath{_+}}

% exponents
\newunicodechar{⁺}{\ensuremath{\textsuperscript{+}}}
\newunicodechar{⁻}{\ensuremath{\textsuperscript{-}}}

\newunicodechar{²}{\ensuremath{^2}}

\newunicodechar{ᵈ}{\ensuremath{^d}}
\newunicodechar{ⁱ}{\ensuremath{^i}}
\newunicodechar{ˡ}{\ensuremath{^l}}
\newunicodechar{ʳ}{\ensuremath{^r}}
\newunicodechar{ˢ}{\ensuremath{^s}}

\newunicodechar{ᴬ}{\ensuremath{^A}}
\newunicodechar{ᴮ}{\ensuremath{^B}}
\newunicodechar{ᴵ}{\ensuremath{^I}}
\newunicodechar{ᴿ}{\ensuremath{^R}}
\newunicodechar{ᵀ}{\ensuremath{^T}}
\newunicodechar{ᵁ}{\ensuremath{^U}}
\newunicodechar{ⱽ}{\ensuremath{^V}}

% Dots
\newunicodechar{⋯}{\ensuremath{\cdots}}
\newunicodechar{∶}{\ensuremath{:}}

% Equality symbols
\newunicodechar{∼}{\ensuremath{\sim}}
\newunicodechar{≡}{\ensuremath{\equiv}}
\newunicodechar{≢}{\ensuremath{\not\equiv}}
\newunicodechar{≟}{\mbox{\tiny\ensuremath{\stackrel{?}{=}}}}
\newunicodechar{≈}{\ensuremath{\approx}}

% Ordering symbols
\newunicodechar{≤}{\ensuremath{\le}}
\newunicodechar{≥}{\ensuremath{\ge}}

% Arrows
\newunicodechar{↦}{\ensuremath{\mapsto}}
\newunicodechar{⇑}{\ensuremath{\Uparrow}}
\newunicodechar{→}{\ensuremath{\rightarrow}}
\newunicodechar{←}{\ensuremath{\leftarrow}}
\newunicodechar{⇒}{\ensuremath{\Rightarrow}}
\newunicodechar{⇉}{\ensuremath{\rightrightarrows}}

% Mathematical symbols
\newunicodechar{∂}{\ensuremath{\partial}}
\newunicodechar{∋}{\ensuremath{\ni}}
\newunicodechar{∞}{\ensuremath{\infty}}
\newunicodechar{∀}{\ensuremath{\forall}}
\newunicodechar{∃}{\ensuremath{\exists}}
\newunicodechar{⊢}{\ensuremath{\vdash}}
\newunicodechar{⟨}{\ensuremath{\langle}}
\newunicodechar{⟩}{\ensuremath{\rangle}}
\newunicodechar{⊤}{\ensuremath{\top}}
\newunicodechar{∘}{\ensuremath{\circ}}
\newunicodechar{⊎}{\ensuremath{\uplus}}
\newunicodechar{×}{\ensuremath{\times}}
\newunicodechar{ℕ}{\ensuremath{\mathbb{N}}}
\newunicodechar{⟦}{\ensuremath{\llbracket}}
\newunicodechar{⟧}{\ensuremath{\rrbracket}}
\newunicodechar{∈}{\ensuremath{\in}}
\newunicodechar{↑}{\ensuremath{\uparrow}}
\newunicodechar{¬}{\ensuremath{\neg}}
\newunicodechar{⊥}{\ensuremath{\bot}}
\newunicodechar{↝}{\ensuremath{\leadsto}}
\newunicodechar{↶}{\ensuremath{\curvearrowleft}}
\newunicodechar{↺}{\ensuremath{\circlearrowleft}}
\newunicodechar{⊔}{\ensuremath{\sqcup}}
\newunicodechar{⨆}{\ensuremath{\bigsqcup}}
\newunicodechar{∩}{\ensuremath{\cap}}
\newunicodechar{∪}{\ensuremath{\cup}}


%%%%%%%%%%%%%%%%% LETTERS

% Misc
\newunicodechar{ℓ}{\ensuremath{\ell}}

% Greek uppercase
\newunicodechar{Δ}{\ensuremath{\Delta}}
\newunicodechar{Γ}{\ensuremath{\Gamma}}
\newunicodechar{Σ}{\ensuremath{\Sigma}}
\newunicodechar{Θ}{\ensuremath{\Theta}}
\newunicodechar{Ω}{\ensuremath{\Omega}}

% Greek lowercase
\newunicodechar{α}{\ensuremath{\alpha}}
\newunicodechar{β}{\ensuremath{\beta}}
\newunicodechar{δ}{\ensuremath{\delta}}
\newunicodechar{ε}{\ensuremath{\varepsilon}}
\newunicodechar{φ}{\ensuremath{\phi}}
\newunicodechar{γ}{\ensuremath{\gamma}}
\newunicodechar{ι}{\ensuremath{\iota}}
\newunicodechar{κ}{\ensuremath{\kappa}}
\newunicodechar{λ}{\ensuremath{\lambda}}
\newunicodechar{μ}{\ensuremath{\mu}}
\newunicodechar{ψ}{\ensuremath{\psi}}
\newunicodechar{η}{\ensuremath{\eta}}
\newunicodechar{ρ}{\ensuremath{\rho}}
\newunicodechar{σ}{\ensuremath{\sigma}}
\newunicodechar{τ}{\ensuremath{\tau}}
\newunicodechar{ξ}{\ensuremath{\xi}}
\newunicodechar{ζ}{\ensuremath{\zeta}}
\newunicodechar{Π}{\ensuremath{\Pi}}


% mathcal
\newunicodechar{𝓒}{\ensuremath{\mathcal{C}}}
\newunicodechar{𝓔}{\ensuremath{\mathcal{E}}}
\newunicodechar{𝓕}{\ensuremath{\mathcal{F}}}
\newunicodechar{𝓡}{\ensuremath{\mathcal{R}}}
\newunicodechar{𝓢}{\ensuremath{\mathcal{S}}}
\newunicodechar{𝓣}{\ensuremath{\mathcal{T}}}
\newunicodechar{𝓥}{\ensuremath{\mathcal{V}}}
\newunicodechar{𝓦}{\ensuremath{\mathcal{W}}}

%%%%%%%%%% AGDA ALIASES

\newcommand{\APT}{\AgdaPrimitiveType}
\newcommand{\AK}{\AgdaKeyword}
\newcommand{\AM}{\AgdaModule}
\newcommand{\AS}{\AgdaSymbol}
\newcommand{\AN}{\AgdaNumber}
\newcommand{\AD}{\AgdaDatatype}
\newcommand{\AF}{\AgdaFunction}
\newcommand{\AR}{\AgdaRecord}
\newcommand{\ARF}{\AgdaField}
\newcommand{\AB}{\AgdaBound}
\newcommand{\AIC}{\AgdaInductiveConstructor}

\newcommand{\model}{$\mathit{𝓜}^{Γ}_{σ}$}


\bibliographystyle{abbrvnat}

\title{Dependent Stringly-Typed Programming}
\author{gallais}

\begin{document}
\maketitle

\section{Introduction}

Static type systems started as a lightweight compile time check enforcing
basic hygiene by preventing users from e.g. adding an integer to a boolean.
Seduced by the promise of ever more guarantees computer scientists have
invented ever more powerful type systems to the point where they could
formalise all of mathematics as types and programs.

In this paper we reclaim this power to the advantage of the honest working
programmer by demonstrating how one can use the ivory tower concepts to
revitalise the age old practice of stringly typed programming. We will use
Agda~\cite{norell2009dependently} as our language of choice
because it provides us with powerful enough
``unsafe'' primitives to conduct our experiment.

This paper is a self-contained literate Agda file so the interested reader
should be able to independently reproduce our results.
You can also find the content on github at
\url{https://github.com/gallais/STRINaGda}.

\section{What even is a type?}

% https://medium.com/@samth/on-typed-untyped-and-uni-typed-languages-8a3b4bedf68c
% https://semantic-domain.blogspot.com/2019/08/on-relationship-between-static-analysis.html
% https://hal.inria.fr/hal-01399694

For our purposes, we will simply say that a type is a function that, given a
linked list of characters, tells us whether we should accept it or not as a
value of that type.
%
Luckily Agda provides us with builtin notions of \AD{List}, \AD{Char},
and \AD{Bool} so this is easily defined.

\begin{code}%
\>[0]\AgdaKeyword{open}\AgdaSpace{}%
\AgdaKeyword{import}\AgdaSpace{}%
\AgdaModule{Agda.Builtin.List}%
\>[31]\AgdaKeyword{using}\AgdaSpace{}%
\AgdaSymbol{(}\AgdaDatatype{List}\AgdaSymbol{)}\<%
\\
\>[0]\AgdaKeyword{open}\AgdaSpace{}%
\AgdaKeyword{import}\AgdaSpace{}%
\AgdaModule{Agda.Builtin.Char}%
\>[31]\AgdaKeyword{using}\AgdaSpace{}%
\AgdaSymbol{(}\AgdaPostulate{Char}\AgdaSymbol{)}\<%
\\
\>[0]\AgdaKeyword{open}\AgdaSpace{}%
\AgdaKeyword{import}\AgdaSpace{}%
\AgdaModule{Agda.Builtin.Bool}%
\>[31]\AgdaKeyword{using}\AgdaSpace{}%
\AgdaSymbol{(}\AgdaDatatype{Bool}\AgdaSymbol{)}\<%
\end{code}
%<*type>
\begin{code}
\\[\AgdaEmptyExtraSkip]%
\>[0]\AgdaFunction{Type}\AgdaSpace{}%
\AgdaSymbol{=}\AgdaSpace{}%
\AgdaDatatype{List}\AgdaSpace{}%
\AgdaPostulate{Char}\AgdaSpace{}%
\AgdaSymbol{→}\AgdaSpace{}%
\AgdaDatatype{Bool}\<%
\end{code}
%</type>

Next we can define what it means to belong to a type. By definition, a
list of characters belongs to a type if the function returns the boolean
true when run on that list. To make this formal we need to define an Agda
function internalising the predicate ``this boolean is true''.

Agda ships with a notion of trivial truthfulness (the unit type) but
unfortunately it does not provide us with a notion of trivial falsity.
So we have to define the empty type ourselves as a sum type with zero
constructor.

\begin{code}%
\>[0]\AgdaKeyword{open}\AgdaSpace{}%
\AgdaKeyword{import}\AgdaSpace{}%
\AgdaModule{Agda.Builtin.Unit}\AgdaSpace{}%
\AgdaKeyword{using}\AgdaSpace{}%
\AgdaSymbol{(}\AgdaRecord{⊤}\AgdaSymbol{)}\<%
\\
%
\\[\AgdaEmptyExtraSkip]%
\>[0]\AgdaKeyword{data}\AgdaSpace{}%
\AgdaDatatype{⊥}\AgdaSpace{}%
\AgdaSymbol{:}\AgdaSpace{}%
\AgdaPrimitive{Set}\AgdaSpace{}%
\AgdaKeyword{where}\<%
\end{code}

Equipped with trivial truthfulness and trivial falsity, we can internalise
what it means for a boolean to be true by pattern matching on it and
returning the unit type if it is \AIC{true}, or the empty one if it is
\AIC{false}.

\begin{code}%
\>[0]\AgdaKeyword{open}\AgdaSpace{}%
\AgdaModule{Agda.Builtin.Bool}\AgdaSpace{}%
\AgdaKeyword{using}\AgdaSpace{}%
\AgdaSymbol{(}\AgdaInductiveConstructor{true}\AgdaSymbol{;}\AgdaSpace{}%
\AgdaInductiveConstructor{false}\AgdaSymbol{)}\<%
\\
%
\\[\AgdaEmptyExtraSkip]%
\>[0]\AgdaFunction{IsTrue}\AgdaSpace{}%
\AgdaSymbol{:}\AgdaSpace{}%
\AgdaDatatype{Bool}\AgdaSpace{}%
\AgdaSymbol{→}\AgdaSpace{}%
\AgdaPrimitive{Set}\<%
\\
\>[0]\AgdaFunction{IsTrue}\AgdaSpace{}%
\AgdaInductiveConstructor{true}%
\>[14]\AgdaSymbol{=}\AgdaSpace{}%
\AgdaRecord{⊤}\<%
\\
\>[0]\AgdaFunction{IsTrue}\AgdaSpace{}%
\AgdaInductiveConstructor{false}%
\>[14]\AgdaSymbol{=}\AgdaSpace{}%
\AgdaDatatype{⊥}\<%
\end{code}

This is precisely what we need to express what it means for a list of
characters to belong to a given type: run the type on the
list of characters and check it returned \AIC{true}.


%<*hastype>
\begin{code}%
\>[0]\AgdaOperator{\AgdaFunction{\AgdaUnderscore{}∈\AgdaUnderscore{}}}\AgdaSpace{}%
\AgdaSymbol{:}\AgdaSpace{}%
\AgdaDatatype{List}\AgdaSpace{}%
\AgdaPostulate{Char}\AgdaSpace{}%
\AgdaSymbol{→}\AgdaSpace{}%
\AgdaFunction{Type}\AgdaSpace{}%
\AgdaSymbol{→}\AgdaSpace{}%
\AgdaPrimitive{Set}\<%
\\
\>[0]\AgdaBound{cs}\AgdaSpace{}%
\AgdaOperator{\AgdaFunction{∈}}\AgdaSpace{}%
\AgdaBound{A}\AgdaSpace{}%
\AgdaSymbol{=}\AgdaSpace{}%
\AgdaFunction{IsTrue}\AgdaSpace{}%
\AgdaSymbol{(}\AgdaBound{A}\AgdaSpace{}%
\AgdaBound{cs}\AgdaSymbol{)}\<%
\end{code}
%</hastype>

We can define a convenient wrapper for elements of a given type by packing
a list of characters together with the proof that it is accepted at that
type. We use a dependent \AK{record} and
%
make the \ARF{check} field an erased instance argument, that is to say
that we never want to have to write these proofs explicitly, expect
Agda to just automatically pass them around without needing our help,
and to forget about them when compiling the code.

\begin{code}%
\>[0]\AgdaKeyword{record}\AgdaSpace{}%
\AgdaRecord{Elt}\AgdaSpace{}%
\AgdaSymbol{(}\AgdaBound{A}\AgdaSpace{}%
\AgdaSymbol{:}\AgdaSpace{}%
\AgdaFunction{Type}\AgdaSymbol{)}\AgdaSpace{}%
\AgdaSymbol{:}\AgdaSpace{}%
\AgdaPrimitive{Set}\AgdaSpace{}%
\AgdaKeyword{where}\<%
\\
\>[0][@{}l@{\AgdaIndent{0}}]%
\>[2]\AgdaKeyword{constructor}\AgdaSpace{}%
\AgdaOperator{\AgdaInductiveConstructor{[\AgdaUnderscore{}]}}\<%
\\
%
\>[2]\AgdaKeyword{field}%
\>[12]\AgdaField{value}%
\>[22]\AgdaSymbol{:}\AgdaSpace{}%
\AgdaDatatype{List}\AgdaSpace{}%
\AgdaPostulate{Char}\<%
\\
%
\>[2]\AgdaKeyword{field}\AgdaSpace{}%
\AgdaSymbol{@}\AgdaSymbol{0}%
\>[12]\AgdaSymbol{\{\{}\AgdaField{check}\AgdaSymbol{\}\}}\AgdaSpace{}%
\AgdaSymbol{:}\AgdaSpace{}%
\AgdaField{value}\AgdaSpace{}%
\AgdaOperator{\AgdaFunction{∈}}\AgdaSpace{}%
\AgdaBound{A}\<%
\\
\>[0]\AgdaKeyword{open}\AgdaSpace{}%
\AgdaModule{Elt}\<%
\end{code}

Agda's string literals are tied to its builtin notion of \AF{String} which
is distinct from \AF{List Char}. We can luckily convert from one to the
other by unpacking the string. We define a convenient helper function to,
given a string and a type, return an element of that type by checking that
the unpacked string is accepted at that type. This will help us write
concrete examples and unit tests.

\begin{code}%
\>[0]\AgdaKeyword{open}\AgdaSpace{}%
\AgdaKeyword{import}\AgdaSpace{}%
\AgdaModule{Agda.Builtin.String}\AgdaSpace{}%
\AgdaKeyword{using}\AgdaSpace{}%
\AgdaSymbol{(}\AgdaPostulate{String}\AgdaSymbol{)}\<%
\\
\>[0][@{}l@{\AgdaIndent{0}}]%
\>[2]\AgdaKeyword{renaming}\AgdaSpace{}%
\AgdaSymbol{(}\AgdaPrimitive{primStringToList}\AgdaSpace{}%
\AgdaSymbol{to}\AgdaSpace{}%
\AgdaPrimitive{unpack}\AgdaSymbol{)}\<%
\\
%
\\[\AgdaEmptyExtraSkip]%
\>[0]\AgdaKeyword{infix}\AgdaSpace{}%
\AgdaNumber{100}\AgdaSpace{}%
\AgdaOperator{\AgdaFunction{\AgdaUnderscore{}∋\AgdaUnderscore{}}}\<%
\\
\>[0]\AgdaOperator{\AgdaFunction{\AgdaUnderscore{}∋\AgdaUnderscore{}}}\AgdaSpace{}%
\AgdaSymbol{:}%
\>[73I]\AgdaSymbol{(}\AgdaBound{A}\AgdaSpace{}%
\AgdaSymbol{:}\AgdaSpace{}%
\AgdaFunction{Type}\AgdaSymbol{)}\AgdaSpace{}%
\AgdaSymbol{(}\AgdaBound{str}\AgdaSpace{}%
\AgdaSymbol{:}\AgdaSpace{}%
\AgdaPostulate{String}\AgdaSymbol{)}\AgdaSpace{}%
\AgdaSymbol{→}\<%
\\
\>[.][@{}l@{}]\<[73I]%
\>[6]\AgdaSymbol{\{\{}\AgdaPrimitive{unpack}\AgdaSpace{}%
\AgdaBound{str}\AgdaSpace{}%
\AgdaOperator{\AgdaFunction{∈}}\AgdaSpace{}%
\AgdaBound{A}\AgdaSymbol{\}\}}\AgdaSpace{}%
\AgdaSymbol{→}\AgdaSpace{}%
\AgdaRecord{Elt}\AgdaSpace{}%
\AgdaBound{A}\<%
\\
\>[0]\AgdaBound{A}\AgdaSpace{}%
\AgdaOperator{\AgdaFunction{∋}}\AgdaSpace{}%
\AgdaBound{str}\AgdaSpace{}%
\AgdaSymbol{=}\AgdaSpace{}%
\AgdaKeyword{record}\AgdaSpace{}%
\AgdaSymbol{\{}\AgdaSpace{}%
\AgdaField{value}\AgdaSpace{}%
\AgdaSymbol{=}\AgdaSpace{}%
\AgdaPrimitive{unpack}\AgdaSpace{}%
\AgdaBound{str}\AgdaSpace{}%
\AgdaSymbol{\}}\<%
\end{code}

We now have a formal definition of what a type is, what it means for a string
to be accepted at a given type and what an element of a type looks like.
Let us look at a concrete example of a type.

\section{Our First Type: ℕ}

As is customary in any document talking about dependent types, we will start
by defining the natural numbers. The customary presentation is that a natural
number is either zero or the successor of a natural natural number. In terms
of strings, we will characterise this as being either the ``Z'' string or a
string starting with `S' and whose tail is itself a natural number.

Agda, being a very inpractical programming language, does not ship with
\AF{\_\&\&\_} and \AF{\_||\_} defined on booleans.
The standard library does provide these definitions but has to be installed
separately and we want this document to be self-contained so we will have
to start by defining them ourselves.

\begin{code}%
\>[0]\AgdaKeyword{infixr}\AgdaSpace{}%
\AgdaNumber{3}\AgdaSpace{}%
\AgdaOperator{\AgdaFunction{\AgdaUnderscore{}\&\&\AgdaUnderscore{}}}\<%
\\
\>[0]\AgdaOperator{\AgdaFunction{\AgdaUnderscore{}\&\&\AgdaUnderscore{}}}\AgdaSpace{}%
\AgdaSymbol{:}\AgdaSpace{}%
\AgdaDatatype{Bool}\AgdaSpace{}%
\AgdaSymbol{→}\AgdaSpace{}%
\AgdaDatatype{Bool}\AgdaSpace{}%
\AgdaSymbol{→}\AgdaSpace{}%
\AgdaDatatype{Bool}\<%
\\
\>[0]\AgdaInductiveConstructor{true}%
\>[7]\AgdaOperator{\AgdaFunction{\&\&}}\AgdaSpace{}%
\AgdaBound{b}\AgdaSpace{}%
\AgdaSymbol{=}\AgdaSpace{}%
\AgdaBound{b}\<%
\\
\>[0]\AgdaInductiveConstructor{false}%
\>[7]\AgdaOperator{\AgdaFunction{\&\&}}\AgdaSpace{}%
\AgdaBound{b}\AgdaSpace{}%
\AgdaSymbol{=}\AgdaSpace{}%
\AgdaInductiveConstructor{false}\<%
\\
%
\\[\AgdaEmptyExtraSkip]%
\>[0]\AgdaKeyword{infixr}\AgdaSpace{}%
\AgdaNumber{2}\AgdaSpace{}%
\AgdaOperator{\AgdaFunction{\AgdaUnderscore{}||\AgdaUnderscore{}}}\<%
\\
\>[0]\AgdaOperator{\AgdaFunction{\AgdaUnderscore{}||\AgdaUnderscore{}}}\AgdaSpace{}%
\AgdaSymbol{:}\AgdaSpace{}%
\AgdaDatatype{Bool}\AgdaSpace{}%
\AgdaSymbol{→}\AgdaSpace{}%
\AgdaDatatype{Bool}\AgdaSpace{}%
\AgdaSymbol{→}\AgdaSpace{}%
\AgdaDatatype{Bool}\<%
\\
\>[0]\AgdaInductiveConstructor{true}%
\>[7]\AgdaOperator{\AgdaFunction{||}}\AgdaSpace{}%
\AgdaBound{b}\AgdaSpace{}%
\AgdaSymbol{=}\AgdaSpace{}%
\AgdaInductiveConstructor{true}\<%
\\
\>[0]\AgdaInductiveConstructor{false}%
\>[7]\AgdaOperator{\AgdaFunction{||}}\AgdaSpace{}%
\AgdaBound{b}\AgdaSpace{}%
\AgdaSymbol{=}\AgdaSpace{}%
\AgdaBound{b}\<%
\end{code}

Next we need a way to test that a list of characters is empty.
The builtin type \AD{List} has two constructors: \AIC{[]} for the
empty list, and \AIC{\_∷\_} for putting together a character as the
head of the linked list and a tail of additional characters. A list is
empty precisely when it is \AIC{[]}.

\begin{code}%
\>[0]\AgdaKeyword{open}\AgdaSpace{}%
\AgdaModule{Agda.Builtin.List}\AgdaSpace{}%
\AgdaKeyword{using}\AgdaSpace{}%
\AgdaSymbol{(}\AgdaInductiveConstructor{[]}\AgdaSymbol{;}\AgdaSpace{}%
\AgdaOperator{\AgdaInductiveConstructor{\AgdaUnderscore{}∷\AgdaUnderscore{}}}\AgdaSymbol{)}\<%
\\
%
\\[\AgdaEmptyExtraSkip]%
\>[0]\AgdaFunction{isNil}\AgdaSpace{}%
\AgdaSymbol{:}\AgdaSpace{}%
\AgdaDatatype{List}\AgdaSpace{}%
\AgdaPostulate{Char}\AgdaSpace{}%
\AgdaSymbol{→}\AgdaSpace{}%
\AgdaDatatype{Bool}\<%
\\
\>[0]\AgdaFunction{isNil}\AgdaSpace{}%
\AgdaInductiveConstructor{[]}%
\>[15]\AgdaSymbol{=}\AgdaSpace{}%
\AgdaInductiveConstructor{true}\<%
\\
\>[0]\AgdaFunction{isNil}\AgdaSpace{}%
\AgdaSymbol{(\AgdaUnderscore{}}\AgdaSpace{}%
\AgdaOperator{\AgdaInductiveConstructor{∷}}\AgdaSpace{}%
\AgdaSymbol{\AgdaUnderscore{})}%
\>[15]\AgdaSymbol{=}\AgdaSpace{}%
\AgdaInductiveConstructor{false}\<%
\end{code}

The last piece of the puzzle is the ability to test two characters
for equality. This is once again provided as a primitive by Agda and
we import it and simply rename it to make the code more readable.

\begin{code}%
\>[0]\AgdaKeyword{open}\AgdaSpace{}%
\AgdaModule{Agda.Builtin.Char}\<%
\\
\>[0][@{}l@{\AgdaIndent{0}}]%
\>[2]\AgdaKeyword{renaming}\AgdaSpace{}%
\AgdaSymbol{(}\AgdaPrimitive{primCharEquality}\AgdaSpace{}%
\AgdaSymbol{to}\AgdaSpace{}%
\AgdaPrimitive{\AgdaUnderscore{}==\AgdaUnderscore{}}\AgdaSymbol{)}\<%
\end{code}

We are now ready to define the type of natural numbers. A beautiful
thing about stringly typed programming is that we can assign a very
precise type to each constructor of a datatype. So we not only define
the type \AF{ℕ} but also mutually introduce the types \AF{isZ} and
\AF{isS} of the zero and successor constructors respectively.

\begin{code}%
\>[0]\AgdaFunction{ℕ}%
\>[5]\AgdaSymbol{:}\AgdaSpace{}%
\AgdaFunction{Type}\<%
\\
\>[0]\AgdaFunction{isZ}%
\>[5]\AgdaSymbol{:}\AgdaSpace{}%
\AgdaFunction{Type}\<%
\\
\>[0]\AgdaFunction{isS}%
\>[5]\AgdaSymbol{:}\AgdaSpace{}%
\AgdaFunction{Type}\<%
\end{code}

The type of natural numbers is exactly the union of the type of zero
and successors.

\begin{code}%
\>[0]\AgdaFunction{ℕ}\AgdaSpace{}%
\AgdaBound{cs}\AgdaSpace{}%
\AgdaSymbol{=}\AgdaSpace{}%
\AgdaFunction{isZ}\AgdaSpace{}%
\AgdaBound{cs}\AgdaSpace{}%
\AgdaOperator{\AgdaFunction{||}}\AgdaSpace{}%
\AgdaFunction{isS}\AgdaSpace{}%
\AgdaBound{cs}\<%
\end{code}

The types of zero and successor are defined by case analysis on the
input list of characters. If the list is empty then it does not belong
to any of these types. If it is non-empty then we check that it is
either `Z'-headed and with an empty tail for the zero type,
or `S'-headed and with a tail that is itself a natural number in the
successor case.

\begin{code}%
\>[0]\AgdaFunction{isZ}\AgdaSpace{}%
\AgdaInductiveConstructor{[]}\AgdaSpace{}%
\AgdaSymbol{=}\AgdaSpace{}%
\AgdaInductiveConstructor{false}\<%
\\
\>[0]\AgdaFunction{isZ}\AgdaSpace{}%
\AgdaSymbol{(}\AgdaBound{c}\AgdaSpace{}%
\AgdaOperator{\AgdaInductiveConstructor{∷}}\AgdaSpace{}%
\AgdaBound{cs}\AgdaSymbol{)}\AgdaSpace{}%
\AgdaSymbol{=}\AgdaSpace{}%
\AgdaBound{c}\AgdaSpace{}%
\AgdaOperator{\AgdaPrimitive{==}}\AgdaSpace{}%
\AgdaString{'Z'}\AgdaSpace{}%
\AgdaOperator{\AgdaFunction{\&\&}}\AgdaSpace{}%
\AgdaFunction{isNil}\AgdaSpace{}%
\AgdaBound{cs}\<%
\\
%
\\[\AgdaEmptyExtraSkip]%
\>[0]\AgdaFunction{isS}\AgdaSpace{}%
\AgdaInductiveConstructor{[]}\AgdaSpace{}%
\AgdaSymbol{=}\AgdaSpace{}%
\AgdaInductiveConstructor{false}\<%
\\
\>[0]\AgdaFunction{isS}\AgdaSpace{}%
\AgdaSymbol{(}\AgdaBound{c}\AgdaSpace{}%
\AgdaOperator{\AgdaInductiveConstructor{∷}}\AgdaSpace{}%
\AgdaBound{cs}\AgdaSymbol{)}\AgdaSpace{}%
\AgdaSymbol{=}\AgdaSpace{}%
\AgdaBound{c}\AgdaSpace{}%
\AgdaOperator{\AgdaPrimitive{==}}\AgdaSpace{}%
\AgdaString{'S'}\AgdaSpace{}%
\AgdaOperator{\AgdaFunction{\&\&}}\AgdaSpace{}%
\AgdaFunction{ℕ}\AgdaSpace{}%
\AgdaBound{cs}\<%
\end{code}

Unsurprisingly we can define the \AF{zero} and \AF{suc} constructors
for \AD{ℕ}. Note that we do not need to write any proofs that the
strings are valid: Agda takes care of the proofs for us by a mix of
computation and implicit proof construction.
%<*constructors>
\begin{code}%
\>[0]\AgdaFunction{zero}\AgdaSpace{}%
\AgdaSymbol{:}\AgdaSpace{}%
\AgdaRecord{Elt}\AgdaSpace{}%
\AgdaFunction{ℕ}\<%
\\
\>[0]\AgdaFunction{zero}\AgdaSpace{}%
\AgdaSymbol{=}\AgdaSpace{}%
\AgdaFunction{ℕ}\AgdaSpace{}%
\AgdaOperator{\AgdaFunction{∋}}\AgdaSpace{}%
\AgdaString{"Z"}\<%
\\
%
\\[\AgdaEmptyExtraSkip]%
\>[0]\AgdaFunction{suc}\AgdaSpace{}%
\AgdaSymbol{:}\AgdaSpace{}%
\AgdaRecord{Elt}\AgdaSpace{}%
\AgdaFunction{ℕ}\AgdaSpace{}%
\AgdaSymbol{→}\AgdaSpace{}%
\AgdaRecord{Elt}\AgdaSpace{}%
\AgdaFunction{ℕ}\<%
\\
\>[0]\AgdaFunction{suc}\AgdaSpace{}%
\AgdaOperator{\AgdaInductiveConstructor{[}}\AgdaSpace{}%
\AgdaBound{n}\AgdaSpace{}%
\AgdaOperator{\AgdaInductiveConstructor{]}}\AgdaSpace{}%
\AgdaSymbol{=}\AgdaSpace{}%
\AgdaOperator{\AgdaInductiveConstructor{[}}\AgdaSpace{}%
\AgdaString{'S'}\AgdaSpace{}%
\AgdaOperator{\AgdaInductiveConstructor{∷}}\AgdaSpace{}%
\AgdaBound{n}\AgdaSpace{}%
\AgdaOperator{\AgdaInductiveConstructor{]}}\<%
\end{code}
%</constructors>

We can define constant numbers either by using our \AF{\_∋\_} gadget or by
using \AF{suc} and \AF{zero}, whatever feels most convenient.

\label{sec:natconstant}
\begin{code}%
\>[0]\AgdaFunction{one}%
\>[7]\AgdaSymbol{=}\AgdaSpace{}%
\AgdaFunction{ℕ}\AgdaSpace{}%
\AgdaOperator{\AgdaFunction{∋}}\AgdaSpace{}%
\AgdaString{"SZ"}\<%
\\
\>[0]\AgdaFunction{two}%
\>[7]\AgdaSymbol{=}\AgdaSpace{}%
\AgdaFunction{suc}\AgdaSpace{}%
\AgdaSymbol{(}\AgdaFunction{suc}\AgdaSpace{}%
\AgdaFunction{zero}\AgdaSymbol{)}\<%
\\
\>[0]\AgdaFunction{three}%
\>[7]\AgdaSymbol{=}\AgdaSpace{}%
\AgdaFunction{ℕ}\AgdaSpace{}%
\AgdaOperator{\AgdaFunction{∋}}\AgdaSpace{}%
\AgdaString{"SSSZ"}\<%
\\
\>[0]\AgdaFunction{four}%
\>[7]\AgdaSymbol{=}\AgdaSpace{}%
\AgdaFunction{suc}\AgdaSpace{}%
\AgdaFunction{three}\<%
\end{code}

We will use these constants again when writing unit tests for the programs
over natural numbers we are now going to develop.

Now that we have our notion of types, a working example and even some
inhabitants, it would be nice to be able to do something with them.


\section{Stringly Typed Programming}

Being able to construct values of a given type is all well and good but
we, as programmers, want to be able to take them apart too.

Induction is the dependently typed generalisation of primitive recursion:
for a predicate \AB{P} on values of type \AF{ℕ},
if we can prove that {\AB{P} \AF{zero}} holds
and that for any natural number \AB{n}, if {\AB{P} \AB{n}} holds
then so does {\AB{P} (\AF{suc} \AB{n})}
then we ought to be able to have a function computing from a natural
number \AB{n} a proof of type {\AB{P} \AB{n}}.

\subsection{Small Scale Reflection}

The tricky part in defining induction for the natural numbers is in
connecting the observations made by the builtin boolean-valued equality
test on characters \AF{\_==\_} with propositional equality.

We introduce a \AD{Reflects} inductive family~\cite{dybjer1994inductive}
indexed by two \AD{Char}s and a \AD{Bool}. Inspired by the architecture of
Coq's small scale reflection library~\cite{assia_mahboubi_2021_4457887},
it formalises the fact that whenever the boolean is \AIC{true} then the two
characters are equal.

We name the \AD{Reflects} constructors the same as the boolean constructor
they are respectively indexed by. This means that matching on such a proof
looks like matching on the original boolean.

\begin{code}%
\>[0]\AgdaKeyword{data}\AgdaSpace{}%
\AgdaDatatype{Reflects}\AgdaSpace{}%
\AgdaSymbol{(}\AgdaBound{c}\AgdaSpace{}%
\AgdaSymbol{:}\AgdaSpace{}%
\AgdaPostulate{Char}\AgdaSymbol{)}\AgdaSpace{}%
\AgdaSymbol{:}\AgdaSpace{}%
\AgdaPostulate{Char}\AgdaSpace{}%
\AgdaSymbol{→}\AgdaSpace{}%
\AgdaDatatype{Bool}\AgdaSpace{}%
\AgdaSymbol{→}\AgdaSpace{}%
\AgdaPrimitive{Set}\AgdaSpace{}%
\AgdaKeyword{where}\<%
\\
\>[0][@{}l@{\AgdaIndent{0}}]%
\>[2]\AgdaInductiveConstructor{true}%
\>[8]\AgdaSymbol{:}\AgdaSpace{}%
\AgdaDatatype{Reflects}\AgdaSpace{}%
\AgdaBound{c}\AgdaSpace{}%
\AgdaBound{c}\AgdaSpace{}%
\AgdaInductiveConstructor{true}\<%
\\
%
\>[2]\AgdaInductiveConstructor{false}\AgdaSpace{}%
\AgdaSymbol{:}\AgdaSpace{}%
\AgdaSymbol{∀}\AgdaSpace{}%
\AgdaSymbol{\{}\AgdaBound{d}\AgdaSymbol{\}}\AgdaSpace{}%
\AgdaSymbol{→}\AgdaSpace{}%
\AgdaDatatype{Reflects}\AgdaSpace{}%
\AgdaBound{c}\AgdaSpace{}%
\AgdaBound{d}\AgdaSpace{}%
\AgdaInductiveConstructor{false}\<%
\end{code}

We can readily prove that if \AB{a} and \AB{b} are known to be the same
according to Agda's builtin notion of propositional equality then we have
that {\AD{Reflects} \AB{a} \AB{b} \AIC{true}}.

\begin{code}%
\>[0]\AgdaKeyword{open}\AgdaSpace{}%
\AgdaKeyword{import}\AgdaSpace{}%
\AgdaModule{Agda.Builtin.Equality}\AgdaSpace{}%
\AgdaKeyword{using}\AgdaSpace{}%
\AgdaSymbol{(}\AgdaOperator{\AgdaDatatype{\AgdaUnderscore{}≡\AgdaUnderscore{}}}\AgdaSymbol{;}\AgdaSpace{}%
\AgdaInductiveConstructor{refl}\AgdaSymbol{)}\<%
\\
%
\\[\AgdaEmptyExtraSkip]%
\>[0]\AgdaFunction{mkTrue}\AgdaSpace{}%
\AgdaSymbol{:}\AgdaSpace{}%
\AgdaSymbol{∀}\AgdaSpace{}%
\AgdaSymbol{\{}\AgdaBound{a}\AgdaSpace{}%
\AgdaBound{b}\AgdaSymbol{\}}\AgdaSpace{}%
\AgdaSymbol{→}\AgdaSpace{}%
\AgdaBound{a}\AgdaSpace{}%
\AgdaOperator{\AgdaDatatype{≡}}\AgdaSpace{}%
\AgdaBound{b}\AgdaSpace{}%
\AgdaSymbol{→}\AgdaSpace{}%
\AgdaDatatype{Reflects}\AgdaSpace{}%
\AgdaBound{a}\AgdaSpace{}%
\AgdaBound{b}\AgdaSpace{}%
\AgdaInductiveConstructor{true}\<%
\\
\>[0]\AgdaFunction{mkTrue}\AgdaSpace{}%
\AgdaInductiveConstructor{refl}\AgdaSpace{}%
\AgdaSymbol{=}\AgdaSpace{}%
\AgdaInductiveConstructor{true}\<%
\end{code}

The only thing missing for us is a proof that whenever the boolean test
{\AB{a} \AF{==} \AB{b}} returns \AIC{true} then the values are indeed
propositionally equal i.e. {\AB{a} \AD{≡} \AB{b}}.
Unfortunately Agda does not provide a primitive proof of this fact.
We will have to use an unsafe primitive called \AF{trustMe} to build
such a proof.

\begin{code}%
\>[0]\AgdaKeyword{open}\AgdaSpace{}%
\AgdaKeyword{import}\AgdaSpace{}%
\AgdaModule{Agda.Builtin.TrustMe}\<%
\\
\>[0][@{}l@{\AgdaIndent{0}}]%
\>[2]\AgdaKeyword{renaming}\AgdaSpace{}%
\AgdaSymbol{(}\AgdaFunction{primTrustMe}\AgdaSpace{}%
\AgdaSymbol{to}\AgdaSpace{}%
\AgdaFunction{trustMe}\AgdaSymbol{)}\<%
\end{code}

By combining \AF{mkTrue} and \AF{trustMe} we can write a function demonstrating
that the {(\AB{a} \AF{==} \AB{b})} test produces a boolean that reflects a test
on propositional equality.

\begin{code}%
\>[0]\AgdaOperator{\AgdaFunction{\AgdaUnderscore{}≟\AgdaUnderscore{}}}\AgdaSpace{}%
\AgdaSymbol{:}\AgdaSpace{}%
\AgdaSymbol{(}\AgdaBound{a}\AgdaSpace{}%
\AgdaBound{b}\AgdaSpace{}%
\AgdaSymbol{:}\AgdaSpace{}%
\AgdaPostulate{Char}\AgdaSymbol{)}\AgdaSpace{}%
\AgdaSymbol{→}\AgdaSpace{}%
\AgdaDatatype{Reflects}\AgdaSpace{}%
\AgdaBound{a}\AgdaSpace{}%
\AgdaBound{b}\AgdaSpace{}%
\AgdaSymbol{(}\AgdaBound{a}\AgdaSpace{}%
\AgdaOperator{\AgdaPrimitive{==}}\AgdaSpace{}%
\AgdaBound{b}\AgdaSymbol{)}\<%
\\
\>[0]\AgdaBound{a}\AgdaSpace{}%
\AgdaOperator{\AgdaFunction{≟}}\AgdaSpace{}%
\AgdaBound{b}\AgdaSpace{}%
\AgdaKeyword{with}\AgdaSpace{}%
\AgdaBound{a}\AgdaSpace{}%
\AgdaOperator{\AgdaPrimitive{==}}\AgdaSpace{}%
\AgdaBound{b}\<%
\\
\>[0]\AgdaSymbol{...}\AgdaSpace{}%
\AgdaSymbol{|}\AgdaSpace{}%
\AgdaInductiveConstructor{false}\AgdaSpace{}%
\AgdaSymbol{=}\AgdaSpace{}%
\AgdaInductiveConstructor{false}\<%
\\
\>[0]\AgdaSymbol{...}\AgdaSpace{}%
\AgdaSymbol{|}\AgdaSpace{}%
\AgdaInductiveConstructor{true}\AgdaSpace{}%
\AgdaSymbol{=}\AgdaSpace{}%
\AgdaFunction{mkTrue}\AgdaSpace{}%
\AgdaFunction{trustMe}\<%
\end{code}

\subsection{Induction principle for ℕ}

And with that in our backpocket we are well equipped to prove induction.
First we use an anonymous module to parametrise all of the following definitions
over the same predicate \AB{P}, proof of the base case \AB{P0} and proof of the
step case \AB{PS}.

\begin{code}%
\>[0]\AgdaKeyword{module}\AgdaSpace{}%
\AgdaModule{\AgdaUnderscore{}}%
\>[289I]\AgdaSymbol{(}\AgdaBound{P}\AgdaSpace{}%
\AgdaSymbol{:}\AgdaSpace{}%
\AgdaRecord{Elt}\AgdaSpace{}%
\AgdaFunction{ℕ}\AgdaSpace{}%
\AgdaSymbol{→}\AgdaSpace{}%
\AgdaPrimitive{Set}\AgdaSymbol{)}\<%
\\
\>[.][@{}l@{}]\<[289I]%
\>[9]\AgdaSymbol{(}\AgdaBound{P0}\AgdaSpace{}%
\AgdaSymbol{:}\AgdaSpace{}%
\AgdaBound{P}\AgdaSpace{}%
\AgdaFunction{zero}\AgdaSymbol{)}\<%
\\
%
\>[9]\AgdaSymbol{(}\AgdaBound{PS}\AgdaSpace{}%
\AgdaSymbol{:}\AgdaSpace{}%
\AgdaSymbol{∀}\AgdaSpace{}%
\AgdaBound{n}\AgdaSpace{}%
\AgdaSymbol{→}\AgdaSpace{}%
\AgdaBound{P}\AgdaSpace{}%
\AgdaBound{n}\AgdaSpace{}%
\AgdaSymbol{→}\AgdaSpace{}%
\AgdaBound{P}\AgdaSpace{}%
\AgdaSymbol{(}\AgdaFunction{suc}\AgdaSpace{}%
\AgdaBound{n}\AgdaSymbol{))}\<%
\\
%
\>[9]\AgdaKeyword{where}\<%
\end{code}

And we then prove the \AF{induction} principle stating that \AB{P} holds for
all of the natural numbers.

\begin{code}%
\>[0][@{}l@{\AgdaIndent{1}}]%
\>[2]\AgdaFunction{induction}\AgdaSpace{}%
\AgdaSymbol{:}\AgdaSpace{}%
\AgdaSymbol{∀}\AgdaSpace{}%
\AgdaBound{n}\AgdaSpace{}%
\AgdaSymbol{→}\AgdaSpace{}%
\AgdaBound{P}\AgdaSpace{}%
\AgdaBound{n}\<%
\end{code}

The details of the proof are not very illuminating but we include them for
completeness' sake. We start by checking
whether the natural number is zero, in which case we can use the base case,
or whether it is a successor in which case we use the step case together
with a recursive call to \AF{induction}.

The stage has been set just right so that things compute where they should,
impossible branches are self-evidently impossible and therefore the proof
goes through. The thing to notice if we want to understand the proof is that
the expression in the \AF{IsTrue} instance argument gets smaller as we make
more and more observations that constrain what the input natural number may
be like.

\begin{code}%
%
\>[2]\AgdaFunction{induction}\AgdaSpace{}%
\AgdaOperator{\AgdaInductiveConstructor{[}}\AgdaSpace{}%
\AgdaBound{ccs}\AgdaSymbol{@(}\AgdaBound{c}\AgdaSpace{}%
\AgdaOperator{\AgdaInductiveConstructor{∷}}\AgdaSpace{}%
\AgdaBound{cs}\AgdaSymbol{)}\AgdaSpace{}%
\AgdaOperator{\AgdaInductiveConstructor{]}}\AgdaSpace{}%
\AgdaSymbol{=}\AgdaSpace{}%
\AgdaFunction{checkZ}\AgdaSpace{}%
\AgdaSymbol{(}\AgdaBound{c}\AgdaSpace{}%
\AgdaOperator{\AgdaFunction{≟}}\AgdaSpace{}%
\AgdaString{'Z'}\AgdaSymbol{)}\AgdaSpace{}%
\AgdaBound{cs}\AgdaSpace{}%
\AgdaInductiveConstructor{refl}\<%
\\
%
\\[\AgdaEmptyExtraSkip]%
\>[2][@{}l@{\AgdaIndent{0}}]%
\>[4]\AgdaKeyword{where}\<%
\\
%
\\[\AgdaEmptyExtraSkip]%
%
\>[4]\AgdaFunction{checkS}\AgdaSpace{}%
\AgdaSymbol{:}\AgdaSpace{}%
\AgdaSymbol{∀}\AgdaSpace{}%
\AgdaSymbol{\{}\AgdaBound{b}\AgdaSymbol{\}}\AgdaSpace{}%
\AgdaSymbol{→}\AgdaSpace{}%
\AgdaDatatype{Reflects}\AgdaSpace{}%
\AgdaBound{c}\AgdaSpace{}%
\AgdaString{'S'}\AgdaSpace{}%
\AgdaBound{b}\AgdaSpace{}%
\AgdaSymbol{→}\AgdaSpace{}%
\AgdaSymbol{∀}\AgdaSpace{}%
\AgdaBound{cs}\AgdaSpace{}%
\AgdaSymbol{→}\<%
\\
\>[4][@{}l@{\AgdaIndent{0}}]%
\>[6]\AgdaSymbol{\{\{@}\AgdaSymbol{0}\AgdaSpace{}%
\AgdaBound{\AgdaUnderscore{}}\AgdaSpace{}%
\AgdaSymbol{:}\AgdaSpace{}%
\AgdaFunction{IsTrue}\AgdaSpace{}%
\AgdaSymbol{(}\AgdaBound{b}\AgdaSpace{}%
\AgdaOperator{\AgdaFunction{\&\&}}\AgdaSpace{}%
\AgdaFunction{ℕ}\AgdaSpace{}%
\AgdaBound{cs}\AgdaSymbol{)\}\}}\AgdaSpace{}%
\AgdaSymbol{→}\<%
\\
%
\>[6]\AgdaSymbol{∀}\AgdaSpace{}%
\AgdaSymbol{\{}\AgdaBound{ccs}\AgdaSymbol{\}}\AgdaSpace{}%
\AgdaSymbol{→}\AgdaSpace{}%
\AgdaBound{c}\AgdaSpace{}%
\AgdaOperator{\AgdaInductiveConstructor{∷}}\AgdaSpace{}%
\AgdaBound{cs}\AgdaSpace{}%
\AgdaOperator{\AgdaDatatype{≡}}\AgdaSpace{}%
\AgdaBound{ccs}\AgdaSpace{}%
\AgdaSymbol{.}\AgdaField{value}\AgdaSpace{}%
\AgdaSymbol{→}\AgdaSpace{}%
\AgdaBound{P}\AgdaSpace{}%
\AgdaBound{ccs}\<%
\\
%
\>[4]\AgdaFunction{checkS}\AgdaSpace{}%
\AgdaInductiveConstructor{true}\AgdaSpace{}%
\AgdaBound{cs}\AgdaSpace{}%
\AgdaInductiveConstructor{refl}\AgdaSpace{}%
\AgdaSymbol{=}\AgdaSpace{}%
\AgdaBound{PS}\AgdaSpace{}%
\AgdaOperator{\AgdaInductiveConstructor{[}}\AgdaSpace{}%
\AgdaBound{cs}\AgdaSpace{}%
\AgdaOperator{\AgdaInductiveConstructor{]}}\AgdaSpace{}%
\AgdaSymbol{(}\AgdaFunction{induction}\AgdaSpace{}%
\AgdaOperator{\AgdaInductiveConstructor{[}}\AgdaSpace{}%
\AgdaBound{cs}\AgdaSpace{}%
\AgdaOperator{\AgdaInductiveConstructor{]}}\AgdaSymbol{)}\<%
\\
%
\\[\AgdaEmptyExtraSkip]%
%
\>[4]\AgdaFunction{checkZ}\AgdaSpace{}%
\AgdaSymbol{:}\AgdaSpace{}%
\AgdaSymbol{∀}\AgdaSpace{}%
\AgdaSymbol{\{}\AgdaBound{b}\AgdaSymbol{\}}\AgdaSpace{}%
\AgdaSymbol{→}\AgdaSpace{}%
\AgdaDatatype{Reflects}\AgdaSpace{}%
\AgdaBound{c}\AgdaSpace{}%
\AgdaString{'Z'}\AgdaSpace{}%
\AgdaBound{b}\AgdaSpace{}%
\AgdaSymbol{→}\AgdaSpace{}%
\AgdaSymbol{∀}\AgdaSpace{}%
\AgdaBound{cs}\AgdaSpace{}%
\AgdaSymbol{→}\<%
\\
\>[4][@{}l@{\AgdaIndent{0}}]%
\>[6]\AgdaSymbol{\{\{@}\AgdaSymbol{0}\AgdaSpace{}%
\AgdaBound{\AgdaUnderscore{}}\AgdaSpace{}%
\AgdaSymbol{:}\AgdaSpace{}%
\AgdaFunction{IsTrue}\AgdaSpace{}%
\AgdaSymbol{(}\AgdaBound{b}\AgdaSpace{}%
\AgdaOperator{\AgdaFunction{\&\&}}\AgdaSpace{}%
\AgdaFunction{isNil}\AgdaSpace{}%
\AgdaBound{cs}\AgdaSpace{}%
\AgdaOperator{\AgdaFunction{||}}\AgdaSpace{}%
\AgdaFunction{isS}\AgdaSpace{}%
\AgdaSymbol{(}\AgdaBound{c}\AgdaSpace{}%
\AgdaOperator{\AgdaInductiveConstructor{∷}}\AgdaSpace{}%
\AgdaBound{cs}\AgdaSymbol{))\}\}}\AgdaSpace{}%
\AgdaSymbol{→}\<%
\\
%
\>[6]\AgdaSymbol{∀}\AgdaSpace{}%
\AgdaSymbol{\{}\AgdaBound{ccs}\AgdaSymbol{\}}\AgdaSpace{}%
\AgdaSymbol{→}\AgdaSpace{}%
\AgdaBound{c}\AgdaSpace{}%
\AgdaOperator{\AgdaInductiveConstructor{∷}}\AgdaSpace{}%
\AgdaBound{cs}\AgdaSpace{}%
\AgdaOperator{\AgdaDatatype{≡}}\AgdaSpace{}%
\AgdaBound{ccs}\AgdaSpace{}%
\AgdaSymbol{.}\AgdaField{value}\AgdaSpace{}%
\AgdaSymbol{→}\AgdaSpace{}%
\AgdaBound{P}\AgdaSpace{}%
\AgdaBound{ccs}\<%
\\
%
\>[4]\AgdaFunction{checkZ}\AgdaSpace{}%
\AgdaInductiveConstructor{true}%
\>[17]\AgdaInductiveConstructor{[]}\AgdaSpace{}%
\AgdaInductiveConstructor{refl}\AgdaSpace{}%
\AgdaSymbol{=}\AgdaSpace{}%
\AgdaBound{P0}\<%
\\
%
\>[4]\AgdaFunction{checkZ}\AgdaSpace{}%
\AgdaInductiveConstructor{false}\AgdaSpace{}%
\AgdaBound{cs}\AgdaSpace{}%
\AgdaBound{eq}%
\>[25]\AgdaSymbol{=}\AgdaSpace{}%
\AgdaFunction{checkS}\AgdaSpace{}%
\AgdaSymbol{(}\AgdaBound{c}\AgdaSpace{}%
\AgdaOperator{\AgdaFunction{≟}}\AgdaSpace{}%
\AgdaString{'S'}\AgdaSymbol{)}\AgdaSpace{}%
\AgdaBound{cs}\AgdaSpace{}%
\AgdaBound{eq}\<%
\end{code}

An induction operator is of course not just one that has
the right type but one that has the right computational behaviour too.
We can readily check that our \AF{induction} function behaves exactly
like the primitive recursor on natural numbers ought to by writing two
unit tests.

First, when applied to \AF{zero}, the recursor immediately returns
its base case.

\begin{code}%
\>[0]\AgdaFunction{\AgdaUnderscore{}}\AgdaSpace{}%
\AgdaSymbol{:}\AgdaSpace{}%
\AgdaSymbol{∀}\AgdaSpace{}%
\AgdaSymbol{\{}\AgdaBound{P}\AgdaSpace{}%
\AgdaBound{P0}\AgdaSpace{}%
\AgdaBound{PS}\AgdaSymbol{\}}\AgdaSpace{}%
\AgdaSymbol{→}\AgdaSpace{}%
\AgdaFunction{induction}\AgdaSpace{}%
\AgdaBound{P}\AgdaSpace{}%
\AgdaBound{P0}\AgdaSpace{}%
\AgdaBound{PS}\AgdaSpace{}%
\AgdaFunction{zero}\AgdaSpace{}%
\AgdaOperator{\AgdaDatatype{≡}}\AgdaSpace{}%
\AgdaBound{P0}\<%
\\
\>[0]\AgdaSymbol{\AgdaUnderscore{}}\AgdaSpace{}%
\AgdaSymbol{=}\AgdaSpace{}%
\AgdaInductiveConstructor{refl}\<%
\end{code}

Second, when applied to the successor of a natural number \AB{n}, the
recursor returns its step case applied to \AB{n} and the result of the
recursive call.

\begin{code}%
\>[0]\AgdaFunction{\AgdaUnderscore{}}\AgdaSpace{}%
\AgdaSymbol{:}%
\>[434I]\AgdaSymbol{∀}\AgdaSpace{}%
\AgdaSymbol{\{}\AgdaBound{P}\AgdaSpace{}%
\AgdaBound{P0}\AgdaSpace{}%
\AgdaBound{PS}\AgdaSpace{}%
\AgdaBound{n}\AgdaSymbol{\}}\AgdaSpace{}%
\AgdaSymbol{→}\<%
\\
\>[.][@{}l@{}]\<[434I]%
\>[4]\AgdaFunction{induction}\AgdaSpace{}%
\AgdaBound{P}\AgdaSpace{}%
\AgdaBound{P0}\AgdaSpace{}%
\AgdaBound{PS}\AgdaSpace{}%
\AgdaSymbol{(}\AgdaFunction{suc}\AgdaSpace{}%
\AgdaBound{n}\AgdaSymbol{)}\<%
\\
%
\>[4]\AgdaOperator{\AgdaDatatype{≡}}\AgdaSpace{}%
\AgdaBound{PS}\AgdaSpace{}%
\AgdaBound{n}\AgdaSpace{}%
\AgdaSymbol{(}\AgdaFunction{induction}\AgdaSpace{}%
\AgdaBound{P}\AgdaSpace{}%
\AgdaBound{P0}\AgdaSpace{}%
\AgdaBound{PS}\AgdaSpace{}%
\AgdaBound{n}\AgdaSymbol{)}\<%
\\
\>[0]\AgdaSymbol{\AgdaUnderscore{}}\AgdaSpace{}%
\AgdaSymbol{=}\AgdaSpace{}%
\AgdaInductiveConstructor{refl}\<%
\end{code}

The fact that both of these unit tests are provable by \AIC{refl} means
that Agda can tell by computation alone that the expressions are equal.

\subsection{Example: Addition, Multiplication}

As is well known, primitive recursion is enough to implement addition and
multiplication on the natural numbers. So induction will be plenty enough
power for us.

Addition of \AB{m} to \AB{n} can be implemented by induction on \AB{m}.
The base case, corresponding to \AF{zero} \AF{+} \AB{n}, amounts to returning
\AB{n}. The step case amounts to taking the successor of the inductive
hypothesis. This gives us the following definition:

\begin{code}%
\>[0]\AgdaOperator{\AgdaFunction{\AgdaUnderscore{}+\AgdaUnderscore{}}}\AgdaSpace{}%
\AgdaSymbol{:}\AgdaSpace{}%
\AgdaRecord{Elt}\AgdaSpace{}%
\AgdaFunction{ℕ}\AgdaSpace{}%
\AgdaSymbol{→}\AgdaSpace{}%
\AgdaRecord{Elt}\AgdaSpace{}%
\AgdaFunction{ℕ}\AgdaSpace{}%
\AgdaSymbol{→}\AgdaSpace{}%
\AgdaRecord{Elt}\AgdaSpace{}%
\AgdaFunction{ℕ}\<%
\\
\>[0]\AgdaBound{m}\AgdaSpace{}%
\AgdaOperator{\AgdaFunction{+}}\AgdaSpace{}%
\AgdaBound{n}\AgdaSpace{}%
\AgdaSymbol{=}\AgdaSpace{}%
\AgdaFunction{induction}\AgdaSpace{}%
\AgdaSymbol{(λ}\AgdaSpace{}%
\AgdaBound{\AgdaUnderscore{}}\AgdaSpace{}%
\AgdaSymbol{→}\AgdaSpace{}%
\AgdaRecord{Elt}\AgdaSpace{}%
\AgdaFunction{ℕ}\AgdaSymbol{)}\AgdaSpace{}%
\AgdaBound{n}\AgdaSpace{}%
\AgdaSymbol{(λ}\AgdaSpace{}%
\AgdaBound{\AgdaUnderscore{}}\AgdaSpace{}%
\AgdaSymbol{→}\AgdaSpace{}%
\AgdaFunction{suc}\AgdaSymbol{)}\AgdaSpace{}%
\AgdaBound{m}\<%
\end{code}

We can test the function thus implemented by writing a unit test reusing
the constants defined in Section~\ref{sec:natconstant}, checking for instance
that $3+1$ evaluates to $4$.

\begin{code}%
\>[0]\AgdaFunction{\AgdaUnderscore{}}\AgdaSpace{}%
\AgdaSymbol{:}\AgdaSpace{}%
\AgdaFunction{three}\AgdaSpace{}%
\AgdaOperator{\AgdaFunction{+}}\AgdaSpace{}%
\AgdaFunction{one}\AgdaSpace{}%
\AgdaOperator{\AgdaDatatype{≡}}\AgdaSpace{}%
\AgdaFunction{four}\<%
\\
\>[0]\AgdaSymbol{\AgdaUnderscore{}}\AgdaSpace{}%
\AgdaSymbol{=}\AgdaSpace{}%
\AgdaInductiveConstructor{refl}\<%
\end{code}

Multiplication is defined in the same way: \AF{zero} \AF{*} \AB{n}
is equal to \AF{zero} and the step case amounts to stating that
(\AF{suc} \AB{m}) \AF{*} \AB{n} should evaluate to
\AB{n} \AF{+} \AB{m} \AF{*} \AB{n}.

\begin{code}%
\>[0]\AgdaOperator{\AgdaFunction{\AgdaUnderscore{}*\AgdaUnderscore{}}}\AgdaSpace{}%
\AgdaSymbol{:}\AgdaSpace{}%
\AgdaRecord{Elt}\AgdaSpace{}%
\AgdaFunction{ℕ}\AgdaSpace{}%
\AgdaSymbol{→}\AgdaSpace{}%
\AgdaRecord{Elt}\AgdaSpace{}%
\AgdaFunction{ℕ}\AgdaSpace{}%
\AgdaSymbol{→}\AgdaSpace{}%
\AgdaRecord{Elt}\AgdaSpace{}%
\AgdaFunction{ℕ}\<%
\\
\>[0]\AgdaBound{m}\AgdaSpace{}%
\AgdaOperator{\AgdaFunction{*}}\AgdaSpace{}%
\AgdaBound{n}\AgdaSpace{}%
\AgdaSymbol{=}\AgdaSpace{}%
\AgdaFunction{induction}\AgdaSpace{}%
\AgdaSymbol{(λ}\AgdaSpace{}%
\AgdaBound{\AgdaUnderscore{}}\AgdaSpace{}%
\AgdaSymbol{→}\AgdaSpace{}%
\AgdaRecord{Elt}\AgdaSpace{}%
\AgdaFunction{ℕ}\AgdaSymbol{)}\AgdaSpace{}%
\AgdaFunction{zero}\AgdaSpace{}%
\AgdaSymbol{(λ}\AgdaSpace{}%
\AgdaBound{\AgdaUnderscore{}}\AgdaSpace{}%
\AgdaSymbol{→}\AgdaSpace{}%
\AgdaBound{n}\AgdaSpace{}%
\AgdaOperator{\AgdaFunction{+\AgdaUnderscore{}}}\AgdaSymbol{)}\AgdaSpace{}%
\AgdaBound{m}\<%
\end{code}

We can check with a unit test that our implementation verifies that
$2*3$ equals $4+2$.

\begin{code}%
\>[0]\AgdaFunction{\AgdaUnderscore{}}\AgdaSpace{}%
\AgdaSymbol{:}\AgdaSpace{}%
\AgdaFunction{two}\AgdaSpace{}%
\AgdaOperator{\AgdaFunction{*}}\AgdaSpace{}%
\AgdaFunction{three}\AgdaSpace{}%
\AgdaOperator{\AgdaDatatype{≡}}\AgdaSpace{}%
\AgdaFunction{four}\AgdaSpace{}%
\AgdaOperator{\AgdaFunction{+}}\AgdaSpace{}%
\AgdaFunction{two}\<%
\\
\>[0]\AgdaSymbol{\AgdaUnderscore{}}\AgdaSpace{}%
\AgdaSymbol{=}\AgdaSpace{}%
\AgdaInductiveConstructor{refl}\<%
\end{code}

Because our \AF{induction} function has the right computational
behaviour, the definitions we just introduced should be well
behaved too.
They did pass a couple of unit tests but given that we are using a
dependently typed host language we ought to do better than that.

\section{Stringly Typed Proving}

This section is dedicated to proving some of the properties of the
functions we have defined. We hope to convince the reader that they
could pick up any proof from the standard library's
\texttt{Data.Nat.Properties} module and reproduce it on our stringly
typed natural numbers.

\subsection{Equality combinators}

Now that we are entering serious proof territory,
we will need to introduce some basic combinators witnessing the
fundamental properties of propositional equality.

We use a block of variables Agda is authorised to implicitly quantify over
to avoid repeating ourselves in this section.

\begin{code}%
\>[0]\AgdaKeyword{variable}\<%
\\
\>[0][@{}l@{\AgdaIndent{0}}]%
\>[2]\AgdaGeneralizable{A}\AgdaSpace{}%
\AgdaGeneralizable{B}\AgdaSpace{}%
\AgdaSymbol{:}\AgdaSpace{}%
\AgdaPrimitive{Set}\<%
\\
%
\>[2]\AgdaGeneralizable{x}\AgdaSpace{}%
\AgdaGeneralizable{y}\AgdaSpace{}%
\AgdaGeneralizable{z}\AgdaSpace{}%
\AgdaSymbol{:}\AgdaSpace{}%
\AgdaGeneralizable{A}\<%
\end{code}

Propositional equality is a congruence. That is to say that if two values
are equal, applying the same function to both will yield equal results.

\begin{code}%
\>[0]\AgdaFunction{cong}\AgdaSpace{}%
\AgdaSymbol{:}\AgdaSpace{}%
\AgdaSymbol{(}\AgdaBound{f}\AgdaSpace{}%
\AgdaSymbol{:}\AgdaSpace{}%
\AgdaGeneralizable{A}\AgdaSpace{}%
\AgdaSymbol{→}\AgdaSpace{}%
\AgdaGeneralizable{B}\AgdaSymbol{)}\AgdaSpace{}%
\AgdaSymbol{→}\AgdaSpace{}%
\AgdaGeneralizable{x}\AgdaSpace{}%
\AgdaOperator{\AgdaDatatype{≡}}\AgdaSpace{}%
\AgdaGeneralizable{y}\AgdaSpace{}%
\AgdaSymbol{→}\AgdaSpace{}%
\AgdaBound{f}\AgdaSpace{}%
\AgdaGeneralizable{x}\AgdaSpace{}%
\AgdaOperator{\AgdaDatatype{≡}}\AgdaSpace{}%
\AgdaBound{f}\AgdaSpace{}%
\AgdaGeneralizable{y}\<%
\\
\>[0]\AgdaFunction{cong}\AgdaSpace{}%
\AgdaBound{f}\AgdaSpace{}%
\AgdaInductiveConstructor{refl}\AgdaSpace{}%
\AgdaSymbol{=}\AgdaSpace{}%
\AgdaInductiveConstructor{refl}\<%
\end{code}

We already know that propositional equality is a reflexive relation as
witnessed by its constructor \AIC{refl} and we can additionally prove that
is is a symmetric and transitive one.

\begin{code}%
\>[0]\AgdaFunction{sym}\AgdaSpace{}%
\AgdaSymbol{:}\AgdaSpace{}%
\AgdaGeneralizable{x}\AgdaSpace{}%
\AgdaOperator{\AgdaDatatype{≡}}\AgdaSpace{}%
\AgdaGeneralizable{y}\AgdaSpace{}%
\AgdaSymbol{→}\AgdaSpace{}%
\AgdaGeneralizable{y}\AgdaSpace{}%
\AgdaOperator{\AgdaDatatype{≡}}\AgdaSpace{}%
\AgdaGeneralizable{x}\<%
\\
\>[0]\AgdaFunction{sym}\AgdaSpace{}%
\AgdaInductiveConstructor{refl}\AgdaSpace{}%
\AgdaSymbol{=}\AgdaSpace{}%
\AgdaInductiveConstructor{refl}\<%
\\
%
\\[\AgdaEmptyExtraSkip]%
\>[0]\AgdaFunction{trans}\AgdaSpace{}%
\AgdaSymbol{:}\AgdaSpace{}%
\AgdaGeneralizable{x}\AgdaSpace{}%
\AgdaOperator{\AgdaDatatype{≡}}\AgdaSpace{}%
\AgdaGeneralizable{y}\AgdaSpace{}%
\AgdaSymbol{→}\AgdaSpace{}%
\AgdaGeneralizable{y}\AgdaSpace{}%
\AgdaOperator{\AgdaDatatype{≡}}\AgdaSpace{}%
\AgdaGeneralizable{z}\AgdaSpace{}%
\AgdaSymbol{→}\AgdaSpace{}%
\AgdaGeneralizable{x}\AgdaSpace{}%
\AgdaOperator{\AgdaDatatype{≡}}\AgdaSpace{}%
\AgdaGeneralizable{z}\<%
\\
\>[0]\AgdaFunction{trans}\AgdaSpace{}%
\AgdaInductiveConstructor{refl}\AgdaSpace{}%
\AgdaBound{eq}\AgdaSpace{}%
\AgdaSymbol{=}\AgdaSpace{}%
\AgdaBound{eq}\<%
\end{code}

We now have the basic building blocks needed to build equality proofs.

\subsection{Properties of Addition}

Given our earlier observation that \AF{induction} immediately returns its
base case when applied to the natural number \AF{zero}, it should not be
any surprise that \AF{zero} is trivially a left neutral for
our definition of addition.

\begin{code}%
\>[0]\AgdaFunction{zero-+}\AgdaSpace{}%
\AgdaSymbol{:}\AgdaSpace{}%
\AgdaSymbol{∀}\AgdaSpace{}%
\AgdaBound{m}\AgdaSpace{}%
\AgdaSymbol{→}\AgdaSpace{}%
\AgdaFunction{zero}\AgdaSpace{}%
\AgdaOperator{\AgdaFunction{+}}\AgdaSpace{}%
\AgdaBound{m}\AgdaSpace{}%
\AgdaOperator{\AgdaDatatype{≡}}\AgdaSpace{}%
\AgdaBound{m}\<%
\\
\>[0]\AgdaFunction{zero-+}\AgdaSpace{}%
\AgdaBound{m}\AgdaSpace{}%
\AgdaSymbol{=}\AgdaSpace{}%
\AgdaInductiveConstructor{refl}\<%
\end{code}

The proof that it is also a right neutral for addition requires a bit more
work. We can use \AF{induction} itself to build such a proof. The base case
corresponding to {\AF{zero} \AF{+} \AF{zero} \AD{≡} \AF{zero}} is trivially
true. The step case is just a matter of using the induction hypothesis together
with the fact that equality is a congruence to add a \AF{suc} to both sides.

\begin{code}%
\>[0]\AgdaFunction{+-zero}\AgdaSpace{}%
\AgdaSymbol{:}\AgdaSpace{}%
\AgdaSymbol{∀}\AgdaSpace{}%
\AgdaBound{m}\AgdaSpace{}%
\AgdaSymbol{→}\AgdaSpace{}%
\AgdaBound{m}\AgdaSpace{}%
\AgdaOperator{\AgdaFunction{+}}\AgdaSpace{}%
\AgdaFunction{zero}\AgdaSpace{}%
\AgdaOperator{\AgdaDatatype{≡}}\AgdaSpace{}%
\AgdaBound{m}\<%
\\
\>[0]\AgdaFunction{+-zero}\AgdaSpace{}%
\AgdaSymbol{=}\<%
\\
\>[0][@{}l@{\AgdaIndent{0}}]%
\>[2]\AgdaFunction{induction}\<%
\\
\>[2][@{}l@{\AgdaIndent{0}}]%
\>[4]\AgdaSymbol{(λ}\AgdaSpace{}%
\AgdaBound{m}\AgdaSpace{}%
\AgdaSymbol{→}\AgdaSpace{}%
\AgdaBound{m}\AgdaSpace{}%
\AgdaOperator{\AgdaFunction{+}}\AgdaSpace{}%
\AgdaFunction{zero}\AgdaSpace{}%
\AgdaOperator{\AgdaDatatype{≡}}\AgdaSpace{}%
\AgdaBound{m}\AgdaSymbol{)}\<%
\\
%
\>[4]\AgdaInductiveConstructor{refl}\<%
\\
%
\>[4]\AgdaSymbol{(λ}\AgdaSpace{}%
\AgdaBound{n}\AgdaSpace{}%
\AgdaSymbol{→}\AgdaSpace{}%
\AgdaFunction{cong}\AgdaSpace{}%
\AgdaFunction{suc}\AgdaSymbol{)}\<%
\end{code}

Similarly, our previous unit test should lead us to anticipate that the
proof that the addition of \AF{suc} \AB{m} to \AB{n} is equal to the
successor of the addition of \AB{m} to \AB{n} is trival. This indeed holds
true by computation alone.

\begin{code}%
\>[0]\AgdaFunction{suc-+}\AgdaSpace{}%
\AgdaSymbol{:}\AgdaSpace{}%
\AgdaSymbol{∀}\AgdaSpace{}%
\AgdaBound{m}\AgdaSpace{}%
\AgdaBound{n}\AgdaSpace{}%
\AgdaSymbol{→}\AgdaSpace{}%
\AgdaFunction{suc}\AgdaSpace{}%
\AgdaBound{m}\AgdaSpace{}%
\AgdaOperator{\AgdaFunction{+}}\AgdaSpace{}%
\AgdaBound{n}\AgdaSpace{}%
\AgdaOperator{\AgdaDatatype{≡}}\AgdaSpace{}%
\AgdaFunction{suc}\AgdaSpace{}%
\AgdaSymbol{(}\AgdaBound{m}\AgdaSpace{}%
\AgdaOperator{\AgdaFunction{+}}\AgdaSpace{}%
\AgdaBound{n}\AgdaSymbol{)}\<%
\\
\>[0]\AgdaFunction{suc-+}\AgdaSpace{}%
\AgdaBound{m}\AgdaSpace{}%
\AgdaBound{n}\AgdaSpace{}%
\AgdaSymbol{=}\AgdaSpace{}%
\AgdaInductiveConstructor{refl}\<%
\end{code}

The statement stating that the addition of \AB{m} to \AF{suc} \AB{n} is
equal to the successor of the addition of \AB{m} to \AB{n} is however a bit
trickier to deal with. It can once again be proven by using induction on \AB{m}.

\begin{code}%
\>[0]\AgdaFunction{+-suc}\AgdaSpace{}%
\AgdaSymbol{:}\AgdaSpace{}%
\AgdaSymbol{∀}\AgdaSpace{}%
\AgdaBound{m}\AgdaSpace{}%
\AgdaBound{n}\AgdaSpace{}%
\AgdaSymbol{→}\AgdaSpace{}%
\AgdaBound{m}\AgdaSpace{}%
\AgdaOperator{\AgdaFunction{+}}\AgdaSpace{}%
\AgdaFunction{suc}\AgdaSpace{}%
\AgdaBound{n}\AgdaSpace{}%
\AgdaOperator{\AgdaDatatype{≡}}\AgdaSpace{}%
\AgdaFunction{suc}\AgdaSpace{}%
\AgdaSymbol{(}\AgdaBound{m}\AgdaSpace{}%
\AgdaOperator{\AgdaFunction{+}}\AgdaSpace{}%
\AgdaBound{n}\AgdaSymbol{)}\<%
\\
\>[0]\AgdaFunction{+-suc}\AgdaSpace{}%
\AgdaBound{m}\AgdaSpace{}%
\AgdaBound{n}\AgdaSpace{}%
\AgdaSymbol{=}\<%
\\
\>[0][@{}l@{\AgdaIndent{0}}]%
\>[2]\AgdaFunction{induction}\<%
\\
\>[2][@{}l@{\AgdaIndent{0}}]%
\>[4]\AgdaSymbol{(λ}\AgdaSpace{}%
\AgdaBound{m}\AgdaSpace{}%
\AgdaSymbol{→}\AgdaSpace{}%
\AgdaSymbol{(}\AgdaBound{m}\AgdaSpace{}%
\AgdaOperator{\AgdaFunction{+}}\AgdaSpace{}%
\AgdaFunction{suc}\AgdaSpace{}%
\AgdaBound{n}\AgdaSymbol{)}\AgdaSpace{}%
\AgdaOperator{\AgdaDatatype{≡}}\AgdaSpace{}%
\AgdaFunction{suc}\AgdaSpace{}%
\AgdaSymbol{(}\AgdaBound{m}\AgdaSpace{}%
\AgdaOperator{\AgdaFunction{+}}\AgdaSpace{}%
\AgdaBound{n}\AgdaSymbol{))}\<%
\\
%
\>[4]\AgdaInductiveConstructor{refl}\<%
\\
%
\>[4]\AgdaSymbol{(λ}\AgdaSpace{}%
\AgdaBound{n}\AgdaSpace{}%
\AgdaSymbol{→}\AgdaSpace{}%
\AgdaFunction{cong}\AgdaSpace{}%
\AgdaFunction{suc}\AgdaSymbol{)}\<%
\\
%
\>[4]\AgdaBound{m}\<%
\end{code}

These auxiliary lemmas are the intermediate results we need to be able to
prove that addition is commutative. We, once again, proceed by induction
and this time make crucial use of the fact that equality is symmetric and
transitive.

\begin{code}%
\>[0]\AgdaFunction{+-comm}\AgdaSpace{}%
\AgdaSymbol{:}\AgdaSpace{}%
\AgdaSymbol{∀}\AgdaSpace{}%
\AgdaBound{m}\AgdaSpace{}%
\AgdaBound{n}\AgdaSpace{}%
\AgdaSymbol{→}\AgdaSpace{}%
\AgdaBound{m}\AgdaSpace{}%
\AgdaOperator{\AgdaFunction{+}}\AgdaSpace{}%
\AgdaBound{n}\AgdaSpace{}%
\AgdaOperator{\AgdaDatatype{≡}}\AgdaSpace{}%
\AgdaBound{n}\AgdaSpace{}%
\AgdaOperator{\AgdaFunction{+}}\AgdaSpace{}%
\AgdaBound{m}\<%
\\
\>[0]\AgdaFunction{+-comm}\AgdaSpace{}%
\AgdaBound{m}\AgdaSpace{}%
\AgdaBound{n}\AgdaSpace{}%
\AgdaSymbol{=}\<%
\\
\>[0][@{}l@{\AgdaIndent{0}}]%
\>[2]\AgdaFunction{induction}\<%
\\
\>[2][@{}l@{\AgdaIndent{0}}]%
\>[4]\AgdaSymbol{(λ}\AgdaSpace{}%
\AgdaBound{m}\AgdaSpace{}%
\AgdaSymbol{→}\AgdaSpace{}%
\AgdaBound{m}\AgdaSpace{}%
\AgdaOperator{\AgdaFunction{+}}\AgdaSpace{}%
\AgdaBound{n}\AgdaSpace{}%
\AgdaOperator{\AgdaDatatype{≡}}\AgdaSpace{}%
\AgdaBound{n}\AgdaSpace{}%
\AgdaOperator{\AgdaFunction{+}}\AgdaSpace{}%
\AgdaBound{m}\AgdaSymbol{)}\<%
\\
%
\>[4]\AgdaSymbol{(}\AgdaFunction{sym}\AgdaSpace{}%
\AgdaSymbol{(}\AgdaFunction{+-zero}\AgdaSpace{}%
\AgdaBound{n}\AgdaSymbol{))}\<%
\\
%
\>[4]\AgdaSymbol{(λ}\AgdaSpace{}%
\AgdaBound{m}\AgdaSpace{}%
\AgdaBound{ih}\AgdaSpace{}%
\AgdaSymbol{→}\AgdaSpace{}%
\AgdaFunction{trans}\AgdaSpace{}%
\AgdaSymbol{(}\AgdaFunction{cong}\AgdaSpace{}%
\AgdaFunction{suc}\AgdaSpace{}%
\AgdaBound{ih}\AgdaSymbol{)}\AgdaSpace{}%
\AgdaSymbol{(}\AgdaFunction{sym}\AgdaSpace{}%
\AgdaSymbol{(}\AgdaFunction{+-suc}\AgdaSpace{}%
\AgdaBound{n}\AgdaSpace{}%
\AgdaBound{m}\AgdaSymbol{)))}\<%
\\
%
\>[4]\AgdaBound{m}\<%
\end{code}

Let us conclude with one last example of a property one can prove of addition
on stringly natural numbers: addition is associative.

\begin{code}%
\>[0]\AgdaFunction{+-assoc}\AgdaSpace{}%
\AgdaSymbol{:}\AgdaSpace{}%
\AgdaSymbol{∀}\AgdaSpace{}%
\AgdaBound{m}\AgdaSpace{}%
\AgdaBound{n}\AgdaSpace{}%
\AgdaBound{p}\AgdaSpace{}%
\AgdaSymbol{→}\AgdaSpace{}%
\AgdaSymbol{(}\AgdaBound{m}\AgdaSpace{}%
\AgdaOperator{\AgdaFunction{+}}\AgdaSpace{}%
\AgdaBound{n}\AgdaSymbol{)}\AgdaSpace{}%
\AgdaOperator{\AgdaFunction{+}}\AgdaSpace{}%
\AgdaBound{p}\AgdaSpace{}%
\AgdaOperator{\AgdaDatatype{≡}}\AgdaSpace{}%
\AgdaBound{m}\AgdaSpace{}%
\AgdaOperator{\AgdaFunction{+}}\AgdaSpace{}%
\AgdaSymbol{(}\AgdaBound{n}\AgdaSpace{}%
\AgdaOperator{\AgdaFunction{+}}\AgdaSpace{}%
\AgdaBound{p}\AgdaSymbol{)}\<%
\\
\>[0]\AgdaFunction{+-assoc}\AgdaSpace{}%
\AgdaBound{m}\AgdaSpace{}%
\AgdaBound{n}\AgdaSpace{}%
\AgdaBound{p}\AgdaSpace{}%
\AgdaSymbol{=}\<%
\\
\>[0][@{}l@{\AgdaIndent{0}}]%
\>[2]\AgdaFunction{induction}\<%
\\
\>[2][@{}l@{\AgdaIndent{0}}]%
\>[4]\AgdaSymbol{(λ}\AgdaSpace{}%
\AgdaBound{m}\AgdaSpace{}%
\AgdaSymbol{→}\AgdaSpace{}%
\AgdaSymbol{((}\AgdaBound{m}\AgdaSpace{}%
\AgdaOperator{\AgdaFunction{+}}\AgdaSpace{}%
\AgdaBound{n}\AgdaSymbol{)}\AgdaSpace{}%
\AgdaOperator{\AgdaFunction{+}}\AgdaSpace{}%
\AgdaBound{p}\AgdaSymbol{)}\AgdaSpace{}%
\AgdaOperator{\AgdaDatatype{≡}}\AgdaSpace{}%
\AgdaSymbol{(}\AgdaBound{m}\AgdaSpace{}%
\AgdaOperator{\AgdaFunction{+}}\AgdaSpace{}%
\AgdaSymbol{(}\AgdaBound{n}\AgdaSpace{}%
\AgdaOperator{\AgdaFunction{+}}\AgdaSpace{}%
\AgdaBound{p}\AgdaSymbol{)))}\<%
\\
%
\>[4]\AgdaInductiveConstructor{refl}\<%
\\
%
\>[4]\AgdaSymbol{(λ}\AgdaSpace{}%
\AgdaBound{m}\AgdaSpace{}%
\AgdaSymbol{→}\AgdaSpace{}%
\AgdaFunction{cong}\AgdaSpace{}%
\AgdaFunction{suc}\AgdaSymbol{)}\<%
\\
%
\>[4]\AgdaBound{m}\<%
\end{code}

We have seen how we can define a type together with its induction principle,
and how we can make use of this induction principle to program and prove our
programs' properties. The next step is to use \AF{induction} on a given type
to define new types.

\section{Our First Indexed Type: Fin}

Given that the only type we have defined thus far is \AF{ℕ}, we are going to
use as the index of our type family. The natural candidate is {\AF{Fin} \AB{n}},
the type of finite numbers strictly smaller than \AB{n}.

This definition proceeds by induction on the index and as such is
characterised by a base and a step case.

\begin{code}%
\>[0]\AgdaFunction{Fin}\AgdaSpace{}%
\AgdaSymbol{:}\AgdaSpace{}%
\AgdaRecord{Elt}\AgdaSpace{}%
\AgdaFunction{ℕ}\AgdaSpace{}%
\AgdaSymbol{→}\AgdaSpace{}%
\AgdaFunction{Type}\<%
\\
\>[0]\AgdaFunction{Fin}\AgdaSpace{}%
\AgdaSymbol{=}\AgdaSpace{}%
\AgdaFunction{induction}\AgdaSpace{}%
\AgdaSymbol{(λ}\AgdaSpace{}%
\AgdaBound{\AgdaUnderscore{}}\AgdaSpace{}%
\AgdaSymbol{→}\AgdaSpace{}%
\AgdaFunction{Type}\AgdaSymbol{)}\AgdaSpace{}%
\AgdaFunction{base}\AgdaSpace{}%
\AgdaSymbol{(λ}\AgdaSpace{}%
\AgdaBound{\AgdaUnderscore{}}\AgdaSpace{}%
\AgdaSymbol{→}\AgdaSpace{}%
\AgdaFunction{step}\AgdaSymbol{)}\<%
\\
%
\\[\AgdaEmptyExtraSkip]%
\>[0][@{}l@{\AgdaIndent{0}}]%
\>[2]\AgdaKeyword{where}\<%
\end{code}

In the base case, corresponding to \AF{Fin} \AF{zero}, the boolean function is
constantly \AIC{false}. The type is empty as there are no finite numbers
strictly smaller than zero.

\begin{code}%
%
\>[2]\AgdaFunction{base}\AgdaSpace{}%
\AgdaSymbol{:}\AgdaSpace{}%
\AgdaFunction{Type}\<%
\\
%
\>[2]\AgdaFunction{base}\AgdaSpace{}%
\AgdaSymbol{\AgdaUnderscore{}}\AgdaSpace{}%
\AgdaSymbol{=}\AgdaSpace{}%
\AgdaInductiveConstructor{false}\<%
\end{code}

In the step case, corresponding to {\AF{Fin} (\AF{suc} \AB{n})}
we recognise a pattern similar to that used in the definition
of \AF{ℕ}: the string of interest is
either `Z'-headed with an empty tail
or `S'-headed with a tail of type \AF{Fin} \AB{n} (this type is
provided to us by the induction hypothesis called \AB{ih} here).

This time we do not bother introducing separate types for each of the
constructors but we could very well do so.

\begin{code}%
%
\>[2]\AgdaFunction{step}\AgdaSpace{}%
\AgdaSymbol{:}\AgdaSpace{}%
\AgdaFunction{Type}\AgdaSpace{}%
\AgdaSymbol{→}\AgdaSpace{}%
\AgdaFunction{Type}\<%
\\
%
\>[2]\AgdaFunction{step}\AgdaSpace{}%
\AgdaBound{ih}\AgdaSpace{}%
\AgdaInductiveConstructor{[]}\AgdaSpace{}%
\AgdaSymbol{=}\AgdaSpace{}%
\AgdaInductiveConstructor{false}\<%
\\
%
\>[2]\AgdaFunction{step}\AgdaSpace{}%
\AgdaBound{ih}\AgdaSpace{}%
\AgdaSymbol{(}\AgdaBound{c}\AgdaSpace{}%
\AgdaOperator{\AgdaInductiveConstructor{∷}}%
\>[765I]\AgdaBound{cs}\AgdaSymbol{)}\AgdaSpace{}%
\AgdaSymbol{=}%
\>[22]\AgdaBound{c}\AgdaSpace{}%
\AgdaOperator{\AgdaPrimitive{==}}\AgdaSpace{}%
\AgdaString{'Z'}\AgdaSpace{}%
\AgdaOperator{\AgdaFunction{\&\&}}\AgdaSpace{}%
\AgdaFunction{isNil}\AgdaSpace{}%
\AgdaBound{cs}\<%
\\
\>[765I][@{}l@{\AgdaIndent{0}}]%
\>[18]\AgdaOperator{\AgdaFunction{||}}%
\>[22]\AgdaBound{c}\AgdaSpace{}%
\AgdaOperator{\AgdaPrimitive{==}}\AgdaSpace{}%
\AgdaString{'S'}\AgdaSpace{}%
\AgdaOperator{\AgdaFunction{\&\&}}\AgdaSpace{}%
\AgdaBound{ih}\AgdaSpace{}%
\AgdaBound{cs}\<%
\end{code}

We can once more define the basic constructors for this type. They
have slightly more complex types, statically enforcing that the return
index is \AF{suc}-headed. ``Z'' gives rise to \AF{fzero}.

\begin{code}%
\>[0]\AgdaFunction{fzero}\AgdaSpace{}%
\AgdaSymbol{:}\AgdaSpace{}%
\AgdaSymbol{∀}\AgdaSpace{}%
\AgdaSymbol{\{}\AgdaBound{n}\AgdaSymbol{\}}\AgdaSpace{}%
\AgdaSymbol{→}\AgdaSpace{}%
\AgdaRecord{Elt}\AgdaSpace{}%
\AgdaSymbol{(}\AgdaFunction{Fin}\AgdaSpace{}%
\AgdaSymbol{(}\AgdaFunction{suc}\AgdaSpace{}%
\AgdaBound{n}\AgdaSymbol{))}\<%
\\
\>[0]\AgdaFunction{fzero}\AgdaSpace{}%
\AgdaSymbol{\{}\AgdaBound{n}\AgdaSymbol{\}}\AgdaSpace{}%
\AgdaSymbol{=}\AgdaSpace{}%
\AgdaFunction{Fin}\AgdaSpace{}%
\AgdaSymbol{(}\AgdaFunction{suc}\AgdaSpace{}%
\AgdaBound{n}\AgdaSymbol{)}\AgdaSpace{}%
\AgdaOperator{\AgdaFunction{∋}}\AgdaSpace{}%
\AgdaString{"Z"}\<%
\end{code}

And extending an existing list of characters with `S' is enough to
compute the successor of a \AF{Fin} \AB{n} element as witnessed by
\AF{fsuc}.

\begin{code}%
\>[0]\AgdaFunction{fsuc}\AgdaSpace{}%
\AgdaSymbol{:}\AgdaSpace{}%
\AgdaSymbol{∀}\AgdaSpace{}%
\AgdaSymbol{\{}\AgdaBound{n}\AgdaSymbol{\}}\AgdaSpace{}%
\AgdaSymbol{→}\AgdaSpace{}%
\AgdaRecord{Elt}\AgdaSpace{}%
\AgdaSymbol{(}\AgdaFunction{Fin}\AgdaSpace{}%
\AgdaBound{n}\AgdaSymbol{)}\AgdaSpace{}%
\AgdaSymbol{→}\AgdaSpace{}%
\AgdaRecord{Elt}\AgdaSpace{}%
\AgdaSymbol{(}\AgdaFunction{Fin}\AgdaSpace{}%
\AgdaSymbol{(}\AgdaFunction{suc}\AgdaSpace{}%
\AgdaBound{n}\AgdaSymbol{))}\<%
\\
\>[0]\AgdaFunction{fsuc}\AgdaSpace{}%
\AgdaOperator{\AgdaInductiveConstructor{[}}\AgdaSpace{}%
\AgdaBound{k}\AgdaSpace{}%
\AgdaOperator{\AgdaInductiveConstructor{]}}\AgdaSpace{}%
\AgdaSymbol{=}\AgdaSpace{}%
\AgdaOperator{\AgdaInductiveConstructor{[}}\AgdaSpace{}%
\AgdaString{'S'}\AgdaSpace{}%
\AgdaOperator{\AgdaInductiveConstructor{∷}}\AgdaSpace{}%
\AgdaBound{k}\AgdaSpace{}%
\AgdaOperator{\AgdaInductiveConstructor{]}}\<%
\end{code}

The definition of the induction principle for \AF{Fin} is left as
an exercise to the reader. It is very similar to the definition of
\AF{induction} for \AF{ℕ}. We will focus instead on a more interesting
observation related to \AF{Fin}.

\subsection{Subtyping: Fin n <: ℕ}

The astute reader will have noticed that the definition of \AF{Fin}
is not only similar to that of \AF{ℕ}, it should be the case that
all of the values of type \AF{Fin} \AB{n} are also stringly natural
number.

This can actually be proven. It should be unsurprising by now that
our tool of choice in this endeavour will be the induction principle
for \AF{ℕ}.

The key ingredient is the step case stating that, provided that
we can already prove that elements of \AF{Fin} \AB{n} are elements
of \AF{ℕ} then we should be able to do the same for elements of
\AF{Fin} (\AF{suc} \AB{n}).

\begin{code}%
\>[0]\AgdaFunction{step}\AgdaSpace{}%
\AgdaSymbol{:}\AgdaSpace{}%
\AgdaSymbol{∀}\AgdaSpace{}%
\AgdaBound{n}\AgdaSpace{}%
\AgdaSymbol{→}%
\>[817I]\AgdaSymbol{(∀}\AgdaSpace{}%
\AgdaBound{str}\AgdaSpace{}%
\AgdaSymbol{→}\AgdaSpace{}%
\AgdaBound{str}\AgdaSpace{}%
\AgdaOperator{\AgdaFunction{∈}}\AgdaSpace{}%
\AgdaFunction{Fin}\AgdaSpace{}%
\AgdaBound{n}\AgdaSpace{}%
\AgdaSymbol{→}\AgdaSpace{}%
\AgdaBound{str}\AgdaSpace{}%
\AgdaOperator{\AgdaFunction{∈}}\AgdaSpace{}%
\AgdaFunction{ℕ}\AgdaSymbol{)}\AgdaSpace{}%
\AgdaSymbol{→}\<%
\\
\>[.][@{}l@{}]\<[817I]%
\>[13]\AgdaSymbol{(∀}\AgdaSpace{}%
\AgdaBound{str}\AgdaSpace{}%
\AgdaSymbol{→}\AgdaSpace{}%
\AgdaBound{str}\AgdaSpace{}%
\AgdaOperator{\AgdaFunction{∈}}\AgdaSpace{}%
\AgdaFunction{Fin}\AgdaSpace{}%
\AgdaSymbol{(}\AgdaFunction{suc}\AgdaSpace{}%
\AgdaBound{n}\AgdaSymbol{)}\AgdaSpace{}%
\AgdaSymbol{→}\AgdaSpace{}%
\AgdaBound{str}\AgdaSpace{}%
\AgdaOperator{\AgdaFunction{∈}}\AgdaSpace{}%
\AgdaFunction{ℕ}\AgdaSymbol{)}\<%
\\
\>[0]\AgdaFunction{step}\AgdaSpace{}%
\AgdaBound{n}\AgdaSpace{}%
\AgdaBound{ih}\AgdaSpace{}%
\AgdaSymbol{(}\AgdaBound{c}\AgdaSpace{}%
\AgdaOperator{\AgdaInductiveConstructor{∷}}\AgdaSpace{}%
\AgdaBound{cs}\AgdaSymbol{)}\AgdaSpace{}%
\AgdaBound{isFin}\AgdaSpace{}%
\AgdaSymbol{=}\<%
\\
\>[0][@{}l@{\AgdaIndent{0}}]%
\>[2]\AgdaFunction{checkZ}\AgdaSpace{}%
\AgdaSymbol{(}\AgdaBound{c}\AgdaSpace{}%
\AgdaOperator{\AgdaFunction{≟}}\AgdaSpace{}%
\AgdaString{'Z'}\AgdaSymbol{)}\AgdaSpace{}%
\AgdaBound{cs}\AgdaSpace{}%
\AgdaSymbol{\{\{}\AgdaBound{isFin}\AgdaSymbol{\}\}}\AgdaSpace{}%
\AgdaKeyword{where}\<%
\end{code}

We include the proof for completness' sake even though it may not
be illuminating for the Agda novice. It proceeds by case analysis
on the input string, concluding immediately if it is ``Z'' and
utilising the induction hypothesis if it is `S'-headed instead.

\begin{code}%
%
\>[2]\AgdaFunction{checkS}\AgdaSpace{}%
\AgdaSymbol{:}%
\>[854I]\AgdaSymbol{∀}\AgdaSpace{}%
\AgdaSymbol{\{}\AgdaBound{b}\AgdaSymbol{\}}\AgdaSpace{}%
\AgdaSymbol{→}\AgdaSpace{}%
\AgdaDatatype{Reflects}\AgdaSpace{}%
\AgdaBound{c}\AgdaSpace{}%
\AgdaString{'S'}\AgdaSpace{}%
\AgdaBound{b}\AgdaSpace{}%
\AgdaSymbol{→}\AgdaSpace{}%
\AgdaSymbol{∀}\AgdaSpace{}%
\AgdaBound{cs}\AgdaSpace{}%
\AgdaSymbol{→}\<%
\\
\>[.][@{}l@{}]\<[854I]%
\>[11]\AgdaSymbol{\{\{}\AgdaFunction{IsTrue}\AgdaSpace{}%
\AgdaSymbol{(}\AgdaBound{b}\AgdaSpace{}%
\AgdaOperator{\AgdaFunction{\&\&}}\AgdaSpace{}%
\AgdaFunction{Fin}\AgdaSpace{}%
\AgdaBound{n}\AgdaSpace{}%
\AgdaBound{cs}\AgdaSymbol{)\}\}}\AgdaSpace{}%
\AgdaSymbol{→}\<%
\\
%
\>[11]\AgdaSymbol{(}\AgdaBound{c}\AgdaSpace{}%
\AgdaOperator{\AgdaInductiveConstructor{∷}}\AgdaSpace{}%
\AgdaBound{cs}\AgdaSymbol{)}\AgdaSpace{}%
\AgdaOperator{\AgdaFunction{∈}}\AgdaSpace{}%
\AgdaFunction{ℕ}\<%
\\
%
\>[2]\AgdaFunction{checkS}\AgdaSpace{}%
\AgdaInductiveConstructor{true}\AgdaSpace{}%
\AgdaBound{cs}\AgdaSpace{}%
\AgdaSymbol{\{\{}\AgdaBound{isFin}\AgdaSymbol{\}\}}\AgdaSpace{}%
\AgdaSymbol{=}\AgdaSpace{}%
\AgdaBound{ih}\AgdaSpace{}%
\AgdaBound{cs}\AgdaSpace{}%
\AgdaBound{isFin}\<%
\\
%
\\[\AgdaEmptyExtraSkip]%
%
\>[2]\AgdaFunction{checkZ}\AgdaSpace{}%
\AgdaSymbol{:}\AgdaSpace{}%
\AgdaSymbol{∀}\AgdaSpace{}%
\AgdaSymbol{\{}\AgdaBound{b}\AgdaSymbol{\}}\AgdaSpace{}%
\AgdaSymbol{→}\AgdaSpace{}%
\AgdaDatatype{Reflects}\AgdaSpace{}%
\AgdaBound{c}\AgdaSpace{}%
\AgdaString{'Z'}\AgdaSpace{}%
\AgdaBound{b}\AgdaSpace{}%
\AgdaSymbol{→}\AgdaSpace{}%
\AgdaSymbol{∀}\AgdaSpace{}%
\AgdaBound{cs}\AgdaSpace{}%
\AgdaSymbol{→}\<%
\\
\>[2][@{}l@{\AgdaIndent{0}}]%
\>[4]\AgdaSymbol{\{\{}\AgdaFunction{IsTrue}\AgdaSpace{}%
\AgdaSymbol{(}\AgdaBound{b}\AgdaSpace{}%
\AgdaOperator{\AgdaFunction{\&\&}}\AgdaSpace{}%
\AgdaFunction{isNil}\AgdaSpace{}%
\AgdaBound{cs}\AgdaSpace{}%
\AgdaOperator{\AgdaFunction{||}}\AgdaSpace{}%
\AgdaBound{c}\AgdaSpace{}%
\AgdaOperator{\AgdaPrimitive{==}}\AgdaSpace{}%
\AgdaString{'S'}\AgdaSpace{}%
\AgdaOperator{\AgdaFunction{\&\&}}\AgdaSpace{}%
\AgdaFunction{Fin}\AgdaSpace{}%
\AgdaBound{n}\AgdaSpace{}%
\AgdaBound{cs}\AgdaSymbol{)\}\}}\AgdaSpace{}%
\AgdaSymbol{→}\<%
\\
%
\>[4]\AgdaSymbol{(}\AgdaBound{c}\AgdaSpace{}%
\AgdaOperator{\AgdaInductiveConstructor{∷}}\AgdaSpace{}%
\AgdaBound{cs}\AgdaSymbol{)}\AgdaSpace{}%
\AgdaOperator{\AgdaFunction{∈}}\AgdaSpace{}%
\AgdaFunction{ℕ}\<%
\\
%
\>[2]\AgdaFunction{checkZ}\AgdaSpace{}%
\AgdaInductiveConstructor{true}%
\>[15]\AgdaInductiveConstructor{[]}\AgdaSpace{}%
\AgdaSymbol{=}\AgdaSpace{}%
\AgdaSymbol{\AgdaUnderscore{}}\<%
\\
%
\>[2]\AgdaFunction{checkZ}\AgdaSpace{}%
\AgdaInductiveConstructor{false}\AgdaSpace{}%
\AgdaBound{cs}\AgdaSpace{}%
\AgdaSymbol{=}\AgdaSpace{}%
\AgdaFunction{checkS}\AgdaSpace{}%
\AgdaSymbol{(}\AgdaBound{c}\AgdaSpace{}%
\AgdaOperator{\AgdaFunction{≟}}\AgdaSpace{}%
\AgdaString{'S'}\AgdaSymbol{)}\AgdaSpace{}%
\AgdaBound{cs}\<%
\end{code}

This can be put together with a trivial base case (remember that
\AF{Fin} \AF{zero} is the empty type so it cannot have any element
in it) to obtain the proof \AF{sub}.

\begin{code}%
\>[0]\AgdaFunction{sub}\AgdaSpace{}%
\AgdaSymbol{:}\AgdaSpace{}%
\AgdaSymbol{∀}\AgdaSpace{}%
\AgdaBound{n}\AgdaSpace{}%
\AgdaBound{str}\AgdaSpace{}%
\AgdaSymbol{→}\AgdaSpace{}%
\AgdaBound{str}\AgdaSpace{}%
\AgdaOperator{\AgdaFunction{∈}}\AgdaSpace{}%
\AgdaFunction{Fin}\AgdaSpace{}%
\AgdaBound{n}\AgdaSpace{}%
\AgdaSymbol{→}\AgdaSpace{}%
\AgdaBound{str}\AgdaSpace{}%
\AgdaOperator{\AgdaFunction{∈}}\AgdaSpace{}%
\AgdaFunction{ℕ}\<%
\\
\>[0]\AgdaFunction{sub}\AgdaSpace{}%
\AgdaSymbol{=}%
\>[936I]\AgdaFunction{induction}\<%
\\
\>[936I][@{}l@{\AgdaIndent{0}}]%
\>[8]\AgdaSymbol{(λ}\AgdaSpace{}%
\AgdaBound{n}\AgdaSpace{}%
\AgdaSymbol{→}\AgdaSpace{}%
\AgdaSymbol{∀}\AgdaSpace{}%
\AgdaBound{str}\AgdaSpace{}%
\AgdaSymbol{→}\AgdaSpace{}%
\AgdaBound{str}\AgdaSpace{}%
\AgdaOperator{\AgdaFunction{∈}}\AgdaSpace{}%
\AgdaFunction{Fin}\AgdaSpace{}%
\AgdaBound{n}\AgdaSpace{}%
\AgdaSymbol{→}\AgdaSpace{}%
\AgdaBound{str}\AgdaSpace{}%
\AgdaOperator{\AgdaFunction{∈}}\AgdaSpace{}%
\AgdaFunction{ℕ}\AgdaSymbol{)}\<%
\\
%
\>[8]\AgdaSymbol{(λ}\AgdaSpace{}%
\AgdaBound{\AgdaUnderscore{}}\AgdaSpace{}%
\AgdaSymbol{())}\<%
\\
%
\>[8]\AgdaFunction{step}\<%
\end{code}

This result allows us to write a \AF{cast} function converting an
element of \AF{Fin} \AB{n} into a stringly natural number. Notice
that the \ARF{value} part is the identity. Given that the \ARF{check}
part of the record will be erased at compile time this means we have
defined a \emph{zero cost coercion} from \AF{Fin} \AB{n} to \AF{ℕ}
which is much better than most state of the art dependently typed
programming languages, save for
Cedille~\cite{DBLP:journals/corr/abs-1802-00787}.

\begin{code}%
\>[0]\AgdaFunction{cast}\AgdaSpace{}%
\AgdaSymbol{:}\AgdaSpace{}%
\AgdaSymbol{∀}\AgdaSpace{}%
\AgdaSymbol{\{}\AgdaBound{n}\AgdaSymbol{\}}\AgdaSpace{}%
\AgdaSymbol{→}\AgdaSpace{}%
\AgdaRecord{Elt}\AgdaSpace{}%
\AgdaSymbol{(}\AgdaFunction{Fin}\AgdaSpace{}%
\AgdaBound{n}\AgdaSymbol{)}\AgdaSpace{}%
\AgdaSymbol{→}\AgdaSpace{}%
\AgdaRecord{Elt}\AgdaSpace{}%
\AgdaFunction{ℕ}\<%
\\
\>[0]\AgdaFunction{cast}\AgdaSpace{}%
\AgdaSymbol{\{}\AgdaBound{n}\AgdaSymbol{\}}\AgdaSpace{}%
\AgdaBound{p}\AgdaSpace{}%
\AgdaSymbol{.}\AgdaField{value}\AgdaSpace{}%
\AgdaSymbol{=}\AgdaSpace{}%
\AgdaBound{p}\AgdaSpace{}%
\AgdaSymbol{.}\AgdaField{value}\<%
\\
\>[0]\AgdaFunction{cast}\AgdaSpace{}%
\AgdaSymbol{\{}\AgdaBound{n}\AgdaSymbol{\}}\AgdaSpace{}%
\AgdaBound{p}\AgdaSpace{}%
\AgdaSymbol{.}\AgdaField{check}\AgdaSpace{}%
\AgdaSymbol{=}\AgdaSpace{}%
\AgdaFunction{sub}\AgdaSpace{}%
\AgdaBound{n}\AgdaSpace{}%
\AgdaSymbol{(}\AgdaBound{p}\AgdaSpace{}%
\AgdaSymbol{.}\AgdaField{value}\AgdaSymbol{)}\AgdaSpace{}%
\AgdaSymbol{(}\AgdaBound{p}\AgdaSpace{}%
\AgdaSymbol{.}\AgdaField{check}\AgdaSymbol{)}\<%
\end{code}


\section{Conclusion \& Future Work}

We have seen that we can take advantage of a dependently typed
host language to seriously consider the prospect of safe and
proven correct stringly typed programming.

We were able to define a notion of type of natural numbers carving out a
subset of well structured strings. This type is closed under the usual
constructors for the natural numbers \AF{zero} and \AF{suc}.

We then proved an induction principle for those strings that represent
natural numbers. This empowered us to start programming over these stringly
typed natural numbers in a way that is guaranteed total.

We demonstrated that our induction principle is strong enough to
not only program on the stringly typed natural numbers but also to prove the
fundamental properties of these programs.

We finally showed how we can use induction to define new types,
and how we can take advantage of the fact we are doing dependent
stringly typed programming to obtain zero cost coercions.

The definition of parametrised types such as the type of linked lists
or binary trees with values stored at the leaves is left to future work.

\bibliography{main}

\end{document}
