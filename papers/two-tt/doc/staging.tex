\documentclass{article}

%%%% Agda code %%%%%%%%%%%%%%%%%%%%%%%%%%%%%%%%
\usepackage{agda}
\usepackage[utf8]{inputenc}
\usepackage{newunicodechar}
\usepackage{newtxmath}
%\usepackage{newtxtext}

% Misc symbols
\newunicodechar{⌞}{\ensuremath{\llcorner}}
\newunicodechar{⌟}{\ensuremath{\lrcorner}}
\newunicodechar{′}{\ensuremath{\prime}}
\newunicodechar{−}{\ensuremath{-}}
\newunicodechar{─}{\ensuremath{-}}
\newunicodechar{◆}{\ensuremath{\Diamondblack}}
\newunicodechar{⧫}{\ensuremath{\blacklozenge}}
\newunicodechar{∷}{\ensuremath{::}}
\newunicodechar{∙}{\ensuremath{\bullet}}
\newunicodechar{□}{\ensuremath{\Box}}
\newunicodechar{∎}{\ensuremath{\blacksquare}}
\newunicodechar{⋆}{\ensuremath{\star}}
\newunicodechar{∣}{\ensuremath{|}}

% indices
\newunicodechar{₀}{\ensuremath{_0}}
\newunicodechar{₁}{\ensuremath{_1}}
\newunicodechar{₂}{\ensuremath{_2}}
\newunicodechar{₃}{\ensuremath{_3}}

\newunicodechar{ₑ}{\ensuremath{_e}}
\newunicodechar{ᵢ}{\ensuremath{_i}}
\newunicodechar{ₖ}{\ensuremath{_k}}
\newunicodechar{ₘ}{\ensuremath{_m}}
\newunicodechar{ₙ}{\ensuremath{_n}}
\newunicodechar{ᵣ}{\ensuremath{_r}}
\newunicodechar{ₛ}{\ensuremath{_s}}

\newunicodechar{₊}{\ensuremath{_+}}

% exponents
\newunicodechar{⁺}{\ensuremath{\textsuperscript{+}}}
\newunicodechar{⁻}{\ensuremath{\textsuperscript{-}}}

\newunicodechar{²}{\ensuremath{^2}}

\newunicodechar{ᵈ}{\ensuremath{^d}}
\newunicodechar{ⁱ}{\ensuremath{^i}}
\newunicodechar{ˡ}{\ensuremath{^l}}
\newunicodechar{ʳ}{\ensuremath{^r}}
\newunicodechar{ˢ}{\ensuremath{^s}}

\newunicodechar{ᴬ}{\ensuremath{^A}}
\newunicodechar{ᴮ}{\ensuremath{^B}}
\newunicodechar{ᴵ}{\ensuremath{^I}}
\newunicodechar{ᴿ}{\ensuremath{^R}}
\newunicodechar{ᵀ}{\ensuremath{^T}}
\newunicodechar{ᵁ}{\ensuremath{^U}}
\newunicodechar{ⱽ}{\ensuremath{^V}}

% Dots
\newunicodechar{⋯}{\ensuremath{\cdots}}
\newunicodechar{∶}{\ensuremath{:}}

% Equality symbols
\newunicodechar{∼}{\ensuremath{\sim}}
\newunicodechar{≡}{\ensuremath{\equiv}}
\newunicodechar{≢}{\ensuremath{\not\equiv}}
\newunicodechar{≟}{\mbox{\tiny\ensuremath{\stackrel{?}{=}}}}
\newunicodechar{≈}{\ensuremath{\approx}}

% Ordering symbols
\newunicodechar{≤}{\ensuremath{\le}}
\newunicodechar{≥}{\ensuremath{\ge}}

% Arrows
\newunicodechar{↦}{\ensuremath{\mapsto}}
\newunicodechar{⇑}{\ensuremath{\Uparrow}}
\newunicodechar{→}{\ensuremath{\rightarrow}}
\newunicodechar{←}{\ensuremath{\leftarrow}}
\newunicodechar{⇒}{\ensuremath{\Rightarrow}}
\newunicodechar{⇉}{\ensuremath{\rightrightarrows}}

% Mathematical symbols
\newunicodechar{∂}{\ensuremath{\partial}}
\newunicodechar{∋}{\ensuremath{\ni}}
\newunicodechar{∞}{\ensuremath{\infty}}
\newunicodechar{∀}{\ensuremath{\forall}}
\newunicodechar{∃}{\ensuremath{\exists}}
\newunicodechar{⊢}{\ensuremath{\vdash}}
\newunicodechar{⟨}{\ensuremath{\langle}}
\newunicodechar{⟩}{\ensuremath{\rangle}}
\newunicodechar{⊤}{\ensuremath{\top}}
\newunicodechar{∘}{\ensuremath{\circ}}
\newunicodechar{⊎}{\ensuremath{\uplus}}
\newunicodechar{×}{\ensuremath{\times}}
\newunicodechar{ℕ}{\ensuremath{\mathbb{N}}}
\newunicodechar{⟦}{\ensuremath{\llbracket}}
\newunicodechar{⟧}{\ensuremath{\rrbracket}}
\newunicodechar{∈}{\ensuremath{\in}}
\newunicodechar{↑}{\ensuremath{\uparrow}}
\newunicodechar{¬}{\ensuremath{\neg}}
\newunicodechar{⊥}{\ensuremath{\bot}}
\newunicodechar{↝}{\ensuremath{\leadsto}}
\newunicodechar{↶}{\ensuremath{\curvearrowleft}}
\newunicodechar{↺}{\ensuremath{\circlearrowleft}}
\newunicodechar{⊔}{\ensuremath{\sqcup}}
\newunicodechar{⨆}{\ensuremath{\bigsqcup}}
\newunicodechar{∩}{\ensuremath{\cap}}
\newunicodechar{∪}{\ensuremath{\cup}}


%%%%%%%%%%%%%%%%% LETTERS

% Misc
\newunicodechar{ℓ}{\ensuremath{\ell}}

% Greek uppercase
\newunicodechar{Δ}{\ensuremath{\Delta}}
\newunicodechar{Γ}{\ensuremath{\Gamma}}
\newunicodechar{Σ}{\ensuremath{\Sigma}}
\newunicodechar{Θ}{\ensuremath{\Theta}}
\newunicodechar{Ω}{\ensuremath{\Omega}}

% Greek lowercase
\newunicodechar{α}{\ensuremath{\alpha}}
\newunicodechar{β}{\ensuremath{\beta}}
\newunicodechar{δ}{\ensuremath{\delta}}
\newunicodechar{ε}{\ensuremath{\varepsilon}}
\newunicodechar{φ}{\ensuremath{\phi}}
\newunicodechar{γ}{\ensuremath{\gamma}}
\newunicodechar{ι}{\ensuremath{\iota}}
\newunicodechar{κ}{\ensuremath{\kappa}}
\newunicodechar{λ}{\ensuremath{\lambda}}
\newunicodechar{μ}{\ensuremath{\mu}}
\newunicodechar{ψ}{\ensuremath{\psi}}
\newunicodechar{η}{\ensuremath{\eta}}
\newunicodechar{ρ}{\ensuremath{\rho}}
\newunicodechar{σ}{\ensuremath{\sigma}}
\newunicodechar{τ}{\ensuremath{\tau}}
\newunicodechar{ξ}{\ensuremath{\xi}}
\newunicodechar{ζ}{\ensuremath{\zeta}}
\newunicodechar{Π}{\ensuremath{\Pi}}


% mathcal
\newunicodechar{𝓒}{\ensuremath{\mathcal{C}}}
\newunicodechar{𝓔}{\ensuremath{\mathcal{E}}}
\newunicodechar{𝓕}{\ensuremath{\mathcal{F}}}
\newunicodechar{𝓡}{\ensuremath{\mathcal{R}}}
\newunicodechar{𝓢}{\ensuremath{\mathcal{S}}}
\newunicodechar{𝓣}{\ensuremath{\mathcal{T}}}
\newunicodechar{𝓥}{\ensuremath{\mathcal{V}}}
\newunicodechar{𝓦}{\ensuremath{\mathcal{W}}}

\usepackage{catchfilebetweentags}
\makeatletter

\newrobustcmd*\OrigExecuteMetaData[2][\jobname]{%
\CatchFileBetweenTags\CatchFBT@tok{#1}{#2}%
\global\expandafter\CatchFBT@tok\expandafter{%
\expandafter}\the\CatchFBT@tok
}%\OrigExecuteMetaData

\newrobustcmd*\ChkExecuteMetaData[2][\jobname]{%
\CatchFileBetweenTags\CatchFBT@tok{#1}{#2}%
\edef\mytokens{\detokenize\expandafter{\the\CatchFBT@tok}}
\ifx\mytokens\empty\PackageError{catchfilebetweentags}{the tag #2 is not found\MessageBreak in file #1 \MessageBreak called from \jobname.tex}{use a different tag}\fi%
}%\ChkExecuteMetaData

\renewrobustcmd*\ExecuteMetaData[2][\jobname]{%
\ChkExecuteMetaData[#1]{#2}%
\OrigExecuteMetaData[#1]{#2}%
}

\makeatother

% do not indent code blocks
\setlength\mathindent{0em}
%%%%%%%%%%%%%%%%%%%%%%%%%%%%%%%%%%%%%%%%%%%%%%%

%%%%%%%%%% AGDA ALIASES

\newcommand{\APT}{\AgdaPrimitiveType}
\newcommand{\AK}{\AgdaKeyword}
\newcommand{\AM}{\AgdaModule}
\newcommand{\AS}{\AgdaSymbol}
\newcommand{\AN}{\AgdaNumber}
\newcommand{\AD}{\AgdaDatatype}
\newcommand{\AF}{\AgdaFunction}
\newcommand{\AR}{\AgdaRecord}
\newcommand{\ARF}{\AgdaField}
\newcommand{\AB}{\AgdaBound}
\newcommand{\AIC}{\AgdaInductiveConstructor}

\newcommand{\model}{$\mathit{𝓜}^{Γ}_{σ}$}


\usepackage{hyperref}
\usepackage[style=alphabetic]{biblatex}
\bibliography{staging.bib}

\title{Scoped and Typed Staging by Evaluation}
\author{Guillaume Allais}

\begin{document}

\maketitle

\section{Introduction}

This paper gives an intrinsically well scoped and well typed
treatment of a simply typed version of the staged compilation
with two-level type theory introduced by
Kov{\'{a}}cs \cite{DBLP:journals/pacmpl/Kovacs22}.
%
We use Agda~\cite{DBLP:conf/afp/Norell08}
as our implementation language and
mobilise standard formalisation
techniques extensively described by Allais, Chapman,
McBride, and McKinna~\cite{DBLP:conf/cpp/Allais0MM17}.

Starting from a language with a static and a dynamic layer,
we perform a model construction and define an evaluation
function turning terms into their semantical counterpart.
%
As a corollary, we obtain a staging function which
fully evaluates all of the static subterms and returns
purely dynamic ones.

\section{An Intrinsically Typed 2 Level Type Theory}

Types are indexed by the stage they live in.

\begin{minipage}[t]{.55\textwidth}
  \ExecuteMetaData[TwoTT.tex]{types}
\end{minipage}\hfill
\begin{minipage}[t]{.4\textwidth}
  \ExecuteMetaData[TwoTT.tex]{stage}
\end{minipage}

We have both static and dynamic natural numbers,
hence the unconstrained index \AB{st} for the
constructor \AIC{`ℕ}.
%
The constructor \AIC{‘⇑\_} allows us to embed dynamic
type into static ones; (\AIC{‘⇑} \AB{A}) is effectively
the type of \emph{programs} that will compute a value of
type \AB{A} at runtime.
%
Function types are available in both layers provided that
they are homogeneous: both the domain and codomain need
to live in the same layer.
%
Last but not least, we have an example of the fact,
highlighted in Kov{\'{a}}cs' original paper, that static
datatypes do not need to have a counterpart at runtime:
pairs are only available in the static layer.

Context are left-nested lists of types.

\ExecuteMetaData[TwoTT.tex]{context}

\ExecuteMetaData[TwoTT.tex]{var}

\ExecuteMetaData[TwoTT.tex]{term}

\section{Staging by Evaluation}

The goal of this section is to define a type of \AF{Value}s
as well as an \AF{eval}uation function which computes the
value associated to each term, provided that we have an
appropriate environment to interpret the term's free variables.

\ExecuteMetaData[TwoTT.tex]{evaldecl}

We start with the model construction describing precisely
what these values ought to be.

\subsection{Model Construction}

The type of values is defined by case analysis on the stage.
%
Static values are given a static meaning (defined below)
while dynamic values are given a meaning as terms
guaranteed not to contain any static subterm.

\ExecuteMetaData[TwoTT.tex]{model}

The family of static values is defined by induction on
the value's type. It is fairly similar to the standard
normalisation by evaluation
construction~\cite{DBLP:conf/lics/BergerS91,DBLP:journals/mscs/CoquandD97,DBLP:journals/lisp/Coquand02}
except that static values at a base types cannot possibly
be neutral terms.

\begin{AgdaSuppressSpace}
\ExecuteMetaData[TwoTT.tex]{modelstadecl}
\ExecuteMetaData[TwoTT.tex]{modelsta}
\end{AgdaSuppressSpace}

\ExecuteMetaData[TwoTT.tex]{thin}

\ExecuteMetaData[TwoTT.tex]{kripke}

\ExecuteMetaData[TwoTT.tex]{thinrefl}

\ExecuteMetaData[TwoTT.tex]{kripkeapp}


\subsection{Evaluation}

\ExecuteMetaData[TwoTT.tex]{zero}
\ExecuteMetaData[TwoTT.tex]{succ}
\ExecuteMetaData[TwoTT.tex]{app}

\begin{AgdaSuppressSpace}
\ExecuteMetaData[TwoTT.tex]{evaldecl}
\ExecuteMetaData[TwoTT.tex]{eval}
\end{AgdaSuppressSpace}


\section{Future Work}

\cite{DBLP:journals/lisp/Coquand02}

\printbibliography

\end{document}
