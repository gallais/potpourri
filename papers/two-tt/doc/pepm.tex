\documentclass[sigplan,screen]{acmart}

\PassOptionsToPackage{x11names}{xcolor}
\usepackage{jigsaw}
\usepackage{tikz}
\usepackage{stmaryrd}
\usepackage{mathtools}
\usepackage{etoolbox}
\usepackage{idris2}
\usepackage{catchfilebetweentags}
\makeatletter

\newrobustcmd*\OrigExecuteMetaData[2][\jobname]{%
\CatchFileBetweenTags\CatchFBT@tok{#1}{#2}%
\global\expandafter\CatchFBT@tok\expandafter{%
\expandafter}\the\CatchFBT@tok
}%\OrigExecuteMetaData

\newrobustcmd*\ChkExecuteMetaData[2][\jobname]{%
\CatchFileBetweenTags\CatchFBT@tok{#1}{#2}%
\edef\mytokens{\detokenize\expandafter{\the\CatchFBT@tok}}
\ifx\mytokens\empty\PackageError{catchfilebetweentags}{the tag #2 is not found\MessageBreak in file #1 \MessageBreak called from \jobname.tex}{use a different tag}\fi%
}%\ChkExecuteMetaData

\renewrobustcmd*\ExecuteMetaData[2][\jobname]{%
\ChkExecuteMetaData[#1]{#2}%
\OrigExecuteMetaData[#1]{#2}%
}

\makeatother

\usepackage{hyperref}
\usepackage{cleveref}
\usepackage{bytefield}

%\usepackage{fullpage}

\usepackage{todonotes}
\setuptodonotes{inline}

\usepackage{listings}

\lstset{ %
  language=C,
  numbers=left,
  numberstyle=\tiny,
  stepnumber=1,
  numbersep=5pt,
  breaklines=true,
  otherkeywords={uint8_t},
}

\usepackage{pgfplots}
\pgfplotsset{compat=1.16, width=\textwidth}

%%%%%%%%%%%%% SUBMISSION PROCESSES

\newtoggle{BLIND}

\newcommand{\usestt}{\small}

%%%%%%%%%%%%%%%%%%%%%%%%%%%

%%% Languages

\newcommand{\agda}{Agda}
\newcommand{\idris}{Idris~2}

%%% Idris highlighting

\newcommand{\assertTotal}{\IdrisPostulate{assert\KatlaUnderscore{}total}}
\newcommand{\believeMe}{\IdrisPostulate{believe\KatlaUnderscore{}me}}

%%% Data highlighting

\newcommand{\hexadesc}[1]{\texttt{\usestt\IdrisType{#1}}}
\newcommand{\hexadata}[1]{\texttt{\usestt\IdrisData{#1}}}
\newcommand{\hexacons}[1]{\texttt{\usestt\IdrisFunction{#1}}}
\newcommand{\hexaoffset}[1]{\texttt{\usestt{\color{lightgray}#1}}}

\newenvironment{hexdump}{\usestt\medskip\ttfamily\obeyspaces\obeylines\noindent}{\medskip}

%%% Links to supplementary material
\newcommand{\suppfile}[1]{File \texttt{#1}}

%%% Remark environment
\usepackage{amsthm}
\newtheorem{remark}{Idris Specifics}

\usepackage{xifthen}% provides \isempty test
%%% Hoare logic
\newcommand{\Pointer}[3]{\ensuremath{#1 \xmapsto{~#2~} #3}}
\newcommand{\Hoare}[4]{\ensuremath{\lbrace~ #1 ~\rbrace~ #2 ~\lbrace~ #3.\, #4 ~\rbrace}}
\newcommand{\Meaning}[2]{\ensuremath{\llbracket \,#1\, \rrbracket\, (\Mu{#2})}}
\newcommand{\Mu}[1]{\ensuremath{\mu \, \mathit{#1}}}


%% Rights management information.  This information is sent to you
%% when you complete the rights form.  These commands have SAMPLE
%% values in them; it is your responsibility as an author to replace
%% the commands and values with those provided to you when you
%% complete the rights form.
\setcopyright{acmcopyright}
\copyrightyear{2024}
\acmYear{2024}
\acmDOI{XXXXXXX.XXXXXXX}

%% These commands are for a PROCEEDINGS abstract or paper.
\acmConference[PEPM '24]{the ACM SIGPLAN Workshop on Partial Evaluation and Program Manipulation}{January 16, 2024}{London, UK}
%%
%%  Uncomment \acmBooktitle if the title of the proceedings is different
%%  from ``Proceedings of ...''!
%%
%%\acmBooktitle{Woodstock '18: ACM Symposium on Neural Gaze Detection,
%%  June 03--05, 2018, Woodstock, NY}
\acmPrice{15.00}
\acmISBN{978-1-4503-XXXX-X/18/06}


%%
%% Submission ID.
%% Use this when submitting an article to a sponsored event. You'll
%% receive a unique submission ID from the organizers
%% of the event, and this ID should be used as the parameter to this command.
%%\acmSubmissionID{123-A56-BU3}

%%
%% For managing citations, it is recommended to use bibliography
%% files in BibTeX format.
%%
%% You can then either use BibTeX with the ACM-Reference-Format style,
%% or BibLaTeX with the acmnumeric or acmauthoryear sytles, that include
%% support for advanced citation of software artefact from the
%% biblatex-software package, also separately available on CTAN.
%%
%% Look at the sample-*-biblatex.tex files for templates showcasing
%% the biblatex styles.
%%

%%
%% The majority of ACM publications use numbered citations and
%% references.  The command \citestyle{authoryear} switches to the
%% "author year" style.
%%
%% If you are preparing content for an event
%% sponsored by ACM SIGGRAPH, you must use the "author year" style of
%% citations and references.
%% Uncommenting
%% the next command will enable that style.
%%\citestyle{acmauthoryear}


%%
%% end of the preamble, start of the body of the document source.
\begin{document}

%%
%% The "title" command has an optional parameter,
%% allowing the author to define a "short title" to be used in page headers.
\title{Scoped and Typed Staging by Evaluation}

%%
%% The "author" command and its associated commands are used to define
%% the authors and their affiliations.
%% Of note is the shared affiliation of the first two authors, and the
%% "authornote" and "authornotemark" commands
%% used to denote shared contribution to the research.
\author{Guillaume Allais}
\email{guillaume.allais@strath.ac.uk}
\orcid{0000-0002-4091-657X}
\affiliation{%
  \institution{University of Strathclyde}
  \streetaddress{26 Richmond Street}
  \city{Glasgow}
  \country{United Kingdom}
  \postcode{G1 1XH}
}

%%
%% By default, the full list of authors will be used in the page
%% headers. Often, this list is too long, and will overlap
%% other information printed in the page headers. This command allows
%% the author to define a more concise list
%% of authors' names for this purpose.
% \renewcommand{\shortauthors}{}

%%
%% The abstract is a short summary of the work to be presented in the
%% article.
\begin{abstract}
\end{abstract}

%%
%% The code below is generated by the tool at http://dl.acm.org/ccs.cfm.
%% Please copy and paste the code instead of the example below.
%%
\begin{CCSXML}
\end{CCSXML}

%%
%% Keywords. The author(s) should pick words that accurately describe
%% the work being presented. Separate the keywords with commas.
\keywords
  { Staging%
  , Dependent Types%
  , Two Level Type Theory%
  , Normalisation by Evaluation%
  , Agda%
  }
%% A "teaser" image appears between the author and affiliation
%% information and the body of the document, and typically spans the
%% page.
%% \begin{teaserfigure}
%%   \includegraphics[width=\textwidth]{sampleteaser}
%%   \caption{Seattle Mariners at Spring Training, 2010.}
%%   \Description{Enjoying the baseball game from the third-base
%%   seats. Ichiro Suzuki preparing to bat.}
%%   \label{fig:teaser}
%% \end{teaserfigure}

\received{18 October 2023}
\received[revised]{TBD}
\received[accepted]{TBD}

\documentclass{article}

%%%% Agda code %%%%%%%%%%%%%%%%%%%%%%%%%%%%%%%%
\usepackage{agda}
\input{unicode}
\usepackage{catchfilebetweentags}
\makeatletter

\newrobustcmd*\OrigExecuteMetaData[2][\jobname]{%
\CatchFileBetweenTags\CatchFBT@tok{#1}{#2}%
\global\expandafter\CatchFBT@tok\expandafter{%
\expandafter}\the\CatchFBT@tok
}%\OrigExecuteMetaData

\newrobustcmd*\ChkExecuteMetaData[2][\jobname]{%
\CatchFileBetweenTags\CatchFBT@tok{#1}{#2}%
\edef\mytokens{\detokenize\expandafter{\the\CatchFBT@tok}}
\ifx\mytokens\empty\PackageError{catchfilebetweentags}{the tag #2 is not found\MessageBreak in file #1 \MessageBreak called from \jobname.tex}{use a different tag}\fi%
}%\ChkExecuteMetaData

\renewrobustcmd*\ExecuteMetaData[2][\jobname]{%
\ChkExecuteMetaData[#1]{#2}%
\OrigExecuteMetaData[#1]{#2}%
}

\makeatother

% do not indent code blocks
\setlength\mathindent{0em}
%%%%%%%%%%%%%%%%%%%%%%%%%%%%%%%%%%%%%%%%%%%%%%%

\usepackage{todonotes}
\setuptodonotes{inline}



%%%%%%%%%%%%%%%%%%%%%%%%%%%%%%%%%%%%%%%%%%%%%%%%%%%%%%%%%%%%%
%%%%%%%%%%%%%%%%%%% Command definitions %%%%%%%%%%%%%%%%%%%%%
%%%%%%%%%%%%%%%%%%%%%%%%%%%%%%%%%%%%%%%%%%%%%%%%%%%%%%%%%%%%%

\newtheorem{theorem}{Theorem}[section]
\newtheorem{lemma}[theorem]{Lemma}
\newtheorem{proposition}[theorem]{Proposition}
\newtheorem{corollary}[theorem]{Corollary}

\newenvironment{definition}[1][Definition]{\begin{trivlist}
\item[\hskip \labelsep {\bfseries #1}]}{\end{trivlist}}
\newenvironment{example}[1][Example]{\begin{trivlist}
\item[\hskip \labelsep {\bfseries #1}]}{\end{trivlist}}
\newenvironment{remark}[1][Remark]{\begin{trivlist}
\item[\hskip \labelsep {\bfseries #1}]}{\end{trivlist}}

\newcommand{\agda}[0]{Agda}
\newcommand{\coq}[0]{Coq}
\newcommand{\lego}[0]{Lego}
\newcommand{\epigram}[0]{Epigram}

%%%%%%%%%%%%%%%%%%%%%%%% Agda stuff %%%%%%%%%%%%%%%%%%%%%%%%%

\newcommand{\set}[0]{{\color{NavyBlue}\mathtt{Set}}}
\newcommand{\fin}[1][n]{{\color{NavyBlue}\mathtt{Fin}}~n}
\DeclareMathOperator{\tyind}{\mathtt{type-induction}}

\newcommandx*\lcni[2][1=\Delta, 2=\Gamma]{#1 ~{\color{NavyBlue}\supseteq}~ #2}
\newcommandx*\incl[2][1=\Gamma, 2=\Delta]{#1 ~{\color{NavyBlue}\subseteq}~ #2}
\DeclareMathOperator{\bni}{~{\color{NavyBlue}\ni}~}
\DeclareMathOperator{\piun}{\pi_1}
\DeclareMathOperator{\pide}{\pi_2}
\DeclareMathOperator{\injun}{\iota_1}
\DeclareMathOperator{\injde}{\iota_2}
\DeclareMathOperator{\teunit}{{\color{ForestGreen}\mathtt{tt}}}

\newcommand{\idinc}[1][\Gamma]{{\color{NavyBlue}\subseteq\hspace{-3pt}-refl}~#1}
\DeclareMathOperator{\gramdecl}{:\hspace{-2pt}:\hspace{-2pt}=~}

%%%%%%%%%%%%%%%%%%%%%%%%%% Types %%%%%%%%%%%%%%%%%%%%%%%%%%%%

\newcommand{\context}[1][\type]{{\color{NavyBlue}\mathtt{Con}}(#1)}
\newcommand{\type}[1][n]{{\color{NavyBlue}\mathtt{type}}_{#1}}

\newcommand{\tybase}[1][k]{{\color{ForestGreen}\mathtt{`b}}~#1}
\newcommand{\tyunit}[0]{{\color{ForestGreen}\mathtt{`1}}}
\newcommandx*\typrod[2][1=\sigma, 2=\tau]{#1 ~{\color{ForestGreen}\mathtt{`}\hspace{-2pt}\times}~ #2}
\newcommandx*\tyarrow[2][1=\sigma, 2=\tau]{#1 ~{\color{ForestGreen}\mathtt{`}\hspace{-3pt}\rightarrow}~ #2}
\newcommand{\tylist}[1][\sigma]{{\color{ForestGreen}\mathtt{`list}}~ #1}

%%%%%%%%%%%%%%%%%%%%%%%%%% Terms %%%%%%%%%%%%%%%%%%%%%%%%%%%%

\newcommandx*\typingrule[3][1=t, 2=\sigma, 3=\Gamma]{#3 ~{\color{NavyBlue}\vdash}~ #1 ~{\color{NavyBlue}\colon} #2}

\newcommandx*\term[2][1=\sigma, 2=\Gamma]{#2 ~{\color{NavyBlue}\vdash}~ #1}
\newcommandx*\subst[2][1=\Gamma, 2=\Delta]{#2 ~{\color{NavyBlue}\vdash\hspace{-3pt}\varepsilon}~ #1}
\newcommand{\idsubst}[1][\Gamma]{{\color{NavyBlue}\mathtt{id}}_{#1}}

\DeclareMathOperator{\base}{{\color{ForestGreen}\mathtt{base}}}
\DeclareMathOperator{\pop}{{\color{ForestGreen}\mathtt{pop!}}}
\DeclareMathOperator{\step}{{\color{ForestGreen}\mathtt{step}}}
\DeclareMathOperator{\ConEmpty}{{\color{ForestGreen}\varepsilon}}
\newcommandx*\ConExtend[2][1=\Gamma, 2=\sigma]{#1~{\color{ForestGreen}\cdot}~#2}

\newcommandx*\whneutral[2][1=\sigma, 2=\Gamma]{#2 ~{\color{NavyBlue}\vdash_{\mathit{whne}}}~ #1}
\newcommandx*\whnormal[2][1=\sigma, 2=\Gamma]{#2 ~{\color{NavyBlue}\vdash_{\mathit{whnf}}}~ #1}
\newcommandx*\neutral[2][1=\sigma, 2=\Gamma]{#2 ~{\color{NavyBlue}\vdash_{\mathit{ne}}}~ #1}
\newcommandx*\normal[2][1=\sigma, 2=\Gamma]{#2 ~{\color{NavyBlue}\vdash_{\mathit{nf}}}~ #1}

\newcommand{\weaken}[1][inc]{{\color{NavyBlue}\mathtt{wk}}_{\mathit{#1}}}
\newcommand{\weakene}[1][inc]{{\color{NavyBlue}\mathtt{wk}\hspace{-2pt}\varepsilon}_{\mathit{#1}}}

\DeclareMathOperator{\tecomp}{{\color{NavyBlue}\cdot}}
\DeclareMathOperator{\tevar}{{\color{ForestGreen}\mathtt{`v}}}
\DeclareMathOperator{\tepair}{~{\color{ForestGreen}\mathtt{`,}}~}
\DeclareMathOperator{\telam}{{\color{ForestGreen}\mathtt{`}\lambda}}
\newcommand{\telamclosure}[1][\rho]{{\color{ForestGreen}\mathtt{`}\lambda[}#1{\color{ForestGreen}]}}
\DeclareMathOperator{\teapp}{{\color{ForestGreen}\mathtt{`\$}}}
\DeclareMathOperator{\tett}{{\color{ForestGreen}\mathtt{`\langle\rangle}}}
\DeclareMathOperator{\tepiun}{{\color{ForestGreen}\mathtt{`\hspace{-2pt}\piun}}}
\DeclareMathOperator{\tepide}{{\color{ForestGreen}\mathtt{`\hspace{-2pt}\pide}}}
\DeclareMathOperator{\temap}{{\color{ForestGreen}\mathtt{`map}}}
\DeclareMathOperator{\teappend}{~{\color{ForestGreen}\mathtt{`\hspace{-3pt}+\hspace{-5pt}+}}~}
\DeclareMathOperator{\tefold}{{\color{ForestGreen}\mathtt{`fold}}}
\DeclareMathOperator{\tenil}{{\color{ForestGreen}\mathtt{`[]}}}
\DeclareMathOperator{\tecons}{~{\color{ForestGreen}\mathtt{`:\hspace{-2pt}:}}~}
\DeclareMathOperator{\telift}{{\color{ForestGreen}`\hspace{-3pt}\uparrow}}
\newcommandx*\temappend[3][1=v_1, 2=n, 3=v_2]
{\temap{\hspace{-2pt}\color{ForestGreen}\langle}#1{~\color{ForestGreen},~}#2{\color{ForestGreen}\rangle\hspace{-5pt}\teappend}#3}

\DeclareMathOperator{\teinjun}{{\color{ForestGreen}`\hspace{-2pt}\injun}}
\DeclareMathOperator{\teinjde}{{\color{ForestGreen}`\hspace{-2pt}\injde}}
\DeclareMathOperator{\tesumelim}{{\color{ForestGreen}\mathtt{`\hspace{-2pt}+\hspace{-2pt}-elim}}}

\DeclareMathOperator{\vappend}{{\color{NavyBlue}\NBE\hspace{-3pt}\mathtt{+\hspace{-5pt}+}}}
\DeclareMathOperator{\vmap}{{\color{NavyBlue}\NBE\hspace{-2pt}\mathtt{map}}}
\DeclareMathOperator{\vcomp}{{\color{NavyBlue}\NBE\hspace{-2pt}\circ}}
\DeclareMathOperator{\vfold}{{\color{NavyBlue}\NBE\hspace{-3pt}\mathtt{fold}}}
\DeclareMathOperator{\lookup}{~{\color{NavyBlue}\mathtt{!!}}~}


%%%%%%%%%%%%%%%%%%%%%%%%% Reductions %%%%%%%%%%%%%%%%%%%%%%%%

\DeclareMathOperator{\rules}{\beta\delta\iota\eta\nu}

\DeclareMathOperator{\betared}{{\color{NavyBlue}\leadsto_{\beta}}}
\DeclareMathOperator{\betareds}{{\color{NavyBlue}\leadsto_{\beta}^*}}
\DeclareMathOperator{\iotared}{{\color{NavyBlue}\leadsto_{\iota}}}
\DeclareMathOperator{\nured}{{\color{NavyBlue}\leadsto_{\nu}}}
\DeclareMathOperator{\etared}{{\color{NavyBlue}\leadsto_{\eta}}}
\DeclareMathOperator{\etareds}{{\color{NavyBlue}\leadsto_{\eta}^*}}
\DeclareMathOperator{\reduce}{{\color{NavyBlue}\leadsto_{\rules}}}
\DeclareMathOperator{\reduces}{{\color{NavyBlue}\leadsto_{\rules}^*}}

\DeclareMathOperator{\conv}{{\color{NavyBlue}\cong_{\rules{}}}}

%%%%%%%%%%%%%%%%%%%%%%%%%% Models %%%%%%%%%%%%%%%%%%%%%%%%%%%

\newcommand{\NBE}[0]{{\color{NavyBlue}\mathcal{M}}}
\newcommand{\NBEL}[0]{{\color{NavyBlue}\mathcal{L}}}
\newcommand{\NBEE}[0]{{\color{NavyBlue}\mathcal{M}\varepsilon}}

\newcommand{\WHNF}[0]{{\color{NavyBlue}\NBE_{wh}}}
\newcommand{\AWHNF}[0]{{\color{NavyBlue}\NBE^{\star}_{wh}}}
\newcommand{\AWHNFL}[0]{{\color{NavyBlue}\NBEL^{\star}_{wh}}}
\newcommand{\WHNFE}[0]{{\color{NavyBlue}\NBEE_{wh}}}

\newcommandx*\whmodel[2][1=\sigma, 2=\Gamma]{{\color{NavyBlue}\WHNF}(#2 , #1)}
\newcommandx*\gluer[3][1=\sigma, 2=\whmodel, 3=\Gamma]{{\color{NavyBlue}\langle} #3 ~{\color{NavyBlue},}~ #1 {\color{NavyBlue}\rangle\hspace{-4pt}\lefttorightarrow} #2}
\newcommandx*\awhmodel[2][1=\sigma, 2=\Gamma]{{\color{NavyBlue}\AWHNF}(#2 , #1)}
\newcommandx*\awhmodellist[3][1=\sigma, 2=\Vdash_{\sigma}, 3=\Gamma]{{\color{NavyBlue}\AWHNFL}(#3 , #1, #2)}
\newcommand{\awhmodellistfun}[1][\sigma]{\lambda \Gamma \rightarrow \whmodel[{#1}]}

\newcommandx*\model[2][1=\sigma, 2=\Gamma]{{\color{NavyBlue}\NBE}(#2 , #1)}
\newcommandx*\modellist[3][1=\sigma, 2=\modellistfunabs, 3=\Gamma]{{\color{NavyBlue}\NBEL}(#3 , #1, #2)}
\newcommand{\modellistfunabs}[1][\sigma]{\mathtt{M}_{#1}}
\newcommand{\modellistfun}[1][\sigma]{\model[#1][.~]}

\newcommandx*\whmodele[2][1=\Gamma, 2=\Delta]{{\color{NavyBlue}\WHNFE}(#2 , #1)}
\newcommandx*\modele[2][1=\Gamma, 2=\Delta]{{\color{NavyBlue}\NBEE}(#2 , #1)}

\newcommand{\idmodele}[1][\Gamma]{{\color{NavyBlue}\mathtt{id}}_{\NBEE ~ #1}}

\newcommand{\reify}[1][\sigma]{{\color{NavyBlue}\uparrow\hspace{-1pt}}_{#1}}
\newcommand{\listreify}[1][\sigma]{{\color{NavyBlue}\uparrow\hspace{-3pt}\uparrow\hspace{-1pt}}_{#1}}
\newcommand{\reflect}[1][\sigma]{{\color{NavyBlue}\downarrow\hspace{-1pt}}_{#1}}
\newcommand{\listreflect}[1][\sigma]{{\color{NavyBlue}\downarrow\hspace{-3pt}\downarrow\hspace{-1pt}}_{#1}}

\DeclareMathOperator{\norm}{{\color{NavyBlue}norm}}
\DeclareMathOperator{\termeval}{{\color{NavyBlue}eval}}

%%%%%%%%%%%%%%%%%%%%% Logical Relation %%%%%%%%%%%%%%%%%%%%%%

\newcommandx*\awhrelation[4][1=\sigma, 2=t, 3=T, 4=\Gamma]{\awhmodel[#1][#4]
     ~{\color{NavyBlue}\ni}~ #2 ~{\color{NavyBlue}\lightning^{\star}_{wh}}~ #3}
\newcommandx*\whrelation[3][1=\sigma, 2=T, 3=\Gamma]{{\color{NavyBlue}\lightning_{wh}}(#1)(#2)}
\newcommand{\whrelationlist}[1][\sigma]{\Vdash_{{#1}}\hspace{-5pt}\lightning_{wh}}

\newcommandx*\sound[4][1=\sigma, 2=t, 3=T, 4=\Gamma]{\model[#1][#4]
     ~{\color{NavyBlue}\ni}~ #2 ~{\color{NavyBlue}\lightning}~ #3}
\newcommandx*\soundlist[6][1=\mathit{xs}, 2=\mathit{XS}, 3=\Gamma, 4=\sigma, 5=\modellistfunabs, 6=\protect{\soundlistfunabs[.][.]}]
{\modellist[#4][#5,#6][#3] ~{\color{NavyBlue}\ni}~ #1 ~{\color{NavyBlue}\lightning}~ #2}
\newcommandx*\soundlistfunabs[3][1=t, 2=T, 3=\sigma]{\modellistfunabs[#3]~#1~\lightning~#2}

\newcommandx*\sounde[4][1=\rho, 2=R, 3=\Gamma, 4=\Delta]{\modele[#3][#4]
     ~{\color{NavyBlue}\ni}~ #1 ~{\color{NavyBlue}\lightning\hspace{-3pt}\lightning}~ #2}


\newcommandx*\exteq[3][1=\sigma, 2=T, 3=U]{#2 ~{\color{NavyBlue}\equiv}_{#1}~ #3}
\newcommandx*\exteqlist[3][1=\sigma, 2=T, 3=U]{#2 ~{\color{NavyBlue}\equiv}_{#1}^{{\color{NavyBlue}\mathtt{`list}}}~ #3}
\newcommandx*\uniform[2][1=\sigma, 2=T]{{\color{NavyBlue}\mathtt{Uni}}_{#1}~ #2}





%%%%%%%%%%%%%%%%%%%%% Unicode characters %%%%%%%%%%%%%%%%%%%%%%






















\usepackage{hyperref}
\usepackage[style=alphabetic]{biblatex}
\bibliography{staging.bib}

\title{Scoped and Typed Staging by Evaluation}
\author{Guillaume Allais}

\begin{document}

\maketitle

\section{Introduction}

This paper gives an intrinsically well scoped and well typed
treatment of a simply typed version of the staged compilation
with two-level type theory introduced by
Kov{\'{a}}cs~\cite{DBLP:journals/pacmpl/Kovacs22}.
%
We use Agda~\cite{DBLP:conf/afp/Norell08}
as our implementation language and
mobilise standard formalisation
techniques extensively described by Allais, Chapman,
McBride, and McKinna~\cite{DBLP:conf/cpp/Allais0MM17}.

Starting from a language with a static and a dynamic layer,
we perform a type-directed model construction
and define an evaluation function turning terms
into their semantical counterpart.
%
As a corollary, we obtain a staging function which
fully evaluates all of the static subterms and returns
purely dynamic ones.

As already observed by Kov{\'{a}}cs, this model construction
is analogous to the one used for normalisation by evaluation,
hence the title of this paper.

\section{An Intrinsically Typed 2 Level Type Theory}

\todo{start with examples}


\subsection{Stages and Types}

Our notion of types is going to be explicitly indexed by the
stage they live in, and so we begin by defining a sum
type \AD{Stage} with two constructors: \AIC{static},
and \AIC{dynamic}.

\ExecuteMetaData[TwoTT.tex]{stage}

We can now define our inductive family of simple types indexed
by their stage.

\ExecuteMetaData[TwoTT.tex]{types}

We have both static and dynamic natural numbers,
hence the unconstrained index \AB{st} for the
constructor \AIC{`ℕ}.
%
The constructor \AIC{‘⇑\_} allows us to embed dynamic
types into static ones; (\AIC{‘⇑} \AB{A}) is effectively
the type of \emph{programs} that will compute a value of
type \AB{A} at runtime.
%
Function types are available in both layers provided that
they are homogeneous: both the domain and codomain need
to live in the same layer.
%
Last but not least, we have an example of the fact,
highlighted in Kov{\'{a}}cs' original paper, that static
datatypes do not need to have a counterpart at runtime:
pairs are only available in the static layer and all traces
of pairs will have completely disappeared after staging.

\subsection{Contexts and Variables}

In order to closely match our usual mathematical notations,
contexts are defined as left-nested lists of types.

\ExecuteMetaData[TwoTT.tex]{context}

Contexts can either be empty (written \AIC{ε})
or take an existing context \AB{Γ} and extend it on the
right with a newly bound variable of type \AB{σ}
(written \AB{Γ} \AIC{-,} \AB{σ}).
Note that the stage of the newly bound variable is
existentially quantified.

Once we have contexts, we can formalise what it means
for a variable of type \AB{σ} to be present in context \AB{Γ}.
This is a standard definition corresponding to a scoped-and-typed
variant of de Bruijn index~\cite{de1972lambda}.

\ExecuteMetaData[TwoTT.tex]{var}

Here we used notations meant to once again mimick inference
rules~\cite{DBLP:conf/cpp/Allais0MM17}.
\todo{Add inference rules}

\subsection{Intrinsically Scoped and Typed Syntax}

\ExecuteMetaData[TwoTT.tex]{term}

\section{Staging by Evaluation}

The goal of this section is to define a type of \AF{Value}s
as well as an \AF{eval}uation function which computes the
value associated to each term, provided that we have an
appropriate environment to interpret the term's free variables.

\ExecuteMetaData[TwoTT.tex]{evaldecl}

We start with the model construction describing precisely
what these values ought to be.

\subsection{Model Construction}

The type of values is defined by case analysis on the stage.
%
Static values are given a static meaning (defined below)
while dynamic values are given a meaning as terms
guaranteed not to contain any static subterm.

\ExecuteMetaData[TwoTT.tex]{model}

The family of static values is defined by induction on
the value's type. It is fairly similar to the standard
normalisation by evaluation
construction~\cite{DBLP:conf/lics/BergerS91,DBLP:journals/mscs/CoquandD97,DBLP:journals/lisp/Coquand02}
except that static values at a base types cannot possibly
be neutral terms.

\begin{AgdaSuppressSpace}
\ExecuteMetaData[TwoTT.tex]{modelstadecl}
\ExecuteMetaData[TwoTT.tex]{modelsta}
\end{AgdaSuppressSpace}

\ExecuteMetaData[TwoTT.tex]{thin}

\ExecuteMetaData[TwoTT.tex]{kripke}

\ExecuteMetaData[TwoTT.tex]{thinrefl}

\ExecuteMetaData[TwoTT.tex]{kripkeapp}


\subsection{Evaluation}

\ExecuteMetaData[TwoTT.tex]{zero}
\ExecuteMetaData[TwoTT.tex]{succ}
\ExecuteMetaData[TwoTT.tex]{app}


\ExecuteMetaData[TwoTT.tex]{extend}

\begin{AgdaSuppressSpace}
\ExecuteMetaData[TwoTT.tex]{evaldecl}
\ExecuteMetaData[TwoTT.tex]{eval}
\end{AgdaSuppressSpace}


\section{Future Work}

\cite{DBLP:journals/lisp/Coquand02}

\newpage
\printbibliography

\end{document}


%%
%% The acknowledgments section is defined using the "acks" environment
%% (and NOT an unnumbered section). This ensures the proper
%% identification of the section in the article metadata, and the
%% consistent spelling of the heading.
\begin{acks}
\iftoggle{BLIND}{
  Thanks to XXX for comments.
}{
  We would like to thank Bob Atkey for his suggestion to index
  stages by a phase, thus allowing us to ensure that a staged
  term does not have any static subterm.

  We would additionally like to thank the reviewers for their
  helpful suggestions of relevant related works.
}

\end{acks}

%%
%% The next two lines define the bibliography style to be used, and
%% the bibliography file.
\bibliographystyle{ACM-Reference-Format}
\bibliography{staging}

\end{document}
\endinput
