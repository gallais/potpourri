
\section{Interlude: Views and Singletons}\label{sec:view}

The precise indexing of pointers by a runtime-irrelevant copy of the value
they are pointing to means that inspecting the buffer's content should
not only return runtime information but also refine the index to reflect
that information at the type-level.
%
As a consequence, the functions we are going to define in the following
subsections are views.

\subsection{Views}

A view
in the sense of Wadler~\citep{DBLP:conf/popl/Wadler87},
and subsequently refined by McBride and McKinna~\citep{DBLP:journals/jfp/McBrideM04}
for a type $T$ is a type family $V$ indexed by $T$ together
with a function which maps values $t$ of type $T$ to values of type
$V\,t$.
%
By inspecting the $V\,t$ values we can learn something about the
$t$ input.
%
The prototypical example is perhaps the `snoc` (`cons' backwards) view
of right-nested lists as if they were left-nested.
We present the \IdrisType{Snoc} family below.

\ExecuteMetaData[Snoc.idr.tex]{Snoc}

By matching on a value of type
(\IdrisType{Snoc} \IdrisBound{xs}) we get to learn
either that \IdrisBound{xs} is empty (\IdrisData{Lin}, nil backwards)
or that it has an initial segment \IdrisBound{init} and a last element
\IdrisBound{last} (\IdrisBound{init} \IdrisData{:<} \IdrisBound{last}).
%
The function \IdrisFunction{unsnoc} demonstrates that we can always
\emph{view} a \IdrisType{List} in a \IdrisType{Snoc}-manner.

\ExecuteMetaData[Snoc.idr.tex]{unsnoc}

Here we defined \IdrisType{Snoc} as an inductive family but it can
sometimes be convenient to define the family recursively instead.
In which case the \IdrisType{Singleton} inductive family can
help us connect runtime values to their
runtime-irrelevant type-level counterparts.


\subsection{The Singleton type}\label{sec:datasingleton}

The \IdrisType{Singleton} family has a single constructor
which takes an argument \IdrisBound{x} of type \IdrisBound{a},
its return type is indexed precisely by this \IdrisBound{x}.

\ExecuteMetaData[Data/Singleton.idr.tex]{singleton}

More concretely this means that a value of type
(\IdrisType{Singleton} $t$) has to be a runtime relevant
copy of the term $t$.
%
Note that \idris{} performs an optimisation similar to Haskell's
\texttt{newtype} unwrapping: every data type that has a single
non-recursive constructor with only one non-erased argument
is unwrapped during compilation.
%
This means that at runtime the
\IdrisType{Singleton} / \IdrisData{MkSingleton} indirections
will have disappeared.

We can define some convenient combinators to manipulate
singletons.
%
We reuse the naming conventions typical of applicative
functors which will allow us to rely on \idris{}'s automatic
desugaring of \emph{idiom brackets}~\cite{DBLP:journals/jfp/McbrideP08}
into expressions using these combinators.

\ExecuteMetaData[Data/Singleton.idr.tex]{pure}

First \IdrisFunction{pure} is a simple alias for \IdrisData{MkSingleton},
it turns a runtime-relevant value \IdrisBound{x} into a singleton for
this value.

\ExecuteMetaData[Data/Singleton.idr.tex]{fmap}

Next, we can `map' a function under a \IdrisType{Singleton} layer: given
a pure function \IdrisBound{f} and a runtime copy of \IdrisBound{t} we
can get a runtime copy of (\IdrisBound{f} \IdrisBound{t}).

\ExecuteMetaData[Data/Singleton.idr.tex]{ap}

Finally, we can apply a runtime copy of a function \IdrisBound{f}
to a runtime copy of an argument \IdrisBound{t}
to get a runtime copy of the result (\IdrisBound{f} \IdrisBound{t}).

As we mentioned earlier, \idris{} automatically desugars idiom brackets
using these combinators. That is to say that
\IdrisKeyword{[|} \IdrisBound{x} \IdrisKeyword{|]} will be elaborated to
(\IdrisFunction{pure} \IdrisBound{x}) while
\IdrisKeyword{[|} \IdrisBound{f} \IdrisBound{t1} $\cdots$ \IdrisBound{tn} \IdrisKeyword{|]}
will become
(\IdrisBound{f} \IdrisFunction{<\$>} \IdrisBound{t1} \IdrisFunction{<*>} $\cdots$ \IdrisFunction{<*>} \IdrisBound{tn}).
%
This lets us apply \IdrisType{Singleton}-wrapped values almost as seamlessly as pure values.


We are now equipped with the appropriate notions and definitions to
look at a buffer's content.
