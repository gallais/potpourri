\subsection{Related Work}

This work sits at the intersection of many domains:
data-generic programming,
the efficient runtime representation of functional data,
programming over serialised values,
and the design of serialisation formats.
%
Correspondingly, a lot of related work is worth discussing.
In many cases the advantage of our approach is precisely that it is
at the intersection of all of these strands of research.

\subsubsection{Data-Generic Programming}

There is a long tradition of data-generic
programming~\citep{DBLP:conf/ssdgp/Gibbons06} and we will mostly focus here
on the approach based on the careful reification of a precise universe of
discourse as an inductive family in a host type theory,
and the definition of generic programs by induction over this family.

One early such instance is Pfeifer and Rue{\ss}'
`polytypic proof construction'~\citep{DBLP:conf/tphol/PfeiferR99}
meant to replace unsafe meta-programs deriving recursors
(be they built-in support, or user-written tactics).

%\todo{See refs. in n-ary polymorphic functions}

In his PhD thesis, Morris~\citep{DBLP:phd/ethos/Morris07} declares various
universes for strictly positive types and families and defines by generic
programming
further types (the type of one-hole contexts),
modalities (the universal and existential predicate lifting over the functors he considers),
and functions (map, boolean equality).

L{\"{o}}h and Magalh{\~{a}}es~\citep{DBLP:conf/icfp/LohM11} define a more
expressive universe over indexed functors that is closed under composition
and fixpoints.
%
They also detail how to define additional generic construction such as a
proof of decidable equality, various recursion schemes, and zippers.
%
This work, quite similar to our own in its presentation, offers a natural
candidate universe for us to use to extend our library.


\subsubsection{Efficient Runtime Representation of Inductive Values}

Although not dealing explicitly with programming over serialised data,
Monnier's work~\citep{DBLP:conf/icfp/Monnier19} with its focus on performance and
in particular on the layout of inductive values at runtime,
partially motivated our endeavour.
%
Provided that we find a way to get the specialisation and partial evaluation
of the generically defined views, we ought to be able to achieve --purely in
user code-- Monnier's vision of a representation where n-ary tuples have
constant-time access to each of their component.

Baudon, Radanne, and Gonnord~\citep{ACM:conf/icfp/Baudon23} adopted
a radically different approach to ours: we provide users with a
uniform representation that requires no setup on their part,
while \textsc{RIBBIT} instead gives power users absolute freedom
to define their own data layouts.
%
The project provides seamless integration: users annotate their
data declarations with the data layout they want.
%
The programming experience is left unchanged: users write nested
pattern-matching programs without ever needing to explicitly talk
about the layout.
%
They cover a universe of monomorphic immutable algebraic datatypes
fairly similar to ours, however the annotations language is expressive
enough to talk about e.g. principled bit stealing, or word alignment.
%
Their approach requires modifying the whole compiler pipeline while
ours lives purely in userland.
%
\todo{Finish description}

\todo{Dargent}

\subsubsection{Working on Serialised Data}

Swierstra and van Geest~\citep{DBLP:conf/icfp/GeestS17}
define in \agda{} a rich universe of data descriptions
and generically implement serialising and deserialising,
proving that the latter is a left inverse to the former.
%
Their universe is powerful enough that later parts of a
description can be constrained using the \emph{value}
associated to ealier ones.
%
We will be able to rely on this work when extending our
current universe to one for type families.
%
In turn, our approach could be ported to their setting
to avoid the need to fully deserialise the data in order
to process it.


LoCal~\citep{DBLP:conf/pldi/VollmerKRS0N19} is the work that originally
motivated the design of this library.
%
We have demonstrated that generic programming within a dependently typed
language can yield the sort of benefits other language can only achieve
by inventing entirely new intermediate languages and compilation schemes.

LoCal was improved upon with a re-thinking of the serialisation scheme
making the approach compatible with parallel
programming~\citep{DBLP:journals/pacmpl/KoparkarRVKN21}.
This impressive improvement is a natural candidate for future work on our
part: the authors demonstrate it is possible to reap the benefits of
both programming over serialised data
and dividing up the work over multiple processors
with almost no additional cost in the case of a purely sequential execution.

\subsubsection{Serialisation Formats}

The PADS project~\citep{DBLP:conf/popl/MandelbaumFWFG07} aims to let users
quickly, correctly, and compositionally describe existing formats they
have no control over.
%
As they reminds us, ad-hoc serialisation formats abound be it in
networking, logging, billing, or as output of measurement equipments
in e.g. gene sequencing or molecular biology.
%
Our current project is not offering this kind of versatility as we have
decided to focus on a specific serialisation format with strong
guarantees about the efficient access to subtrees.
%
But our approach to defining correct-by-construction components could
be leveraged in that setting too and bring users strong guarantees about
the traversals they write.

ASN.1~\citep{MANUAL:book/larmouth1999} gives users the ability to define
a high-level specification of the exchange format (the `abstract syntax')
to be used in communications without the need to concern themselves with
the actual encoding as bit patterns (the `transfer syntax').
%
This separation between specification and implementation means that parsing
and encoding can be defined once and for all by generic programming
(here, a compiler turning specifications into code in the user's host
language of choice).
%
The main difference is once again our ability to program in a
correct-by-construction manner over the values thus represented.


Yallop's automatic derivation of serialisers using an OCaml
preprocessor~\citep{DBLP:conf/ml/Yallop07} highlights the importance
of empowering domain experts to take advantage of the specifics of
the problem they are solving to minimise the size of the encoded data.
%
By detecting sharing using a custom equality function respecting
$\alpha$-equivalence instead of the default one, he was able to
serialise large lambda terms using only
a quarter of the bytes OCaml's standard library marshaller.

\subsubsection{Type Theory for Data Layout}

Petersen, Robert, Crary, and Pfenning~\citeyearpar{DBLP:conf/popl/PetersenHCP03}
design an elegant type theory based on ordered linear logic
in which users can carefully control their data's memory
layout.
%
This logic-based approach supports the explicit reservation,
initialisation, and allocation of new memory locations all
without needing to explicitly talk about an underlying
heap-based memory model.
%
This is however only possible by requiring implementors to
write a dedicated language implementation while our definitions
live purely in client code, defining an embedded domain
specific language as a library.
%
Lastly, their language provides no support for internally
proving the properties of the programs one defines.
