\section{Meaning as Trees}\label{sec:trees}

We now see descriptions as functors and, correspondingly,
datatypes as the initial objects of the associated functor-algebras.
%
This is a standard construction derived from Malcolm's
work~\citep{DBLP:journals/scp/Malcolm90},
itself building on Hagino's categorically-inspired
definition of a lambda calculus
with a generic notion of datatypes~\citep{DBLP:conf/ctcs/Hagino87}.

In our work these trees will be used primarily to allow users to
give a precise specification of the IO procedures they actually want
to write in order to process values stored in buffers.
%
We expect these inductive trees and the associated generic programs
consuming them to be mostly used at the \IdrisKeyword{0}
modality i.e. to be erased during compilation.

\subsection{Descs as Functors}

We define the meaning of descriptions as strictly positive
endofunctors on \IdrisType{Type} by induction on said descriptions.
%
\IdrisFunction{Meaning} gives us the action of the functors on objects.

\ExecuteMetaData[Serialised/Desc.idr.tex]{meaning}

Both \IdrisData{None} and \IdrisData{Byte} are interpreted by constant
functors (respectively the one returning the unit type, and the one returning
the type of bytes).
%
\IdrisData{Rec} is the identity functor.
%
Finally (\IdrisData{Prod} \IdrisBound{d} \IdrisBound{e})
is interpreted as the pairing of the interpretation of
\IdrisBound{d} and \IdrisBound{e} respectively.
We use our own definition of pairing rather than the
standard library's because it gives us better syntactic sugar.

\ExecuteMetaData[Lib.idr.tex]{pair}

This gives us the action of descriptions on types, let us now
see their action on morphisms.
%
We once again proceed by induction on the description.

\ExecuteMetaData[Serialised/Desc.idr.tex]{fmap}\label{def:fmap}

All cases but the one for \IdrisData{Rec} are structural.
%
Verifying that these definitions respect the functor laws is left as
an exercise for the reader.

\subsection{Data as Trees}

Given a datatype description \IdrisBound{cs}, our first goal is
to define what it means to pick a constructor.
%
The \IdrisType{Index} record is a thin wrapper around a finite
natural number known to be smaller than the number of constructors
this type provides.

\ExecuteMetaData[Serialised/Desc.idr.tex]{index}

We use this type rather than \IdrisType{Fin} directly because it
plays well with inference.
%
In the following code snippet, implementing a function returning
the description corresponding to a given index,
we use this to our advantage:
the \IdrisBound{cs} argument can be left implicit because it already
shows up in the type of the \IdrisType{Index} and can thus be
reconstructed by unification.

\ExecuteMetaData[Serialised/Desc.idr.tex]{description}

This type of indices also allows us to provide users with
syntactic sugar enabling them to use the constructors' names
directly rather than confusing numeric indices.
%
The following function runs a decision procedure
\IdrisFunction{isConstructor} at the type level
in order to turn any raw string \IdrisBound{str}
into the corresponding \IdrisType{Index}.

\ExecuteMetaData[Serialised/Desc.idr.tex]{fromString}

If the name is valid then \IdrisFunction{isConstructor} will
return a valid \IdrisType{Index} and \idris{} will be able to
\IdrisKeyword{auto}matically fill-in the implicit proof.
%
If the name is not valid then \idris{} will not
find the index and will raise a compile time error.
%
We include a successful example on the left and a failing test
on the right hand side (\IdrisKeyword{failing} blocks are only
accepted in \idris{} if their body leads to an error).

\noindent
\begin{minipage}[t]{0.35\textwidth}
  \ExecuteMetaData[Serialised/Desc.idr.tex]{indexleaf}
\end{minipage}\hfill
\begin{minipage}[t]{0.42\textwidth}
\ExecuteMetaData[Serialised/Desc.idr.tex]{notindexcons}
\end{minipage}

Once equipped with the ability to pick constructors, we can define
the type of algebras for the functor described by a \IdrisType{Data}
description. For each possible constructor, we demand an algebra for
the functor corresponding to the meaning of the  constructor's description.

\ExecuteMetaData[Serialised/Desc.idr.tex]{alg}

We can then introduce the fixpoint of data descriptions as the initial
algebra, defined as the following inductive type.

\ExecuteMetaData[Serialised/Desc.idr.tex]{mu}


Note that here we are forced to use \assertTotal{} to convince \idris{}
to accept the definition.
%
Indeed, unlike Agda, \idris{} does not (yet!) track whether a function's
arguments are used in a strictly positive manner.
%
Consequently the positivity checker
is unable to see that \IdrisFunction{Meaning} uses its second
argument in a strictly positive manner
and that this is therefore a legal definition.

Now that we can build trees as fixpoints of the
meaning of descriptions, we can define convenient aliases for
the \IdrisFunction{Tree} constructors.
%
Note that the leftmost \IdrisData{(\#)} use in each definition corresponds
to the \IdrisType{Mu} constructor while later ones are \IdrisType{Tuple}
constructors.
%
\idris{}'s type-directed disambiguation of constructors allows us to use
this uniform notation for all of these pairing notions.

\noindent
\begin{minipage}[t]{.3\textwidth}
  \ExecuteMetaData[Serialised/Desc.idr.tex]{leaf}
\end{minipage}\hfill
\begin{minipage}[t]{.65\textwidth}
  \ExecuteMetaData[Serialised/Desc.idr.tex]{node}
\end{minipage}

This enables us to define our running example as an inductive value:

\ExecuteMetaData[Serialised/Desc.idr.tex]{longexample}

\subsection{Generic Fold}\label{sec:genericfoldinductive}

\IdrisType{Mu} gives us the initial fixpoint for these algebras i.e.
given any other algebra over a type \IdrisBound{a}, from a term of
type (\IdrisType{Mu} \IdrisBound{cs}), we can compute an \IdrisBound{a}.
%
We define the generic \IdrisFunction{fold} function over inductive values
as follows:

\ExecuteMetaData[Serialised/Desc.idr.tex]{fold}

We first match on the term's top constructor, use \IdrisFunction{fmap}
(defined in \Cref{def:fmap})
to recursively apply the fold to all the node's subterms and finally
apply the algebra to the result.

Here we only use \assertTotal{} because \idris{} does not see that
\IdrisFunction{fmap} only applies its argument to strict subterms.
This limitation could easily be bypassed by mutually defining
an inlined and specialised version of
(\IdrisFunction{fmap} \IdrisKeyword{\KatlaUnderscore} (\IdrisFunction{fold} \IdrisBound{alg})).
as we demonstrate in \Cref{appendix:safefold}.
%
In an ideal type theory these supercompilation steps, whose sole
purpose is to satisfy the totality checker, would be automatically
performed by the compiler~\citep{MANUAL:phd/dublin/Mendel12}.

%\todo{Include induction example \& move \IdrisFunction{All} here?}

Further generic programming can yield other useful programs e.g. a
generic proof that tree equality is decidable or a generic definition
of zippers~\citep{DBLP:conf/icfp/LohM11}.
