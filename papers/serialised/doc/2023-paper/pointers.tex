\section{Meaning as Pointers into a Buffer}\label{sec:pointers}

Now that we know the serialisation format, we can give a meaning
to constructor and data descriptions as pointers into a buffer.

\subsection{Tracking Buffer Positions}

We start with the definition of the counterpart to \IdrisType{Mu}
for serialised values. A tree sitting in a buffer is represented
by a record packing the buffer, the position at which the tree's
root node is stored, and the size of the tree.
%
Note that according to our serialisation format the size is not stored
in the file but using the size of the buffer, the stored offsets,
and the size of the static data we will always
be able to compute a value corresponding to it.

\ExecuteMetaData[SaferIndexed.idr.tex]{pointermu}

For reasons that will become apparent in \Cref{sec:bufferfold}
when we start programming over serialised data in a correct-by-construction
manner the record \IdrisType{Mu} is parameterised not only by the description
of the type of the data stored but also by a runtime-irrelevant inductive value of
that type.

\ExecuteMetaData[SaferIndexed.idr.tex]{elem}

The counterpart to a \IdrisFunction{Meaning} stores additional information.
For a description of type (\IdrisType{Desc} \IdrisBound{r} \IdrisBound{s} \IdrisBound{o})
on top of the buffer, the position at which the root of the meaning resides,
and the size of the layer we additionally have a vector of \IdrisBound{o} offsets.

Now that we have pointers, we can use them to look at the content
they are referring to. But first we need to introduce some basic tools
to be able to talk precisely about this stored content.

\subsection{Writing a Tree to a File}\label{sec:writetofile}

Once we have a pointer to a tree in a buffer, we can easily write it to a
file be it for safekeeping or sending.

\ExecuteMetaData[SaferIndexed.idr.tex]{writeToFile}

We first start by reading the size of the header stored in the buffer.
%
This allows us to compute both the start of the data block as well as the
size of the buffer that will contain the header followed by the tree we want
to write to a file.


We then check whether the position of the pointer is exactly the beginning
of the data block.

If it is then we are pointing to the whole tree and the current buffer can
be written to a file as is.

Otherwise we are pointing to a subtree and need to separate it from its
surrounding context first.
%
To do so we allocate a new buffer of the right size and copy the raw bytes
corresponding to the header first, and the tree of interest second.

We can then write the buffer we have picked to a file and happily succeed.
