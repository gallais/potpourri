
\section{Inspecting the Buffer's Content}\label{sec:poking}

We can now describe the combinators allowing our users to inspect
the value they have a pointer for.
%
We are going to define the most basic of building blocks
(\IdrisFunction{poke} and \IdrisFunction{out}),
combine them to derive useful higher-level combinators
(\IdrisFunction{layer} and \IdrisFunction{view}),
and ultimately use these to implement a generic correct-by-construction
version of \IdrisFunction{fold} operating over trees stored in a buffer
(cf. \Cref{sec:bufferfold}).

Coming back to our analogy to the `points to' assertions used in
separation logic~\cite{DBLP:conf/lics/Reynolds02} when verifying
imperative programs,
the basic building blocks correspond to the axiomatisation of
the behaviour of the underlying language constructs (hence their
necessarily unsafe implementation in this shallow embedding)
while the high-level combinators are derived reasoning principles
allowing us to implemement more complex programs.

\subsection{Poking the Buffer}

Our most basic operation consists in poking the buffer to unfold
the description by exactly one step.

Thinking in terms of Hoare triples, if we have a pointer
$\ell$ to a term $t$ known to be a single byte then
(\IdrisFunction{poke} $\ell$)
will return a byte $bs$ and allow us to observe that $t$ is
equal to that byte.

\[
\Hoare
    {\Pointer{\ell}{\text{\IdrisData{Byte}}}{t}}
    {\text{\IdrisFunction{poke}}\,\ell}
    {\mathit{bs}.\, t = \mathit{bs} \ast \ell \mapsto t}
\]

Similarly, if the pointer $\ell$ is for a pair then
(\IdrisFunction{poke} $\ell$) will
reveal that the term $t$ can be taken apart into the pairing
of two terms $t_1$ and $t_2$ and
return a pointer for each of these components.

\[
\Hoare
    {\Pointer{\ell}{\text{\IdrisData{Prod}}\, d_1 d_2}{t}}
    {\text{\IdrisFunction{poke}}\,\ell}
    {(\ell_1, \ell_2).\, \exists t_1.\, \exists t_2.\, t = (t_1, t_2)
        \ast \ell \mapsto t \ast \Pointer{\ell_1}{d_1}{t_1} \ast \Pointer{\ell_2}{d_2}{t_2}}
\]

The exact type of the function is as follows: provider a pointer for
a meaning \IdrisBound{t}, we return an \IdrisType{IO} process computing
the one step unfolding of the meaning.

\ExecuteMetaData[SaferIndexed.idr.tex]{pokefun}

The result type of this operation is defined by case-analysis on the
description. In order to keep the notations user-friendly, we
mutually define
a recursive function \IdrisFunction{Poke} interpreting the straightforward type constructors
and an inductive family \IdrisType{Poke'} with interesting return indices.

\ExecuteMetaData[SaferIndexed.idr.tex]{pokedatafun}

Poking a buffer containing \IdrisData{None} will return a value of
the unit type as no information whatsoever is stored there.

If we access a \IdrisData{Byte} then we expect that inspecting the
buffer will yield a runtime-relevant copy of the type-level byte we
have for reference. Hence the use of \IdrisType{Singleton}.

If we access a \IdrisData{Prod} of two descriptions then the type-level term
better be a pair and we better be able to obtain a \IdrisType{Pointer.Meaning}
for each of the sub-meanings.
%
Because \idris{} does not currently support definitional eta equality
for records, it will me more ergonomic for users if we introduce
\IdrisType{Poke'} rather than yielding a \IdrisType{Tuple} of values.
By matching on \IdrisType{Poke'} at the value level, they will see the
pair at the type level also reduced to a constructor-headed tuple.

Last but not least if the description is \IdrisData{Rec} this means
we have a substructure. In this case we simply demand a pointer to it.

\ExecuteMetaData[SaferIndexed.idr.tex]{pokedatadata}

The implementation of this operation proceeds by case analysis
on the description.
%
As we are going to see shortly, it is necessarily somewhat unsafe
as we claim to be able to connect a type-level value to whatever
it is that we read from the buffer.
%
Let us go through each case one-by-one.

\ExecuteMetaData[SaferIndexed.idr.tex]{pokefunNone}

If the description is \IdrisData{None} we do not need to fetch any
information from the buffer and can immediately return \IdrisData{()}.

\ExecuteMetaData[SaferIndexed.idr.tex]{pokefunByte}

If the description is \IdrisData{Byte} then we read a byte at the
determined position. The only way we can connect this value we just
read to the type index is to use the unsafe combinator
\IdrisPostulate{unsafeMkSingleton} to manufacture a value of type
(\IdrisType{Singleton} \IdrisBound{t}) instead of the value of type
(\IdrisType{Singleton} \IdrisBound{bs})
we would expect from wrapping \IdrisBound{bs} in the \IdrisData{MkSingleton} constructor.


\ExecuteMetaData[SaferIndexed.idr.tex]{pokefunProd}

If the description is the product of two sub-descriptions then we
want to compute the \IdrisType{Pointer.Meaning} corresponding to
each of them.
%
We start by splitting the vector of offsets to distribute them between
the left and right subtrees.

We can readily build the pointer for the \IdrisBound{left} subdescription:
it takes the left offsets, the buffer, and has the same starting position
as the whole description of the product as they are stored one after the other.
Its size is the sum of the space reserverd by all of the left offsets
(\IdrisFunction{sum} \IdrisBound{subl}) as well as the static size occupied
by the rest of the content  (\IdrisBound{sl}).

We then compute the starting position of the right subdescription: we need to
move past the whole of the left subdescription, that is to say that the starting
position is the sum of the starting position for the whole product and \IdrisBound{sizel}.
%
The size of the right subdescription is then easily computed by subtracting
\IdrisBound{sizel} from the overall \IdrisBound{size} of the paired subdescriptions.

We can finally use the lemma \IdrisFunction{etaTuple} saying that a tuple
is equal to the pairing of its respective projections
in order to turn \IdrisBound{t} into
(\IdrisFunction{fst} \IdrisBound{t} \IdrisData{\#} \IdrisFunction{snd} \IdrisBound{t})
which lets us use the \IdrisType{Poke'} constructor \IdrisData{(\#)} to return our
pair of pointers.

\ExecuteMetaData[SaferIndexed.idr.tex]{pokefunRec}

Lastly, when we reach a \IdrisData{Rec} description, we can discard the
vector of offsets and return a \IdrisType{Pointer.Mu} with the same buffer,
starting position and size as our input pointer.

\subsection{Extracting one layer}

By repeatedly poking the buffer, we can unfold a full layer.
The result of this operation is defined by induction
on the description. It is identical to the definition of
\IdrisFunction{Poke} except for the \IdrisData{Prod} case:
here, instead of being content with a pointer for each of the
subdescriptions, we demand a full layer for them too.

\ExecuteMetaData[SaferIndexed.idr.tex]{layerdata}

This function can easily be implemented by induction on the description
and repeatedly calling \IdrisFunction{poke} to expose the values one by
one.

\ExecuteMetaData[SaferIndexed.idr.tex]{layerfun}

\subsection{Exposing the top constructor}

Now that we can deserialise an entire layer of \IdrisFunction{Meaning},
the only thing we are missing to be able to generically manipulate trees
is the ability to expose the top constructor of a tree stored at a
\IdrisType{Pointer.Mu} position.

\[
\Hoare
    {\Pointer{\ell}{\text{\IdrisType{$\mu$}}\, d}{t}}
    {\text{\IdrisFunction{out}}\, \ell}
    {(k, \ell_1).\, \exists t_1.\, t = (k \,\text{\IdrisData{\#}}\, t_1)
        \ast \ell \mapsto t \ast \Pointer{\ell_1}{d}{t_1}}
\]


Remembering the data layout detailed in \Cref{fig:data-layout}, this will
amount to inspecting the tag used by the node and then deserialising the
offsets stored immeditately after it.

The \IdrisType{Out} family describes the typed point of view:
to get your hands on the index of a tree's constructor means
obtaining an \IdrisType{Index},
and a \IdrisType{Pointer.Meaning} to the constructor's payload
(remember that these high-level `pointers' store a vector of offsets).
%
The family's index (\IdrisBound{k} \IdrisData{\#} \IdrisBound{t})
ensures that the structure of the runtime irrelevant
tree is adequately described by
the index (\IdrisBound{k}) and
the \IdrisFunction{Data.Meaning} (\IdrisBound{t}) the \IdrisType{Pointer.Meaning} is for.

\ExecuteMetaData[SaferIndexed.idr.tex]{outdata}

The type of the \IdrisFunction{out} function is as expected: given a pointer
to a tree \IdrisBound{t} of type \IdrisBound{cs} we can get a value
of type (\IdrisType{Out} \IdrisBound{cs} \IdrisBound{t}).
%
That is to say, we can get a view allowing us to reveal what the
index of the tree's head constructor is.

\ExecuteMetaData[SaferIndexed.idr.tex]{outfun}

The implementation is fairly straightforward except for another
unsafe step meant to reconcile the information we read in the buffer
with the runtime-irrelevant tree index.

\ExecuteMetaData[SaferIndexed.idr.tex]{outfunbody}

We start by reading the tag \IdrisBound{k}
corresponding to the constructor choice:
we obtain a byte by calling \IdrisFunction{getBits8}, cast it to a
natural number and then make sure that it is in the range
$[0 \cdots \text{\IdrisFunction{consNumber} \IdrisBound{cs}}[$ using
\IdrisFunction{natToFin}.
%
We then use the unsafe \IdrisPostulate{unfoldAs} primitive to step the
type-level \IdrisBound{t} to something of the form
(\IdrisBound{k} \IdrisData{\#} \IdrisBound{val}).

\ExecuteMetaData[Serialised/Desc.idr.tex]{unfoldAs}

The declaration of \IdrisPostulate{unfoldAs} is marked as runtime
irrelevant because it cannot possibly be implemented
(\IdrisBound{t} is runtime irrelevant and so cannot be inspected)
and so its output should not be relied upon in runtime-relevant
computations.
%
Its type states that there exists a \IdrisFunction{Meaning} called
\IdrisBound{val} such that \IdrisBound{t} is equal to
(\IdrisBound{k} \IdrisData{\#} \IdrisBound{val})

Now that we know the head constructor we want to deserialise and that
we have the ability to step the runtime irrelevant tree to match the
actual content of the buffer, we can use \IdrisFunction{getConstructor}
to build such a value.

\ExecuteMetaData[SaferIndexed.idr.tex]{getConstructor}

To get a constructor, we start by getting the vector of offsets stored
immediately after the tag. We then compute the size of the remaining
\IdrisFunction{Meaning} description: it is the size of the overall tree,
minus $1$ (for the index) and $8$ times the number of offsets (because
each offset is stored as an 8 bytes number).
%
We can then use the record constructor \IdrisData{MkMeaning} to pack
together the vector of offsets, the buffer, the position past the offsets
and the size we just computed.

\ExecuteMetaData[SaferIndexed.idr.tex]{getOffsets}

The implementation of \IdrisType{getOffsets} is straightforward: given
a continuation that expect \IdrisBound{n} offsets as well as the
position past the last of these offsets, we read the 8-bytes-long
offsets one by one and pass them to the continuation, making sure
that we move the current position accordingly before every recursive call.

\subsection{Offering a convenient \IdrisType{View}}\label{sec:dataview}

We can combine \IdrisFunction{out} and \IdrisFunction{layer} to obtain
the \IdrisFunction{view} function we used in our introductory examples
in~\Cref{sec:seamless}.
%
A (\IdrisType{View} \IdrisBound{cs} \IdrisBound{t}) value gives us
access to the (\IdrisType{Index} \IdrisBound{cs}) of
\IdrisBound{t}'s top constructor together with the corresponding
\IdrisFunction{Layer} of deserialised values and pointers to subtrees.

\ExecuteMetaData[SaferIndexed.idr.tex]{viewdata}

The implementation of \IdrisFunction{view} is unsurprising: we use
\IdrisFunction{out} to expose the top constructor index and a
\IdrisType{Pointer.Meaning} to the constructor's payload.
%
We then user \IdrisFunction{layer} to extract the full
\IdrisFunction{Layer} of deserialised values that the
pointer references.

\ExecuteMetaData[SaferIndexed.idr.tex]{viewfun}
\ExecuteMetaData[SaferIndexed.idr.tex]{viewfunbody}

It is worth noting that although a \IdrisFunction{view} may be
convenient to consume, a performance-minded user may decide to
directly use the \IdrisFunction{out} and \IdrisFunction{poke}
combinators to avoid deserialising values that they do not need.
%
We present a case study in \Cref{appendix:rightmost} comparing the
access patterns of two implementations of the function fetching the
byte stored in a tree's rightmost node depending on whether we use
\IdrisFunction{view} or the lower level \IdrisFunction{poke} combinator.

By repeatedly calling \IdrisFunction{view}, we can define the
correct-by-construction generic deserialisation function that
turns a pointer to a tree into a runtime value equal to this tree.

\ExecuteMetaData[SaferIndexed.idr.tex]{deserialise}

We can measure the benefits of our approach by comparing the runtime
of a function directly operating on buffers to its pure counterpart
composed with a deserialisation step.
%
For functions like \IdrisFunction{rightmost} that only explore a
very small part of the full tree, the gains are spectacular: the
process operating on buffers is exponentially faster than its
counterpart which needs to deserialise the entire tree first
(cf. \Cref{sec:timing}).
