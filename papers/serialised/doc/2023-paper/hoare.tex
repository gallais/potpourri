\section{Analogy to Separation Logic}\label{appendix:hoare}

This is in no way an exhaustive introduction to separation logic.

\newcommand{\Pointer}[3]{\ensuremath{#1 \xmapsto{#2} #3}}
\newcommand{\Hoare}[4]{\ensuremath{\lbrace~ #1 ~\rbrace~ #2 ~\lbrace~ #3.\, #4 ~\rbrace}}
\newcommand{\Meaning}[2]{\ensuremath{\llbracket \,#1\, \rrbracket\, (\Mu{#2})}}
\newcommand{\Mu}[1]{\ensuremath{\mu \, \mathit{#1}}}

\subsection{Interlude: Separation Logic}

A Hoare triple~\cite{DBLP:journals/cacm/Hoare69} of the form
\[ \Hoare{P}{e}{v}{Q} \]
states that under the precondition
$P$, and binding the result of evaluating the expression $e$ as $v$,
we can prove that $Q$ holds.
%
One of the basic predicates of separation logic~\cite{DBLP:conf/lics/Reynolds02}
is a `points to' assertion (\Pointer{\ell}{}{t}) stating that
the label $\ell$ points to a memory location containing $t$.

A separation logic proof system then typically consists in defining
a language and providing axioms characterising the behaviour of
each language construct.
%
The simplest example involving memory is perhaps a language
with pointers to bytes and a single \texttt{deref} construct
dereferencing a pointer.
%
We can then give the following axiom
\[ \Hoare{\Pointer{\ell}{}{\mathit{bs}}}{\texttt{deref}\,\ell}{v}{\mathit{bs} = v \ast \Pointer{\ell}{}{\mathit{bs}}} \]
to characterise \texttt{deref} by stating that the value
it returns is precisely the one the pointer is referencing,
and that the pointer is still valid and still referencing
the same value after it has been dereferenced.

The axioms can be combined to prove statements about more
complex programs such as the following silly one for instance.
Here we state that if we dereference the pointer a first time,
discard the result and then dereference it once more then we
end up in the same situation as if we had dereferenced it only once.
\[ \Hoare{\Pointer{\ell}{}{\mathit{bs}}}{\texttt{deref}\,\ell;\, \texttt{deref}\,\ell}{v}{\mathit{bs} = v \ast \Pointer{\ell}{}{\mathit{bs}}} \]

Note that in all of these rules $\mathit{bs}$ is only present
in the specification layer. \texttt{deref} itself cannot possibly
return $\mathit{bs}$ directly, it needs to actually perform an
effectful operation that will read the memory cell's content.

\subsection{Characterising Our Library}

We are going to explain that we can see our library as a small
embedded Domain Specific
Language~\cite{DBLP:journals/csur/Hudak96}
that has \IdrisFunction{poke} and \IdrisFunction{out} as
sole language constructs.
%
Our main departure from separation logic is that we want to
program in a correct-by-construction fashion and so the types
of \IdrisFunction{poke} and \IdrisFunction{out} have to be
just as informative as the axioms we would postulate
in separation logic.

We will write \Pointer{\ell}{\Meaning{d}{cs}}{t} for the assumption
that we own a pointer $\ell$ of type
(\IdrisType{Pointer.Meaning} \IdrisBound{d} \IdrisBound{cs} \IdrisBound{t}),
and \Pointer{\ell}{\Mu{cs}}{t} for the assumption that we own
a pointer $\ell$ of type (\IdrisType{Pointer.Mu} \IdrisBound{cs} \IdrisBound{t}).
%
In case we do not care about the type of the pointer at hand
(e.g. because it can be easily inferred from the context), we
will simply write $\ell \mapsto t$.

the basic building blocks correspond to the axiomatisation of
the behaviour of the underlying language constructs (hence their
necessarily unsafe implementation in this shallow embedding)
while the high-level combinators are derived reasoning principles
allowing us to implemement more complex programs.

\subsubsection{Axioms for \IdrisFunction{poke}}


Thinking in terms of Hoare triples, if we have a pointer
$\ell$ to a term $t$ known to be a single byte then
(\IdrisFunction{poke} $\ell$)
will return a byte $bs$ and allow us to observe that $t$ is
equal to that byte.
\[
\Hoare
    {\Pointer{\ell}{\Meaning{\text{\IdrisData{Byte}}}{cs}}{t}}
    {\text{\IdrisFunction{poke}}\,\ell}
    {\mathit{bs}}
    {t = \mathit{bs} \ast \ell \mapsto t}
\]

Similarly, if the pointer $\ell$ is for a pair then
(\IdrisFunction{poke} $\ell$) will
reveal that the term $t$ can be taken apart into the pairing
of two terms $t_1$ and $t_2$ and
return a pointer for each of these components.
\begin{gather*}
\Hoare
    {\Pointer{\ell}{\Meaning{\text{\IdrisData{Prod}}\, d_1 d_2}{cs}}{t}}
    {\\\text{\IdrisFunction{poke}}\,\ell\\}
    {(\ell_1, \ell_2)}
    {\exists t_1.\, \exists t_2.\, t = (t_1, t_2)
        \ast \ell \mapsto t \ast \Pointer{\ell_1}{\Meaning{d_1}{cs}}{t_1} \ast \Pointer{\ell_2}{\Meaning{d_2}{cs}}{t_2}}
\end{gather*}

Last but not least, poking a pointer with the \IdrisData{Rec}
description will return another pointer for the same value but
at a different type.
\[
\Hoare
    {\Pointer{\ell}{\Meaning{\text{\IdrisData{Rec}}}{cs}}{t}}
    {\text{\IdrisFunction{poke}}\,\ell}
    {\ell_1}
    {\Pointer{\ell_1}{\Mu{cs}}{t} \ast \Pointer{\ell}{}{t}}
\]

\subsubsection{Example of a Derived Rule for \IdrisFunction{layer}}

Given that \IdrisFunction{layer} is defined in terms of \IdrisFunction{poke},
we do not need to postulate any axioms to characterise it and can insteads
prove lemmas.

Using the description of a node in our running example of binary trees,
\IdrisFunction{layer}'s behaviour would be characterised by the following
statement.
\begin{gather*}
\Hoare
    {\Pointer{\ell}{\Meaning{\text{\IdrisData{Prod} \IdrisData{Rec} (\IdrisData{Prod} \IdrisData{Byte} \IdrisData{Rec})}}{cs}}{t}}
    {\\\text{\IdrisFunction{layer}}\,\ell\\}
    {(\ell_1, bs, \ell_2)}
    {\exists t_1.\, \exists t_2.\,
      t = (t_1 \,\text{\IdrisData{\#}}\,  bs \,\text{\IdrisData{\#}}\, t_2)
      \ast \Pointer{\ell_1}{\Mu{cs}}{t_1}
      \ast \Pointer{\ell_2}{\Mu{cs}}{t_2}
      \ast \Pointer{\ell}{}{t}}
\end{gather*}
Provided a pointer to such a layer, calling \IdrisFunction{layer} would
return a triple of
a pointer $\ell_1$ for the left subtree,
the byte $\mathit{bs}$,
and a pointer $\ell_2$ for the right subtree.

\subsubsection{Axiom for \IdrisFunction{out}}

Things are a lot simpler for \IdrisFunction{out} as the return type
is not defined by induction on the description.
\[
\Hoare
    {\Pointer{\ell}{\Mu{cs}}{t}}
    {\text{\IdrisFunction{out}}\,\ell}
    {(k, \ell_1)}
    {\exists t_1.\, t = (k \,\text{\IdrisData{\#}}\, t_1)
      \ast \Pointer{\ell}{}{t} \ast \Pointer{\ell_1}{\Meaning{cs_k}{cs}}{t_1}}
\]
