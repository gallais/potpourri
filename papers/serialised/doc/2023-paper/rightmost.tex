\section{Access Patterns: Viewing vs. Poking}\label{appendix:rightmost}

In this example we implement \IdrisFunction{rightmost}, the function walking
down the rightmost branch of our type of binary trees and returning the
content of its rightmost node (if it exists).

The first implementation is the most straightforward: use \IdrisFunction{view}
to obtain the top constructor as well as an entire layer of deserialised values
and pointers to substructures and inspect the constructor.
%
If we have a leaf then there is no byte to return.
%
If we have a node then call \IdrisFunction{rightmost} recursively and inspect the
result: if we got \IdrisData{Nothing} back we are at the rightmost node and can
return the current byte, otherwise simply propagate the result.

\ExecuteMetaData[SaferIndexed.idr.tex]{viewrightmost}

In the alternative implementation we use \IdrisFunction{out} to expose the top
constructor and then, in the node case, call \IdrisFunction{poke} multiple times
to get our hands on the pointer to the right subtree.
%
We inspect the result of recursively calling \IdrisFunction{rightmost} on this
subtree and only deserialise the byte contained in the current node if the result
we get back is \IdrisData{Nothing}.

\ExecuteMetaData[SaferIndexed.idr.tex]{pokerightmost}

This will give rise to two different access patterns: the first function will
have deserialised all of the bytes stored in the nodes along the tree's
rightmost path whereas the second will only have deserialised the rightmost byte.
%
Admittedly deserialising a byte is not extremely expensive but in a more realistic
example we could have for instance been storing arbitrarily large values in
these nodes. In that case it may be worth trading convenience for making sure we
are not doing any unnecessary work.
