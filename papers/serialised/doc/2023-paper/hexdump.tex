\section{Serialised Representation}\label{sec:hexdump}

Before we can give a meaning to descriptions as pointers into a buffer we
need to decide on a serialisation format.
%
The format we have opted for is split in two parts: a header containing
data that can be used to check that a user's claim that a given file
contains a serialised tree of a given type is correct, followed by the
actual representation of the tree.


For instance, the following binary snippet is a hex dump of a file
containing the serialised representation of a binary tree belonging to
the type we have been using as our running example.
%
The raw data is semantically highlighted:
8-bytes-long \hexaoffset{offsets},
a \hexadesc{type} description of the stored data,
some \hexacons{nodes} of the tree
and the \hexadata{data} stored in the nodes.

\begin{center}

\begin{hexdump}
87654321\hphantom{:} 00 11 22 33 44 55 66 77 88 99 AA BB CC DD EE FF
00000000: \hexaoffset{07 00 00 00 00 00 00 00} \hexadesc{02 00 02 03 02 01 03} \hexacons{01}
00000010: \hexaoffset{17 00 00 00 00 00 00 00} \hexacons{01} \hexaoffset{0c 00 00 00 00 00 00}
00000020: \hexaoffset{00} \hexacons{01} \hexaoffset{01 00 00 00 00 00 00 00} \hexacons{00} \hexadata{01} \hexacons{00} \hexadata{05} \hexacons{00} \hexadata{0a}
00000030: \hexacons{01} \hexaoffset{01 00 00 00 00 00 00 00} \hexacons{00} \hexadata{14} \hexacons{00}
\end{hexdump}

\end{center}

More specifically, this block is the encoding of the \IdrisFunction{example}
given in the previous section and,
%
knowing that a \IdrisFunction{leaf} is represented here by \hexacons{00}
and a \IdrisFunction{node} is represented by \hexacons{01}
%
the careful reader can check
(modulo ignoring the type description and offsets for now)
that the data is stored in a depth-first, left-to-right traversal of the tree
(i.e. we get exactly the bit pattern we saw in the naïve encoding
presented in \Cref{sec:intro}).


\subsection{Header}

In our example, the header is as follows:
\begin{hexdump}
\hexaoffset{07 00 00 00 00 00 00 00} \hexadesc{02 00 02 03 02 01 03}
\end{hexdump}

The header consists of an offset allowing us to jump past it in case we do
not care to inspect it, followed by a binary representation of the
\IdrisType{Data} description of the value stored in the buffer.
%
This can be useful in a big project where different components produce
and consume such serialised values: if we change the format in one place
but forget to update it in another, we want the program to gracefully
fail to load the file using an unexpected format.
%
We detail in \Cref{sec:limitation-robust} how dependent type providers
can help structure a software project to prevent such issues.

The encoding of a data description starts with a byte giving us the number
of constructors, followed by these constructors' respective descriptions
serialised one after the other.
%
\IdrisData{None} is represented by \hexadesc{00},
\IdrisData{Byte} is represented by \hexadesc{01},
(\IdrisData{Prod} \IdrisBound{d} \IdrisBound{e}) is represented by
\hexadesc{02} followed by the representation of \IdrisBound{d} and then that of \IdrisBound{e},
and \IdrisData{Rec} is represented by \hexadesc{03}.


Looking once more at the header in the running example,
the \IdrisType{Data} description is indeed 7 bytes long like the offset states.
The \IdrisType{Data} description starts with \hexadesc{02}
meaning that the type has two constructors.
The first one is \hexadesc{00} i.e. \IdrisData{None}
(this is the encoding of the type of \IdrisFunction{Leaf}),
and the second one is \hexadesc{02 03 02 01 03} i.e.
\IdrisKeyword{(}\IdrisData{Prod} \IdrisData{Rec}
\IdrisKeyword{(}\IdrisData{Prod} \IdrisData{Byte} \IdrisData{Rec}\IdrisKeyword{))}
(that is to say the encoding of the type of \IdrisFunction{Node}).
%
According to the header, this file does contain a \IdrisFunction{Tree}.

\subsection{Tree Serialisation}\label{sec:tree-serialisation}

Our main focus in the definition of this format is that we should be able
to process any of a node's subtrees without having to first traverse the
subtrees that come before it.
%
This will allow us to, for instance, implement a function looking up the
value stored in the rightmost node in our running example type of binary
trees in time linear in the depth of the tree rather than exponential.
%
To this end each node needs to store an offset measuring the size of the
subtrees that are to the left of any relevant information.

If a given tag is associated to a description of type
(\IdrisType{Desc} \IdrisData{True} \IdrisBound{s} \IdrisBound{o})
then the representation in memory of the associated node will look something
like the following.

\label{fig:data-layout}
\begin{center}
\begin{bytefield}[bitwidth=6mm, bitheight=7mm]{4}
  \bitbox{2}[bgcolor=Chartreuse4!40]{tag}
  & \bitbox{4}[bgcolor=lightgray!30]{$o$ offsets}
  & \bitbox[ltb]{2}{tree$_1$}
  & \bitbox[tb]{1}{$\cdots$}
  & \bitbox[tb]{2}[bgcolor=IndianRed1!40]{byte$_1$}
  & \bitbox[tb]{1}{$\cdots$}
  & \bitbox[tb]{2}{tree$_k$}
  & \bitbox[tb]{1}{$\cdots$}
  & \bitbox[tb]{2}[bgcolor=IndianRed1!40]{byte$_s$}
  & \bitbox[tbr]{2}{tree$_{o+1}$} \\
  \bitbox[l]{1}{\small$0$}
  & \bitbox[]{1}{}
  & \bitbox[l]{1}{\small$1$}
  & \bitbox[]{3}{}
  & \bitbox[l]{3}{\small$1+8\times{}o$}
  & \bitbox[]{6}{}
  & \bitbox[l]{5}{\small$8\times{}o+s+\Sigma_{i=1}^{o}o_i$}
\end{bytefield}

\end{center}

On the first line we have a description of the data layout and on the
second line we have the offset of various positions in the block with
respect to the tag's address.

For the data layout,
we start with the tag
then we have $o$ offsets,
and finally we have a block contiguously storing an interleaving of
subtrees and $s$ bytes
dictated by the description.
%
In this example the rightmost value in the description is a subtree and
so even though we have $o$ offsets, we actually have $(o+1)$ subtrees stored.

The offsets of the tag with respect to its own address is $0$.
The tag occupies one byte and so the offset of the block of offsets is $1$.
Each offset occupies 8 bytes and so the constructor's arguments are stored at offset $(1+8*o)$.
Finally each value's offset can be computed by adding up
the offset of the start of the block of constructor arguments,
the offsets corresponding to all of the subtrees that come before it,
and the number of bytes stored before it;
in the case of the last byte that gives $1+8*o + \Sigma_{i=1}^{o}o_i + s-1$
hence the formula included in the diagram.

Going back to our running example, this translates to the following
respective data layouts and offsets for a leaf and a node.

\begin{center}
  \begin{bytefield}[bitwidth=6mm, bitheight=7mm]{4}
  \bitbox[]{2}{Leaf} \\
  \bitbox{2}[bgcolor=Chartreuse4!40]{00} \\
  \bitbox[l]{1}{\small$0$}
\end{bytefield}\hfill
\begin{bytefield}[bitwidth=.05\linewidth, bitheight=7mm]{4}
  \bitbox[]{2}{Node} \\
  \bitbox{2}[bgcolor=Chartreuse4!40]{01}
  & \bitbox{4}[bgcolor=lightgray!30]{offset}
  & \bitbox{4}{left subtree}
  & \bitbox{2}[bgcolor=IndianRed1!40]{byte}
  & \bitbox{4}{right subtree} \\
  \bitbox[l]{1}{\small$0$}
  & \bitbox[]{1}{}
  & \bitbox[l]{1}{\small$1$}
  & \bitbox[]{3}{}
  & \bitbox[l]{1}{\small$9$}
  & \bitbox[]{3}{}
  & \bitbox[l]{2}{\small$9+o_1$\hfill}
  & \bitbox[l]{2}{\small$10+o_1$}
\end{bytefield}

\end{center}

Now that we understand the format we want, we ought to be able to implement pointers
and the functions manipulating them.
