\section{Serialising Data}\label{sec:serialising}

So far all of our example programs involved taking an inductive value
apart and computing a return value in the host language.
%
But we may instead want to compute another value in serialised form.
%
We include below one such example: a \IdrisFunction{map} function which
takes a function \IdrisBound{f} acting on bytes
and applies it to all of the ones stored
in the nodes of our type of \IdrisFunction{Tree}s.

\ExecuteMetaData[SaferIndexed.idr.tex]{serialisedmap}\label{fig:serialised-map}

It calls the \IdrisFunction{view} we just defined to observe whether
the tree is a leaf or a node.
If it's a leaf, it returns a leaf.
If it's a node, it returns a node where the map has been recursively applied
to the left and right subtrees while the function \IdrisBound{f} has been applied
to the byte \IdrisBound{b}.

%% \subsection{Generic Traverse}
%% \ExecuteMetaData[SaferIndexed.idr.tex]{traverse}

In this section we are going to spell out how we can define high-level
constructs allowing users to write these correct-by-construction
serialisers.

\subsection{The Type of Serialisation Processes}

A serialisation process for a tree \IdrisBound{t} that belongs to the
datatype \IdrisBound{cs} is a function that takes a buffer
and a starting offset and returns an \IdrisType{IO} process that
serialises the term in the buffer at that offset and computes the
offset of the first byte past the serialised tree.

\ExecuteMetaData[SaferIndexed.idr.tex]{serialising}

We do not expect users to define such processes by hand and in fact
prevent them from doing so by not exporting the \IdrisData{MkSerialising}
constructor.
%
Instead, we provide high-level, invariant-respecting combinators to safely
construct such serialisation processes.

\subsection{Building Serialisation Processes}

Our main combinator is \IdrisFunction{(\#)}: by providing
a node's constructor index
and a way to serialise all of the node's subtrees,
we obtain a serialisation process for said node.
%
We will give a detailed explanation of \IdrisFunction{All} below.

\ExecuteMetaData[SaferIndexed.idr.tex]{serialisemu}

The keen reader may refer to the accompanying code to see the implementation.
Informally (cf. \Cref{sec:tree-serialisation} for the description of the format):
first we write the tag corresponding to the choice of constructor,
then we leave some space for the offsets,
in the meantime we write all of the constructor's arguments and collect the offsets
associated to each subtree while doing so,
and finally we fill in the space we had left blank with the offsets
we have thus collected.

The \IdrisFunction{All} quantifier performs the pointwise lifting of a predicate over
the functor described by a \IdrisType{Desc}. It is defined by induction over
the description.

\ExecuteMetaData[SaferIndexed.idr.tex]{allquant}

If the description is \IdrisData{None} then there is nothing to apply
the predicate to and so we return the unit type.

If the description is \IdrisData{Byte} we only demand that we have a runtime
copy of the byte so that we may write it inside a buffer. This is done using
the \IdrisType{Singleton} family discussed in \Cref{sec:datasingleton}.

If the description is a the \IdrisData{Prod} of two subdescriptions, we once
again use an auxiliary family purely for ergonomics. It is defined mutually
with \IdrisFunction{All} and does the expected structural operation.

\ExecuteMetaData[SaferIndexed.idr.tex]{dataallquant}

Finally, if the description is \IdrisData{Rec} then we demand that the
predicate holds.

It should now be clear that
(\IdrisFunction{All} \IdrisKeyword{(}\IdrisFunction{description} \IdrisBound{k}\IdrisKeyword{)}
(\IdrisType{Serialising} \IdrisBound{cs}\IdrisKeyword{)}) indeed corresponded
to having already defined a serialisation process for each subtree.

This very general combinator should be enough to define all the
serialisers we may ever want.
%
By repeatedly pattern-matching on the input tree and using \IdrisFunction{(\#)},
we can for instance define the correct-by-construction generic serialisation function.

\ExecuteMetaData[SaferIndexed.idr.tex]{serialise}

We nonetheless include other combinators purely for performance reasons.

\subsection{Copying Entire Trees}

We introduce a \IdrisFunction{copy} combinator for trees that we want to
serialise as-is and have a pointer for.
%
Equipped with this combinator, we are able to easily write e.g.
the \IdrisFunction{swap} function which takes a binary tree apart
and swaps its left and right branches (if the tree is non-empty).

\ExecuteMetaData[SaferIndexed.idr.tex]{swap}

We could define this \IdrisFunction{copy} combinator at a high level
either by composing
\IdrisFunction{deserialise} and \IdrisFunction{serialise},
or by interleaving calls to \IdrisFunction{view} and \IdrisFunction{(\#)}.
%
This would however lead to a slow implementation that needs to
traverse the entire tree in order to simply copy it.

Instead, we implement \IdrisFunction{copy} by using the
\IdrisFunction{copyData} primitive for \IdrisType{Buffer}s
present in \idris{}'s standard library.
%
This primitive allows us to grab a slice of the source buffer
corresponding to the tree
and to copy the raw bytes directly into the target buffer.

\ExecuteMetaData[SaferIndexed.idr.tex]{copy}

This is the one combinator that crucially relies
on our format only using offsets and not absolute addresses
and on the accuracy of the size information we have been keeping
in \IdrisType{Pointer.Mu} and \IdrisType{Pointer.Meaning}.
%
As we can see in \Cref{sec:timing}, this is spectacularly faster than
a deep copying process traversing the tree.

\subsection{Executing a Serialisation Action}

Now that we can describe actions serialising a value to a buffer,
the last basic building block we are still missing is a function actually
performning such actions.
%
This is provided by the \IdrisFunction{execSerialising} function
declared below.

\ExecuteMetaData[SaferIndexed.idr.tex]{execSerialising}

By executing a (\IdrisType{Serialising} \IdrisBound{cs} \IdrisBound{t}), we
obtain an \IdrisType{IO} process returning a pointer to the tree \IdrisBound{t}
stored in a buffer.
%
We can then either compute further with this tree (e.g. by calling
\IdrisFunction{sum} on it), or write it to a file for safekeeping
using the function \IdrisFunction{writeToFile}
introduced in \Cref{sec:writetofile}.

\subsection{Evaluation Order}

The careful reader may have noticed that we can and do run arbitrary \IdrisType{IO}
operations when building a value of type \IdrisType{Serialising}
(cf. the \IdrisFunction{map} example in \Cref{fig:serialised-map} where we perform
a call to \IdrisFunction{view} to inspect the input's shape).

This is possible thanks to \idris{} elaborating \IdrisKeyword{do}-blocks using
whichever appropriate bind operator is in scope. In particular, we have defined
the following one to use when building a serialisation process:

\ExecuteMetaData[SaferIndexed.idr.tex]{serialisingbind}

By using this bind we can temporarily pause writing to the buffer to make
arbitrary \IdrisType{IO} requests to the outside world.
%
In particular, this allows us to interleave reading from the original buffer
and writing into the target one thus having a much better memory footprint than
if we were first using the \IdrisType{IO} monad to build in one go the whole
serialisation process for a given tree and then executing it.
