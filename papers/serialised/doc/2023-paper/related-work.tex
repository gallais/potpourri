\subsection{Related Work}

This work sits at the intersection of many domains:
data-generic programming,
the efficient runtime representation of functional data,
programming over serialised values,
and the design of serialisation formats.
%
Correspondingly, a lot of related work is worth discussing.
In many cases the advantage of our approach is precisely that it is
at the intersection of all of these strands of research.

\subsubsection{Data-Generic Programming}

There is a long tradition of data-generic
programming~\cite{DBLP:conf/ssdgp/Gibbons06} and we will mostly focus here
on the approach based on the careful reification of a precise universe of
discourse as an inductive family in a host type theory,
and the definition of generic programs by induction over this family.

One early such instance is Pfeifer and Rue{\ss}'
`polytypic proof construction'~\cite{DBLP:conf/tphol/PfeiferR99}
meant to replace unsafe meta-programs deriving recursors
(be they built-in support, or user-written tactics).

\todo{See refs. in n-ary polymorphic functions}

In his PhD thesis, Morris~\cite{DBLP:phd/ethos/Morris07} declares various
universes for strictly positive types and families and defines by generic
programming
further types (the type of one-hole contexts),
modalities (the universal and existential predicate lifting over the functors he considers),
and functions (map, boolean equality).


\todo{\cite{DBLP:conf/icfp/LohM11}}

\subsubsection{Efficient Runtime Representation of Inductive Values}

Although not dealing explicitly with programming over serialised data,
Monnier's work~\cite{DBLP:conf/icfp/Monnier19} with its focus on performance and
in particular on the layout of inductive values at runtime,
partially motivated our endeavour.
%
Provided that we find a way to get the specialisation and partial evaluation
of the generically defined views, we ought to be able to achieve --purely in
user code-- Monnier's vision of a representation where n-ary tuples have
constant-time access to each of their component.

\subsubsection{Working on Serialised Data}

LoCal~\cite{DBLP:conf/pldi/VollmerKRS0N19} is the work that originally
motivated the design of this library.
%
We have demonstrated that generic programming within a dependently typed
language can yield the sort of benefits other language can only achieve
by inventing entirely new intermediate languages and compilation schemes.

LoCal was improved upon with a re-thinking of the serialisation scheme
making the approach compatible with parallel
programming~\cite{DBLP:journals/pacmpl/KoparkarRVKN21}.
This impressive improvement is a natural candidate for future work on our
part: the authors demonstrate it is possible to reap the benefits of
both programming over serialised data
and dividing up the work over multiple processors
with almost no additional cost in the case of a purely sequential execution.

\subsubsection{Serialisation Formats}

The PADS project~\cite{DBLP:conf/popl/MandelbaumFWFG07} aims to let users
quickly, correctly, and compositionally describe existing formats they
have no control over.
%
As they reminds us, ad-hoc serialisation formats abound be it in
networking, logging, billing, or as output of measurement equipments
in e.g. gene sequencing or molecular biology.
%
Our current project is not offering this kind of versatility as we have
decided to focus on a specific serialisation format with strong
guarantees about the efficient access to subtrees.
%
But our approach to defining correct-by-construction components could
be leveraged in that setting too and bring users strong guarantees about
the traversals they write.

ASN.1~\cite{MANUAL:book/larmouth1999} gives users the ability to define
a high-level specification of the exchange format (the `abstract syntax')
to be used in communications without the need to concern themselves with
the actual encoding as bit patterns (the `transfer syntax').
%
This separation between specification and implementation means that parsing
and encoding can be defined once and for all by generic programming
(here, a compiler turning specifications into code in the user's host
language of choice).
%
The main difference is once again our ability to program in a
correct-by-construction manner over the values thus represented.


Yallop's automatic derivation of serialisers using an OCaml
preprocessor~\cite{DBLP:conf/ml/Yallop07} highlights the importance
of empowering domain experts to take advantage of the specifics of
the problem they are solving to minimise the size of the encoded data.
%
By detecting sharing using a custom equality function respecting
$\alpha$-equivalence instead of the default one, he was able to
serialise large lambda terms using only
a quarter of the bytes OCaml's standard library marshaller.

\todo{protobuf}
