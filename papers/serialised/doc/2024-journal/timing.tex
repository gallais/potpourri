
\newcommand{\graphOf}[1]{\begin{tikzpicture}
  \begin{axis}
    [ xlabel = depth
    , ylabel = #1 (ns)
    , ymode = log
    , legend pos = north west
    ]
    \addplot table
      [ grid = major
      , x = size
      , y = serialised
      , col sep = comma
      ] {#1.csv};
    \addplot table
      [ grid = major
      , x = size
      , y = deserialised
      , col sep = comma
      ] {#1.csv};
  \legend{pointer, data}
  \end{axis}
\end{tikzpicture}}


\section{Benchmarks}\label{sec:timing}

Now that we have the ability to read, write, and program directly
over trees stored in a buffer we can run some experiments to see
whether this allows us to gain anything over the purely functional
programming style.

For all of these tests we generate full trees for each depth,
labeling its nodes with the bytes 0, 1, etc. in a depth-first
left-to-right manner.

Each test is run 25 times in a row, and the duration averaged.
%
We manually run chezscheme's garbage collector before the start of
each time measurement.

\todo{Describe laptop characteristics}

\subsection{Pointer vs. Data}

We compare the time it takes to run the composition of deserialising
the tree and applying the pure function to the time it takes to run
its pointer-based counterpart.
%
All of our plots use a logarithmic y axis because the runtime of the
deserialisation-based function is necessarily exponential in the depth
of the full tree.

The \IdrisFunction{sum} function explores the entirety of the tree
and as such the difference between the deserialisation-based and the
pointer-based functions is minimal.

\noindent
\begin{minipage}{.5\textwidth}
  \ExecuteMetaData[SaferIndexed.idr.tex]{tsum}
\end{minipage}\hfill
\begin{minipage}{.45\textwidth}
  \graphOf{pointer}{data}{sum}
\end{minipage}

The \IdrisFunction{rightmost} function only explores the rightmost
branch of the tree and we correspondingly see an exponential speedup
for the pointer-based function which is able to efficiently skip past
every left subtree.

\noindent
\begin{minipage}{.5\textwidth}
  \ExecuteMetaData[SaferIndexed.idr.tex]{rightmost}
\end{minipage}\hfill
\begin{minipage}{.45\textwidth}
  \graphOf{pointer}{data}{rightmost}
\end{minipage}

The \IdrisFunction{find} function looks for a given byte and stops
as soon as it has found it. We run it using the byte 120.
%
Due to the way our test trees are generated, we will need to explore
at most 121 of the tree's nodes.
%
Unsurprisingly we observe that the pointer-based function is
constant time while the one operating over data is exponential
due to the deserialisation step.

\noindent
\begin{minipage}{.5\textwidth}
  \ExecuteMetaData[SaferIndexed.idr.tex]{tfind}
\end{minipage}\hfill
\begin{minipage}{.45\textwidth}
  \graphOf{pointer}{data}{find}
\end{minipage}

%% The deep \IdrisFunction{copy} is unsurprisingly also exponential in
%% the depth of the tree being copied whereas the version based on the
%% \IdrisFunction{copyData} primitive for buffers is vastly faster.

%% \begin{center}
%% \begin{minipage}{.5\textwidth}
%% \graphOf{primitive}{deep}{copy}
%% \end{minipage}
%% \end{center}

\subsection{Serialising}

\noindent
\begin{minipage}{.5\textwidth}
  \ExecuteMetaData[SaferIndexed.idr.tex]{swap}
\end{minipage}\hfill
\begin{minipage}{.45\textwidth}
  \graphOf{primitive}{deep}{swap}
\end{minipage}
