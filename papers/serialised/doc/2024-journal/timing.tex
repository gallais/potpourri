
\newcommand{\graphOf}[1]{\begin{tikzpicture}
  \begin{axis}
    [ xlabel = depth
    , ylabel = #1 (ns)
    , ymode = log
    , legend pos = north west
    ]
    \addplot table
      [ grid = major
      , x = size
      , y = serialised
      , col sep = comma
      ] {#1.csv};
    \addplot table
      [ grid = major
      , x = size
      , y = deserialised
      , col sep = comma
      ] {#1.csv};
  \legend{pointer, data}
  \end{axis}
\end{tikzpicture}}


\section{Benchmarks}\label{sec:timing}

Now that we have the ability to read, write, and program directly
over trees stored in a buffer we can run some experiments to see
whether this allows us to gain anything over the purely functional
programming style.

For all of these tests we generate full trees for each depth,
labeling its nodes with the bytes 0, 1, etc. in a depth-first
left-to-right manner.

Each test is run 25 times in a row, and the duration averaged.
%
We manually run chezscheme's garbage collector before the start of
each time measurement.

All of our plots use a logarithmic y axis because the runtime of the
deserialisation-based functions is necessarily exponential in the depth
of the full tree.

%\todo{Describe laptop characteristics?}

\subsection{Pointer vs. Data}

We will first look at the runtime characteristics of various functions
implemented using the views defined in \Cref{sec:poking}.
%
We compare the time it takes to run the composition of deserialising
the tree and applying the pure function to the time it takes to run
its pointer-based counterpart.

\subsubsection{Traversing the Full Tree}

The \IdrisFunction{sum} function visits all of the tree's nodes to
add up all of the bytes that are stored.

\smallskip\noindent
\begin{minipage}{.5\textwidth}
  \ExecuteMetaData[SaferIndexed.idr.tex]{tsum}
\end{minipage}\hfill
\begin{minipage}{.45\textwidth}
  \graphOf{data}{pointer}{sum}
\end{minipage}\smallskip

Because this function explores the entirety of the tree, the
difference between the deserialisation-based and the pointer-based
functions is minimal.

\subsubsection{Skipping Most of the Tree}

The \IdrisFunction{rightmost} function goes down the tree's right hand
and returns the byte stored in its rightmost node.

\smallskip\noindent
\begin{minipage}{.5\textwidth}
  \ExecuteMetaData[SaferIndexed.idr.tex]{rightmost}
\end{minipage}\hfill
\begin{minipage}{.45\textwidth}
  \graphOf{data}{pointer}{rightmost}
\end{minipage}\smallskip

Because this function only traverses the rightmost branch of the tree,
the pointer based implementation is able to run in linear time by
efficiently skipping past every left subtree.
%
This is effectively an exponential speedup compared to the implementation
that first fully deserialises the tree.

\subsubsection{Exploring a Bounded Fragment of the Tree}

%% \smallskip\noindent
%% \begin{minipage}{.5\textwidth}
%%   \ExecuteMetaData[SaferIndexed.idr.tex]{tfindB}
%% \end{minipage}\hfill
%% \begin{minipage}{.45\textwidth}
%%   \graphOf{data}{pointer}{findB}
%% \end{minipage}

Let us for a change look at a function returning a richly typed result.
%
The \IdrisFunction{find} function looks for a given byte in a tree
and returns a path to it if it can.
%
The notion of \IdrisType{Path} is defined inductively: if the
\IdrisBound{tgt} byte we are looking for is in the node then
we can use \IdrisData{Here}; otherwise we can take a turn down
the left or right subtree using \IdrisData{TurnL} and \IdrisData{TurnR}
respectively and provide a path to the target byte in the corresponding
subtree.

\ExecuteMetaData[SaferIndexed.idr.tex]{path}

To save space we only present a semi-decision procedure but this could
be generalised to a full decision procedure.
%
Given that \IdrisType{Path} is a very informative type, in this test
case the version manipulating pointers does not bother using the
\IdrisType{Singleton} family.

\smallskip\noindent
\begin{minipage}{.5\textwidth}
  \ExecuteMetaData[SaferIndexed.idr.tex]{tfind}
\end{minipage}\hfill
\begin{minipage}{.45\textwidth}
  \graphOf{data}{pointer}{find}
\end{minipage}\smallskip

We run the test using the byte 120.
%
Due to the way our test trees are generated, we will only need to
explore at most 121 of the tree's nodes before finding this byte.
%
Unsurprisingly we observe that the pointer-based function is
constant time while the one operating over data is exponential
due to the deserialisation step.

%% The deep \IdrisFunction{copy} is unsurprisingly also exponential in
%% the depth of the tree being copied whereas the version based on the
%% \IdrisFunction{copyData} primitive for buffers is vastly faster.

%% \begin{center}
%% \begin{minipage}{.5\textwidth}
%% \graphOf{primitive}{deep}{copy}
%% \end{minipage}
%% \end{center}

\subsection{Serialising}

Let us now turn to the time characteristics of the serialisation primitives
defined in \Cref{sec:serialising}.
%
We compare the time it takes to run the composition of deserialising
the tree, applying the pure function, and serialising the result
to the time it takes to run its pointer-based counterpart using the
\IdrisType{Serialising} framework.

\subsubsection{Traversing the Full Tree}

The \IdrisFunction{map} function applies a byte-to-byte function
to all of the bytes stored in a tree's node.
%
We run it using the \IdrisKeyword{(}\IdrisFunction{+} \IdrisData{100}\IdrisKeyword{)}
function.

\noindent
\begin{minipage}{.5\textwidth}
  \ExecuteMetaData[SaferIndexed.idr.tex]{serialisedmapcompact}
\end{minipage}\hfill
\begin{minipage}{.45\textwidth}
  \graphOf{data}{pointer}{map}
\end{minipage}

We can see that both approaches yield a similar runtime: everything is
dominated by traversing the whole tree and building the resulting output.
%
This is where a framework allowing for destructive updates may help
write faster functions.
%
This also has untapped opportunities for parallelism.

\subsubsection{Using the Copy Primitive}

The \IdrisFunction{swap} function takes a tree and, if it is non-empty,
swaps its left and right subtrees.
%
For this test case we compare an implementation using the
\emph{primitive} \IdrisFunction{copy} operator we introduced in \Cref{sec:copy},
one using a \emph{pointer}-based copy that interleaves exposing the head
constructor and serialising it,
and a copy going via the \emph{data} representation by fully deserialising
the tree before re-serialising it.

\noindent
\begin{minipage}{.5\textwidth}
  \ExecuteMetaData[SaferIndexed.idr.tex]{swap}
\end{minipage}\hfill
\begin{minipage}{.45\textwidth}
  \graphOfThree{data}{pointer}{primitive}{swap}
\end{minipage}

We can see that, the \emph{pointer} and \emph{data} variants yield a similar
runtime as they need to traverse the full tree whereas the primitive
one based on copying raw buffer slices is significantly faster.
