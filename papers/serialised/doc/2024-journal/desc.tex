\section{Our Universe of Descriptions}\label{sec:desc}

We first need to pin down the domain of our discourse.
%
To talk generically about an entire class of datatypes
without needing to modify the host language
we have decided to perform a universe
construction~\citep{DBLP:journals/njc/BenkeDJ03, DBLP:phd/ethos/Morris07, DBLP:conf/icfp/LohM11}.
%
That is to say that we are going to introduce an inductive type
defining a set of codes together
with an interpretation of these codes as bona fide
host-language types.
%
We will then be able to program generically over the universe of
datatypes by performing induction on the type of
codes~\citep{DBLP:conf/tphol/PfeiferR99}.

The universe we define is in the tradition of
a sums-of-products vision of inductive types~\citep{DBLP:conf/popl/JanssonJ97}
where the data description records additional information about
the static and dynamic size of the data being stored.
%
In our setting, constructors are essentially arbitrarily nested tuples of
values of type unit,
bytes,
and recursive substructures.
%
A datatype is given by listing a choice of constructors.

\subsection{Descriptions}

We start with these constructor descriptions;
they are represented internally by an inductive family \IdrisType{Desc}
declared below.

\ExecuteMetaData[Serialised/Desc.idr.tex]{desctype}

This family has three indices corresponding to three crucial
invariants being tracked.
%
First, an index telling us whether the current description
is being used in the \IdrisBound{rightmost} branch of the overall
constructor description.
%
Second, the \IdrisBound{static}ally known size of the described data
in the number of bytes it occupies.
%
Third, the number of \IdrisBound{offsets} that need to be stored to
compensate for subterms not having a statically known size.
%
The reader should think of \IdrisBound{rightmost} as an `input' index
(the context in which the family is used tells it whether it is in
a rightmost branch)
whereas \IdrisBound{static} and \IdrisBound{offsets} are `output' indices
(the family's own constructors each determine their respective sizes).

Next we define the family proper by giving its four constructors.

\ExecuteMetaData[Serialised/Desc.idr.tex]{desc}

Each constructor can be used anywhere in a description so their return
\IdrisBound{rightmost} index can be an arbitrary boolean.

\IdrisData{None} is the description of values of type unit. The static
size of these values is zero as no data is stored in a value of type unit.
Similarly, they do not require an offset to be stored as we statically
know their size.

\IdrisData{Byte} is the description of bytes.
%
Their static size is precisely one byte, and they do not require an
offset to be stored either.

\IdrisData{Prod} gives us the ability to pair two descriptions together.
Its static size and the number of offsets are the respective sums of the
static sizes and numbers of offsets of each subdescription.
%
The description of the left element of the pair will never be in the
rightmost branch of the overall constructors description and so its
index is \IdrisData{False} while the description of the right element
of the pair is in the rightmost branch precisely whenever the whole pair
is; hence the propagation of the \IdrisBound{r} arbitrary value from the
return index into the description of the right component.

Last but not least, \IdrisData{Rec} is a position for a subtree.
We cannot know its size in bytes statically and so we decide to store
an offset unless we are in the rightmost branch of the overall description.
%
Indeed, there are no additional constructor arguments behind the rightmost
one and so we have no reason to skip past the subterm. Consequently we
do not bother recording an offset for it.


\subsection{Constructors}

We represent a constructor as a record packing together
a name for the constructor,
the description of its arguments (which is, by virtue of
being used at the toplevel, in rightmost position),
and the values of the \IdrisFunction{static} and
\IdrisFunction{offsets} invariants.

\ExecuteMetaData[Serialised/Desc.idr.tex]{constructor}

\begin{remark}[Implicit Record Fields]
  Record fields whose type declaration is surrounded by
  curly braces are implicit: they do not need to be explicitly
  mentioned when constructing the record, or when pattern-matching
  on it.
\end{remark}

Here the two invariants \IdrisFunction{static} and
\IdrisFunction{offsets} are stored as implicit fields
because their value is easily reconstructed using
unification~\citep{DBLP:conf/tlca/AbelP11}.
%
Note that we used \IdrisData{(::)} as the name of the
constructor for records of type \IdrisType{Constructor}.
This allows us to define constructors by forming an
expression reminiscent of Haskell's type declarations:
\IdrisBound{name} \IdrisData{::} \IdrisBound{type}.
%
Returning to our running example, this gives us the following encodings for
leaves that do not store anything
and nodes that contain a left branch, a byte, and a right branch.

\noindent
\begin{minipage}[t]{.38\textwidth}
  \ExecuteMetaData[Serialised/Desc.idr.tex]{treeleaf}
\end{minipage}\hfill
\begin{minipage}[t]{.58\textwidth}
  \ExecuteMetaData[Serialised/Desc.idr.tex]{treenode}
\end{minipage}

\subsection{Datatypes}

A datatype description is given by a vector (also known as
a length-indexed list) of constructors.
%
For convenience, we store the length separately so that it
does not need to be recomputed every time it is needed.

\ExecuteMetaData[Serialised/Desc.idr.tex]{data}

We can then encode our running example as a simple \IdrisType{Data}
declaration: a binary tree whose node stores bytes is described by the choice
of either a \IdrisFunction{Leaf} or \IdrisFunction{Node}, as defined above.

\ExecuteMetaData[Serialised/Desc.idr.tex]{treedesc}

Now that we have a language that allows us to give a description of our
inductive types, we are going to give these descriptions a meaning as trees.
