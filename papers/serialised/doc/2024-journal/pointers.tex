\section{Meaning as Pointers into a Buffer}\label{sec:pointers}

For reasons that will become apparent in \Cref{sec:bufferfold}
when we start programming over serialised data in a correct-by-construction
manner, our types of `pointers' will be parameterised not only
by the description of the type of the data stored but also by a
runtime-irrelevant inductive value of that type.

Note that we will \emph{not} export the constructors to the various
`pointer' types defined in the following section.
%
Consequently, the only way for a user to get their hands on such a
pointer is to use the library functions we provide.
%
By only providing invariant-respecting functions, we can ensure
that the assumptions encoded in the phantom parameters are never
violated.

\subsection{Tracking Buffer Positions}

We start with the definition of \IdrisType{Pointer.Mu},
the counterpart to \IdrisType{Data.Mu} for serialised values.

\ExecuteMetaData[SaferIndexed.idr.tex]{pointermu}

A tree sitting in a buffer is represented
by a record packing the buffer, the position at which the tree's
root node is stored, and the size of the tree.
%
The record is indexed by \IdrisBound{cs} a \IdrisType{Data} description
and \IdrisBound{t} the tree of type (\IdrisType{Data.Mu} \IdrisBound{cs})
which is represented by the buffer's content.
Neither are mentioned in the types of the record's fields, making them
\emph{phantom types}~\citep{DBLP:conf/dsl/LeijenM99}.
%
A term of type (\IdrisType{Pointer.Mu} \IdrisBound{cs} \IdrisBound{t})
plays a double role in our library: it acts both as a label $\ell$ in
the runtime relevant layer, and a proof that said label points to a
buffer-bound value equal to $t$
(i.e. \Pointer{\ell}{\text{\IdrisType{Data.Mu} \IdrisBound{cs}}}{t}).


%% Note that according to our serialisation format the size is not stored
%% in the file but using the size of the buffer, the stored offsets,
%% and the size of the static data we will always
%% be able to compute a value corresponding to it.

The pointer counterpart to a \IdrisFunction{Meaning} stores
additional information.

\ExecuteMetaData[SaferIndexed.idr.tex]{elem}

For a description of type (\IdrisType{Desc} \IdrisBound{r} \IdrisBound{s} \IdrisBound{o})
on top of the buffer, the position at which the root of the meaning resides,
and the size of the layer we additionally have a vector of \IdrisBound{o} offsets
corresponding to the sizes of the subtrees we may want to skip past.
This will allow us to efficiently access any constructor argument we want.

\subsection{Writing a Tree to a File}\label{sec:writetofile}

Once we have a pointer to a tree \IdrisBound{t} of type \IdrisBound{cs}
(\IdrisType{Pointer.Mu} \IdrisBound{cs} \IdrisBound{t} in the type below)
in a buffer, we can easily write it to a file be it for safekeeping
or sending over the network.

\ExecuteMetaData[SaferIndexed.idr.tex]{writeToFile}

\begin{remark}[Forall Quantifier]
  The \IdrisKeyword{forall} quantifier is sugar for an implicit
  binder at quantity \IdrisKeyword{0}.
  %
  It can be useful to introduce variables that cannot be automatically
  bound in a prenex manner because they have a type dependency over an
  explicitly bound argument.
\end{remark}


We first start by reading the size of the header stored in the buffer.
%
This allows us to compute both the \IdrisBound{start} of the data block
as well as the size of the buffer (\IdrisBound{bufSize}) that will
contain the header followed by the tree we want to write to a file.
%
We then check whether the position of the pointer is exactly the beginning
of the data block.
%
If it is then we are pointing to the whole tree and the current buffer can
be written to a file as is.
%
Otherwise we are pointing to a subtree and need to separate it from its
surrounding context first.
%
To do so we allocate a new buffer of the right size and use the
standard library's \IdrisFunction{copyData} primitive to copy the raw bytes
corresponding to the header first, and the tree of interest second.
%
We can then write the buffer we have picked to a file and happily succeed.

\subsection{Reading a Tree from a File}

We can also go in the other direction: using the data description
\IdrisBound{cs}, we can load the content of the file located at
\IdrisBound{fp} as a pointer to the root of the tree of type
(\IdrisType{Mu} \IdrisBound{cs}) we claim is contained in it.

\ExecuteMetaData[SaferIndexed.idr.tex]{readFromFile}

This function takes a default argument \IdrisBound{safe} controlling
whether we should attempt to check that the file starts with a header
containing a type descripton matching the one passed as an argument
by the caller.

\begin{remark}[Default Arguments]
  An implicit argument can be assigned a \IdrisKeyword{default} value.
  It will take this value unless explicitly overwritten by the caller.
\end{remark}

First, we create a buffer from the content of the file.
We then read the offset giving us the size of the header.
If we want to be \IdrisBound{safe} we then read the type description
contained in the header using \IdrisFunction{getData} (not shown here)
and check it for equality against the one we were passed using \IdrisFunction{eqData}
(not shown here).
%
If the check fails, we emit an error and fail.
%
If the check succeeds, we postulate the existence of a runtime irrelevant
tree meant to represent the file's content and put together a pointer
for that asbtract tree.
%
This postulate cements the user's claim about the file's content;
naturally if the file is not in fact contain a serialised representation
of a tree this can lead to fatal errors later on when attempting to
inspect the buffer's content.
%
We return a pair of the runtime irrelevant tree and a pointer to it
thus ensuring users cannot directly attempt to match on the inductive
tree; they will have to use the combinators we are about to define to
inspect it by reading into the buffer.

Note that, in order to save space in the paper, we never checked whether
the buffer reads we performed in both \IdrisFunction{writeToFile} and
\IdrisFunction{readFromFile} were within bounds.
%
A released version of the library would naturally need to include such
checks.


Now that we have pointers,
and use files to read and write the trees they are standing for,
we are only missing the ability to look at the content they are pointing to.
%
But first we need to introduce some basic tools
to be able to talk precisely about this stored content.
