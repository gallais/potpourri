\begin{abstract}
In typed functional languages, one can typically only manipulate data
in a type-safe manner if it first has been deserialised into an in-memory
tree internally represented as a graph of nodes-as-structs and subterms-as-pointers.
%
We demonstrate how we can use Quantitative Type Theory
as implemented in the dependently typed programming language \idris{} to define
a small universe of serialised datatypes, and provide generic programs
allowing users to process values stored contiguously in buffers.
%
The code manipulating buffer-bound values is extremely similar
to its pure counterpart processing inductive structures thus
allowing a seamless user experience.
%
Our approach allows implementors to prove the full functional
correctness by construction of the IO functions processing the
data stored in the buffer.
%
We finally observe how this approach gives us a significant speedup
for functions that do not need to explore the full tree.

This work has been implemented in Idris 2 and fully ported to Agda,
allowing programs written in the two languages to easily exchange
data.
\end{abstract}
