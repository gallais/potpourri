\section{Interlude: Logics for Imperative Programs}

Let us first look back at historical frameworks built to explicitly
talk about memory locations and their contents.
%
Their design and properties will then inform our choices.

\subsection{Hoare Logic}

Hoare logic~\citeyearpar{DBLP:journals/cacm/Hoare69}  is a framework in
which we can state and prove the properties of imperative programs.
%
Its central notion is that of Hoare triples. They are statements of
the form
\[ \Hoare{P}{e}{v}{Q} \]
declaring that under the precondition
$P$, and binding the result of evaluating the expression $e$ as $v$,
we can prove that $Q$ holds.

The most basic of predicates to express knowledge about the memory
state is a `points to' assertion (\Pointer{\ell}{}{w}) stating that
the label $\ell$ points to a memory location containing the word $w$.
%
\label{sec:deref}
This can be used for instance to specify the behaviour of a \texttt{deref}
primitive: if $\ell$ is known to point to a value $w$ then evaluating
the program (\texttt{deref}($\ell$)) will return a value $v$ equal to $w$.
\[ \Hoare{\Pointer{\ell}{}{w}}{\texttt{deref}(\ell)}{v}{v = w} \]

Provided that we only have read-only operations, evaluating a program cannot
invalidate assumptions about the contents of the memory and so we enjoy a
form of weakening: for any proposition $R$, if we can specify the behaviour
of a program $e$ using the precondition $P$ and the postcondition $Q$,
then the same program will also abide by the specification using the
precondition ($P \wedge R$) and the postcondition ($Q \wedge R$).
\[ \Hoare{P}{e}{v}{Q} \quad \Longrightarrow \quad \Hoare{P \wedge R}{e}{v}{Q \wedge R} \]
In particular, the validity of (\Pointer{\ell}{}{w}) can be propagated even
after having dereferenced it.

Using further structural rules such as the one below for a
\texttt{let} construct, we can combine the axioms to prove
statements about more complex programs.

\newcommand{\letin}[3]{\texttt{let}\,#1\,=\,#2~\texttt{in}~#3}

\begin{gather*}
  \Hoare{P}{e}{v}{Q} \wedge \Hoare{Q(x)}{e^{\prime}}{v}{R} \\
  \Longrightarrow \Hoare{P}{\letin{x}{e}{e^{\prime}}}{v}{R}
\end{gather*}

We can for instance prove that we can chase pointers:
if $\ell$ points to a value $\ell^{\prime}$ itself corresponding
to a label pointing to $w$, nested calls to \texttt{deref}
will return a value equal to $w$:
\[ \Hoare
     {\Pointer{\ell}{}{\mathit{\ell^{\prime}}} \wedge \Pointer{\ell^{\prime}}{}{\mathit{w}}}
     {\letin{x}{\texttt{deref}(\ell)}{\texttt{deref}(x)}}{v}{v = w}
\]

Once we have these building blocks, it becomes interesting to introduce
some abstractions. As we explained above, \Pointer{\ell}{}{w} states that
the memory location called $\ell$ contains the word $w$.
%
The programs we are interested in do not however manipulate isolated words,
they talk about full blown inductive trees.
%
The key idea is to use this base `points to' assertion as a building
block to compositionally give a meaning to richly typed pointers.
We will write (\Pointer{\ell}{\mathit{ty}}{v}) for the pointer
to a value $v$ of type $\mathit{ty}$.
%
We can naturally define (\Pointer{\ell}{\text{\IdrisType{Bits8}}}{w}) as
simply (\Pointer{\ell}{}{w}).
%
We can then decide that, assuming that we know the size $s_a$
of values of type $a$, (\Pointer{\ell}{(a,~ b)}{(v_1, v_2)})
means that $\ell$ points to two contiguous values $v_1$ and $v_2$.
%
In other words it is an alias for
($\Pointer{\ell}{a}{v_1} \wedge \Pointer{\ell+s_a}{b}{v_2}$).


These richly typed pointers are the kind of pointers we are going to
define in the next section.
%
But before, let us have a look at how things get more complicated
once we are given access to primitives that can destructively update
the contents of a memory location.

\subsection{Separation Logic}

It is tempting to add a new primitive \texttt{assign} that takes a
label $\ell$ and a word $w$ and updates the memory cell so that
(\Pointer{\ell}{}{w}) now holds true. Its specification would be
\[ \Hoare{\Pointer{\ell}{}{\_}}{\texttt{assign}\,(\ell,\,w)}{\_}{\Pointer{\ell}{}{w}} \]
However with such a primitive things that were true before a program
was run may have been invalidated in the process.
%
Consequently the weakening principle given above does not hold anymore.
Otherwise the following reasoning step would be valid and the memory cell
corresponding to the label $\ell$ would need to contain both $0$ and $1$.
\begin{gather*}
  \Hoare{\Pointer{\ell}{}{0}}{\texttt{assign}\,(\ell,\,1)}{\_}{\Pointer{\ell}{}{1}}
  \\\quad \centernot\Longrightarrow \quad
   \Hoare{\Pointer{\ell}{}{0} \wedge \Pointer{\ell}{}{0}}{\texttt{assign}\,(\ell,\,1)}{\_}{\Pointer{\ell}{}{1} \wedge \Pointer{\ell}{}{0}}
\end{gather*}

This makes Hoare logic highly non-compositional in the presence of
destructive updates: having proven a program
fragment's specification using predicates mentioning the memory locations
it interacts with, this result cannot be reused when proving the specification
of a larger program involving additional memory locations.

The state-of-the-art solution is to move away from Hoare logic and use
separation logic~\citep{DBLP:conf/lics/Reynolds02,DBLP:journals/cacm/OHearn19,DBLP:books/hal/Chargueraud23,MANUAL:book/sfoundations/Chargueraud23}
whose core idea is to have a \emph{separating} conjunction: ($P \ast Q$)
states that both $P$ and $Q$ hold true but that they are talking about
\emph{separate} slices of the memory.
%
This allows for the safe inclusion of a similar weakening principle,
the \emph{frame rule} which states that for an arbitrary predicate $R$
if $e$ can be specified using precondition $P$ and postcondition $Q$
then it also can for precondition ($P \ast R$) and postcondition ($Q \ast R$)
precisely because $R$ is talking about memory locations not impacted by $e$.

As Rouvoet\citeyearpar{DBLP:phd/basesearch/Rouvoet21}
demonstrated, it is possible to embed separation logic in
type theory but, even with best efforts, it remains a somewhat
heavy process. Luckily for us, we do not need to.

Our goal with this work is to process data serialised in buffers
without first deserialising it as an in-memory tree.
To do so we only need to be able to read from the buffer to
analyse the structure of the inductive data stored in it.
%
Although we may \emph{want} to have destructive updates in future
work, for this specific task we do not actually \emph{need} them.
%
Correspondingly, we will happily stick to a Hoare-style logic and
its really powerful weakening principle.
%
As a matter of fact, we will go even further and entirely bypass the
usual explicit embedding of such logics in the host language.
%
We will use \idris{}'s abstraction facilities to introduce richly typed
`points to' predicates as first class values that can just be passed
around, and trusted primitives that can take these pointers and reveal
the shape of the values they are pointing to.
