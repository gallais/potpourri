
\subsection{Generic Fold}\label{sec:bufferfold}

The implementation of the generic \IdrisFunction{fold} over a tree stored
in a buffer is going to have the same structure as the generic fold over
inductive values: first match on the top constructor, then use \IdrisFunction{fmap}
to apply the fold to all the substructures and, finally, apply the algebra to
the result.
%
We start by implementing the buffer-based counterpart to \IdrisFunction{fmap}.
Let us go through the details of its type first.

\ExecuteMetaData[SaferIndexed.idr.tex]{fmaptype}

The first two arguments to \IdrisFunction{fmap} are similar to its pure
counterpart:
a description \IdrisBound{d}
and a (here runtime-irrelevant) function \IdrisBound{f}
to map over a \IdrisFunction{Meaning}.
%
Next we take a function which is the buffer-aware counterpart to \IdrisBound{f}:
given any runtime-irrelevant term \IdrisBound{t} and a pointer to it in a buffer,
it returns an \IdrisType{IO} process computing the value (\IdrisBound{f} \IdrisBound{t}).
%
Finally, we take a runtime-irrelevant meaning \IdrisBound{t}
as well as a pointer to its representation in a buffer and compute
an \IdrisType{IO} process which will return a value equal to
(\IdrisFunction{Data.fmap} \IdrisBound{d} \IdrisBound{f} \IdrisBound{t}).

We can now look at the definition of \IdrisFunction{fmap}.

\ExecuteMetaData[SaferIndexed.idr.tex]{fmapfun}

We poke the buffer to reveal the value the \IdrisType{Pointer.Meaning}
named \IdrisBound{ptr} is pointing at and then dispatch over the description
\IdrisBound{d} using the \IdrisFunction{go} auxiliary function.

If the description is \IdrisData{None} we use \IdrisFunction{byIrrelevance}
which happily builds any (\IdrisType{Singleton} \IdrisBound{t}) provided
that \IdrisBound{t}'s type is proof irrelevant.

If the description is \IdrisData{Byte}, the value is left untouched and so
we can simply return it immediately.

If we have a \IdrisData{Prod} of two descriptions, we recursively apply
\IdrisFunction{fmap} to each of them and pair the results back.

Finally, if we have a \IdrisData{Rec} we apply the function operating
on buffers that we know performs the same computation as \IdrisBound{f}.


We can now combine \IdrisFunction{out} and \IdrisFunction{fmap} to compute
the correct-by-construction \IdrisFunction{fold}: provided an algebra for
a datatype \IdrisBound{cs} and a pointer to a tree of type \IdrisBound{cs}
stored in a buffer, we return an \IdrisType{IO} process computing the fold.

\ExecuteMetaData[SaferIndexed.idr.tex]{foldtype}

We first use \IdrisFunction{out} to reveal the constructor choice in the
tree's top node, we then recursively apply (\IdrisFunction{fold} \IdrisBound{alg})
to all the substructures by calling \IdrisFunction{fmap}, and we conclude by
applying the algebra to this result.

\ExecuteMetaData[SaferIndexed.idr.tex]{foldfun}

We once again (cf. \Cref{sec:genericfoldinductive}) had to
use \assertTotal{} because it is not obvious to
\idris{} that \IdrisFunction{fmap} only uses its argument on subterms.
%
This could have also been avoided by mutually defining \IdrisFunction{fold}
and a specialised version of
(\IdrisFunction{fmap} \IdrisKeyword{(}\IdrisFunction{fold} \IdrisBound{alg}\IdrisKeyword{)})
at the cost of code duplication and obfuscation.
%
We once again include such a definition in \Cref{appendix:safefold}.
