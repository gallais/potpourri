\section{Meaning as Trees}\label{sec:trees}

We now see descriptions as functors and, correspondingly,
datatypes as the initial objects of the associated functor-algebras.
%
This is a standard construction derived from Malcolm's
work~\citeyearpar{DBLP:journals/scp/Malcolm90},
itself building on Hagino's categorically-inspired
definition of a lambda calculus
with a generic notion of datatypes~\citep{DBLP:conf/ctcs/Hagino87}.

In our work these trees will be used primarily to allow users to
give a precise specification of the IO procedures they actually want
to write in order to process values stored in buffers.
%
We expect these inductive trees and the associated generic programs
consuming them to be mostly used at the \IdrisKeyword{0}
modality i.e. to be erased during compilation.

\subsection{Descs as Functors}

We are going to define the meaning of descriptions as
strictly positive endofunctors on \IdrisType{Type} by
induction on said descriptions.
%
\IdrisData{None} and \IdrisData{Prod} will respectively
be interpreted by a unit type (\IdrisType{True}) and a
a product type (\IdrisType{Tuple}) defined below.
%
We do not use the unit and product from the standard
library purely to offer better syntactic sugar to
our users.

\noindent
\begin{minipage}[t]{.4\textwidth}
  \ExecuteMetaData[Lib.idr.tex]{true}
\end{minipage}\hfill
\begin{minipage}[t]{.5\textwidth}
  \ExecuteMetaData[Lib.idr.tex]{pair}
\end{minipage}

\begin{remark}[Lack of Eta-rules for Records]\label{rmk:etarecords}
  \IdrisType{True} and \IdrisType{Tuple} are both defined
  as \IdrisKeyword{record}s.
  %
  However in \idris{}, this is currently only syntactic sugar
  to declare a datatype together with a projection for each
  of the record's fields.
  %
  Conversion checking does not incorporate eta rules for records
  and so we have to manually state, prove, and deploy eta rules
  whenever necessary.
  %
  We include the two lemmas below as we will need them later.

  \ExecuteMetaData[Lib.idr.tex]{etaTrue}
  \ExecuteMetaData[Lib.idr.tex]{etaTuple}
\end{remark}

Now that we have these two type constructors, we can
define \IdrisFunction{Meaning}, the function giving
us the action on objects of the \IdrisType{Desc}-encoded
endofunctors.
%
In its type, all of \IdrisBound{r}, \IdrisBound{s},
and \IdrisBound{o} are implicitly universally quantified,
a feature of \idris{} we explain below.

\ExecuteMetaData[Serialised/Desc.idr.tex]{meaning}

\begin{remark}[Implicit Prenex Polymorphism]\label{rmk:prenexpoly}
  Lowercase names that are seemingly unbound are automatically
  quantified over in a prenex manner reminiscent of other functional
  languages like OCaml or Haskell.
  These variables are bound at quantity 0, meaning that they will
  be automatically erased during compilation.
\end{remark}

Both \IdrisData{None} and \IdrisData{Byte} are interpreted by constant
functors (respectively the one returning the unit type \IdrisType{True},
and the one returning the type of bytes).
%
(\IdrisData{Prod} \IdrisBound{d} \IdrisBound{e})
is interpreted as the pairing of the interpretation of
\IdrisBound{d} and \IdrisBound{e} respectively.
%
Finally \IdrisData{Rec} is the identity functor.

Now that we have the action of descriptions on types,
let us see their action on morphisms: provided a
function from \IdrisBound{a} to \IdrisBound{b}, we can
build one from the \IdrisFunction{Meaning} of \IdrisBound{d}
at type \IdrisBound{a} to its meaning at type \IdrisBound{b}.
%
We once again proceed by induction on the description.

\ExecuteMetaData[Serialised/Desc.idr.tex]{fmap}\label{def:fmap}

All cases but the one for \IdrisData{Rec} are structural.
%
We leave out the proofs verifying that these definitions
respect the functor laws up to pointwise equality.
They are included in the supplementary material
(\suppfile{Data.Serialisable.Data}).

\subsection{Data as Trees}

Given a datatype description \IdrisBound{cs}, our first goal is
to define what it means to pick a constructor.
%
The \IdrisType{Index} record is a thin wrapper around a finite
natural number known to be smaller than the number of constructors
this type provides.

\ExecuteMetaData[Serialised/Desc.idr.tex]{index}

We use this type rather than \IdrisType{Fin} directly because it
plays well with inference.
%
In the following code snippet, implementing a function returning
the description corresponding to a given index,
we use this to our advantage:
the \IdrisBound{cs} argument can be left implicit because it already
shows up in the type of the \IdrisType{Index} and can thus be
reconstructed by unification~\citep{DBLP:conf/tlca/AbelP11}.

\ExecuteMetaData[Serialised/Desc.idr.tex]{description}

The \IdrisFunction{index} function is provided by the standard
library: given a position of type (\IdrisType{Fin} \IdrisBound{n})
and a vector of size \IdrisBound{n}, it returns the value located
at that position.
%
Note that we seem to define \IdrisFunction{description}
in terms of itself, in a manner that would create an
infinite loop.
%
But the occurrence on the right-hand side is actually
referring to the projection out of the
\IdrisType{Constructor} record.
%
This is possible thanks to Idris' type-directed disambiguation.

\begin{remark}[Type-directed Disambiguation]
  If multiple definitions in scope have the same name,
  Idris performs type-directed disambiguation to pick
  the only one that would work in that context.
\end{remark}

This type of indices also allows us to provide users with
syntactic sugar enabling them to use the constructors' names
directly rather than confusing numeric indices.
%
The following function runs a semi-decision procedure
\IdrisFunction{isConstructor} (whose implementation is
not given here) at the type level
in order to turn any raw string \IdrisBound{str}
into the corresponding \IdrisType{Index}.

\ExecuteMetaData[Serialised/Desc.idr.tex]{fromString}

\begin{remark}[Ad-hoc Polymorphisms for Literals]
  String, numeric, and floating point literals are respectively
  desugared using \IdrisFunction{fromString}, \IdrisFunction{fromInteger}, and
  \IdrisFunction{fromDouble}.
  Combined with the type-directed disambiguation (see above)
  of overloaded symbols, this allows users to compute data from literals.
\end{remark}

In this instance, this allows us to use string literals as
proxies for constructor names.
%
If the string literal stands for a valid name then
\IdrisFunction{isConstructor} will
return a valid \IdrisType{Index} and the compiler's proof
search mechanism will be able to
\IdrisKeyword{auto}matically fill-in the implicit proof.
%
If the name is not valid then \idris{} will not
find the index and will raise a compile time error.
%
We include a successful example on the left and a failing test
on the right hand side where the compiler is not able to find
a proof of (\IdrisFunction{isJust} \IdrisKeyword{(}
\IdrisFunction{isConstructor} \IdrisData{"Cons"}
\IdrisFunction{Tree}\IdrisKeyword{)}).

\noindent
\begin{minipage}[t]{0.35\textwidth}
  \ExecuteMetaData[Serialised/Desc.idr.tex]{indexleaf}
\end{minipage}\hfill
\begin{minipage}[t]{0.42\textwidth}
\ExecuteMetaData[Serialised/Desc.idr.tex]{notindexcons}
\end{minipage}

\begin{remark}[Failing Blocks]
  A \IdrisKeyword{failing} block is a way to document (and
  enforce) that some code leads to an error.
  Such blocks are only accepted if their body parses but
  leads to an error during elaboration.
\end{remark}

Once equipped with the ability to pick constructors, we can define
the type of algebras for the functor described by a \IdrisType{Data}
description. For each possible constructor, we demand an algebra for
the functor corresponding to the meaning of the  constructor's description.

\ExecuteMetaData[Serialised/Desc.idr.tex]{alg}

We can then introduce the fixpoint of data descriptions as the initial
algebra, defined as the following inductive type.

\ExecuteMetaData[Serialised/Desc.idr.tex]{mu}

\begin{remark}[Escape hatches]
  \idris provides some escape hatches to use when the author
  knows a usage is safe but the compiler is not able to
  determine that it is.

  A function call of the form (\assertTotal{} $e$) will
  circumvent the termination and positivity checkers in
  the expression $e$.
  This is only safe if the function is actually terminating
  or the type strictly positive.

  A function call of the form (\believeMe{} $e$) will be
  usable at any type. This is only safe if $e$'s actual type
  and the type it is being used at have the same runtime
  representation.
\end{remark}

Note that here we are forced to use \assertTotal{} to convince \idris{}
to accept the definition.
%
Indeed, unlike Agda, \idris{} does not (yet!) track whether a function's
arguments are used in a strictly positive manner.
%
Consequently the positivity checker
is unable to see that \IdrisFunction{Meaning} uses its second
argument in a strictly positive manner
and that this is therefore a legal definition.

Now that we can build trees as fixpoints of the
meaning of descriptions, we can define convenient aliases for
the \IdrisFunction{Tree} constructors.
%
Note that the leftmost \IdrisData{(\#)} use in each definition corresponds
to the \IdrisType{Mu} constructor while later ones are \IdrisType{Tuple}
constructors.
%
\idris{}'s type-directed disambiguation of constructors allows us to use
this uniform notation for all of these pairing notions.

\noindent
\begin{minipage}[t]{.3\textwidth}
  \ExecuteMetaData[Serialised/Desc.idr.tex]{leaf}
\end{minipage}\hfill
\begin{minipage}[t]{.65\textwidth}
  \ExecuteMetaData[Serialised/Desc.idr.tex]{node}
\end{minipage}

This enables us to define our running example as an inductive value:

\ExecuteMetaData[Serialised/Desc.idr.tex]{longexample}

\subsection{Generic Fold}\label{sec:genericfoldinductive}

\IdrisType{Mu} gives us the initial fixpoint for these algebras i.e.
given any other algebra over a type \IdrisBound{a}, from a term of
type (\IdrisType{Mu} \IdrisBound{cs}), we can compute an \IdrisBound{a}.
%
We define the generic \IdrisFunction{fold} function over inductive values
as follows:

\ExecuteMetaData[Serialised/Desc.idr.tex]{fold}

We first match on the term's top constructor, use \IdrisFunction{fmap}
(defined in \Cref{def:fmap})
to recursively apply the fold to all the node's subterms and finally
apply the algebra to the result.
%
Here we only use \assertTotal{} because \idris{} does not see that
\IdrisFunction{fmap} only applies its argument to strict subterms.
This limitation could easily be bypassed by mutually defining
an inlined and specialised version of
(\IdrisFunction{fmap} \IdrisKeyword{\KatlaUnderscore} (\IdrisFunction{fold} \IdrisBound{alg}))ma
as we demonstrate in \Cref{appendix:safefold}.
%
In an ideal type theory these supercompilation steps, whose sole
purpose is to satisfy the totality checker, would be automatically
performed by the compiler~\citep{MANUAL:phd/dublin/Mendel12}.

%\todo{Include induction example \& move \IdrisFunction{All} here?}

Further generic programming can yield other useful programs e.g. a
generic proof that tree equality is decidable or a generic definition
of zippers~\citep{DBLP:conf/icfp/LohM11}.
