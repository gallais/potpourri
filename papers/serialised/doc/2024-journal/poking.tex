\section{Inspecting a Buffer's Content}\label{sec:poking}

Let us now describe the combinators allowing our users to
take apart the values they have a pointer for.
%
These functions will read bytes in the buffer and reflect
the observations thus made at the type level by refining
the pointer's indices.

There will be two separate tiers of definitions: the most
basic building blocks (\IdrisFunction{poke} and \IdrisFunction{out})
will be a trusted core of primitives implemented using
escape hatches.
%
This is inevitable given that we are reflecting the content
of buffer reads at the type level.
%
We will clearly specify their behaviour by explaining what
benign Hoare-style axioms they correspond to.

We will then show how we can use these low-level trusted
primitives to define higher level combinators
(\IdrisFunction{layer} and \IdrisFunction{view}).
%
Crucially these definitions will not need to use further
escape hatches: provided that the trusted core is correct,
then so will they.
%
This will culminate in the implementation of a generic
correct-by-construction version of \IdrisFunction{fold}
operating over trees stored in a buffer (cf. \Cref{sec:bufferfold}).

\subsection{Poking the Buffer}\label{sec:poke}

Our most basic operation consists in poking the buffer to gain
access to the head constructor of the underlying layer of
\IdrisFunction{Data.Meaning} we have a pointer to.
%
This operation is description-directed and so its result (called
\IdrisFunction{Poke}) is defined by case analysis on the description
associated to the pointer.

Concretely, the type of the function is as follows: provided a pointer for
a description \IdrisBound{d}, subtrees of type \IdrisBound{cs} and an
associated meaning \IdrisBound{t} of type
(\IdrisFunction{Meaning} \IdrisBound{cs} \IdrisBound{t})
we return an \IdrisType{IO} process computing the one step
unfolding of the meaning.

\ExecuteMetaData[SaferIndexed.idr.tex]{pokefun}

As we explained, \IdrisFunction{Poke} is defined by case-analysis
on the description.
%
However, in order to keep the notations user-friendly, we
are forced by Idris' lack of eta-rules (cf. \cref{rmk:etarecords})
to mutually define
an inductive family \IdrisType{Poke'} with interesting return
indices.
%
It will allow users to, by matching on \IdrisType{Poke'}
constructors, automatically refine the associated meaning
present at the type level into a term with a head constructor.
%
This will ensure that functions defined by pattern-matching
can reduce in types based on observations made at the term
level.

\ExecuteMetaData[SaferIndexed.idr.tex]{pokedatafun}

Poking a buffer containing a \IdrisData{Byte} will yield a
runtime-relevant copy of the type-level byte we have for
reference, hence the use of \IdrisType{Singleton}.
This corresponds to adding the following Hoare-style axiom
for \IdrisFunction{poke}. Remembering that
(\IdrisFunction{Meaning} \IdrisData{Byte} (\IdrisType{Mu} \IdrisBound{cs}))
computes to \IdrisType{Bits8} and so that
(\Pointer{\ell}{\text{\IdrisFunction{Meaning} \IdrisData{Byte} (\IdrisType{Mu} \IdrisBound{cs})}}{w})
is essentially (\Pointer{\ell}{}{w}), we
note that this Hoare-style axiom looks eerily
similar to the axiom for \texttt{deref} we gave in
\Cref{sec:deref}):
\[ \Hoare
     {\Pointer{\ell}{\text{\IdrisFunction{Meaning} \IdrisData{Byte} (\IdrisType{Mu} \IdrisBound{cs})}}{w}}
     {\text{\IdrisFunction{poke}}~\ell}
     {v}{v = w}
\]
If the description is \IdrisData{Rec} this means
we have a substructure. In this case we simply demand
a pointer to it. This amounts to adding the following
axiom:
\[ \Hoare
     {\Pointer{\ell}{\text{\IdrisFunction{Meaning} \IdrisData{Rec} (\IdrisType{Mu} \IdrisBound{cs})}}{t}}
     {\text{\IdrisFunction{poke}}~\ell}
     {\ell^{\prime}}{\Pointer{\ell^{\prime}}{\text{\IdrisType{Mu} \IdrisBound{cs}}}{t}}
\]

Last but not least, if we are accessing a value of a
record type (either \IdrisData{None} or a \IdrisData{Prod} of two descriptions)
then we describe the resulting observation using the \IdrisType{Poke'} family.

\ExecuteMetaData[SaferIndexed.idr.tex]{pokedatadata}

\IdrisType{Poke'} has two constructors corresponding to the two
descriptions it covers.
%
If the description of the buffer's content is \IdrisData{None}
then we do not expect to get a value back, only the knowledge
that the type-level meaning is \IdrisData{I}. This corresponds
to adding the following axiom.
\[ \Hoare
     {\Pointer{\ell}{\text{\IdrisFunction{Meaning} \IdrisData{None} (\IdrisType{Mu} \IdrisBound{cs})}}{t}}
     {\text{\IdrisFunction{poke}}~\ell}
     {\_}{t = \text{\IdrisData{I}}}
\]
If the description is (\IdrisData{Prod} \IdrisBound{d} \IdrisBound{e})
then we demand to learn that the type-level term is \IdrisData{(\#)}-headed
with two substructures $t_1$ and $t_2$ and we expect
\IdrisFunction{poke} to give us a pointer to each of these substructures.
This corresponds to the following axiom.
\begin{gather*}
  \Hoare
     {\Pointer{\ell}{\text{\IdrisFunction{Meaning} (\IdrisData{Prod} \IdrisBound{d} \IdrisBound{e}) (\IdrisType{Mu} \IdrisBound{cs})}}{t}}
     {\\\text{\IdrisFunction{poke}}~\ell\\}
     {(\ell_1,~\ell_2)}
     {\exists t_1.~ \exists t_2.~
       t = (t_1 ~\text{\IdrisData{\#}}~ t_2)
       \wedge \Pointer{\ell_1}{\text{\IdrisFunction{Meaning} \IdrisBound{d} (\IdrisType{Mu} \IdrisBound{cs})}}{t_1}
       \wedge \Pointer{\ell_2}{\text{\IdrisFunction{Meaning} \IdrisBound{e} (\IdrisType{Mu} \IdrisBound{cs})}}{t_2}}
\end{gather*}


As we mentioned earlier, the \idris{} implementation of the
\IdrisFunction{poke} function
is necessarily using escape hatches as we are essentially
giving a computational content to the axioms listed above.
%
We proceed by case analysis on the description.
%
Let us go through each case one-by-one.

\ExecuteMetaData[SaferIndexed.idr.tex]{pokefunNone}

If the description is \IdrisData{None} we do not need to fetch any
information from the buffer but we do need to deploy the eta rule
for \IdrisType{True} (cf. \Cref{fig:etatrue} for the definition
of \IdrisFunction{etaTrue})
in order to be able to use the \IdrisType{Poke'}
constructor \IdrisData{I}.

\ExecuteMetaData[SaferIndexed.idr.tex]{pokefunByte}

If the description is \IdrisData{Byte} then we read a byte at the
determined position. The only way we can connect this value we just
read to the runtime irrelevant type index is to use the unsafe combinator
\IdrisPostulate{unsafeMkSingleton} to manufacture a value of type
(\IdrisType{Singleton} \IdrisBound{t}) instead of the value of type
(\IdrisType{Singleton} \IdrisBound{bs})
we would expect from wrapping \IdrisBound{bs} in the \IdrisData{MkSingleton} constructor.
%
As we explained earlier, this amounts to realising the
Hoare-style axiom specifying the act of dereferencing a pointer.

\ExecuteMetaData[SaferIndexed.idr.tex]{pokefunProd}
If the description is the product of two sub-descriptions then we
want to compute the \IdrisType{Pointer.Meaning} corresponding to
each of them.
%
We do so by following the serialisation format we detailed in
\Cref{sec:hexdump}.
%
We start by splitting the vector of offsets to distribute them between
the left and right subtrees.

We can readily build the pointer for the \IdrisBound{left} subdescription:
it takes the left offsets, the buffer, and has the same starting position
as the whole description of the product as the submeanings are stored one after the other.
Its size (called \IdrisBound{sizel}) is the sum of the space reserved
by all of the left offsets (\IdrisFunction{sum} \IdrisBound{subl})
as well as the static size occupied by the rest of the content
(\IdrisBound{sl}).

We then compute the starting position of the right subdescription: we need to
move past the whole of the left subdescription, that is to say that the starting
position is the sum of the starting position for the whole product and \IdrisBound{sizel}.
%
The size of the right subdescription is then easily computed by subtracting
\IdrisBound{sizel} from the overall \IdrisBound{size} of the paired subdescriptions.

We can finally use the lemma \IdrisFunction{etaTuple}
(defined in \Cref{fig:etatuple}) saying that a tuple
is equal to the pairing of its respective projections
in order to turn \IdrisBound{t} into
(\IdrisFunction{fst} \IdrisBound{t} \IdrisData{\#} \IdrisFunction{snd} \IdrisBound{t})
which lets us use the \IdrisType{Poke'} constructor \IdrisData{(\#)} to return our
pair of pointers.

Although we did not need to use escape hatches here, the implementation
is still part of the trusted core in that we are computing offsets in
(we claim!) accordance with the serialisation format.

\ExecuteMetaData[SaferIndexed.idr.tex]{pokefunRec}

Lastly, when we reach a \IdrisData{Rec} description, we can discard the
vector of offsets and return a \IdrisType{Pointer.Mu} with the same buffer,
starting position and size as our input pointer.

\subsection{Extracting One Layer}

By repeatedly poking the buffer, we can unfold a full layer of term.
This operation is not part of the trusted core: provided that
\IdrisFunction{poke} is correct then it will automatically be
correct-by-construction.
The result type is once again defined by induction on the description.
It is essentially identical to the definition of
\IdrisFunction{Poke} except for the \IdrisData{Prod} case:
instead of being content with a pointer for each of the
subdescriptions, we demand a \IdrisFunction{Layer} for them too.

\ExecuteMetaData[SaferIndexed.idr.tex]{layerdata}

This function can easily be implemented by induction on the description
and repeatedly calling \IdrisFunction{poke} to expose the values one by
one.
%
We call \IdrisFunction{poke} and use the \IdrisType{IO} monad's bind
operator (\IdrisFunction{>>=}) to pass the result to \IdrisFunction{go},
the auxiliary function recursively going under record constructors to
perform further poking.

\ExecuteMetaData[SaferIndexed.idr.tex]{layerfun}

If the description is \IdrisData{None}, \IdrisData{Byte}, or
\IdrisData{Rec} then poking the buffer was already enough to
reveal the full layer and we can simply return the result.
%
If we have a (\IdrisData{Prod} \IdrisBound{d} \IdrisBound{e})
then poking the buffer revealed that the term is \IdrisData{(\#)}-headed
and handed us a pointer to each of its components.
%
We call \IdrisFunction{layer} recursively on each of these pointers and
use idiom brackets to combine the \IdrisType{IO}-wrapped results
using \IdrisType{Layer'}'s \IdrisData{(\#)} constructor.

We can readily use this function to inspect meanings
we have a pointer to. In the following artificial example,
looking at the goal \texttt{?hole}, we learn that
\IdrisBound{v} has the shape {\IdrisKeyword{(}\IdrisBound{t} \IdrisData{\#} \IdrisKeyword{(}\IdrisData{I} \IdrisData{\#} \IdrisBound{u}\IdrisKeyword{))}}
where \IdrisBound{t} is a tree of type (\IdrisType{Mu} \IdrisBound{cs})
and \IdrisBound{u} is a byte, and that
\IdrisBound{q} is a pointer to \IdrisBound{t}
and \IdrisBound{w} has type (\IdrisType{Singleton} \IdrisBound{u}).
\ExecuteMetaData[SaferIndexed.idr.tex]{layerexample}
We have effectively managed to take \IdrisBound{v} apart in a
type-directed manner and get a handle on its subterms.
In that sense we have just defined a view for meanings stored in buffers!

\subsection{Exposing a Tree's Top Node}\label{sec:out}

Now that we can deserialise an entire layer of \IdrisFunction{Meaning},
the only thing we are missing to be able to generically manipulate trees
is the ability to expose the top node of a tree stored at a
\IdrisType{Pointer.Mu} position.
%
This will require the addition of a new function to the trusted core:
the function \IdrisFunction{out}.
%
Its type states that given a pointer
to a tree \IdrisBound{t} of type \IdrisBound{cs} we can get an
\IdrisType{IO} process revealing the top node of \IdrisBound{t}.

\ExecuteMetaData[SaferIndexed.idr.tex]{outfun}

The \IdrisType{Out} family formally describes what revealing the
top node means:
obtaining an \IdrisType{Index} named \IdrisBound{k},
and a \IdrisType{Pointer.Meaning} to the meaning
\IdrisBound{t} of the description associated to \IdrisBound{k}.
%
The family's index (\IdrisBound{k} \IdrisData{\#} \IdrisBound{t})
(where the overloaded \IdrisData{(\#)} is here the
\IdrisType{Data.Mu} constructor)
ensures that the structure of the runtime irrelevant
tree is adequately described by this index and meaning.

\ExecuteMetaData[SaferIndexed.idr.tex]{outdata}

This amounts to introducing the following Hoare-style axiom for
\IdrisFunction{out}:
\[
\Hoare
    {\Pointer{\ell}{\text{\IdrisType{Mu} \IdrisBound{cs}}}{t}}
    {\text{\IdrisFunction{out}}\,\ell}
    {(k,\, \ell_1)}
    {\exists t_1.~ t = (k \,\text{\IdrisData{\#}}\, t_1)
      \wedge \Pointer{\ell_1}{\text{\IdrisFunction{Meaning} \IdrisBound{cs$_k$} (\IdrisType{Mu} \IdrisBound{cs})}}{t_1}}
\]

The function \IdrisFunction{out} being part of the trusted
core means that its implementation will once again need to
use escape hatches to reconcile the buffer's observed content
with the type-level indices.
%
Let us recall the data layout detailed in \Cref{fig:data-layout}:
\begin{center}
\begin{bytefield}[bitwidth=6mm, bitheight=7mm]{4}
  \bitbox{2}[bgcolor=Chartreuse4!40]{tag}
  & \bitbox{4}[bgcolor=lightgray!30]{$o$ offsets}
  & \bitbox[ltb]{2}{tree$_1$}
  & \bitbox[tb]{1}{$\cdots$}
  & \bitbox[tb]{2}[bgcolor=IndianRed1!40]{byte$_1$}
  & \bitbox[tb]{1}{$\cdots$}
  & \bitbox[tb]{2}{tree$_k$}
  & \bitbox[tb]{1}{$\cdots$}
  & \bitbox[tb]{2}[bgcolor=IndianRed1!40]{byte$_s$}
  & \bitbox[tbr]{2}{tree$_{o+1}$} \\
  \bitbox[l]{1}{\small$0$}
  & \bitbox[]{1}{}
  & \bitbox[l]{1}{\small$1$}
  & \bitbox[]{3}{}
  & \bitbox[l]{3}{\small$1+8\times{}o$}
  & \bitbox[]{6}{}
  & \bitbox[l]{5}{\small$8\times{}o+s+\Sigma_{i=1}^{o}o_i$}
\end{bytefield}

\end{center}
Operationally, \IdrisFunction{out}
will amount to inspecting the tag used by the node,
deserialising the offsets stored immediately after it,
and forming a pointer to the start of the meaning block.
%
As a first step, let us get our hands on the index of the head constructor.
\ExecuteMetaData[SaferIndexed.idr.tex]{getIndex}
We obtain a byte by calling \IdrisFunction{getBits8}, cast it to a
natural number and then make sure that it is in the range
$[\text{\IdrisData{0}} \cdots \text{\IdrisFunction{consNumber} \IdrisBound{cs}}[$ using
\IdrisFunction{natToFin}. If the check fails then we return
a hard fail: the buffer contains an invalid representation
and so the precondition that we only have pointers to valid
values was violated.

We can now describe the \IdrisFunction{out} function's implementation.
\ExecuteMetaData[SaferIndexed.idr.tex]{outfunbody}
We start by reading the index \IdrisBound{k}
corresponding to the constructor choice.
%
We then use the unsafe \IdrisPostulate{unfoldAs} postulate to step the
type-level \IdrisBound{t} to something of the form
(\IdrisBound{k} \IdrisData{\#} \IdrisBound{val}).
%
We then conclude using the \IdrisFunction{getConstructor} function
(defined later) to gather the required offsets and put together
the pointer to the meaning of the description associated to \IdrisBound{k}.

\ExecuteMetaData[Serialised/Desc.idr.tex]{unfoldAs}

The declaration of \IdrisPostulate{unfoldAs} is marked as runtime
irrelevant because it cannot possibly be implemented
(\IdrisBound{t} is runtime irrelevant and so cannot be inspected)
and so its output should not be relied upon in runtime-relevant
computations.
%
Its type states that there exists a \IdrisFunction{Meaning} called
\IdrisBound{val} such that \IdrisBound{t} is equal to
(\IdrisBound{k} \IdrisData{\#} \IdrisBound{val}).
%
This is of course untrue in general: we cannot take an arbitrary
\IdrisBound{t} and declare that it is \IdrisBound{k}-headed.
This use-case is however justified in that it reflects at the
type-level the observation we made by reading the buffer.

Now that we know the head constructor we want to deserialise and that
we have the ability to step the runtime irrelevant tree to match the
actual content of the buffer, we can use \IdrisFunction{getConstructor}
to build such a value.
%
Given a pointer to a tree (\IdrisBound{k} \IdrisData{\#} \IdrisBound{t}),
it will read enough information from the buffer to assemble a pointer to
the meaning \IdrisBound{t}.

\ExecuteMetaData[SaferIndexed.idr.tex]{getConstructor}

To get a pointer to the meaning \IdrisBound{t},
we start by getting the vector of offsets stored
immediately after the tag. We then compute the size of the remaining
\IdrisFunction{Meaning} description: it is the size of the overall tree,
minus $1$ (for the tag) and $8$ times the number of offsets (because
each offset is stored as an 8 bytes number).
%
We can then use the record constructor \IdrisData{MkMeaning} to pack
together the vector of offsets, the buffer, the position past the offsets
and the size we just computed.

\ExecuteMetaData[SaferIndexed.idr.tex]{getOffsets}

The implementation of \IdrisType{getOffsets} is straightforward: given
a continuation that expect \IdrisBound{n} offsets as well as the
position past the last of these offsets, we read the 8-bytes-long
offsets one by one and pass them to the continuation, making sure
that we move the current position by 8 bytes before every recursive call.

We can readily use this function to inspect a top-level
constructor in a correct-by-construction manner like in
the following example.

\noindent
\begin{minipage}[t]{.4\textwidth}
  \ExecuteMetaData[SaferIndexed.idr.tex]{pureisleaf}
\end{minipage}\hfill\begin{minipage}[t]{.5\textwidth}
  \ExecuteMetaData[SaferIndexed.idr.tex]{isleaf}
\end{minipage}\medskip

Re-using the \IdrisFunction{Tree} description we
introduced in \Cref{fig:treedesc}, we defined a
pure \IdrisFunction{isLeaf} function checking that
an inductive tree is a leaf, together with its
effectful equivalent using \IdrisFunction{out}
to reveal the tree's top node thus allowing the
type-level call to the pure \IdrisFunction{isLeaf}
to reduce in each of the lambda-case's branches.
%
The use of \IdrisType{Singleton} guarantees that
we indeed return the appropriate boolean for the
tree we are pointing to.

\subsection{Offering a Convenient View}\label{sec:dataview}

We can combine \IdrisFunction{out} and \IdrisFunction{layer} to obtain
the \IdrisFunction{view} function we used in our introductory examples
in~\Cref{sec:seamless}.
%
This operation is not part of the trusted core: provided that
\IdrisFunction{poke} and \IdrisFunction{out} are correct then
it will automatically be correct-by-construction.
%
A (\IdrisType{View} \IdrisBound{cs} \IdrisBound{t}) value gives us
access to the (\IdrisType{Index} \IdrisBound{cs}) of
\IdrisBound{t}'s top constructor together with the corresponding
\IdrisFunction{Layer} of deserialised values and pointers to subtrees.

\ExecuteMetaData[SaferIndexed.idr.tex]{viewdata}

The implementation of \IdrisFunction{view} is unsurprising: we use
\IdrisFunction{out} to expose the top constructor index and a
\IdrisType{Pointer.Meaning} to the constructor's payload.
%
We then user \IdrisFunction{layer} to extract the full
\IdrisFunction{Layer} of deserialised values that the
pointer references.

\ExecuteMetaData[SaferIndexed.idr.tex]{view}

It is worth noting that although a \IdrisFunction{view} may be
convenient to consume, a performance-minded user may decide to
directly use the \IdrisFunction{out} and \IdrisFunction{poke}
combinators to avoid deserialising values that they do not need.
%
We present a case study in \Cref{appendix:rightmost} comparing the
access patterns of two implementations of the function fetching the
byte stored in a tree's rightmost node depending on whether we use
\IdrisFunction{view} or the lower level \IdrisFunction{poke} combinator.

By repeatedly calling \IdrisFunction{view}, we can define the
correct-by-construction generic deserialisation function: by
using \IdrisType{Singleton}, its type guarantees that we turn
a pointer to a tree \IdrisBound{t} into a runtime value known
to be equal to \IdrisBound{t}.

\ExecuteMetaData[SaferIndexed.idr.tex]{deserialise}

We can measure the benefits of our approach by comparing the runtime
of a function directly operating on buffers to its pure counterpart
composed with a deserialisation step.
%
For functions like \IdrisFunction{rightmost} that only explore a
very small part of the full tree, the gains are spectacular: the
process operating on buffers is exponentially faster than its
counterpart which needs to deserialise the entire tree first
(cf. \Cref{sec:timing}).
